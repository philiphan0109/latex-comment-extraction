\documentclass{article}

% if you need to pass options to natbib, use, e.g.:
\PassOptionsToPackage{numbers, compress}{natbib}
% before loading nips_2017
%
% to avoid loading the natbib package, add option nonatbib:
% \usepackage[nonatbib]{nips_2017}

% Keep final for now to remove annoying line numbers.
%\usepackage[final]{nips_2017}
% Keep final for now to remove annoying line numbers.


% to compile a camera-ready version, add the [final] option, e.g.:
\usepackage[final]{nips_2017}

\usepackage[utf8]{inputenc} % allow utf-8 input
\usepackage[T1]{fontenc}    % use 8-bit T1 fonts
\usepackage{hyperref}       % hyperlinks
\usepackage{url}            % simple URL typesetting
\usepackage{booktabs}       % professional-quality tables
\usepackage{amsfonts}       % blackboard math symbols
\usepackage{amsmath}
\usepackage{nicefrac}       % compact symbols for 1/2, etc.
\usepackage{microtype}      % microtypography
\usepackage{subfiles}
\usepackage{xcolor}
\usepackage{multirow}
\usepackage{enumerate}
\usepackage{subfiles}
\usepackage{multirow}
\usepackage{graphicx}
\usepackage{subfiles}


% % Custom Commands:
% \newcommand\blfootnote[1]{%
%   \begingroup
%   \renewcommand\thefootnote{}\footnote{#1}%
%   \addtocounter{footnote}{-1}%
%   \endgroup
% }

\newcommand\todo[1]{\textcolor{red}{[[#1]]}}
\newcommand\mc[1]{\mathcal{#1}}
\newcommand*\samethanks[1][\value{footnote}]{\footnotemark[#1]}
%keys for memory and values. Can be changed if needed
\newcommand{\kq}{q}
\newcommand{\km}{k}
\newcommand{\vq}{o}
\newcommand{\vm}{m}
\newcommand{\Wkq}{W_q}
\newcommand{\Wkm}{W_k}
\newcommand{\Wvq}{W_o}
\newcommand{\Wvm}{W_m}
\newcommand{\dmodel}{d_{\text{model}}}
\newcommand{\dffn}{d_{\text{ffn}}}
\newcommand{\dff}{d_{\text{ff}}}
\newcommand{\mbf}[1]{\mathbf{#1}}
%\newcommand{\kq}{{q}_k}
%\newcommand{\km}{{m}_k}
%\newcommand{\vq}{{q}_v}
%\newcommand{\vm}{{m}_v}
%\newcommand{\Wkq}{{W_q}_k}
%\newcommand{\Wkm}{{W_m}_k}
%\newcommand{\Wvq}{{W_q}_v}
%\newcommand{\Wvm}{{W_m}_v}
\newcommand\concat[3]{\left[#1 \parallel_#3 #2\right]}

\title{Attention Is All You Need}

% The \author macro works with any number of authors. There are two
% commands used to separate the names and addresses of multiple
% authors: \And and \AND.
%
% Using \And between authors leaves it to LaTeX to determine where to
% break the lines. Using \AND forces a line break at that point. So,
% if LaTeX puts 3 of 4 authors names on the first line, and the last
% on the second line, try using \AND instead of \And before the third
% author name.
\author{
  \AND
  Ashish Vaswani\thanks{Equal contribution. Listing order is random. Jakob proposed replacing RNNs with self-attention and started the effort to evaluate this idea.
Ashish, with Illia, designed and implemented the first Transformer models and has been crucially involved in every aspect of this work. Noam proposed scaled dot-product attention, multi-head attention and the parameter-free position representation and became the other person involved in nearly every detail. Niki designed, implemented, tuned and evaluated countless model variants in our original codebase and tensor2tensor. Llion also experimented with novel model variants, was responsible for our initial codebase, and efficient inference and visualizations. Lukasz and Aidan spent countless long days designing various parts of and implementing tensor2tensor, replacing our earlier codebase, greatly improving results and massively accelerating our research.
}\\
  Google Brain\\
  \texttt{avaswani@google.com}\\
  \And
  Noam Shazeer\footnotemark[1]\\
  Google Brain\\
  \texttt{noam@google.com}\\
  \And
  Niki Parmar\footnotemark[1]\\
  Google Research\\
  \texttt{nikip@google.com}\\  
  \And
  Jakob Uszkoreit\footnotemark[1]\\
  Google Research\\
  \texttt{usz@google.com}\\
  \And  
  Llion Jones\footnotemark[1]\\
  Google Research\\
  \texttt{llion@google.com}\\   
  \And
  Aidan N. Gomez\footnotemark[1] \hspace{1.7mm}\thanks{Work performed while at Google Brain.}\\
  University of Toronto\\
  \texttt{aidan@cs.toronto.edu}
  \And
  {\L}ukasz Kaiser\footnotemark[1]\\
  Google Brain\\
  \texttt{lukaszkaiser@google.com}\\
  \And
  Illia Polosukhin\footnotemark[1]\hspace{1.7mm} \thanks{Work performed while at Google Research.}\\
  \texttt{illia.polosukhin@gmail.com}\\  
}

\begin{document}
\begin{center}
    \color{red}
    \large Provided proper attribution is provided, Google hereby grants permission to reproduce the tables and figures in this paper solely for use in journalistic or scholarly works.
\end{center}

\maketitle

\begin{abstract}
The dominant sequence transduction models are based on complex recurrent or convolutional neural networks that include an encoder and a decoder. The best performing models also connect the encoder and decoder through an attention mechanism. We propose a new simple network architecture, the Transformer, based solely on attention mechanisms, dispensing with recurrence and convolutions entirely. Experiments on two machine translation tasks show these models to be superior in quality while being  more parallelizable and requiring significantly less time to train. Our model achieves 28.4 BLEU on the WMT 2014 English-to-German translation task, improving over the existing best results, including ensembles, by over 2 BLEU.  On the WMT 2014 English-to-French translation task, our model establishes a new single-model state-of-the-art BLEU score of 41.8 after training for 3.5 days on eight GPUs, a small fraction of the training costs of the best models from the literature. We show that the Transformer generalizes well to other tasks by applying it successfully to English constituency parsing  both with large and limited training data.
% \blfootnote{Code available at \url{https://github.com/tensorflow/tensor2tensor}}

%TODO(noam): update results for new models.

%llion@: FAIR's paper seems to concentrate solely on the convolutional aspect of their model and have the attention as an after thought almost, this gives us a good opportunity to differentiate ourselves from their paper.

%We are simpler in a number of ways and should have the simplicity as a big selling point:
%\begin{itemize}
%\item No convolutions
%\item No need for such careful initializations and %normalization.
%\item Simpler non-lineararities, they use the gated linear %units.
%\item Less layers?
%\end{itemize}
%One thing we do more is that we have self attention.
%Another selling point is the increased interpretability as %shown with the visualizations. Which comes from the %simplicity and use of only attentions.
\end{abstract}

\section{Introduction}

\section{Introduction}
\label{sec:introduction}

Sequential decision making is generally considered an essential ingredient for generally capable agents. The ability to plan ahead and adapt to changing circumstances is synonymous with the concept of {\em agency}. For decades, the field of reinforcement learning (RL) has worked on developing methods, or agents, for precisely this purpose. This research has borne impressive results, such as developing agents which can play difficult Atari games \citep{mnih2015humanlevel}, control stratospheric balloons \citep{Bellemare2020AutonomousNO}, control a tokamak fusion reactor \citep{Degrave2022MagneticCO}, among others. These are all examples of {\em deep reinforcement learning} (DRL), which combines the theory of reinforcement learning with the expressiveness and flexibility of deep neural networks.

The success of these methods built on years of academic research, where novel algorithms and techniques were introduced and showcased on academic benchmarks such as the ALE \citep{bellemare2012ale}, MuJoCo \citep{todorov2012mujoco}, and others. These benchmarks typically consist of a suite of environments that have varied transition and reward dynamics. Their common usage provides us with a familiarity which affords us a sense of interpretability, a consistency in evaluation that grants us a sense of reliability, and their variety yields a sense of generalizability. Unfortunately, this promise often fails to materialize: their reliability has been brought into question by numerous works which demonstrate their fickleness \citep{Henderson2017DeepRL,agarwal2021deep}, while there is a general sentiment that researchers have ``overfit’’ to these benchmarks, bringing into question their generalizability. A critical aspect to these challenges is the difficulty in training neural networks in an RL setting \citep{ostrovski2021the,lyle2022learning,sokar2023dormant}.

Although the successes above built on prior methods, they were not taken ``as is’’: it took large teams of researchers many months and lots of compute to adapt prior work to their specific problem. These adaptations include changes to the network architectures, designing reward functions to induce the desired behaviours, and careful tuning of the many hyper-parameters. This last point is indeed {\em essential} to the success of any DRL method: improper hyper-parameter choices can cause a theoretically sound method to drastically underperform, while careful hyper-parameter selection can dramatically increase the performance of an otherwise sub-optimal method.

As an example of this dichotomy, we examine how DER \citep{hasselt19when}, a method that has become a common baseline for the Atari $100$k benchmark \citep{kaiser2020modelbased}, came to be. DQN, considered to be the start of the field of DRL research, was introduced by showcasing its super-human performance on the ALE \citep{bellemare2012ale}, a suite of $57$ Atari $2600$ games. This suite became one of the most popular benchmarks on which to evaluate new methods over $200$ million environment frames\footnote{See \citep{machado2018revisiting} for more details on ALE evaluation standards.}. A few years later, when \citet{kaiser2020modelbased} introduced the SiMPLe algorithm as a sample-efficient method, they argued for evaluating it only on $100$k agent actions\footnote{The standard for ALE agents is to use frame-skipping, where $4$ environment frames occur for every agent action. This results in frustratingly confusing nomenclature, as $200$M is specified in environment frames (or $500$k agent actions), while $100$k is specified in agent actions (or $400$k environment frames).} with a subset of $26$ games, so as to properly test the sample-efficiency of new methods. The authors demonstrated that their proposed method outperformed Rainbow \citep{Hessel2018RainbowCI}, the state-of-the-art method of the time. In response, \citet{hasselt19when} introduced Data Efficient Rainbow (DER), which outperformed SiMPLe even though it was the same Rainbow algorithm, but {\em with a careful tuning of the hyper-parameters for the $100$k training regime}.

One could argue that the hyper-parameters of Rainbow were overly-tuned to the $200$M benchmark, while the hyper-parameters of DER were overly-tuned to the $100$k benchmark. More importantly, what this story highlights is that, despite careful evaluation it is quite likely that a new method {\em will not work as intended when deployed on a different environment from which it was trained on}, and that a significant  amount of hyper-parameter tuning will be necessary. This flies in the face of the supposed generalizability of DRL academic research, and makes it difficult for groups without large computational budgets to successfully apply prior work to applied problems.

It thus behooves the community to develop a better understanding of the {\em transferability} and {\em consistency} of hyper-parameter selection across different training regimes, and to build a better shared understanding of the relative importance of the many possible hyper-parameters to tune. In this work, we take a stride towards this by conducting an exhaustive empirical investigation of the various hyper-parameters affecting DRL agents. We focus our attention on two value-based agents developed for the Atari $100$k suite: DER mentioned above, and DrQ($\epsilon$), a variant of DQN that was optimized for the $100$k suite. Although developed for the $100$k suite, we also train these agents for $40$M million environment frames. Our intent is to examine the transferability of various hyper-parameter choices across different training regimes. Specifically, we investigate:
{\bf Across data regimes:} Do hyper-parameters selected in the $100$k regime work well in a larger data regime? {\bf Across agents:} Do hyper-parameters selected for one agent work well in another? {\bf Across environments:} Do hyper-parameters tuned in one set of environments work well in others?

In total, we investigated $12$ hyper-parameters with different values for $2$ agents over $26$ environments, each for $5$ seeds, resulting in a total of $108$k independent training runs. This breadth of experimentation results in an overwhelming amount of data which complicates their analyses. We address this challenge in two ways: \textit{(i)} We introduce a new score which provides us with an aggregate value for the considerations mentioned above. \textit{(ii)} We provide an interactive website where others may easily navigate the large number of experimental figures we have generated.

The score provides us with a high-level overview of our findings, while the website grants us a fine-grained mechanism to analyze the results. We hope this effort provides the community with useful tools so as to develop not just better DRL algorithms, but better methodologies to evaluate their interpretability, reliability, and generalizability.

\section{Background}

The goal of reducing sequential computation also forms the foundation of the Extended Neural GPU \citep{extendedngpu}, ByteNet \citep{NalBytenet2017} and ConvS2S \citep{JonasFaceNet2017}, all of which use convolutional neural networks as basic building block, computing hidden representations in parallel for all input and output positions. In these models, the number of operations required to relate signals from two arbitrary input or output positions grows in the distance between positions, linearly for ConvS2S and logarithmically for ByteNet. This makes it more difficult to learn dependencies between distant positions \citep{hochreiter2001gradient}. In the Transformer this is reduced to a constant number of operations, albeit at the cost of reduced effective resolution due to averaging attention-weighted positions, an effect we counteract with Multi-Head Attention as described in section~\ref{sec:attention}. 

Self-attention, sometimes called intra-attention is an attention mechanism relating different positions of a single sequence in order to compute a representation of the sequence. Self-attention has been used successfully in a variety of tasks including reading comprehension, abstractive summarization, textual entailment and learning task-independent sentence representations \citep{cheng2016long, decomposableAttnModel, paulus2017deep, lin2017structured}.

End-to-end memory networks are based on a recurrent attention mechanism instead of sequence-aligned recurrence and have been shown to perform well on simple-language question answering and language modeling tasks \citep{sukhbaatar2015}.

To the best of our knowledge, however, the Transformer is the first transduction model relying entirely on self-attention to compute representations of its input and output without using sequence-aligned RNNs or convolution.
In the following sections, we will describe the Transformer, motivate self-attention and discuss its advantages over models such as \citep{neural_gpu, NalBytenet2017} and \citep{JonasFaceNet2017}.


%\citep{JonasFaceNet2017} report new SOTA on machine translation for English-to-German (EnDe), Enlish-to-French (EnFr) and English-to-Romanian language pairs. 

%For example,! in MT, we must draw information from both input and previous output words to translate an output word accurately. An attention layer \citep{bahdanau2014neural} can connect a very large number of positions at low computation cost, making it an essential ingredient in competitive recurrent models for machine translation.

%A natural question to ask then is, "Could we replace recurrence with attention?". \marginpar{Don't know if it's the most natural question to ask given the previous statements. Also, need to say that the complexity table summarizes these statements} Such a model would be blessed with the computational efficiency of attention and the power of cross-positional communication. In this work, show that pure attention models work remarkably well for MT, achieving new SOTA results on EnDe and EnFr, and can be trained in under $2$ days on xyz architecture. 

%After the seminal models introduced in \citep{sutskever14, bahdanau2014neural, cho2014learning}, recurrent models have become the dominant solution for both sequence modeling and sequence-to-sequence transduction. Many efforts such as \citep{wu2016google,luong2015effective,jozefowicz2016exploring} have pushed the boundaries of machine translation (MT) and language modeling with recurrent endoder-decoder and recurrent language models. Recent effort \citep{shazeer2017outrageously} has successfully combined the power of conditional computation with sequence models to train very large models for MT, pushing SOTA at lower computational cost.

%Recurrent models compute a vector of hidden states $h_t$, for each time step $t$ of computation. $h_t$ is a function of both the input at time $t$ and the previous hidden state $h_t$. This dependence on the previous hidden state precludes processing all timesteps at once, instead requiring long sequences of sequential operations.  In practice, this results in greatly reduced computational efficiency, as on modern computing hardware, a single operation on a large batch is much faster than a large number of operations on small batches.  The problem gets worse at longer sequence lengths. Although sequential computation is not a severe bottleneck at inference time, as autoregressively generating each output requires all previous outputs, the inability to compute scores at all output positions at once hinders us from rapidly training our models over large datasets. Although impressive work such as \citep{Kuchaiev2017Factorization} is able to significantly accelerate the training of LSTMs with factorization tricks, we are still bound by the linear dependence on sequence length.

%If the model could compute hidden states at each time step using only the inputs and outputs,  it would be liberated from the dependence on results from previous time steps during training. This line of thought is the foundation of recent efforts such as the Markovian neural GPU \citep{neural_gpu}, ByteNet \citep{NalBytenet2017} and ConvS2S \citep{JonasFaceNet2017}, all of which use convolutional neural networks as a building block to compute hidden representations simultaneously for all timesteps, resulting in $O(1)$ sequential time complexity. \citep{JonasFaceNet2017} report new SOTA on machine translation for English-to-German (EnDe), Enlish-to-French (EnFr) and English-to-Romanian language pairs. 

%A crucial component for accurate sequence prediction is modeling cross-positional communication. For example, in MT, we must draw information from both input and previous output words to translate an output word accurately. An attention layer \citep{bahdanau2014neural} can connect a very large number of positions at a low computation cost, also $O(1)$ sequential time complexity, making it an essential ingredient in recurrent encoder-decoder architectures for MT. A natural question to ask then is, "Could we replace recurrence with attention?". \marginpar{Don't know if it's the most natural question to ask given the previous statements. Also, need to say that the complexity table summarizes these statements} Such a model would be blessed with the computational efficiency of attention and the power of cross-positional communication. In this work, show that pure attention models work remarkably well for MT, achieving new SOTA results on EnDe and EnFr, and can be trained in under $2$ days on xyz architecture. 



%Note: Facebook model is no better than RNNs in this regard, since it requires a number of layers proportional to the distance you want to communicate.  Bytenet is more promising, since it requires a logarithmnic number of layers (does bytenet have SOTA results)?   

%Note: An attention  layer can connect a very large number of positions at a low computation cost in O(1) sequential operations.  This is why encoder-decoder attention has been so successful in seq-to-seq models so far.  It is only natural, then, to also use attention to connect the timesteps of the same sequence.

%Note: I wouldn't say that long sequences are not a problem during inference.  It would be great if we could infer with no long sequences.  We could just say later on that, while our training graph is constant-depth, our model still requires sequential operations in the decoder part during inference due to the autoregressive nature of the model.   

%\begin{table}[h!]
%\caption{Attention models are quite efficient for cross-positional communications when sequence length is smaller than channel depth. $n$ represents the sequence length and $d$ represents the channel depth.}
%\label{tab:op_complexities}
%\begin{center}
%\vspace{-5pt}
%\scalebox{0.75}{

%\begin{tabular}{l|c|c|c}
%\hline \hline
%Layer Type & Receptive & Complexity & Sequential  \\
%           & Field     &            & Operations  \\
%\hline
%Pointwise Feed-Forward & $1$ & $O(n \cdot d^2)$ & $O(1)$ \\
%\hline
%Recurrent & $n$ & $O(n \cdot d^2)$ & $O(n)$ \\
%\hline
%Convolutional & $r$ & $O(r \cdot n \cdot d^2)$ & $O(1)$ \\
%\hline
%Convolutional (separable) & $r$ & $O(r \cdot n \cdot d + n %\cdot d^2)$ & $O(1)$ \\
%\hline
%Attention & $r$ & $O(r \cdot n \cdot d)$ & $O(1)$ \\
%\hline \hline
%\end{tabular}
%}
%\end{center}
%\end{table}

\section{Model Architecture}

\begin{figure}
  \centering
  \includegraphics[scale=0.6]{Figures/ModalNet-21}
  \caption{The Transformer - model architecture.}
  \label{fig:model-arch}
\end{figure}

% Although the primary workhorse of our model is attention, 
%Our model maintains the encoder-decoder structure that is common to many so-called sequence-to-sequence models \citep{bahdanau2014neural,sutskever14}.  As in all such architectures, the encoder computes a representation of the input sequence, and the decoder consumes these representations along with the output tokens to autoregressively produce the output sequence.  Where, traditionally, the encoder and decoder contain stacks of recurrent or convolutional layers, our encoder and decoder stacks are composed of attention layers and position-wise feed-forward layers (Figure~\ref{fig:model-arch}).  The following sections describe the gross architecture and these particular components in detail.

Most competitive neural sequence transduction models have an encoder-decoder structure \citep{cho2014learning,bahdanau2014neural,sutskever14}. Here, the encoder maps an input sequence of symbol representations $(x_1, ..., x_n)$ to a sequence of continuous representations $\mathbf{z} = (z_1, ..., z_n)$. Given $\mathbf{z}$, the decoder then generates an output sequence $(y_1,...,y_m)$ of symbols one element at a time. At each step the model is auto-regressive \citep{graves2013generating}, consuming the previously generated symbols as additional input when generating the next.

The Transformer follows this overall architecture using stacked self-attention and point-wise, fully connected layers for both the encoder and decoder, shown in the left and right halves of Figure~\ref{fig:model-arch}, respectively.

\subsection{Encoder and Decoder Stacks}

\paragraph{Encoder:}The encoder is composed of a stack of $N=6$ identical layers. Each layer has two sub-layers. The first is a multi-head self-attention mechanism, and the second is a simple, position-wise fully connected feed-forward network.   We employ a residual connection \citep{he2016deep} around each of the two sub-layers, followed by layer normalization \cite{layernorm2016}.  That is, the output of each sub-layer is $\mathrm{LayerNorm}(x + \mathrm{Sublayer}(x))$, where $\mathrm{Sublayer}(x)$ is the function implemented by the sub-layer itself.  To facilitate these residual connections, all sub-layers in the model, as well as the embedding layers, produce outputs of dimension $\dmodel=512$.

\paragraph{Decoder:}The decoder is also composed of a stack of $N=6$ identical layers.  In addition to the two sub-layers in each encoder layer, the decoder inserts a third sub-layer, which performs multi-head attention over the output of the encoder stack.  Similar to the encoder, we employ residual connections around each of the sub-layers, followed by layer normalization.  We also modify the self-attention sub-layer in the decoder stack to prevent positions from attending to subsequent positions.  This masking, combined with fact that the output embeddings are offset by one position, ensures that the predictions for position $i$ can depend only on the known outputs at positions less than $i$.

% In our model (Figure~\ref{fig:model-arch}), the encoder and decoder are composed of stacks of alternating self-attention layers (for cross-positional communication) and position-wise feed-forward layers (for in-place computation).  In addition, the decoder stack contains encoder-decoder attention layers.  Since attention is agnostic to the distances between words, our model requires a "positional encoding" to be added to the encoder and decoder input. The following sections describe all of these components in detail.

\subsection{Attention} \label{sec:attention}
An attention function can be described as mapping a query and a set of key-value pairs to an output, where the query, keys, values, and output are all vectors.  The output is computed as a weighted sum of the values, where the weight assigned to each value is computed by a compatibility function of the query with the corresponding key.

\subsubsection{Scaled Dot-Product Attention} \label{sec:scaled-dot-prod}

% \begin{figure}
%   \centering
%   \includegraphics[scale=0.6]{Figures/ModalNet-19}
%   \caption{Scaled Dot-Product Attention.}
%   \label{fig:multi-head-att}
% \end{figure}

We call our particular attention "Scaled Dot-Product Attention" (Figure~\ref{fig:multi-head-att}).   The input consists of queries and keys of dimension $d_k$, and values of dimension $d_v$.  We compute the dot products of the query with all keys, divide each by $\sqrt{d_k}$, and apply a softmax function to obtain the weights on the values.

In practice, we compute the attention function on a set of queries simultaneously, packed together into a matrix $Q$.   The keys and values are also packed together into matrices $K$ and $V$.  We compute the matrix of outputs as:

\begin{equation}
   \mathrm{Attention}(Q, K, V) = \mathrm{softmax}(\frac{QK^T}{\sqrt{d_k}})V
\end{equation}

The two most commonly used attention functions are additive attention \citep{bahdanau2014neural}, and dot-product (multiplicative) attention.  Dot-product attention is identical to our algorithm, except for the scaling factor of $\frac{1}{\sqrt{d_k}}$. Additive attention computes the compatibility function using a feed-forward network with a single hidden layer.  While the two are similar in theoretical complexity, dot-product attention is much faster and more space-efficient in practice, since it can be implemented using highly optimized matrix multiplication code. 

%We scale the dot products by $1/\sqrt{d_k}$ to limit the magnitude of the dot products, which works well in practice. Otherwise, we found applying the softmax to often result in weights very close to 0 or 1, and hence minuscule gradients.

% Already described in the subsequent section
%When used as part of decoder self-attention, an optional mask function is applied just before the softmax to prevent positions from attending to subsequent positions.   This mask simply sets the logits corresponding to all illegal connections (those outside of the lower triangle) to $-\infty$.

%\paragraph{Comparison to Additive Attention: } We choose dot product attention over additive attention \citep{bahdanau2014neural} since it can be computed using highly optimized matrix multiplication code.  This optimization is particularly important to us, as we employ many attention layers in our model.

While for small values of $d_k$ the two mechanisms perform similarly, additive attention outperforms dot product attention without scaling for larger values of $d_k$ \citep{DBLP:journals/corr/BritzGLL17}. We suspect that for large values of $d_k$, the dot products grow large in magnitude, pushing the softmax function into regions where it has extremely small gradients  \footnote{To illustrate why the dot products get large, assume that the components of $q$ and $k$ are independent random variables with mean $0$ and variance $1$.  Then their dot product, $q \cdot k = \sum_{i=1}^{d_k} q_ik_i$, has mean $0$ and variance $d_k$.}. To counteract this effect, we scale the dot products by $\frac{1}{\sqrt{d_k}}$.


%We suspect this to be caused by the dot products growing too large in magnitude to result in useful gradients after applying the softmax function.  To counteract this, we scale the dot product by $1/\sqrt{d_k}$.


\subsubsection{Multi-Head Attention} \label{sec:multihead}

\begin{figure}
\begin{minipage}[t]{0.5\textwidth}
  \centering
  Scaled Dot-Product Attention \\
  \vspace{0.5cm}
  \includegraphics[scale=0.6]{Figures/ModalNet-19}
\end{minipage}
\begin{minipage}[t]{0.5\textwidth}
  \centering 
  Multi-Head Attention \\
  \vspace{0.1cm}
  \includegraphics[scale=0.6]{Figures/ModalNet-20}  
\end{minipage}


  % \centering

  \caption{(left) Scaled Dot-Product Attention. (right) Multi-Head Attention consists of several attention layers running in parallel.}
  \label{fig:multi-head-att}
\end{figure}

Instead of performing a single attention function with $\dmodel$-dimensional keys, values and queries, we found it beneficial to linearly project the queries, keys and values $h$ times with different, learned linear projections to $d_k$, $d_k$ and $d_v$ dimensions, respectively.
On each of these projected versions of queries, keys and values we then perform the attention function in parallel, yielding $d_v$-dimensional output values. These are concatenated and once again projected, resulting in the final values, as depicted in Figure~\ref{fig:multi-head-att}.

Multi-head attention allows the model to jointly attend to information from different representation subspaces at different positions. With a single attention head, averaging inhibits this.

\begin{align*}
    \mathrm{MultiHead}(Q, K, V) &= \mathrm{Concat}(\mathrm{head_1}, ..., \mathrm{head_h})W^O\\
%    \mathrm{where} \mathrm{head_i} &= \mathrm{Attention}(QW_Q_i^{\dmodel \times d_q}, KW_K_i^{\dmodel \times d_k}, VW^V_i^{\dmodel \times d_v})\\
    \text{where}~\mathrm{head_i} &= \mathrm{Attention}(QW^Q_i, KW^K_i, VW^V_i)\\
\end{align*}

Where the projections are parameter matrices $W^Q_i \in \mathbb{R}^{\dmodel \times d_k}$, $W^K_i \in \mathbb{R}^{\dmodel \times d_k}$, $W^V_i \in \mathbb{R}^{\dmodel \times d_v}$ and $W^O \in \mathbb{R}^{hd_v \times \dmodel}$.


%find it better (and no more expensive) to have multiple parallel attention layers (each over the full set of positions) with proportionally lower-dimensional keys, values and queries.  We call this "Multi-Head Attention" (Figure~\ref{fig:multi-head-att}).  The keys, values, and queries for each of these parallel attention layers are computed by learned linear transformations of the inputs to the multi-head attention.  We use different linear transformations across different parallel attention layers.  The output of the parallel attention layers are concatenated, and then passed through a final learned linear transformation. 

In this work we employ $h=8$ parallel attention layers, or heads. For each of these we use $d_k=d_v=\dmodel/h=64$.
Due to the reduced dimension of each head, the total computational cost is similar to that of single-head attention with full dimensionality.

\subsubsection{Applications of Attention in our Model}

The Transformer uses multi-head attention in three different ways: 
\begin{itemize}
 \item In "encoder-decoder attention" layers, the queries come from the previous decoder layer, and the memory keys and values come from the output of the encoder.   This allows every position in the decoder to attend over all positions in the input sequence.  This mimics the typical encoder-decoder attention mechanisms in sequence-to-sequence models such as \citep{wu2016google, bahdanau2014neural,JonasFaceNet2017}.

 \item The encoder contains self-attention layers.  In a self-attention layer all of the keys, values and queries come from the same place, in this case, the output of the previous layer in the encoder.   Each position in the encoder can attend to all positions in the previous layer of the encoder.

 \item Similarly, self-attention layers in the decoder allow each position in the decoder to attend to all positions in the decoder up to and including that position.  We need to prevent leftward information flow in the decoder to preserve the auto-regressive property.  We implement this inside of scaled dot-product attention by masking out (setting to $-\infty$) all values in the input of the softmax which correspond to illegal connections.  See Figure~\ref{fig:multi-head-att}.

\end{itemize}

\subsection{Position-wise Feed-Forward Networks}\label{sec:ffn}

In addition to attention sub-layers, each of the layers in our encoder and decoder contains a fully connected feed-forward network, which is applied to each position separately and identically.  This consists of two linear transformations with a ReLU activation in between.

\begin{equation}
   \mathrm{FFN}(x)=\max(0, xW_1 + b_1) W_2 + b_2
\end{equation}

While the linear transformations are the same across different positions, they use different parameters from layer to layer. Another way of describing this is as two convolutions with kernel size 1.  The dimensionality of input and output is $\dmodel=512$, and the inner-layer has dimensionality $d_{ff}=2048$.



%In the appendix, we describe how the position-wise feed-forward network can also be seen as a form of attention.

%from Jakob: The number of operations required for the model to relate signals from two arbitrary input or output positions grows in the distance between positions in input or output, linearly for ConvS2S and logarithmically for ByteNet, making it harder to learn dependencies between these positions \citep{hochreiter2001gradient}. In the transformer this is reduced to a constant number of operations, albeit at the cost of effective resolution caused by averaging attention-weighted positions, an effect we aim to counteract with multi-headed attention.


%Figure~\ref{fig:simple-att} presents a simple attention function, $A$, with a single head, that forms the basis of our multi-head attention. $A$ takes a query key vector $\kq$, matrices of memory keys $\km$ and memory values $\vm$ ,and produces a query value vector $\vq$ as 
%\begin{equation*} \label{eq:attention}
%    A(\kq, \km, \vm) = {\vm}^T (Softmax(\km \kq).
%\end{equation*}
%We linearly transform $\kq,\,\km$, and $\vm$ with learned matrices ${\Wkq \text{,} \, \Wkm}$, and ${\Wvm}$ before calling the attention function, and transform the output query with $\Wvq$ before handing it to the feed forward layer. Each attention layer has it's own set of transformation matrices, which are shared across all query positions. $A$ is applied in parallel for each query position, and is implemented very efficiently as a batch of matrix multiplies. The self-attention and encoder-decoder attention layers use $A$, but with different arguments. For example, in encdoder self-attention, queries in encoder layer $i$ attention to memories in encoder layer $i-1$. To ensure that decoder self-attention layers do not look at future words, we add $- \inf$ to the softmax logits in positions $j+1$ to query length for query position $l$.  

%In simple attention, the query value is a weighted combination of the memory values where the attention weights sum to one. Although this function performs well in practice, the constraint on attention weights can restrict the amount of information that flows from memories to queries because the query cannot focus on multiple memory positions at once, which might be desirable when translating long sequences. \marginpar{@usz, could you think of an example of this ?} We remedy this by maintaining multiple attention heads at each query position that attend to all memory positions in parallel, with a different set of parameters  per attention head $h$. 
%\marginpar{}

\subsection{Embeddings and Softmax}
Similarly to other sequence transduction models, we use learned embeddings to convert the input tokens and output tokens to vectors of dimension $\dmodel$.  We also use the usual learned linear transformation and softmax function to convert the decoder output to predicted next-token probabilities.  In our model, we share the same weight matrix between the two embedding layers and the pre-softmax linear transformation, similar to \citep{press2016using}.   In the embedding layers, we multiply those weights by $\sqrt{\dmodel}$.


\subsection{Positional Encoding}
Since our model contains no recurrence and no convolution, in order for the model to make use of the order of the sequence, we must inject some information about the relative or absolute position of the tokens in the sequence.  To this end, we add "positional encodings" to the input embeddings at the bottoms of the encoder and decoder stacks.  The positional encodings have the same dimension $\dmodel$ as the embeddings, so that the two can be summed.   There are many choices of positional encodings, learned and fixed \citep{JonasFaceNet2017}.

In this work, we use sine and cosine functions of different frequencies:

\begin{align*}
    PE_{(pos,2i)} = sin(pos / 10000^{2i/\dmodel}) \\
    PE_{(pos,2i+1)} = cos(pos / 10000^{2i/\dmodel})
\end{align*}

where $pos$ is the position and $i$ is the dimension.  That is, each dimension of the positional encoding corresponds to a sinusoid.  The wavelengths form a geometric progression from $2\pi$ to $10000 \cdot 2\pi$.  We chose this function because we hypothesized it would allow the model to easily learn to attend by relative positions, since for any fixed offset $k$, $PE_{pos+k}$ can be represented as a linear function of $PE_{pos}$.

We also experimented with using learned positional embeddings \citep{JonasFaceNet2017} instead, and found that the two versions produced nearly identical results (see Table~\ref{tab:variations} row (E)).  We chose the sinusoidal version because it may allow the model to extrapolate to sequence lengths longer than the ones encountered during training.

 
\section{Why Self-Attention}
% This is a full line comment

Hello this is a test file. 

% This is a block comment
% This comment is so cool!

I like to spend my time playing chess! % Even though I hate chess. 

This is a funky line. % Comment comment % Blah Blah Blah %%%

I hope this works 100\% of the time!

% Final remarks:
% Yippee!

\section{Training}
This section describes the training regime for our models. 

%In order to speed up experimentation, our ablations are performed relative to a smaller base model described in detail in Section \ref{sec:results}.

\subsection{Training Data and Batching}
We trained on the standard WMT 2014 English-German dataset consisting of about 4.5 million sentence pairs.  Sentences were encoded using byte-pair encoding \citep{DBLP:journals/corr/BritzGLL17}, which has a shared source-target vocabulary of about 37000 tokens. For English-French, we used the significantly larger WMT 2014 English-French dataset consisting of 36M sentences and split tokens into a 32000 word-piece vocabulary \citep{wu2016google}.  Sentence pairs were batched together by approximate sequence length.  Each training batch contained a set of sentence pairs containing approximately 25000 source tokens and 25000 target tokens.  

\subsection{Hardware and Schedule}

We trained our models on one machine with 8 NVIDIA P100 GPUs.  For our base models using the hyperparameters described throughout the paper, each training step took about 0.4 seconds.  We trained the base models for a total of 100,000 steps or 12 hours.  For our big models,(described on the bottom line of table \ref{tab:variations}), step time was 1.0 seconds.  The big models were trained for 300,000 steps (3.5 days).

\subsection{Optimizer} We used the Adam optimizer~\citep{kingma2014adam} with $\beta_1=0.9$, $\beta_2=0.98$ and $\epsilon=10^{-9}$.  We varied the learning rate over the course of training, according to the formula:

\begin{equation}
lrate = \dmodel^{-0.5} \cdot
  \min({step\_num}^{-0.5},
    {step\_num} \cdot {warmup\_steps}^{-1.5})
\end{equation}

This corresponds to increasing the learning rate linearly for the first $warmup\_steps$ training steps, and decreasing it thereafter proportionally to the inverse square root of the step number.  We used $warmup\_steps=4000$.

\subsection{Regularization} \label{sec:reg}

We employ three types of regularization during training: 
\paragraph{Residual Dropout} We apply dropout \citep{srivastava2014dropout} to the output of each sub-layer, before it is added to the sub-layer input and normalized.   In addition, we apply dropout to the sums of the embeddings and the positional encodings in both the encoder and decoder stacks.  For the base model, we use a rate of $P_{drop}=0.1$.

% \paragraph{Attention Dropout} Query to key attentions are structurally similar to hidden-to-hidden weights in a feed-forward network, albeit across positions. The softmax activations yielding attention weights can then be seen as the analogue of hidden layer activations. A natural possibility is to extend dropout \citep{srivastava2014dropout} to attention. We implement attention dropout by dropping out attention weights as,
% \begin{equation*}
%   \mathrm{Attention}(Q, K, V) = \mathrm{dropout}(\mathrm{softmax}(\frac{QK^T}{\sqrt{d}}))V
% \end{equation*}
% In addition to residual dropout, we found attention dropout to be beneficial for our parsing experiments.  

%\paragraph{Symbol Dropout} In the source and target embedding layers, we replace a random subset of the token ids with a sentinel id.  For the base model, we use a rate of $symbol\_dropout\_rate=0.1$.  Note that this applies only to the auto-regressive use of the target ids - not their use in the cross-entropy loss. 

%\paragraph{Attention Dropout} Query to memory attentions are structurally similar to hidden-to-hidden weights in a feed-forward network, albeit across positions. The softmax activations yielding attention weights can then be seen as the analogue of hidden layer activations. A natural possibility is to extend dropout \citep{srivastava2014dropout} to attentions. We implement attention dropout by dropping out attention weights as,
%\begin{equation*}
%   A(Q, K, V) = \mathrm{dropout}(\mathrm{softmax}(\frac{QK^T}{\sqrt{d}}))V
%\end{equation*}
%As a result, the query will not be able to access the memory values at the dropped out position. In our experiments, we tried attention dropout rates of 0.2, and 0.3, and found it to work favorably for English-to-German translation.
%$attention\_dropout\_rate=0.2$.

\paragraph{Label Smoothing} During training, we employed label smoothing of value $\epsilon_{ls}=0.1$ \citep{DBLP:journals/corr/SzegedyVISW15}.  This hurts perplexity, as the model learns to be more unsure, but improves accuracy and BLEU score.

 
\section{Results} \label{sec:results}
\section{Hyper-parameters considered} 
\label{sec:hyper-parameter_selection}

We describe the set of hyper-parameters explored in this work, with the values used for each listed in \autoref{sec:list_hyperparameters}. Unless otherwise specified, these are examined for both Conv and Dense layers.

{\bf Activation functions:} 
Non-linear activation functions are a fundamental part of neural networks, as their removal effectively turns the network into a linear function approximator.
While various activation functions have been proposed \citep{devlin2019bert, Elfwing2018SigmoidWeightedLU, 10.5555/3305381.3305478}, there have been few works comparing their performance \citep{Shamir2020SmoothAA}; to the best of our knowledge, there are no previous examples of such a comparison in the RL setting.


{\bf Normalization: }
Normalization plays an important role in supervised learning \citep{tan2020efficientnet, xie2017aggregated} but is relatively rare in deep reinforcement learning, with a few exceptions \citep{gogianu2021spectral, bhatt2019crossnorm, arpit2019initialize, alphaZero}. We explore {\em batch normalization} \citep{ioffe2015batch} and {\em layer normalization} \citep{ba2016layer}.

{\bf Network capacity: } 
``Scaling laws'' have been central to the growth of capabilities in large language/vision models, but have mostly eluded reinforcement learning agents, with a few exceptions \citep{schwarzer23a, taiga2022investigating, farebrother2022proto,obando2024mixtures,obandoceron2024pruned,farebrother2024stop}. 
To investigate the impact of network size, we vary the {\em depth} (e.g. the number of hidden layers) and the {\em width} (e.g. the number of neurons of each hidden layer).

{\bf Optimizer hyper-parameters: }
\label{sec:optimizerHypers}
We explore three hyper-parameters of Adam \citep{kingma15adam}, which has become the standard optimizer used by most: {\em learning rate}, {\em epsilon} and {\em weight decay}.
\emph{Learning rate} determines the step size at which the algorithm adjusts the model's parameters during each iteration.
$\epsilon$ represents a small constant value that is added to the denominator of the update rule to avoid numerical instabilities.
\emph{Weight decay} adds a penalty term to the loss function during training that discourages the model from assigning excessively increasing weight magnitudes.


{\bf $\epsilon$-greedy exploration: } 
$\epsilon$-greedy exploration is a simple and popular exploration technique which picks actions greedily with probability $1-\epsilon$, and a random action with probability $\epsilon$. Traditionally, experiments on the ALE use a linear decay strategy to decay $\epsilon$ from $1.0$ to its target value.

{\bf Reward clipping: } 
Most ALE experiments clip rewards at $(-1, 1)$ \citep{mnih2015humanlevel}.

{\bf Discount factor: } 
The multiplicative factor $\gamma$ discounts future rewards and its importance has been observed in a number of recent works \citep{amit2020discount, hessel19inductive, gelada2019off, vanseijen2019using, francoislavet2016discount,schwarzer23a}.

{\bf Replay buffer: }  
DRL agents  store past experiences in a replay buffer, to sample from during learning. The {\em replay capacity} parameter refers to the amount of data experiences stored in the buffer. 
It is common practice to only begin sampling from the replay buffer when a minimum number of transitions have been stored, referred to as the {\em minimum replay history}.


{\bf Batch size: } 
The number of stored transitions that are sampled for learning at each training step.

{\bf Update horizon: }
Multi-step learning \citep{sutton88learning} computes the temporal difference error using multi-step transitions, instead of a single step. DQN uses a single-step update by default, whereas Rainbow chose a 3-step update \citep{Hessel2018RainbowCI}. The update horizon has been argued to trade-off between the bias and the variance of the return estimate \citep{biasandvariance_kea}. 


{\bf Target Update periods: }
Value based agents often employ an online and a {\em target} Q-network, the latter which is updated less frequently by directly syncing (or Polyak-averaging) from the online network; the {\em target updated period} determines how frequently this occurs.


{\bf Update periods: }
The online network parameters are updated after every {\em update period} environment steps, with a value of $4$ used in standard ALE training.

{\bf Number of atoms: } 
In distributional reinforcement learning \citep{Bellemare2017ADP}, the output layer predicts the distribution of the returns for each action $a$ in a state $s$, instead of the mean $Q^{\pi}(s, a)$. A popular approach is to model the return as a categorical distribution parameterized by a certain number of 'atoms' over a pre-specified support. 


\begin{figure}[!t]
    \centering
  \includegraphics[width=\linewidth]{figures/this_score_all.pdf}%
    \caption{Tuning hyper-parameter Consistency (THC Score, see \cref{sec:thc_metric}) evaluated across agents (\textbf{left panel}), data regimes (\textbf{center panel}), and environments  (\textbf{right panel}). Different colors indicate different data regimes (left panel) and different agents (center and right panels); grey bars/titles indicate hyper-parameters which are not comparable across the considered transfer settings.
    \label{fig:this_score_all}%
    }%
\end{figure}



\section{Experimental results} 
\label{exp_results}
As mentioned in the introduction, there already exist two data regimes for evaluating agents on the ALE suite: the (low-data regime) $100$k \citep{kaiser2020modelbased} and the original $200$M benchmark \citep{mnih2015humanlevel}. The $100$k benchmark includes only $26$ games from the original suite, so we focus on these for our evaluation. For computational considerations, we follow \citet{graesser2022state} and use $40$M million environment frames as our large-data regime.
We use the settings of DrQ($\epsilon$) (introduced by \citet{agarwal2021deep} as an improvement over the DrQ of \citet{yarats2021image}), and 
Data Efficient Rainbow (DER) introduced by \citet{hasselt19when}. All experiments were run on a Tesla P100 GPU and took around $2$-$4$ hours ($100$k) and $1$-$2$ days ($40$M) per run.
Both algorithms are implemented in the Dopamine library \citep{castro18dopamine}. Since the $100$k setting is cheaper, we evaluated a larger set of hyper-parameter values there and manually picked the most informative subset for running in the $40$M setting. For all our experiments we ran 5 independent seeds and followed the guidelines suggested by \citet{agarwal2021deep} for more statistically meaningful comparisons. Specifically, we computed aggregate human-normalized scores and report interquantile mean (IQM) with $95\%$ stratified bootstrap CIs. 

In \autoref{fig:this_score_all} we present the computed THC score for all the hyper-parameters discussed in \cref{sec:hyper-parameter_selection}, and we discuss their consistency across agents in Section~\ref{sec:acrossAlgorithms}, across data regimes in Section~\ref{sec:acrossData}, and  across environments in Section~\ref{sec:acrossEnvironments}. More detailed discussions are provided in \autoref{sec:finerGrainedExperiments} and a set of interesting findings in \autoref{sec:imf}. It is worth recalling that higher THC scores indicate less consistency, which suggests a likely need to re-tune the respective hyper-parameters when changing training configurations.


\subsection{Optimal hyper-parameters mostly Transfer Across Agents}
\label{sec:acrossAlgorithms}
We find that optimal hyper-parameters for DrQ($\epsilon$) agree quite often with DER, which is somewhat expected given that they're based on the same classical RL algorithm of Q-learning, and have the same number of updates in the same environments. Looking at THC values between the two agents for different data regimes we see that all values are below $0.5$, and in the $100$k regime tend to be even lower. Nevertheless, comparing the results of the two rows in \cref{fig:drq_eps_batch_sizes,fig:per_game} demonstrate that there can still be strong differences between the two. In the $40$M regime, the hyper-parameters with the highest THC are batch size and update horizon, consistent with the findings of \cite{obandoceron2023small}, where these two hyper-parameters proved crucial to boosting agent performance.


\begin{figure}[!t]
    \centering
  \includegraphics[width=0.8\linewidth]{figures/DER_adam_eps.pdf}%
    \caption{
     \textbf{Measured IQM of human-normalized scores on the $26$ $100$k benchmark games, with varying Adam's $\epsilon$} for DER. We evaluate performance at 100k agent steps (or 400k environment frames), and at $40$ million environment frames. The ordering of the best hyper-parameters switches between the two data regimes.
    }
    \label{fig:der_adam_eps}
\end{figure}

\subsection{Optimal hyper-parameters mostly do not Transfer Across Data Regimes}
\label{sec:acrossData}
We find that optimal hyper-parameters for Atari 100k mostly do not transfer once you move to 40M updates, showing that even when keeping algorithms and environment constant one may still need to tune hyper-parameters should they change the amount of data their agent can train on. Of the hyper-parameters considered, {\em Adam's $\epsilon$} and {\em update period} seem to be the most critical to re-tune (see \autoref{fig:der_adam_eps} for results on DER for Adam's $\epsilon$). The results with Adam's $\epsilon$ are surprising, as the purpose of this hyper-parameter is mostly for numerical stability. The update horizon results are consistent with what is done in practice between these two data regimes (e.g. Rainbow uses an update horizon of $3$, while DER uses $10$).

\begin{figure}[!h]
    \centering
  \includegraphics[width=0.8\linewidth]{figures/DrQ_eps_subs.pdf}
  \includegraphics[width=0.8\linewidth]{figures/DER_subs.pdf}
    \caption{\textbf{Measured returns with varying batch size} for DrQ($\epsilon$) (top) and DER (bottom) at $40$M environment frames for four representative games, demonstrating that the ranking of the hyper-parameter values can drastically change from one game to the next. All results averaged over $5$ seeds, shaded areas represent $95\%$ confidence intervals.
    }%
    \label{fig:drq_eps_batch_sizes}%
\end{figure}


\subsection{Optimal hyper-parameters do not Transfer Across Environments}
\label{sec:acrossEnvironments}
Our experiments show that hyper-parameters that perform well on some games lead to lackluster final performance in others. Indeed, in \autoref{fig:this_score_all} we can see that the THC score is highest when evaluating across environments. This strongly suggests that, when using an existing agent in a new environment, most of the hyper-parameters would need extra tuning.
\autoref{fig:drq_eps_batch_sizes} displays the results when varying batch size, where we can see that the rankings can sometimes be complete opposites across games (compare Kangaroo and Gopher).
 


\section{A web-based appendix} 
\label{web_results}
We have run an extensive number of experiments (around 108k) for this work, which would render a traditional appendix unwieldy. Instead, we provide an interactive website\footnote{Website available at \href{https://consistent-hyperparameters.streamlit.app/}{\emph{https://consistent-hparams.streamlit.app/}}.} which facilitates navigating the full set of results. Presenting empirical research results in this manner offers a range of benefits that enhance accessibility, engagement, and comprehension. 
This dynamic presentation allows readers to more easily make comparisons over different games, agents, and parameters. 


The website's main page presents aggregate IQM results for all hyper-parameters investigated in both data regimes (e.g. \autoref{fig:der_adam_eps}), while sub-pages present detailed performance comparisons when sliced by game (\autoref{fig:drq_eps_batch_sizes} presents a subset of this) and hyper-parameter (\autoref{fig:per_game} presents a subset of this).
The added level of granularity provided by the sub-pages can be crucial for understanding the specific strengths and weaknesses of an algorithm in various scenarios. All results averaged over 5 seeds, shaded areas represent 95\% confidence intervals.

\begin{figure}[!t]
    \centering
   \includegraphics[width=\textwidth]{figures/DrQ_eps_game_subs.pdf}
   \includegraphics[width=\textwidth]{figures/DER_game_subs.pdf}
  
    \caption{\textbf{Measured returns with various hyper-parameter variations on Asterix} for DrQ($\epsilon$) (top) and DER (bottom) at 40M environment frames. Displaying eight representative hyper-parameters, enabling per-game analyses for hyper-parameter selection.}%
    \label{fig:per_game}%
    \vspace{-1em}
\end{figure}

\section{Conclusion}
In this work, we presented the Transformer, the first sequence transduction model based entirely on attention, replacing the recurrent layers most commonly used in encoder-decoder architectures with multi-headed self-attention.

For translation tasks, the Transformer can be trained significantly faster than architectures based on recurrent or convolutional layers. On both WMT 2014 English-to-German and WMT 2014 English-to-French translation tasks, we achieve a new state of the art. In the former task our best model outperforms even all previously reported ensembles. %We also provide an indication of the broader applicability of our models through experiments on English constituency parsing.

We are excited about the future of attention-based models and plan to apply them to other tasks. We plan to extend the Transformer to problems involving input and output modalities other than text and to investigate local, restricted attention mechanisms to efficiently handle large inputs and outputs such as images, audio and video.
Making generation less sequential is another research goals of ours.

The code we used to train and evaluate our models is available at \url{https://github.com/tensorflow/tensor2tensor}.

\paragraph{Acknowledgements} We are grateful to Nal Kalchbrenner and Stephan Gouws for
their fruitful comments, corrections and inspiration.

\bibliographystyle{plain}
%\bibliography{deeplearn}
\begin{thebibliography}{10}

\bibitem{layernorm2016}
Jimmy~Lei Ba, Jamie~Ryan Kiros, and Geoffrey~E Hinton.
\newblock Layer normalization.
\newblock {\em arXiv preprint arXiv:1607.06450}, 2016.

\bibitem{bahdanau2014neural}
Dzmitry Bahdanau, Kyunghyun Cho, and Yoshua Bengio.
\newblock Neural machine translation by jointly learning to align and
  translate.
\newblock {\em CoRR}, abs/1409.0473, 2014.

\bibitem{DBLP:journals/corr/BritzGLL17}
Denny Britz, Anna Goldie, Minh{-}Thang Luong, and Quoc~V. Le.
\newblock Massive exploration of neural machine translation architectures.
\newblock {\em CoRR}, abs/1703.03906, 2017.

\bibitem{cheng2016long}
Jianpeng Cheng, Li~Dong, and Mirella Lapata.
\newblock Long short-term memory-networks for machine reading.
\newblock {\em arXiv preprint arXiv:1601.06733}, 2016.

\bibitem{cho2014learning}
Kyunghyun Cho, Bart van Merrienboer, Caglar Gulcehre, Fethi Bougares, Holger
  Schwenk, and Yoshua Bengio.
\newblock Learning phrase representations using rnn encoder-decoder for
  statistical machine translation.
\newblock {\em CoRR}, abs/1406.1078, 2014.

\bibitem{xception2016}
Francois Chollet.
\newblock Xception: Deep learning with depthwise separable convolutions.
\newblock {\em arXiv preprint arXiv:1610.02357}, 2016.

\bibitem{gruEval14}
Junyoung Chung, {\c{C}}aglar G{\"{u}}l{\c{c}}ehre, Kyunghyun Cho, and Yoshua
  Bengio.
\newblock Empirical evaluation of gated recurrent neural networks on sequence
  modeling.
\newblock {\em CoRR}, abs/1412.3555, 2014.

\bibitem{dyer-rnng:16}
Chris Dyer, Adhiguna Kuncoro, Miguel Ballesteros, and Noah~A. Smith.
\newblock Recurrent neural network grammars.
\newblock In {\em Proc. of NAACL}, 2016.

\bibitem{JonasFaceNet2017}
Jonas Gehring, Michael Auli, David Grangier, Denis Yarats, and Yann~N. Dauphin.
\newblock Convolutional sequence to sequence learning.
\newblock {\em arXiv preprint arXiv:1705.03122v2}, 2017.

\bibitem{graves2013generating}
Alex Graves.
\newblock Generating sequences with recurrent neural networks.
\newblock {\em arXiv preprint arXiv:1308.0850}, 2013.

\bibitem{he2016deep}
Kaiming He, Xiangyu Zhang, Shaoqing Ren, and Jian Sun.
\newblock Deep residual learning for image recognition.
\newblock In {\em Proceedings of the IEEE Conference on Computer Vision and
  Pattern Recognition}, pages 770--778, 2016.

\bibitem{hochreiter2001gradient}
Sepp Hochreiter, Yoshua Bengio, Paolo Frasconi, and J{\"u}rgen Schmidhuber.
\newblock Gradient flow in recurrent nets: the difficulty of learning long-term
  dependencies, 2001.

\bibitem{hochreiter1997}
Sepp Hochreiter and J{\"u}rgen Schmidhuber.
\newblock Long short-term memory.
\newblock {\em Neural computation}, 9(8):1735--1780, 1997.

\bibitem{huang-harper:2009:EMNLP}
Zhongqiang Huang and Mary Harper.
\newblock Self-training {PCFG} grammars with latent annotations across
  languages.
\newblock In {\em Proceedings of the 2009 Conference on Empirical Methods in
  Natural Language Processing}, pages 832--841. ACL, August 2009.

\bibitem{jozefowicz2016exploring}
Rafal Jozefowicz, Oriol Vinyals, Mike Schuster, Noam Shazeer, and Yonghui Wu.
\newblock Exploring the limits of language modeling.
\newblock {\em arXiv preprint arXiv:1602.02410}, 2016.

\bibitem{extendedngpu}
{\L}ukasz Kaiser and Samy Bengio.
\newblock Can active memory replace attention?
\newblock In {\em Advances in Neural Information Processing Systems, ({NIPS})},
  2016.

\bibitem{neural_gpu}
\L{}ukasz Kaiser and Ilya Sutskever.
\newblock Neural {GPU}s learn algorithms.
\newblock In {\em International Conference on Learning Representations
  ({ICLR})}, 2016.

\bibitem{NalBytenet2017}
Nal Kalchbrenner, Lasse Espeholt, Karen Simonyan, Aaron van~den Oord, Alex
  Graves, and Koray Kavukcuoglu.
\newblock Neural machine translation in linear time.
\newblock {\em arXiv preprint arXiv:1610.10099v2}, 2017.

\bibitem{structuredAttentionNetworks}
Yoon Kim, Carl Denton, Luong Hoang, and Alexander~M. Rush.
\newblock Structured attention networks.
\newblock In {\em International Conference on Learning Representations}, 2017.

\bibitem{kingma2014adam}
Diederik Kingma and Jimmy Ba.
\newblock Adam: A method for stochastic optimization.
\newblock In {\em ICLR}, 2015.

\bibitem{Kuchaiev2017Factorization}
Oleksii Kuchaiev and Boris Ginsburg.
\newblock Factorization tricks for {LSTM} networks.
\newblock {\em arXiv preprint arXiv:1703.10722}, 2017.

\bibitem{lin2017structured}
Zhouhan Lin, Minwei Feng, Cicero Nogueira~dos Santos, Mo~Yu, Bing Xiang, Bowen
  Zhou, and Yoshua Bengio.
\newblock A structured self-attentive sentence embedding.
\newblock {\em arXiv preprint arXiv:1703.03130}, 2017.

\bibitem{multiseq2seq}
Minh-Thang Luong, Quoc~V. Le, Ilya Sutskever, Oriol Vinyals, and Lukasz Kaiser.
\newblock Multi-task sequence to sequence learning.
\newblock {\em arXiv preprint arXiv:1511.06114}, 2015.

\bibitem{luong2015effective}
Minh-Thang Luong, Hieu Pham, and Christopher~D Manning.
\newblock Effective approaches to attention-based neural machine translation.
\newblock {\em arXiv preprint arXiv:1508.04025}, 2015.

\bibitem{marcus1993building}
Mitchell~P Marcus, Mary~Ann Marcinkiewicz, and Beatrice Santorini.
\newblock Building a large annotated corpus of english: The penn treebank.
\newblock {\em Computational linguistics}, 19(2):313--330, 1993.

\bibitem{mcclosky-etAl:2006:NAACL}
David McClosky, Eugene Charniak, and Mark Johnson.
\newblock Effective self-training for parsing.
\newblock In {\em Proceedings of the Human Language Technology Conference of
  the NAACL, Main Conference}, pages 152--159. ACL, June 2006.

\bibitem{decomposableAttnModel}
Ankur Parikh, Oscar Täckström, Dipanjan Das, and Jakob Uszkoreit.
\newblock A decomposable attention model.
\newblock In {\em Empirical Methods in Natural Language Processing}, 2016.

\bibitem{paulus2017deep}
Romain Paulus, Caiming Xiong, and Richard Socher.
\newblock A deep reinforced model for abstractive summarization.
\newblock {\em arXiv preprint arXiv:1705.04304}, 2017.

\bibitem{petrov-EtAl:2006:ACL}
Slav Petrov, Leon Barrett, Romain Thibaux, and Dan Klein.
\newblock Learning accurate, compact, and interpretable tree annotation.
\newblock In {\em Proceedings of the 21st International Conference on
  Computational Linguistics and 44th Annual Meeting of the ACL}, pages
  433--440. ACL, July 2006.

\bibitem{press2016using}
Ofir Press and Lior Wolf.
\newblock Using the output embedding to improve language models.
\newblock {\em arXiv preprint arXiv:1608.05859}, 2016.

\bibitem{sennrich2015neural}
Rico Sennrich, Barry Haddow, and Alexandra Birch.
\newblock Neural machine translation of rare words with subword units.
\newblock {\em arXiv preprint arXiv:1508.07909}, 2015.

\bibitem{shazeer2017outrageously}
Noam Shazeer, Azalia Mirhoseini, Krzysztof Maziarz, Andy Davis, Quoc Le,
  Geoffrey Hinton, and Jeff Dean.
\newblock Outrageously large neural networks: The sparsely-gated
  mixture-of-experts layer.
\newblock {\em arXiv preprint arXiv:1701.06538}, 2017.

\bibitem{srivastava2014dropout}
Nitish Srivastava, Geoffrey~E Hinton, Alex Krizhevsky, Ilya Sutskever, and
  Ruslan Salakhutdinov.
\newblock Dropout: a simple way to prevent neural networks from overfitting.
\newblock {\em Journal of Machine Learning Research}, 15(1):1929--1958, 2014.

\bibitem{sukhbaatar2015}
Sainbayar Sukhbaatar, Arthur Szlam, Jason Weston, and Rob Fergus.
\newblock End-to-end memory networks.
\newblock In C.~Cortes, N.~D. Lawrence, D.~D. Lee, M.~Sugiyama, and R.~Garnett,
  editors, {\em Advances in Neural Information Processing Systems 28}, pages
  2440--2448. Curran Associates, Inc., 2015.

\bibitem{sutskever14}
Ilya Sutskever, Oriol Vinyals, and Quoc~VV Le.
\newblock Sequence to sequence learning with neural networks.
\newblock In {\em Advances in Neural Information Processing Systems}, pages
  3104--3112, 2014.

\bibitem{DBLP:journals/corr/SzegedyVISW15}
Christian Szegedy, Vincent Vanhoucke, Sergey Ioffe, Jonathon Shlens, and
  Zbigniew Wojna.
\newblock Rethinking the inception architecture for computer vision.
\newblock {\em CoRR}, abs/1512.00567, 2015.

\bibitem{KVparse15}
{Vinyals {\&} Kaiser}, Koo, Petrov, Sutskever, and Hinton.
\newblock Grammar as a foreign language.
\newblock In {\em Advances in Neural Information Processing Systems}, 2015.

\bibitem{wu2016google}
Yonghui Wu, Mike Schuster, Zhifeng Chen, Quoc~V Le, Mohammad Norouzi, Wolfgang
  Macherey, Maxim Krikun, Yuan Cao, Qin Gao, Klaus Macherey, et~al.
\newblock Google's neural machine translation system: Bridging the gap between
  human and machine translation.
\newblock {\em arXiv preprint arXiv:1609.08144}, 2016.

\bibitem{DBLP:journals/corr/ZhouCWLX16}
Jie Zhou, Ying Cao, Xuguang Wang, Peng Li, and Wei Xu.
\newblock Deep recurrent models with fast-forward connections for neural
  machine translation.
\newblock {\em CoRR}, abs/1606.04199, 2016.

\bibitem{zhu-EtAl:2013:ACL}
Muhua Zhu, Yue Zhang, Wenliang Chen, Min Zhang, and Jingbo Zhu.
\newblock Fast and accurate shift-reduce constituent parsing.
\newblock In {\em Proceedings of the 51st Annual Meeting of the ACL (Volume 1:
  Long Papers)}, pages 434--443. ACL, August 2013.

\end{thebibliography}
%\newpage
\pagebreak
\section*{Attention Visualizations}\label{sec:viz-att}
\begin{figure*}[h]
{\includegraphics[width=\textwidth, trim=0 0 0 36, clip]{./vis/making_more_difficult5_new.pdf}}
\caption{An example of the attention mechanism following long-distance dependencies in the encoder self-attention in layer 5 of 6. Many of the attention heads attend to a distant dependency of the verb `making', completing the phrase `making...more difficult'.  Attentions here shown only for the word `making'. Different colors represent different heads. Best viewed in color.}
\end{figure*}

\begin{figure*}
{\includegraphics[width=\textwidth, trim=0 0 0 45, clip]{./vis/anaphora_resolution_new.pdf}}
{\includegraphics[width=\textwidth, trim=0 0 0 37, clip]{./vis/anaphora_resolution2_new.pdf}}
\caption{Two attention heads, also in layer 5 of 6, apparently involved in anaphora resolution. Top: Full attentions for head 5. Bottom: Isolated attentions from just the word `its' for attention heads 5 and 6. Note that the attentions are very sharp for this word.}
\end{figure*}

\begin{figure*}
{\includegraphics[width=\textwidth, trim=0 0 0 36, clip]{./vis/attending_to_head_new.pdf}}
{\includegraphics[width=\textwidth, trim=0 0 0 36, clip]{./vis/attending_to_head2_new.pdf}}
\caption{Many of the attention heads exhibit behaviour that seems related to the structure of the sentence. We give two such examples above, from two different heads from the encoder self-attention at layer 5 of 6. The heads clearly learned to perform different tasks.}
\end{figure*}

%\appendix
%\newpage
%\pagebreak
\section*{Two Feed-Forward Layers = Attention over Parameters}\label{sec:parameter_attention}

In addition to attention layers, our model contains position-wise feed-forward networks (Section \ref{sec:ffn}), which consist of two linear transformations with a ReLU activation in between.  In fact, these networks too can be seen as a form of attention.  Compare the formula for such a network with the formula for a simple dot-product attention layer (biases and scaling factors omitted):

\begin{align*}
    FFN(x, W_1, W_2) = ReLU(xW_1)W_2 \\
    A(q, K, V) = Softmax(qK^T)V
\end{align*}

Based on the similarity of these formulae, the two-layer feed-forward network can be seen as a kind of attention, where the keys and values are the rows of the trainable parameter matrices $W_1$ and $W_2$, and where we use ReLU instead of Softmax in the compatibility function.

% the compatablity function is $compat(q, k_i) = ReLU(q \cdot k_i)$ instead of $Softmax(qK_T)_i$.

Given this similarity, we experimented with replacing the position-wise feed-forward networks with attention layers similar to the ones we use everywhere else our model. The multi-head-attention-over-parameters sublayer is identical to the multi-head attention described in \ref{sec:multihead}, except that the "keys" and "values" inputs to each attention head are trainable model parameters, as opposed to being linear projections of a previous layer.  These parameters are scaled up by a factor of $\sqrt{d_{model}}$ in order to be more similar to activations.

In our first experiment, we replaced each position-wise feed-forward network with a multi-head-attention-over-parameters sublayer with $h_p=8$ heads, key-dimensionality $d_{pk}=64$, and value-dimensionality $d_{pv}=64$, using $n_p=1536$ key-value pairs for each attention head.  The sublayer has a total of $2097152$ parameters, including the parameters in the query projection and the output projection.  This matches the number of parameters in the position-wise feed-forward network that we replaced.  While the theoretical amount of computation is also the same, in practice, the attention version caused the step times to be about 30\% longer.

In our second experiment, we used $h_p=8$ heads, and $n_p=512$ key-value pairs for each attention head, again matching the total number of parameters in the base model.

Results for the first experiment were slightly worse than for the base model, and results for the second experiment were slightly better, see Table~\ref{tab:parameter_attention}.

\begin{table}[h]
\caption{Replacing the position-wise feed-forward networks with multihead-attention-over-parameters produces similar results to the base model.  All metrics are on the English-to-German translation development set, newstest2013.}
\label{tab:parameter_attention}
\begin{center}
\vspace{-2mm}
%\scalebox{1.0}{
\begin{tabular}{c|cccccc|cccc}
\hline\rule{0pt}{2.0ex}
 & \multirow{2}{*}{$\dmodel$} & \multirow{2}{*}{$\dff$} &
\multirow{2}{*}{$h_p$} & \multirow{2}{*}{$d_{pk}$} & \multirow{2}{*}{$d_{pv}$} &
 \multirow{2}{*}{$n_p$} &
 PPL & BLEU & params & training\\
 & & & & & &  & (dev) & (dev) & $\times10^6$ & time \\
\hline\rule{0pt}{2.0ex}
base & 512 & 2048 & & & & & 4.92 & 25.8 & 65 & 12 hours\\
\hline\rule{0pt}{2.0ex}
AOP$_1$ & 512 & & 8 & 64 & 64 & 1536 & 4.92& 25.5  & 65 & 16 hours\\
AOP$_2$ & 512 & & 16 & 64 & 64 & 512 & \textbf{4.86} & \textbf{25.9}  & 65 & 16 hours \\
\hline
\end{tabular}
%}
\end{center}
\end{table}


%\section*{Justfication of the Scaling Factor in Dot-product Attention}

In Section~\ref{sec:scaled-dot-prod}, we introduced Scaled dot-product attention, where we scale down the dot products by $\sqrt{d_k}$.   In this section, we will give a rough justification of this scaling factor.  If we assume that $q$ and $k$ are $d_k$-dimensional vectors whose components are independent random variables with mean $0$ and variance $1$, then their dot product, $q \cdot k = \sum_{i=1}^{d_k} u_iv_i$, has mean $0$ and variance $d_k$.  Since we would prefer these values to have variance $1$, we divide by $\sqrt{d_k}$.  



%For any two $d_k$-dimension vectors $\vec{u}$ and $\vec{v}$, whose dimensions are independent, the mean and variance of the dot product will be the summation of the product of means and variances over the dimensions, that is, $E[<\vec{u},\vec{v}>] = \sum_{i=1}^{d_k} E[u_i]E[v_i]$, and $E[(<\vec{u},\vec{v}>-E[<\vec{u},\vec{v}>])^2] = \sum_{i=1}^{d_k} E[({u_i}-E[u_i])^2] E[({v_i}-E[v_i])^2]$. Layer norm encourages the mean and variance of each dimension to be $0$ and $1$ respectively, resultig in the dot product having mean $0$ and $d_k$ respectively. Therefore, scaling by $\sqrt{d_k}$ encourages the logits to be normalized as well. 

\iffalse

In this section, we will give a rough justification of this scaling factor, that is, we will show that for any two vectors, $\vec{u}$ and $\vec{v}$, whose variance and mean are $1$ and $0$ respectively, the variance and the mean of the dot product are $d_k$ and $0$ respectively. Therefore, dividing by $\sqrt{d_k}$ ensures that each component of the attention logits are normalized. The repeated layer norms at each transformer layer encourage $\vec{u}$ and $\vec{v}$ to be normalized. 


\begin{align*}
    E[<\vec{u},\vec{v}>] & =  \sum_k E[u_i v_i] &\text{By linearity of expectation} \\
    & =\sum_k E[u_i]E[v_i] & \text{Assuming independence} \\
    & = 0
\end{align*}

\begin{align*}
    E[(<\vec{u},\vec{v}>-E[<\vec{u},\vec{v}>])^2]  & = E[(<\vec{u},\vec{v}>)^2] - E[<\vec{u},\vec{v}>]^2 \\
    & = E[(<\vec{u},\vec{v}>)^2] \\
    & =  \sum_k E[{u_i}^2] E[{v_i}^2] &\text{By linearity of expectation and indepedence} \\
    & = d_k
\end{align*}


\fi

\end{document}
