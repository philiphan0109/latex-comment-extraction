% RLC main.tex Version 2024.4

\documentclass[10pt]{article} % For LaTeX2e
%\usepackage{rlc}
% If accepted, instead use the following line for the camera-ready submission:
\usepackage[accepted]{rlc}
% To de-anonymize and remove mentions to RLC (for example, for posting to preprint servers), instead use the following:
% \usepackage[preprint]{rlc}

\usepackage{booktabs} % for professional tables
\newcommand{\algorithmautorefname}{Algorithm}



% Optional math commands from https://github.com/goodfeli/dlbook_notation.
%%%%% NEW MATH DEFINITIONS %%%%%

\usepackage{amsmath,amsfonts,bm}

% Mark sections of captions for referring to divisions of figures
\newcommand{\figleft}{{\em (Left)}}
\newcommand{\figcenter}{{\em (Center)}}
\newcommand{\figright}{{\em (Right)}}
\newcommand{\figtop}{{\em (Top)}}
\newcommand{\figbottom}{{\em (Bottom)}}
\newcommand{\captiona}{{\em (a)}}
\newcommand{\captionb}{{\em (b)}}
\newcommand{\captionc}{{\em (c)}}
\newcommand{\captiond}{{\em (d)}}

% Highlight a newly defined term
\newcommand{\newterm}[1]{{\bf #1}}


% Figure reference, lower-case.
\def\figref#1{figure~\ref{#1}}
% Figure reference, capital. For start of sentence
\def\Figref#1{Figure~\ref{#1}}
\def\twofigref#1#2{figures \ref{#1} and \ref{#2}}
\def\quadfigref#1#2#3#4{figures \ref{#1}, \ref{#2}, \ref{#3} and \ref{#4}}
% Section reference, lower-case.
\def\secref#1{section~\ref{#1}}
% Section reference, capital.
\def\Secref#1{Section~\ref{#1}}
% Reference to two sections.
\def\twosecrefs#1#2{sections \ref{#1} and \ref{#2}}
% Reference to three sections.
\def\secrefs#1#2#3{sections \ref{#1}, \ref{#2} and \ref{#3}}
% Reference to an equation, lower-case.
\def\eqref#1{equation~\ref{#1}}
% Reference to an equation, upper case
\def\Eqref#1{Equation~\ref{#1}}
% A raw reference to an equation---avoid using if possible
\def\plaineqref#1{\ref{#1}}
% Reference to a chapter, lower-case.
\def\chapref#1{chapter~\ref{#1}}
% Reference to an equation, upper case.
\def\Chapref#1{Chapter~\ref{#1}}
% Reference to a range of chapters
\def\rangechapref#1#2{chapters\ref{#1}--\ref{#2}}
% Reference to an algorithm, lower-case.
\def\algref#1{algorithm~\ref{#1}}
% Reference to an algorithm, upper case.
\def\Algref#1{Algorithm~\ref{#1}}
\def\twoalgref#1#2{algorithms \ref{#1} and \ref{#2}}
\def\Twoalgref#1#2{Algorithms \ref{#1} and \ref{#2}}
% Reference to a part, lower case
\def\partref#1{part~\ref{#1}}
% Reference to a part, upper case
\def\Partref#1{Part~\ref{#1}}
\def\twopartref#1#2{parts \ref{#1} and \ref{#2}}

\def\ceil#1{\lceil #1 \rceil}
\def\floor#1{\lfloor #1 \rfloor}
\def\1{\bm{1}}
\newcommand{\train}{\mathcal{D}}
\newcommand{\valid}{\mathcal{D_{\mathrm{valid}}}}
\newcommand{\test}{\mathcal{D_{\mathrm{test}}}}

\def\eps{{\epsilon}}


% Random variables
\def\reta{{\textnormal{$\eta$}}}
\def\ra{{\textnormal{a}}}
\def\rb{{\textnormal{b}}}
\def\rc{{\textnormal{c}}}
\def\rd{{\textnormal{d}}}
\def\re{{\textnormal{e}}}
\def\rf{{\textnormal{f}}}
\def\rg{{\textnormal{g}}}
\def\rh{{\textnormal{h}}}
\def\ri{{\textnormal{i}}}
\def\rj{{\textnormal{j}}}
\def\rk{{\textnormal{k}}}
\def\rl{{\textnormal{l}}}
% rm is already a command, just don't name any random variables m
\def\rn{{\textnormal{n}}}
\def\ro{{\textnormal{o}}}
\def\rp{{\textnormal{p}}}
\def\rq{{\textnormal{q}}}
\def\rr{{\textnormal{r}}}
\def\rs{{\textnormal{s}}}
\def\rt{{\textnormal{t}}}
\def\ru{{\textnormal{u}}}
\def\rv{{\textnormal{v}}}
\def\rw{{\textnormal{w}}}
\def\rx{{\textnormal{x}}}
\def\ry{{\textnormal{y}}}
\def\rz{{\textnormal{z}}}

% Random vectors
\def\rvepsilon{{\mathbf{\epsilon}}}
\def\rvtheta{{\mathbf{\theta}}}
\def\rva{{\mathbf{a}}}
\def\rvb{{\mathbf{b}}}
\def\rvc{{\mathbf{c}}}
\def\rvd{{\mathbf{d}}}
\def\rve{{\mathbf{e}}}
\def\rvf{{\mathbf{f}}}
\def\rvg{{\mathbf{g}}}
\def\rvh{{\mathbf{h}}}
\def\rvu{{\mathbf{i}}}
\def\rvj{{\mathbf{j}}}
\def\rvk{{\mathbf{k}}}
\def\rvl{{\mathbf{l}}}
\def\rvm{{\mathbf{m}}}
\def\rvn{{\mathbf{n}}}
\def\rvo{{\mathbf{o}}}
\def\rvp{{\mathbf{p}}}
\def\rvq{{\mathbf{q}}}
\def\rvr{{\mathbf{r}}}
\def\rvs{{\mathbf{s}}}
\def\rvt{{\mathbf{t}}}
\def\rvu{{\mathbf{u}}}
\def\rvv{{\mathbf{v}}}
\def\rvw{{\mathbf{w}}}
\def\rvx{{\mathbf{x}}}
\def\rvy{{\mathbf{y}}}
\def\rvz{{\mathbf{z}}}

% Elements of random vectors
\def\erva{{\textnormal{a}}}
\def\ervb{{\textnormal{b}}}
\def\ervc{{\textnormal{c}}}
\def\ervd{{\textnormal{d}}}
\def\erve{{\textnormal{e}}}
\def\ervf{{\textnormal{f}}}
\def\ervg{{\textnormal{g}}}
\def\ervh{{\textnormal{h}}}
\def\ervi{{\textnormal{i}}}
\def\ervj{{\textnormal{j}}}
\def\ervk{{\textnormal{k}}}
\def\ervl{{\textnormal{l}}}
\def\ervm{{\textnormal{m}}}
\def\ervn{{\textnormal{n}}}
\def\ervo{{\textnormal{o}}}
\def\ervp{{\textnormal{p}}}
\def\ervq{{\textnormal{q}}}
\def\ervr{{\textnormal{r}}}
\def\ervs{{\textnormal{s}}}
\def\ervt{{\textnormal{t}}}
\def\ervu{{\textnormal{u}}}
\def\ervv{{\textnormal{v}}}
\def\ervw{{\textnormal{w}}}
\def\ervx{{\textnormal{x}}}
\def\ervy{{\textnormal{y}}}
\def\ervz{{\textnormal{z}}}

% Random matrices
\def\rmA{{\mathbf{A}}}
\def\rmB{{\mathbf{B}}}
\def\rmC{{\mathbf{C}}}
\def\rmD{{\mathbf{D}}}
\def\rmE{{\mathbf{E}}}
\def\rmF{{\mathbf{F}}}
\def\rmG{{\mathbf{G}}}
\def\rmH{{\mathbf{H}}}
\def\rmI{{\mathbf{I}}}
\def\rmJ{{\mathbf{J}}}
\def\rmK{{\mathbf{K}}}
\def\rmL{{\mathbf{L}}}
\def\rmM{{\mathbf{M}}}
\def\rmN{{\mathbf{N}}}
\def\rmO{{\mathbf{O}}}
\def\rmP{{\mathbf{P}}}
\def\rmQ{{\mathbf{Q}}}
\def\rmR{{\mathbf{R}}}
\def\rmS{{\mathbf{S}}}
\def\rmT{{\mathbf{T}}}
\def\rmU{{\mathbf{U}}}
\def\rmV{{\mathbf{V}}}
\def\rmW{{\mathbf{W}}}
\def\rmX{{\mathbf{X}}}
\def\rmY{{\mathbf{Y}}}
\def\rmZ{{\mathbf{Z}}}

% Elements of random matrices
\def\ermA{{\textnormal{A}}}
\def\ermB{{\textnormal{B}}}
\def\ermC{{\textnormal{C}}}
\def\ermD{{\textnormal{D}}}
\def\ermE{{\textnormal{E}}}
\def\ermF{{\textnormal{F}}}
\def\ermG{{\textnormal{G}}}
\def\ermH{{\textnormal{H}}}
\def\ermI{{\textnormal{I}}}
\def\ermJ{{\textnormal{J}}}
\def\ermK{{\textnormal{K}}}
\def\ermL{{\textnormal{L}}}
\def\ermM{{\textnormal{M}}}
\def\ermN{{\textnormal{N}}}
\def\ermO{{\textnormal{O}}}
\def\ermP{{\textnormal{P}}}
\def\ermQ{{\textnormal{Q}}}
\def\ermR{{\textnormal{R}}}
\def\ermS{{\textnormal{S}}}
\def\ermT{{\textnormal{T}}}
\def\ermU{{\textnormal{U}}}
\def\ermV{{\textnormal{V}}}
\def\ermW{{\textnormal{W}}}
\def\ermX{{\textnormal{X}}}
\def\ermY{{\textnormal{Y}}}
\def\ermZ{{\textnormal{Z}}}

% Vectors
\def\vzero{{\bm{0}}}
\def\vone{{\bm{1}}}
\def\vmu{{\bm{\mu}}}
\def\vtheta{{\bm{\theta}}}
\def\va{{\bm{a}}}
\def\vb{{\bm{b}}}
\def\vc{{\bm{c}}}
\def\vd{{\bm{d}}}
\def\ve{{\bm{e}}}
\def\vf{{\bm{f}}}
\def\vg{{\bm{g}}}
\def\vh{{\bm{h}}}
\def\vi{{\bm{i}}}
\def\vj{{\bm{j}}}
\def\vk{{\bm{k}}}
\def\vl{{\bm{l}}}
\def\vm{{\bm{m}}}
\def\vn{{\bm{n}}}
\def\vo{{\bm{o}}}
\def\vp{{\bm{p}}}
\def\vq{{\bm{q}}}
\def\vr{{\bm{r}}}
\def\vs{{\bm{s}}}
\def\vt{{\bm{t}}}
\def\vu{{\bm{u}}}
\def\vv{{\bm{v}}}
\def\vw{{\bm{w}}}
\def\vx{{\bm{x}}}
\def\vy{{\bm{y}}}
\def\vz{{\bm{z}}}

% Elements of vectors
\def\evalpha{{\alpha}}
\def\evbeta{{\beta}}
\def\evepsilon{{\epsilon}}
\def\evlambda{{\lambda}}
\def\evomega{{\omega}}
\def\evmu{{\mu}}
\def\evpsi{{\psi}}
\def\evsigma{{\sigma}}
\def\evtheta{{\theta}}
\def\eva{{a}}
\def\evb{{b}}
\def\evc{{c}}
\def\evd{{d}}
\def\eve{{e}}
\def\evf{{f}}
\def\evg{{g}}
\def\evh{{h}}
\def\evi{{i}}
\def\evj{{j}}
\def\evk{{k}}
\def\evl{{l}}
\def\evm{{m}}
\def\evn{{n}}
\def\evo{{o}}
\def\evp{{p}}
\def\evq{{q}}
\def\evr{{r}}
\def\evs{{s}}
\def\evt{{t}}
\def\evu{{u}}
\def\evv{{v}}
\def\evw{{w}}
\def\evx{{x}}
\def\evy{{y}}
\def\evz{{z}}

% Matrix
\def\mA{{\bm{A}}}
\def\mB{{\bm{B}}}
\def\mC{{\bm{C}}}
\def\mD{{\bm{D}}}
\def\mE{{\bm{E}}}
\def\mF{{\bm{F}}}
\def\mG{{\bm{G}}}
\def\mH{{\bm{H}}}
\def\mI{{\bm{I}}}
\def\mJ{{\bm{J}}}
\def\mK{{\bm{K}}}
\def\mL{{\bm{L}}}
\def\mM{{\bm{M}}}
\def\mN{{\bm{N}}}
\def\mO{{\bm{O}}}
\def\mP{{\bm{P}}}
\def\mQ{{\bm{Q}}}
\def\mR{{\bm{R}}}
\def\mS{{\bm{S}}}
\def\mT{{\bm{T}}}
\def\mU{{\bm{U}}}
\def\mV{{\bm{V}}}
\def\mW{{\bm{W}}}
\def\mX{{\bm{X}}}
\def\mY{{\bm{Y}}}
\def\mZ{{\bm{Z}}}
\def\mBeta{{\bm{\beta}}}
\def\mPhi{{\bm{\Phi}}}
\def\mLambda{{\bm{\Lambda}}}
\def\mSigma{{\bm{\Sigma}}}

% Tensor
\DeclareMathAlphabet{\mathsfit}{\encodingdefault}{\sfdefault}{m}{sl}
\SetMathAlphabet{\mathsfit}{bold}{\encodingdefault}{\sfdefault}{bx}{n}
\newcommand{\tens}[1]{\bm{\mathsfit{#1}}}
\def\tA{{\tens{A}}}
\def\tB{{\tens{B}}}
\def\tC{{\tens{C}}}
\def\tD{{\tens{D}}}
\def\tE{{\tens{E}}}
\def\tF{{\tens{F}}}
\def\tG{{\tens{G}}}
\def\tH{{\tens{H}}}
\def\tI{{\tens{I}}}
\def\tJ{{\tens{J}}}
\def\tK{{\tens{K}}}
\def\tL{{\tens{L}}}
\def\tM{{\tens{M}}}
\def\tN{{\tens{N}}}
\def\tO{{\tens{O}}}
\def\tP{{\tens{P}}}
\def\tQ{{\tens{Q}}}
\def\tR{{\tens{R}}}
\def\tS{{\tens{S}}}
\def\tT{{\tens{T}}}
\def\tU{{\tens{U}}}
\def\tV{{\tens{V}}}
\def\tW{{\tens{W}}}
\def\tX{{\tens{X}}}
\def\tY{{\tens{Y}}}
\def\tZ{{\tens{Z}}}


% Graph
\def\gA{{\mathcal{A}}}
\def\gB{{\mathcal{B}}}
\def\gC{{\mathcal{C}}}
\def\gD{{\mathcal{D}}}
\def\gE{{\mathcal{E}}}
\def\gF{{\mathcal{F}}}
\def\gG{{\mathcal{G}}}
\def\gH{{\mathcal{H}}}
\def\gI{{\mathcal{I}}}
\def\gJ{{\mathcal{J}}}
\def\gK{{\mathcal{K}}}
\def\gL{{\mathcal{L}}}
\def\gM{{\mathcal{M}}}
\def\gN{{\mathcal{N}}}
\def\gO{{\mathcal{O}}}
\def\gP{{\mathcal{P}}}
\def\gQ{{\mathcal{Q}}}
\def\gR{{\mathcal{R}}}
\def\gS{{\mathcal{S}}}
\def\gT{{\mathcal{T}}}
\def\gU{{\mathcal{U}}}
\def\gV{{\mathcal{V}}}
\def\gW{{\mathcal{W}}}
\def\gX{{\mathcal{X}}}
\def\gY{{\mathcal{Y}}}
\def\gZ{{\mathcal{Z}}}

% Sets
\def\sA{{\mathbb{A}}}
\def\sB{{\mathbb{B}}}
\def\sC{{\mathbb{C}}}
\def\sD{{\mathbb{D}}}
% Don't use a set called E, because this would be the same as our symbol
% for expectation.
\def\sF{{\mathbb{F}}}
\def\sG{{\mathbb{G}}}
\def\sH{{\mathbb{H}}}
\def\sI{{\mathbb{I}}}
\def\sJ{{\mathbb{J}}}
\def\sK{{\mathbb{K}}}
\def\sL{{\mathbb{L}}}
\def\sM{{\mathbb{M}}}
\def\sN{{\mathbb{N}}}
\def\sO{{\mathbb{O}}}
\def\sP{{\mathbb{P}}}
\def\sQ{{\mathbb{Q}}}
\def\sR{{\mathbb{R}}}
\def\sS{{\mathbb{S}}}
\def\sT{{\mathbb{T}}}
\def\sU{{\mathbb{U}}}
\def\sV{{\mathbb{V}}}
\def\sW{{\mathbb{W}}}
\def\sX{{\mathbb{X}}}
\def\sY{{\mathbb{Y}}}
\def\sZ{{\mathbb{Z}}}

% Entries of a matrix
\def\emLambda{{\Lambda}}
\def\emA{{A}}
\def\emB{{B}}
\def\emC{{C}}
\def\emD{{D}}
\def\emE{{E}}
\def\emF{{F}}
\def\emG{{G}}
\def\emH{{H}}
\def\emI{{I}}
\def\emJ{{J}}
\def\emK{{K}}
\def\emL{{L}}
\def\emM{{M}}
\def\emN{{N}}
\def\emO{{O}}
\def\emP{{P}}
\def\emQ{{Q}}
\def\emR{{R}}
\def\emS{{S}}
\def\emT{{T}}
\def\emU{{U}}
\def\emV{{V}}
\def\emW{{W}}
\def\emX{{X}}
\def\emY{{Y}}
\def\emZ{{Z}}
\def\emSigma{{\Sigma}}

% entries of a tensor
% Same font as tensor, without \bm wrapper
\newcommand{\etens}[1]{\mathsfit{#1}}
\def\etLambda{{\etens{\Lambda}}}
\def\etA{{\etens{A}}}
\def\etB{{\etens{B}}}
\def\etC{{\etens{C}}}
\def\etD{{\etens{D}}}
\def\etE{{\etens{E}}}
\def\etF{{\etens{F}}}
\def\etG{{\etens{G}}}
\def\etH{{\etens{H}}}
\def\etI{{\etens{I}}}
\def\etJ{{\etens{J}}}
\def\etK{{\etens{K}}}
\def\etL{{\etens{L}}}
\def\etM{{\etens{M}}}
\def\etN{{\etens{N}}}
\def\etO{{\etens{O}}}
\def\etP{{\etens{P}}}
\def\etQ{{\etens{Q}}}
\def\etR{{\etens{R}}}
\def\etS{{\etens{S}}}
\def\etT{{\etens{T}}}
\def\etU{{\etens{U}}}
\def\etV{{\etens{V}}}
\def\etW{{\etens{W}}}
\def\etX{{\etens{X}}}
\def\etY{{\etens{Y}}}
\def\etZ{{\etens{Z}}}

% The true underlying data generating distribution
\newcommand{\pdata}{p_{\rm{data}}}
% The empirical distribution defined by the training set
\newcommand{\ptrain}{\hat{p}_{\rm{data}}}
\newcommand{\Ptrain}{\hat{P}_{\rm{data}}}
% The model distribution
\newcommand{\pmodel}{p_{\rm{model}}}
\newcommand{\Pmodel}{P_{\rm{model}}}
\newcommand{\ptildemodel}{\tilde{p}_{\rm{model}}}
% Stochastic autoencoder distributions
\newcommand{\pencode}{p_{\rm{encoder}}}
\newcommand{\pdecode}{p_{\rm{decoder}}}
\newcommand{\precons}{p_{\rm{reconstruct}}}

\newcommand{\laplace}{\mathrm{Laplace}} % Laplace distribution

\newcommand{\E}{\mathbb{E}}
\newcommand{\Ls}{\mathcal{L}}
\newcommand{\R}{\mathbb{R}}
\newcommand{\emp}{\tilde{p}}
\newcommand{\lr}{\alpha}
\newcommand{\reg}{\lambda}
\newcommand{\rect}{\mathrm{rectifier}}
\newcommand{\softmax}{\mathrm{softmax}}
\newcommand{\sigmoid}{\sigma}
\newcommand{\softplus}{\zeta}
\newcommand{\KL}{D_{\mathrm{KL}}}
\newcommand{\Var}{\mathrm{Var}}
\newcommand{\standarderror}{\mathrm{SE}}
\newcommand{\Cov}{\mathrm{Cov}}
% Wolfram Mathworld says $L^2$ is for function spaces and $\ell^2$ is for vectors
% But then they seem to use $L^2$ for vectors throughout the site, and so does
% wikipedia.
\newcommand{\normlzero}{L^0}
\newcommand{\normlone}{L^1}
\newcommand{\normltwo}{L^2}
\newcommand{\normlp}{L^p}
\newcommand{\normmax}{L^\infty}

\newcommand{\parents}{Pa} % See usage in notation.tex. Chosen to match Daphne's book.

\DeclareMathOperator*{\argmax}{arg\,max}
\DeclareMathOperator*{\argmin}{arg\,min}

\DeclareMathOperator{\sign}{sign}
\DeclareMathOperator{\Tr}{Tr}
\let\ab\allowbreak

\newcommand\blfootnote[1]{%
  \begingroup
  \renewcommand\thefootnote{}\footnote{#1}%
  \addtocounter{footnote}{-1}%
  \endgroup
}

\usepackage{hyperref}
\hypersetup{colorlinks,linkcolor={blue},citecolor={magenta},urlcolor={red}}  

\newcommand{\psc}[1]{\textcolor{blue}{\textbf{[psc: }#1\textbf{]}}}
\newcommand{\johan}[1]{\textcolor{red}{\textbf{[johan: }#1\textbf{]}}}
\newcommand{\joao}[1]{\textcolor{green}{\textbf{[joao: }#1\textbf{]}}}
\newcommand\sbullet[1][.5]{\mathbin{\vcenter{\hbox{\scalebox{#1}{$\bullet$}}}}}
\usepackage{graphicx}
\usepackage{subfigure}
\usepackage{wrapfig}

\usepackage{algorithm}
\usepackage{algpseudocode}

\usepackage{bbm}

\usepackage{hyperref}
\usepackage{cleveref}
\usepackage{url}


\title{On the consistency of hyper-parameter selection in value-based deep reinforcement learning}

% Authors must not appear in the submitted version. They should be hidden
% as long as the tmlr package is used without the [accepted] or [preprint] options.
% Non-anonymous submissions will be rejected without review.
% \author{Johan Obando-Ceron\footnotemark[1]\\%\thanks{Equal contribution}
%       jobando0730@gmail.com \\
%       Mila - Québec AI Institute \\
%       Universit\'e de Montr\'eal \\
%       Google DeepMind
%       \And
%       João G.M. Araújo \thanks{Equal contribution}\\%\printfnsymbol{1}\\
%       joaogui@google.com \\
%       Google DeepMind
%       \And
%       Aaron Courville \\
%       aaron.courville@umontreal.ca \\
%       Mila - Québec AI Institute, Universit\'e de Montr\'eal \\
%       \And
%       Pablo Samuel Castro\\
%       psc@google.com\\
%       Google DeepMind\\
%       Mila - Québec AI Institute, Universit\'e de Montr\'eal \\
%       }


\author{Johan Obando-Ceron\(^{*1,2,3}\), João G.M. Araújo\(^{*3}\), Aaron Courville\(^{1,2}\), \\\textbf{Pablo Samuel Castro\(^{1,2,3}\)
}\\\\
Mila - Québec AI Institute\(^{1}\) \\
Universit\'e de Montr\'eal\(^{2}\) \\
Google DeepMind\(^{3}\)  \\
}


% The \author macro works with any number of authors. Use \AND 
% to separate the names and addresses of multiple authors.

\newcommand{\fix}{\marginpar{FIX}}
\newcommand{\new}{\marginpar{NEW}}

\def\month{June}  % Insert correct month for camera-ready version
\def\year{2024} % Insert correct year for camera-ready version
\def\openreview{\url{https://openreview.net/pdf?id=szUyvvwoZB}} % Insert correct link to OpenReview for camera-ready version


\begin{document}


\maketitle
\blfootnote{*Authors contributed equally. Correspondence to \texttt{jobando0730@gmail.com},\texttt{[joaogui,psc]@google.com}}

\begin{abstract}
Deep reinforcement learning (deep RL) has achieved tremendous success on various domains through a combination of algorithmic design and careful selection of hyper-parameters. Algorithmic improvements are often the result of iterative enhancements built upon prior approaches, while hyper-parameter choices are typically inherited from previous methods or fine-tuned specifically for the proposed technique. Despite their crucial impact on performance, hyper-parameter choices are frequently overshadowed by algorithmic advancements. This paper conducts an extensive empirical study focusing on the reliability of hyper-parameter selection for value-based deep reinforcement learning agents, including the introduction of a new score to quantify the consistency and reliability of various hyper-parameters. Our findings not only help establish which hyper-parameters are most critical to tune, but also help clarify which tunings remain {\em consistent} across different training regimes.

\end{abstract}

\section{Introduction}
\label{sec:introduction}

Sequential decision making is generally considered an essential ingredient for generally capable agents. The ability to plan ahead and adapt to changing circumstances is synonymous with the concept of {\em agency}. For decades, the field of reinforcement learning (RL) has worked on developing methods, or agents, for precisely this purpose. This research has borne impressive results, such as developing agents which can play difficult Atari games \citep{mnih2015humanlevel}, control stratospheric balloons \citep{Bellemare2020AutonomousNO}, control a tokamak fusion reactor \citep{Degrave2022MagneticCO}, among others. These are all examples of {\em deep reinforcement learning} (DRL), which combines the theory of reinforcement learning with the expressiveness and flexibility of deep neural networks.

The success of these methods built on years of academic research, where novel algorithms and techniques were introduced and showcased on academic benchmarks such as the ALE \citep{bellemare2012ale}, MuJoCo \citep{todorov2012mujoco}, and others. These benchmarks typically consist of a suite of environments that have varied transition and reward dynamics. Their common usage provides us with a familiarity which affords us a sense of interpretability, a consistency in evaluation that grants us a sense of reliability, and their variety yields a sense of generalizability. Unfortunately, this promise often fails to materialize: their reliability has been brought into question by numerous works which demonstrate their fickleness \citep{Henderson2017DeepRL,agarwal2021deep}, while there is a general sentiment that researchers have ``overfit’’ to these benchmarks, bringing into question their generalizability. A critical aspect to these challenges is the difficulty in training neural networks in an RL setting \citep{ostrovski2021the,lyle2022learning,sokar2023dormant}.

Although the successes above built on prior methods, they were not taken ``as is’’: it took large teams of researchers many months and lots of compute to adapt prior work to their specific problem. These adaptations include changes to the network architectures, designing reward functions to induce the desired behaviours, and careful tuning of the many hyper-parameters. This last point is indeed {\em essential} to the success of any DRL method: improper hyper-parameter choices can cause a theoretically sound method to drastically underperform, while careful hyper-parameter selection can dramatically increase the performance of an otherwise sub-optimal method.

As an example of this dichotomy, we examine how DER \citep{hasselt19when}, a method that has become a common baseline for the Atari $100$k benchmark \citep{kaiser2020modelbased}, came to be. DQN, considered to be the start of the field of DRL research, was introduced by showcasing its super-human performance on the ALE \citep{bellemare2012ale}, a suite of $57$ Atari $2600$ games. This suite became one of the most popular benchmarks on which to evaluate new methods over $200$ million environment frames\footnote{See \citep{machado2018revisiting} for more details on ALE evaluation standards.}. A few years later, when \citet{kaiser2020modelbased} introduced the SiMPLe algorithm as a sample-efficient method, they argued for evaluating it only on $100$k agent actions\footnote{The standard for ALE agents is to use frame-skipping, where $4$ environment frames occur for every agent action. This results in frustratingly confusing nomenclature, as $200$M is specified in environment frames (or $500$k agent actions), while $100$k is specified in agent actions (or $400$k environment frames).} with a subset of $26$ games, so as to properly test the sample-efficiency of new methods. The authors demonstrated that their proposed method outperformed Rainbow \citep{Hessel2018RainbowCI}, the state-of-the-art method of the time. In response, \citet{hasselt19when} introduced Data Efficient Rainbow (DER), which outperformed SiMPLe even though it was the same Rainbow algorithm, but {\em with a careful tuning of the hyper-parameters for the $100$k training regime}.

One could argue that the hyper-parameters of Rainbow were overly-tuned to the $200$M benchmark, while the hyper-parameters of DER were overly-tuned to the $100$k benchmark. More importantly, what this story highlights is that, despite careful evaluation it is quite likely that a new method {\em will not work as intended when deployed on a different environment from which it was trained on}, and that a significant  amount of hyper-parameter tuning will be necessary. This flies in the face of the supposed generalizability of DRL academic research, and makes it difficult for groups without large computational budgets to successfully apply prior work to applied problems.

It thus behooves the community to develop a better understanding of the {\em transferability} and {\em consistency} of hyper-parameter selection across different training regimes, and to build a better shared understanding of the relative importance of the many possible hyper-parameters to tune. In this work, we take a stride towards this by conducting an exhaustive empirical investigation of the various hyper-parameters affecting DRL agents. We focus our attention on two value-based agents developed for the Atari $100$k suite: DER mentioned above, and DrQ($\epsilon$), a variant of DQN that was optimized for the $100$k suite. Although developed for the $100$k suite, we also train these agents for $40$M million environment frames. Our intent is to examine the transferability of various hyper-parameter choices across different training regimes. Specifically, we investigate:
{\bf Across data regimes:} Do hyper-parameters selected in the $100$k regime work well in a larger data regime? {\bf Across agents:} Do hyper-parameters selected for one agent work well in another? {\bf Across environments:} Do hyper-parameters tuned in one set of environments work well in others?

In total, we investigated $12$ hyper-parameters with different values for $2$ agents over $26$ environments, each for $5$ seeds, resulting in a total of $108$k independent training runs. This breadth of experimentation results in an overwhelming amount of data which complicates their analyses. We address this challenge in two ways: \textit{(i)} We introduce a new score which provides us with an aggregate value for the considerations mentioned above. \textit{(ii)} We provide an interactive website where others may easily navigate the large number of experimental figures we have generated.

The score provides us with a high-level overview of our findings, while the website grants us a fine-grained mechanism to analyze the results. We hope this effort provides the community with useful tools so as to develop not just better DRL algorithms, but better methodologies to evaluate their interpretability, reliability, and generalizability.
The goal of reducing sequential computation also forms the foundation of the Extended Neural GPU \citep{extendedngpu}, ByteNet \citep{NalBytenet2017} and ConvS2S \citep{JonasFaceNet2017}, all of which use convolutional neural networks as basic building block, computing hidden representations in parallel for all input and output positions. In these models, the number of operations required to relate signals from two arbitrary input or output positions grows in the distance between positions, linearly for ConvS2S and logarithmically for ByteNet. This makes it more difficult to learn dependencies between distant positions \citep{hochreiter2001gradient}. In the Transformer this is reduced to a constant number of operations, albeit at the cost of reduced effective resolution due to averaging attention-weighted positions, an effect we counteract with Multi-Head Attention as described in section~\ref{sec:attention}. 

Self-attention, sometimes called intra-attention is an attention mechanism relating different positions of a single sequence in order to compute a representation of the sequence. Self-attention has been used successfully in a variety of tasks including reading comprehension, abstractive summarization, textual entailment and learning task-independent sentence representations \citep{cheng2016long, decomposableAttnModel, paulus2017deep, lin2017structured}.

End-to-end memory networks are based on a recurrent attention mechanism instead of sequence-aligned recurrence and have been shown to perform well on simple-language question answering and language modeling tasks \citep{sukhbaatar2015}.

To the best of our knowledge, however, the Transformer is the first transduction model relying entirely on self-attention to compute representations of its input and output without using sequence-aligned RNNs or convolution.
In the following sections, we will describe the Transformer, motivate self-attention and discuss its advantages over models such as \citep{neural_gpu, NalBytenet2017} and \citep{JonasFaceNet2017}.


%\citep{JonasFaceNet2017} report new SOTA on machine translation for English-to-German (EnDe), Enlish-to-French (EnFr) and English-to-Romanian language pairs. 

%For example,! in MT, we must draw information from both input and previous output words to translate an output word accurately. An attention layer \citep{bahdanau2014neural} can connect a very large number of positions at low computation cost, making it an essential ingredient in competitive recurrent models for machine translation.

%A natural question to ask then is, "Could we replace recurrence with attention?". \marginpar{Don't know if it's the most natural question to ask given the previous statements. Also, need to say that the complexity table summarizes these statements} Such a model would be blessed with the computational efficiency of attention and the power of cross-positional communication. In this work, show that pure attention models work remarkably well for MT, achieving new SOTA results on EnDe and EnFr, and can be trained in under $2$ days on xyz architecture. 

%After the seminal models introduced in \citep{sutskever14, bahdanau2014neural, cho2014learning}, recurrent models have become the dominant solution for both sequence modeling and sequence-to-sequence transduction. Many efforts such as \citep{wu2016google,luong2015effective,jozefowicz2016exploring} have pushed the boundaries of machine translation (MT) and language modeling with recurrent endoder-decoder and recurrent language models. Recent effort \citep{shazeer2017outrageously} has successfully combined the power of conditional computation with sequence models to train very large models for MT, pushing SOTA at lower computational cost.

%Recurrent models compute a vector of hidden states $h_t$, for each time step $t$ of computation. $h_t$ is a function of both the input at time $t$ and the previous hidden state $h_t$. This dependence on the previous hidden state precludes processing all timesteps at once, instead requiring long sequences of sequential operations.  In practice, this results in greatly reduced computational efficiency, as on modern computing hardware, a single operation on a large batch is much faster than a large number of operations on small batches.  The problem gets worse at longer sequence lengths. Although sequential computation is not a severe bottleneck at inference time, as autoregressively generating each output requires all previous outputs, the inability to compute scores at all output positions at once hinders us from rapidly training our models over large datasets. Although impressive work such as \citep{Kuchaiev2017Factorization} is able to significantly accelerate the training of LSTMs with factorization tricks, we are still bound by the linear dependence on sequence length.

%If the model could compute hidden states at each time step using only the inputs and outputs,  it would be liberated from the dependence on results from previous time steps during training. This line of thought is the foundation of recent efforts such as the Markovian neural GPU \citep{neural_gpu}, ByteNet \citep{NalBytenet2017} and ConvS2S \citep{JonasFaceNet2017}, all of which use convolutional neural networks as a building block to compute hidden representations simultaneously for all timesteps, resulting in $O(1)$ sequential time complexity. \citep{JonasFaceNet2017} report new SOTA on machine translation for English-to-German (EnDe), Enlish-to-French (EnFr) and English-to-Romanian language pairs. 

%A crucial component for accurate sequence prediction is modeling cross-positional communication. For example, in MT, we must draw information from both input and previous output words to translate an output word accurately. An attention layer \citep{bahdanau2014neural} can connect a very large number of positions at a low computation cost, also $O(1)$ sequential time complexity, making it an essential ingredient in recurrent encoder-decoder architectures for MT. A natural question to ask then is, "Could we replace recurrence with attention?". \marginpar{Don't know if it's the most natural question to ask given the previous statements. Also, need to say that the complexity table summarizes these statements} Such a model would be blessed with the computational efficiency of attention and the power of cross-positional communication. In this work, show that pure attention models work remarkably well for MT, achieving new SOTA results on EnDe and EnFr, and can be trained in under $2$ days on xyz architecture. 



%Note: Facebook model is no better than RNNs in this regard, since it requires a number of layers proportional to the distance you want to communicate.  Bytenet is more promising, since it requires a logarithmnic number of layers (does bytenet have SOTA results)?   

%Note: An attention  layer can connect a very large number of positions at a low computation cost in O(1) sequential operations.  This is why encoder-decoder attention has been so successful in seq-to-seq models so far.  It is only natural, then, to also use attention to connect the timesteps of the same sequence.

%Note: I wouldn't say that long sequences are not a problem during inference.  It would be great if we could infer with no long sequences.  We could just say later on that, while our training graph is constant-depth, our model still requires sequential operations in the decoder part during inference due to the autoregressive nature of the model.   

%\begin{table}[h!]
%\caption{Attention models are quite efficient for cross-positional communications when sequence length is smaller than channel depth. $n$ represents the sequence length and $d$ represents the channel depth.}
%\label{tab:op_complexities}
%\begin{center}
%\vspace{-5pt}
%\scalebox{0.75}{

%\begin{tabular}{l|c|c|c}
%\hline \hline
%Layer Type & Receptive & Complexity & Sequential  \\
%           & Field     &            & Operations  \\
%\hline
%Pointwise Feed-Forward & $1$ & $O(n \cdot d^2)$ & $O(1)$ \\
%\hline
%Recurrent & $n$ & $O(n \cdot d^2)$ & $O(n)$ \\
%\hline
%Convolutional & $r$ & $O(r \cdot n \cdot d^2)$ & $O(1)$ \\
%\hline
%Convolutional (separable) & $r$ & $O(r \cdot n \cdot d + n %\cdot d^2)$ & $O(1)$ \\
%\hline
%Attention & $r$ & $O(r \cdot n \cdot d)$ & $O(1)$ \\
%\hline \hline
%\end{tabular}
%}
%\end{center}
%\end{table}
\section{THC Score}
\label{sec:thc_metric}

Statistical metrics play a crucial role in assessing and evaluating the performance of DRL algorithms. 
They provide valuable insights into the strengths and weaknesses of different approaches, guiding researchers and practitioners in the development of more effective reinforcement learning systems.
For example, some the metrics focus on the mean reward obtained by an agent per time step (Average Reward), the percentage of episodes in which the agent achieves a predefined goal or task (success rate) among others \citep{agarwal2021deep, chan2020measuring, Henderson2017DeepRL}. 

Measuring the transferability/consistency of hyper-parameters in DRL is challenging, as existing metrics fall short in capturing the nuanced aspects of how well hyper-parameter settings generalize across different environments or agents. Developing such a metric would enhance the ability to systematically compare and select hyper-parameter configurations that exhibit robust performance across a range of application domains.

To understand the consistency of hyper-parameters we focus on their ranking consistency across experimental settings. Put another way: if a given hyper-parameter value is optimal/pessimal in a setting, is it still optimal/pessimal in another? And so we analyse, for each hyper-parameter, whether their values lead to the same ranking order for different experimental settings, where the ranking is on final performance. 

We compute ranking agreement for three setups: 
{\bf $1$) Varying algorithms} while keeping the environment and data regime fixed (e.g. when proposing a new value-based algorithm but not having enough compute to run a comprehensive hyper-parameter search). {\bf $2$) Varying environments} while keeping the algorithm and data regime fixed (e.g. when using a state of the art algorithm in a new domain).
{\bf $3$) Varying data regimes} while keeping the environment and algorithm fixed (e.g. when adapting a new algorithm to a new data regime \citep{hasselt19when}).
Concretely, our desire is to have a metric that yields a high value score would indicate that the hyper-parameter in question is {\em important}, in the sense that it will likely require retuning; conversely, a low score suggests the hyper-parameter value can likely be kept as is.

Kendall's Tau \citep{kendall38measure} and Kendall's W \citep{10.1214/aoms/1177732186} are natural choices, but these metrics were developed for situations where the rankings were based on a single score, instead of a range of possible scores, and they can result in degenerate values when two settings have similar performance or when two settings alternate between optimal and pessimal rankings. For these reasons, we introduce the \textbf{T}uning \textbf{H}yperparameter \textbf{C}onsistency ({\bf THC}) score. Consider a set of $n$ hyper-parameters $\lbrace H_1,\ldots,H_n\rbrace$, each with its set of values $\lbrace\lbrace h_{11},h_{12},\ldots,h_{1m_1}\rbrace, \ldots,\lbrace h_{n1},h_{n2},\ldots,h_{nm_n}\rbrace\rbrace$ (e.g. hyper-parameter $H_i$ has $m_i$ values). The THC score involves three computations: (i) rankings for each hyper-parameter setting (\autoref{alg:computeRankings}); (ii) normalized peak-to-peak value for each hyper-parameter setting (Eqn.~\ref{eqn:ptp} below); and (iii) overall THC score for the hyper-parameter (see Eqn.~\ref{eqn:thc} below).

If we run multiple independent runs for each hyper-parameter setting $h_{ij}$, we can compute the mean $\mu_{ij}$ and standard deviation $\sigma_{ij}$ for these runs\footnote{One may also use confidence intervals instead of standard deviations.}. For each hyper-parameter setting $h_{ij}$ we then compute an initial ranking $r'_{ij}$ based on the upper bound ($\mu_{ij}+\sigma_{ij}$), with the lower bound ($\mu_{ij}-\sigma_{ij}$) used to break ties. We then define a set containing hyper-parameter settings with overlapping values:
\begin{align*}
    I_{ij} := \{k \vert (\mu_{ij} - \sigma_{ij} < \mu_{ik} + \sigma_{ik} &\text{ and } \mu_{ij} - \sigma_{ij} > \mu_{ik} - \sigma_{ik}) \\ &\text{ or } \\ \break (\mu_{ij} + \sigma_{ij} > \mu_{ik} - \sigma_{ik} &\text{ and } \mu_{ij} + \sigma_{ij} < \mu_{ik} + \sigma_{ik}) \}
\end{align*}



\begin{algorithm}[!t]
\caption{Compute rankings}\label{alg:computeRankings}
\begin{algorithmic}[1]
\Require Multiple runs for various settings of hyper-parameter $H_i$: $\lbrace h_{i1},h_{i2},\ldots,h_{im_i}\rbrace$, aggregate metrics $\mu_i$: $\lbrace \mu_{i1},\mu_{i2},\ldots,\mu_{im_i}\rbrace$ and measure of spread $\sigma_i$: $\lbrace \sigma_{i1},\sigma_{i2},\ldots,\sigma_{im_i}\rbrace$
\For{$i$ in $1 \ldots n$}
    \State $r'_{i} = \textrm{argsort}(\mu_i + \sigma_i)$ \Comment{Gets the index of each value as if the array was sorted}
    \State $\mu'_i, \sigma'_i = \mu_i[r'_{i}], \sigma_i[r'_{i}]$ \Comment{Sorted versions of aggregate and spread metrics}
    \For{$j$ in $1 \ldots m_{i}$} 
        \State $u_{j} = \textrm{binary\_search}(\mu'_i - \sigma'_i, \mu_{ij} + \sigma_{ij})$ \Comment{highest rank whose lower bound overlaps with j}
        \State $l_{j} = \textrm{binary\_search}(\mu'_i + \sigma'_i, \mu_{ij} - \sigma_{ij})$ \Comment{lowest rank whose upper bound overlaps with j}
    \EndFor
    \State $\bf{r_{i}} = \frac{u + l}{2}$ \Comment{The average rank in $l_j,l_j+1, \ldots, u_j$ is the average of $l_j$ and $u_j$}
\EndFor

\end{algorithmic}
\end{algorithm}

The final ranking of each hyper-parameter is $r_{ij} = \frac{\sum_{k \in I_{ij}} r
'_{ik}}{\vert I_{ij} \vert}$, 
as \autoref{alg:computeRankings} details. These rankings are for {\em one} training regime; however, as mentioned in the introduction, we are interested in quantifying the {\em consistency} of a hyper-parameter $H$ across varying training regimes. Consider four training regimes $A, B, C, D$, and let $\lbrace \mathfrak{R}^A,\ldots,\mathfrak{R}^D\rbrace$ denote their respective rankings. For each hyper-parameter value $h_x\in H$ we compute its normalized ``peak-to-peak''\footnote{Inspired by numpy's peak-to-peak function numpy.ptp \citep{harris2020array}.} value $\overline{\textrm{ptp}}$, which quantifies its variance in ranking, as follows: First compute the $\textrm{ptp}$ value $\textrm{ptp}(h_x) = \max\left(\lbrace \mathfrak{R}^A(h_x),\ldots,\mathfrak{R}^D(h_x)\rbrace\right) - \min\left(\lbrace \mathfrak{R}^A(h_x),\ldots,\mathfrak{R}^D(h_x)\rbrace\right)$, then normalize:
\begin{align}
    \overline{\textrm{ptp}}(h_x) = \frac{\textrm{ptp}(h_x)}{\sum_{h_y\in H}\textrm{ptp}(h_y)}
    \label{eqn:ptp}
\end{align}

Notably, hyper-parameter settings that have consistent rankings across training regimes will have a normalized $\textrm{ptp}$ value of zero. Finally, the $\textrm{THC}$ score for hyper-parameter $H$ is defined as:
\begin{align}
    \textrm{THC}(H) = \frac{\sum_{h_x\in H}\overline{\textrm{ptp}}(h_x)}{|H|}.
    \label{eqn:thc}
\end{align}

This score will result in low values for hyper-parameters whose varying settings have consistent ranking across various training regimes, and high values when these rankings vary. Intuitively, {\em hyper-parameters with high values will most likely require re-tuning when switching training regimes}. See \autoref{sec:appendixTHC} for more examples of computing the score, as well as the source code provided with this submission.



\section{Hyper-parameters considered} 
\label{sec:hyper-parameter_selection}

We describe the set of hyper-parameters explored in this work, with the values used for each listed in \autoref{sec:list_hyperparameters}. Unless otherwise specified, these are examined for both Conv and Dense layers.

{\bf Activation functions:} 
Non-linear activation functions are a fundamental part of neural networks, as their removal effectively turns the network into a linear function approximator.
While various activation functions have been proposed \citep{devlin2019bert, Elfwing2018SigmoidWeightedLU, 10.5555/3305381.3305478}, there have been few works comparing their performance \citep{Shamir2020SmoothAA}; to the best of our knowledge, there are no previous examples of such a comparison in the RL setting.


{\bf Normalization: }
Normalization plays an important role in supervised learning \citep{tan2020efficientnet, xie2017aggregated} but is relatively rare in deep reinforcement learning, with a few exceptions \citep{gogianu2021spectral, bhatt2019crossnorm, arpit2019initialize, alphaZero}. We explore {\em batch normalization} \citep{ioffe2015batch} and {\em layer normalization} \citep{ba2016layer}.

{\bf Network capacity: } 
``Scaling laws'' have been central to the growth of capabilities in large language/vision models, but have mostly eluded reinforcement learning agents, with a few exceptions \citep{schwarzer23a, taiga2022investigating, farebrother2022proto,obando2024mixtures,obandoceron2024pruned,farebrother2024stop}. 
To investigate the impact of network size, we vary the {\em depth} (e.g. the number of hidden layers) and the {\em width} (e.g. the number of neurons of each hidden layer).

{\bf Optimizer hyper-parameters: }
\label{sec:optimizerHypers}
We explore three hyper-parameters of Adam \citep{kingma15adam}, which has become the standard optimizer used by most: {\em learning rate}, {\em epsilon} and {\em weight decay}.
\emph{Learning rate} determines the step size at which the algorithm adjusts the model's parameters during each iteration.
$\epsilon$ represents a small constant value that is added to the denominator of the update rule to avoid numerical instabilities.
\emph{Weight decay} adds a penalty term to the loss function during training that discourages the model from assigning excessively increasing weight magnitudes.


{\bf $\epsilon$-greedy exploration: } 
$\epsilon$-greedy exploration is a simple and popular exploration technique which picks actions greedily with probability $1-\epsilon$, and a random action with probability $\epsilon$. Traditionally, experiments on the ALE use a linear decay strategy to decay $\epsilon$ from $1.0$ to its target value.

{\bf Reward clipping: } 
Most ALE experiments clip rewards at $(-1, 1)$ \citep{mnih2015humanlevel}.

{\bf Discount factor: } 
The multiplicative factor $\gamma$ discounts future rewards and its importance has been observed in a number of recent works \citep{amit2020discount, hessel19inductive, gelada2019off, vanseijen2019using, francoislavet2016discount,schwarzer23a}.

{\bf Replay buffer: }  
DRL agents  store past experiences in a replay buffer, to sample from during learning. The {\em replay capacity} parameter refers to the amount of data experiences stored in the buffer. 
It is common practice to only begin sampling from the replay buffer when a minimum number of transitions have been stored, referred to as the {\em minimum replay history}.


{\bf Batch size: } 
The number of stored transitions that are sampled for learning at each training step.

{\bf Update horizon: }
Multi-step learning \citep{sutton88learning} computes the temporal difference error using multi-step transitions, instead of a single step. DQN uses a single-step update by default, whereas Rainbow chose a 3-step update \citep{Hessel2018RainbowCI}. The update horizon has been argued to trade-off between the bias and the variance of the return estimate \citep{biasandvariance_kea}. 


{\bf Target Update periods: }
Value based agents often employ an online and a {\em target} Q-network, the latter which is updated less frequently by directly syncing (or Polyak-averaging) from the online network; the {\em target updated period} determines how frequently this occurs.


{\bf Update periods: }
The online network parameters are updated after every {\em update period} environment steps, with a value of $4$ used in standard ALE training.

{\bf Number of atoms: } 
In distributional reinforcement learning \citep{Bellemare2017ADP}, the output layer predicts the distribution of the returns for each action $a$ in a state $s$, instead of the mean $Q^{\pi}(s, a)$. A popular approach is to model the return as a categorical distribution parameterized by a certain number of 'atoms' over a pre-specified support. 


\begin{figure}[!t]
    \centering
  \includegraphics[width=\linewidth]{figures/this_score_all.pdf}%
    \caption{Tuning hyper-parameter Consistency (THC Score, see \cref{sec:thc_metric}) evaluated across agents (\textbf{left panel}), data regimes (\textbf{center panel}), and environments  (\textbf{right panel}). Different colors indicate different data regimes (left panel) and different agents (center and right panels); grey bars/titles indicate hyper-parameters which are not comparable across the considered transfer settings.
    \label{fig:this_score_all}%
    }%
\end{figure}



\section{Experimental results} 
\label{exp_results}
As mentioned in the introduction, there already exist two data regimes for evaluating agents on the ALE suite: the (low-data regime) $100$k \citep{kaiser2020modelbased} and the original $200$M benchmark \citep{mnih2015humanlevel}. The $100$k benchmark includes only $26$ games from the original suite, so we focus on these for our evaluation. For computational considerations, we follow \citet{graesser2022state} and use $40$M million environment frames as our large-data regime.
We use the settings of DrQ($\epsilon$) (introduced by \citet{agarwal2021deep} as an improvement over the DrQ of \citet{yarats2021image}), and 
Data Efficient Rainbow (DER) introduced by \citet{hasselt19when}. All experiments were run on a Tesla P100 GPU and took around $2$-$4$ hours ($100$k) and $1$-$2$ days ($40$M) per run.
Both algorithms are implemented in the Dopamine library \citep{castro18dopamine}. Since the $100$k setting is cheaper, we evaluated a larger set of hyper-parameter values there and manually picked the most informative subset for running in the $40$M setting. For all our experiments we ran 5 independent seeds and followed the guidelines suggested by \citet{agarwal2021deep} for more statistically meaningful comparisons. Specifically, we computed aggregate human-normalized scores and report interquantile mean (IQM) with $95\%$ stratified bootstrap CIs. 

In \autoref{fig:this_score_all} we present the computed THC score for all the hyper-parameters discussed in \cref{sec:hyper-parameter_selection}, and we discuss their consistency across agents in Section~\ref{sec:acrossAlgorithms}, across data regimes in Section~\ref{sec:acrossData}, and  across environments in Section~\ref{sec:acrossEnvironments}. More detailed discussions are provided in \autoref{sec:finerGrainedExperiments} and a set of interesting findings in \autoref{sec:imf}. It is worth recalling that higher THC scores indicate less consistency, which suggests a likely need to re-tune the respective hyper-parameters when changing training configurations.


\subsection{Optimal hyper-parameters mostly Transfer Across Agents}
\label{sec:acrossAlgorithms}
We find that optimal hyper-parameters for DrQ($\epsilon$) agree quite often with DER, which is somewhat expected given that they're based on the same classical RL algorithm of Q-learning, and have the same number of updates in the same environments. Looking at THC values between the two agents for different data regimes we see that all values are below $0.5$, and in the $100$k regime tend to be even lower. Nevertheless, comparing the results of the two rows in \cref{fig:drq_eps_batch_sizes,fig:per_game} demonstrate that there can still be strong differences between the two. In the $40$M regime, the hyper-parameters with the highest THC are batch size and update horizon, consistent with the findings of \cite{obandoceron2023small}, where these two hyper-parameters proved crucial to boosting agent performance.


\begin{figure}[!t]
    \centering
  \includegraphics[width=0.8\linewidth]{figures/DER_adam_eps.pdf}%
    \caption{
     \textbf{Measured IQM of human-normalized scores on the $26$ $100$k benchmark games, with varying Adam's $\epsilon$} for DER. We evaluate performance at 100k agent steps (or 400k environment frames), and at $40$ million environment frames. The ordering of the best hyper-parameters switches between the two data regimes.
    }
    \label{fig:der_adam_eps}
\end{figure}

\subsection{Optimal hyper-parameters mostly do not Transfer Across Data Regimes}
\label{sec:acrossData}
We find that optimal hyper-parameters for Atari 100k mostly do not transfer once you move to 40M updates, showing that even when keeping algorithms and environment constant one may still need to tune hyper-parameters should they change the amount of data their agent can train on. Of the hyper-parameters considered, {\em Adam's $\epsilon$} and {\em update period} seem to be the most critical to re-tune (see \autoref{fig:der_adam_eps} for results on DER for Adam's $\epsilon$). The results with Adam's $\epsilon$ are surprising, as the purpose of this hyper-parameter is mostly for numerical stability. The update horizon results are consistent with what is done in practice between these two data regimes (e.g. Rainbow uses an update horizon of $3$, while DER uses $10$).

\begin{figure}[!h]
    \centering
  \includegraphics[width=0.8\linewidth]{figures/DrQ_eps_subs.pdf}
  \includegraphics[width=0.8\linewidth]{figures/DER_subs.pdf}
    \caption{\textbf{Measured returns with varying batch size} for DrQ($\epsilon$) (top) and DER (bottom) at $40$M environment frames for four representative games, demonstrating that the ranking of the hyper-parameter values can drastically change from one game to the next. All results averaged over $5$ seeds, shaded areas represent $95\%$ confidence intervals.
    }%
    \label{fig:drq_eps_batch_sizes}%
\end{figure}


\subsection{Optimal hyper-parameters do not Transfer Across Environments}
\label{sec:acrossEnvironments}
Our experiments show that hyper-parameters that perform well on some games lead to lackluster final performance in others. Indeed, in \autoref{fig:this_score_all} we can see that the THC score is highest when evaluating across environments. This strongly suggests that, when using an existing agent in a new environment, most of the hyper-parameters would need extra tuning.
\autoref{fig:drq_eps_batch_sizes} displays the results when varying batch size, where we can see that the rankings can sometimes be complete opposites across games (compare Kangaroo and Gopher).
 


\section{A web-based appendix} 
\label{web_results}
We have run an extensive number of experiments (around 108k) for this work, which would render a traditional appendix unwieldy. Instead, we provide an interactive website\footnote{Website available at \href{https://consistent-hyperparameters.streamlit.app/}{\emph{https://consistent-hparams.streamlit.app/}}.} which facilitates navigating the full set of results. Presenting empirical research results in this manner offers a range of benefits that enhance accessibility, engagement, and comprehension. 
This dynamic presentation allows readers to more easily make comparisons over different games, agents, and parameters. 


The website's main page presents aggregate IQM results for all hyper-parameters investigated in both data regimes (e.g. \autoref{fig:der_adam_eps}), while sub-pages present detailed performance comparisons when sliced by game (\autoref{fig:drq_eps_batch_sizes} presents a subset of this) and hyper-parameter (\autoref{fig:per_game} presents a subset of this).
The added level of granularity provided by the sub-pages can be crucial for understanding the specific strengths and weaknesses of an algorithm in various scenarios. All results averaged over 5 seeds, shaded areas represent 95\% confidence intervals.

\begin{figure}[!t]
    \centering
   \includegraphics[width=\textwidth]{figures/DrQ_eps_game_subs.pdf}
   \includegraphics[width=\textwidth]{figures/DER_game_subs.pdf}
  
    \caption{\textbf{Measured returns with various hyper-parameter variations on Asterix} for DrQ($\epsilon$) (top) and DER (bottom) at 40M environment frames. Displaying eight representative hyper-parameters, enabling per-game analyses for hyper-parameter selection.}%
    \label{fig:per_game}%
    \vspace{-1em}
\end{figure}
%\section{Environment properties}
\label{sec:environment_properties}

The Arcade Learning Environment (ALE) is a platform designed for evaluating and comparing the performance of reinforcement learning algorithms on classic arcade games. ALE has emerged as the benchmark for evaluating the capabilities of reinforcement learning (RL) algorithms in tackling intricate discrete control tasks. Since its release in 2013 \citep{bellemare2012ale}, the benchmark has gained thousands of citations and almost all state-of-the-art RL algorithms have featured it in their work. However, results generated from the full benchmark have typically been limited to a few large research groups.

The cost of producing evaluations on the full dataset is not feasible for many researchers and not necessary if you want to evaluate some specific algorithm capabilities, like long context games. Therefore, here we provide some key properties of the environments based on our previous findings which will allow the RL community select a small but representative subsets of environments when evaluating specific algorithm capabilities.
\section{Related work}
\label{related_work}

While RL as a field has seen many innovations in the last
years, small changes to the algorithm or its implementation can have a big impact on its results \citep{engstrom2020implementation, joajo2021lifting}.  Deep reinforcement learning approaches are often notoriously sensitive to their hyperparamaters and demonstrate brittle convergence properties \citep{haarnoja2018soft}. This is particularly true for off-policy approaches that use a replay buffer to leverage past experiences \citep{duan2016benchmarking}.


\cite{Henderson2017DeepRL} investigate the effects of existing degrees of variability between various RL setups and their effects on algorithm performance. Although restricted to the domain of existing environments, \cite{Henderson2017DeepRL} propose more robust performance estimators for RL learning algorithms. \cite{islam2017reproducibility} and \cite{shengyi2022the37implementation} have shown the difficulty in reproducing policy gradient algorithms due to the variance.
\cite{andrychowicz2020matters} did a deep dive in algorithmic choices on policy-based algorithms. Their analyses covered differences in hyper-parameters, algorithms, and implementation details.


In an effort to consolidate innovations in deep RL, several papers have examined the effect of smaller design decisions like the loss function or policy regularization for on-policy algorithms \cite{andrychowicz2020matters}, DQN agents \citep{obando2020revisiting}, imitation learning \citep{hussenot2021hyperparameter} and offline RL \citep{paine2020hyperparameter, lu2021revisiting}. AutoRL methods, on the other hand, have focused on automating and abstracting some of these decisions \citep{parker2022automated, eimer2023hyperparameters} by using data-driven approaches to learn various algorithmic components or even entire RL algorithms \citep{co2021evolving,lu2022discovered}. All these works have demonstrated that hyperparameters in deep reinforcement learning warrant more attention from the research community than they currently receive. Underreported tuning practices can distort algorithm evaluations, and overlooked hyperparameters may lead to suboptimal performance.


% Despite all the efforts, there are still some mysteries that have been not understood yet. Many of these new and unexpected discoveries have remained concealed due to limited parameter exploration caused by the substantial computational resources they demand. Investigate the impact of design choices and hyper-parameter in deep RL algorithms with large state environments is very challenging and almost impossible to explore on academic labs. Therefore, we decide to focus on exploring the relationship between hyparparemeters and  value-based methods, as opposed to the greater focus in Actor-Critic and Policy-based of the previous works \citep{andrychowicz2020matters}.

\section{Discussion}
\label{sec:discussion}
One of the central challenges in reinforcement learning research is the non-stationarity during training in the inputs (due to self-collected data) and targets (due to bootstrapping). This is in direct contrast with supervised learning settings, where datasets and labels are typically fixed throughout training. This non-stationarity may be largely to blame for some of the ranking inconsistencies observed under different training regimes (e.g. \autoref{fig:der_adam_eps}), and why different hyper-parameter tunings are required for different settings (e.g. DER versus Rainbow).

Hyper-parameters are commonly tuned on a subset of environments (e.g. 3-5 games) and then evaluated on the full suite. Our findings suggest that this approach may not be the most rigorous, as hyper-parameter selection can vary dramatically from one game to the next (c.f. \cref{fig:drq_eps_batch_sizes,fig:per_game}).
While aggregate results (e.g. IQM) provide a succinct summary of performance, they unfortunately gloss over substantial differences in the individual environments. If our hope as researchers is to be able to use these algorithms beyond academic benchmarks, understanding these differences is {\em essential}, in particular in real-world applications such as healthcare and autonomous driving.

We have conducted a large number of experiments to investigate the impact of various hyper-parameter choices. While the THC score (\autoref{fig:this_score_all}) provides a high-level view of the transferability of hyper-parameter choices, our collective results suggest that a {\em single} set of hyper-parameter choices will never suffice to achieve strong performance across all environments. The ability to dynamically adjust hyper-parameter values during training is one way to address this; to properly do so would require quantifiable measures of environment characteristics that go beyond coarse specifications (such as sparse versus dense reward systems). The per-game results we present here may serve as an initial step in this direction. In Appendix~\ref{sec:gopher} we provide a fine-grained analysis of DER on Gopher as an example of the type of analyses enabled by our website. We hope our analyses, results, and website prove useful to RL researchers in developing robust and  transferable algorithms to handle increasingly complex problems.\\

\subsubsection*{Acknowledgements}

The authors would like to thank Jesse Farebrother, Gopeshh Subbaraj, Doina Precup, Hugo Larochelle, and the rest of the Google DeepMind Montreal team for valuable discussions during the preparation of this work.  Jesse Farebrother deserves a special mention for providing us valuable feed-back on an early draft of the paper. We thank the anonymous reviewers for their valuable help in improving our manuscript. We would also like to thank the Python community \cite{van1995python, 4160250} for developing tools that enabled this work, including NumPy \cite{harris2020array}, Matplotlib \cite{hunter2007matplotlib}, Jupyter \cite{2016ppap}, Pandas \cite{McKinney2013Python} and JAX \cite{bradbury2018jax}.

\subsubsection*{Broader Impact Statement}

Although the work presented here is mostly academic, it aids in the development of more capable and reliable autonomous agents. While our contributions do not directly contribute to any negative societal impacts, we urge the community to consider these when building on our research.

\bibliography{main}
\bibliographystyle{rlc}

\newpage
\appendix

\section{Prompts for \ourmethod}
\label{app:prompts}
In this section, we list all prompts that we use for each step of our method.

\subsection{Fact Updating}
\label{app:update_prompts}
In practice, we implement these operations by performing \textit{two passes} over the retrieved facts.\bzl{TODO potentially abstract away this detail} In the first pass, we prompt the LM with the input $\doc$ and each fact $\fact\in R$ and prompt it to \textit{classify} the fact into one of \textit{reinforce, no change, make false}.
From this first pass, we divide the retrieved facts into two sets: $R_\text{true}$, comprising facts that remain true (\textit{reinforce, no change}), and $R_\text{false}$, comprised of facts that have become false (\textit{make false}).
In the second pass, we iterate through $R_\text{false}$, and prompt the LM to rewrite the fact into a true fact (if possible), conditioned on the new document $\doc$ and $R_\text{true}$. This serves a few purposes:
\begin{enumerate}
\item If $\fact$ is only made partially false by $\doc$, we may retain information expressed in $\fact$ but not $\doc$. For example, if $f$ is \textit{Mary and Bob work at UPS}, and $\doc$ is \textit{Mary got fired from UPS}, we may rewrite $\fact$ as \textit{Bob works at UPS}, rather than negating the entire fact.
\item Conditioning on $R_\text{true}$ allows the LM to make \textit{multi-hop} edits. For example, if $\fact$ is \textit{Mary is coworkers with Bob}, and $\doc$ is \textit{Mary changed workplaces to Amazon}, if $R_\text{true}$ contains \textit{Quinn works at Amazon}, then we can rewrite $\fact$ as \textit{Mary is coworkers with Quinn}.
\end{enumerate}


First round: classifying facts as becoming more or less likely to be true. 
\begin{lstlisting}
[Input] [Timestamp: {ts}] {context} [End Input]

The fact "{fact}" was previously true. In light of the input, is "{fact}" likely still true as of {ts}? Begin by summarizing the changes we learned from the input, then reasoning briefly about them to give your final answer with "Answer: Reinforce" (if the input makes the fact more likely) or "Answer: Make False" (if the input makes the fact less likely) or "Answer: No Change" (if the input doesn't affect the fact, e.g. if the input is irrelevant to the fact). Assume that the fact is still true (keep true) if nothing in the input contradicts it.
\end{lstlisting}

Second round: extracting rewrites
\begin{lstlisting}
[Input] [Timestamp: {ts}] {context}
Other True Facts at {ts}: {", ".join(still_true_facts)}
[End Input]

The fact "{fact}" was previously true but no longer. Given the above input and true facts, can you rewrite it into one that is true as of {ts}? Output your answer in form "rewrite: rewritten fact" or "no rewrite possible".
\end{lstlisting}

\subsection{Fact Extraction}
\label{app:fact_extraction}
\begin{lstlisting}
Extract all facts from the input text, with each fact on a new line and without bullet points or numbered lists. Facts should be simple, independent, standalone, and decontextualized. Break up long facts into smaller facts. Resolve all references (e.g. pronouns, definite articles, etc.) by copying full reference object everywhere it is referenced. Only include facts referring to the current world state (what is true *now*), as opposed to facts true in the past. If there are no facts, please output "No new facts." Do not include any other text.
\end{lstlisting}

\subsection{Inference}
\label{app:infer_prompt}
Given a question \texttt{question} at timestep \texttt{ts} (and choices \texttt{answer\_choices}), 
We first retrieve facts $\fact_i, [(\tau_{i0}, v_{i0}), (\tau_{i1}, v_{i1}), \cdots]$ from the knowledge base with similarity threshold $> 0.7$ to \texttt{question}. We then prompt a LM with the following: 
\begin{lstlisting}
Read the statements/passages below then answer the question below

***BEGIN STATEMENTS***
{f_i} ({v_{i0}} at {tau_{i0}}, {v_{i1}} at {tau_{i1}}, ...)
{f_j} ({v_{j0}} at {tau_{j0}}, {v_{j1}} at {tau_{j1}}, ...)
...
***END STATEMENTS***

Given the above statements are true and any prior knowledge you have, answer the following question at timestep {ts}?:
{question}

Briefly reason then answer with one of: {answer_choices}.
\end{lstlisting}

For questions requiring list answers (e.g. list all the siblings of Rachel), we replace the last line with:
\begin{lstlisting}
Briefly reason then answer with a JSON list, ["item1", "item2", ...], of zero or more of the following items: {answer_choices}. If you include any of the above items, make sure to copy their names exactly as is from the list. Your list may be empty, [], if none of the answers are true.
\end{lstlisting}




\section{Dataset Construction Details}
\subsection{News Articles}
\label{app:wikidata}

We construct this dataset in three stages:
\paragraph{Extracting World States $W$.} We retrieve \texttt{(subj,rel)} pairs from Wikidata for which there are at least two distinct fact relations at different timestamps, e.g. \verb|(subj,rel,obj1,start_ts1,end_ts1)| and \verb|(subj,rel,obj2,start_ts2,end_ts2)|. These timestamped facts are used to ``represent'' $W$. We filter for subjects \texttt{subj} located in English-speaking countries to ensure we can find English-language sources.
We use SPARQL\footnote{\url{https://www.w3.org/TR/sparql11-query/}} to obtain a set of \verb|(subj,rel)| pairs. 
    
\paragraph{Obtaining Documents $L$.} We annotate each timestamped relation, \verb|(subj,rel,obj,start_ts, end_ts)| with a source written between \verb|start_ts| and \verb|end_ts| (preferably close to the \verb|start_ts|) stating that the \verb|(subj,rel,obj)| relation is true. We crowdsource annotations from Prolific in two stages. In the first stage, Prolific annotators were presented with an interface which scraped candidate news articles off of Google\footnote{In particular, we set the to-be-matched parameter of the search to ``news'', i.e. \url{https://www.google.com/?tbm=nws}}, and were asked to select sources which stated that the fact \verb|(subj,rel,obj,start_ts, end_ts)| is true, but \textbf{did not} state that any succeeding fact, \verb|(subj,rel,obj2,start_ts2, end_ts2)| where \verb|start_ts2| $>$ \verb|start_ts|, is true.
In the second stage, we validated Prolific annotations from the first stage by presenting articles from the first round of annotations to annotators in the second round, and asking users whether those articles contained the fact in question. If second annotator does not affirm the fact is present in the article, we throw out the fact and the associated annotation.
\bzl{TODO: inter-annotator agreement}
We do an additional third round of filtration with a language model, asking the language model to affirm that the text of an article contains \verb|(subj,rel,obj,start_ts, end_ts)| but not any succeeding facts \verb|(subj,rel,obj2,start_ts2, end_ts2)|.
We only include articles and facts that pass all three rounds of annotation.
We recruited English-speaking participants from the US for annotations for all annotations. The full set of instructions we give annotators can be found in~\Cref{tab:annotator_instructions_1,tab:annotator_instructions_2}.
Screenshots of the interface can be found in~\Cref{fig:annotator_screenshot_1,fig:annotator_screenshot_2}.

\paragraph{Generating Question-Answers Pairs $(q,\{a\})$.} We automate generation of questions and answers from $W$ by writing templates for each relation and generating questions and answers from those templates. The full list of templates can be found in~\Cref{tab:wiki_qs_templates}.

\begin{table*}[]
    \centering
    \small
    \begin{tabular}{p{15cm}}
     \textbf{Please read these instructions carefully and only proceed once you have understood them. Once you start the task, you will have 10 minutes to get through as many questions as possible.}

    For each question, you will be presented a fact. Please find a news article that implies that the fact is true, according to the below requirements:
    \begin{enumerate}
    \item The article implies the fact, such that a reasonable person, without any prior knowledge, can infer that the fact is true from reading the article.

    Example: For fact Emad Mostaque is CEO of Stability AI (was True from 2020 to 2024-03-23)
        
    Good Sources: This startup is setting a DALL-E 2-like AI free, consequences be damned: Article says "...Stability AI CEO and founder Emad Mostaque wrote in a blog post"
    
    Bad Sources:  Artists can now opt out of the next version of Stable Diffusion: Cannot conclude fact from text of article

    \item The article is a news article or blog post.

    Example: For fact Taylor Aylmer is a member of the Racing Louisville FC sports tea

    Good Sources: Team News: Aylmer to make first regular season start

    Bad Sources: Taylor Aylmer - Racing Louisville FC Midfielder - ESPN, Taylor Aylmer - Instagram
         	
        
    \item The fact is stated in the main body of the article text, not in a table, list, image, image caption, embedded tweet, etc.

    Example: For fact Taylor Aylmer is a member of the Racing Louisville FC sports team

    Good Sources: Team News: Aylmer to make first regular season start, Recap: Racing rallies to beat Orlando, keep playoff hopes alive: Fact is in a list at the end, not the main text

    Bad Sources: Jaelin Howell, Racing Louisville bring community together to help people with Down syndrome: Fact is in an image caption but nowhere in the main text

    \item The article is a web page, not a PDF or other file format.

    Example: For fact Ali Shojaie is a IMS Fellow

    Good Sources: Ali Shojaie elected fellow of the Institute of Mathematical Statistics 

    Bad Sources: IMS Carver Award 2023: Source is a PDF file, not a web page
    
    \item The article is written in English.

    Example: For fact Emad Mostaque is CEO of Stability AI (was True from 2020 to 2024-03-23)

    Good Sources: This startup is setting a DALL-E 2-like AI free, consequences be damned

    Bad Sources: [Bengali article]: Article is not in English
    
    \item Avoid articles that state that the fact is or is about to become false. These are generally written near or past the end date of a fact being true.

    Example: For fact Emad Mostaque is CEO of Stability AI (was True from 2020 to 2024-03-23)

    Good Sources: This startup is setting a DALL-E 2-like AI free, consequences be damned

    Bad Sources: Stability AI founder Emad Mostaque plans to resign as CEO, sources say: Article is about the fact being about to be false
    \end{enumerate}

If no listed articles satisfy these requirements, you have the option to either find a news article that satisfies the requirements (a google search link is provided for reference, you may need to manually adjust the query or date parameters) or selecting "cannot find source" if you cannot find any source in a reasonable amount of time.

There may also be a second fact that you need to avoid. If you see this fact in the article, do not select it as a source.

\textbf{Tip}: You may use "ctrl-f" (find tool) to quickly validate whether or not a fact is in the article.
    \end{tabular}
    \caption{Instructions for round 1 of annotation for news article.}
    \label{tab:annotator_instructions_1}
\end{table*}
\bzl{non-English not showing up!}

\begin{table*}[]
\small
    \centering
    \begin{tabular}{p{15cm}}
\textbf{Please read these instructions carefully and only proceed once you have understood them. Once you start the task, you will have 12 minutes to get through as many questions as possible.}

For each question, you will be presented a fact and a news article. Please confirm that the news article implies that the fact is true, and conforms to the below requirements:
\begin{enumerate}
    \item The article implies the fact, such that a reasonable person, without any prior knowledge, can infer that the fact is true from reading the article.

    Example: For fact Emad Mostaque is CEO of Stability AI (was True from 2020 to 2024-03-23)
        
    Good Sources: This startup is setting a DALL-E 2-like AI free, consequences be damned: Article says "...Stability AI CEO and founder Emad Mostaque wrote in a blog post"
    
    Bad Sources:  Artists can now opt out of the next version of Stable Diffusion: Cannot conclude fact from text of article

    \item The article is written in English.

    Example: For fact Emad Mostaque is CEO of Stability AI (was True from 2020 to 2024-03-23)

    Good Sources: This startup is setting a DALL-E 2-like AI free, consequences be damned

    Bad Sources: [Bengali article]: Article is not in English

    \item Avoid articles that state that the fact is or is about to become false. These are generally written near or past the end date of a fact being true.

    Example: For fact Emad Mostaque is CEO of Stability AI (was True from 2020 to 2024-03-23)

    Good Sources: This startup is setting a DALL-E 2-like AI free, consequences be damned

    Bad Sources: Stability AI founder Emad Mostaque plans to resign as CEO, sources say: Article is about the fact being about to be false
\end{enumerate}

If the provided article does not satisfy these requirements, you have the option to either find a news article that satisfies the requirements (a google search link is provided for reference, you may need to manually adjust the query or date parameters) or selecting "cannot find source" if you cannot find any source in a reasonable amount of time.

There may also be a second fact that you need to avoid. If you see this fact in the article, do not select it as a source.

\textbf{Tip}: You may use "ctrl-f" (find tool) to quickly validate whether or not a fact is in the article.

    \end{tabular}
    \caption{Instructions for round 2 of annotation for news article.}
    \label{tab:annotator_instructions_2}
\end{table*}

\begin{figure*}
    \centering
    \includegraphics[width=\linewidth]{figures/round1_interface_screenshot.png}
    \caption{Screenshot of round 1 of annotation for news article.}
    \label{fig:annotator_screenshot_1}
\end{figure*}

\begin{figure*}
    \centering
    \includegraphics[width=\linewidth]{figures/round2_interface_screenshot.png}
    \caption{Screenshot of round 2 of annotation for news article.}
    \label{fig:annotator_screenshot_2}
\end{figure*}

\begin{table*}[]
\small
    \centering
    \begin{tabular}{p{6cm}p{12cm}}
\multirow{2}{6cm}{
    \texttt{(\{subj\}, employer, \{obj\})}
} & \verb|Who is the employer of {subject}?| \\
& \verb|Is {subject} an employee of {object}?| \\
\multirow{3}{6cm}{
    \texttt{(\{subj\}, chief executive officer, \{obj\})}
} & \verb|Who is the CEO of {subject}?| \\
& \verb|What company is {object} the CEO of?| \\
& \verb|Is {object} the CEO of {subject}?| \\
\multirow{3}{6cm}{
    \texttt{(\{subj\}, chairperson, \{obj\})}
} & \verb|Who is the chairperson of {subject}?| \\
& \verb|What organization is {object} the chairperson of?| \\
& \verb|Is {object} the chairperson of {subject}?| \\
\multirow{3}{6cm}{
    \texttt{(\{subj\}, head of state, \{obj\})}
} & \verb|Who is the head of state of {subject}?| \\
& \verb|Where is {object} the head of state of?| \\
& \verb|Is {object} the head of state of {subject}?| \\
\multirow{2}{6cm}{
    \texttt{(\{subj\}, position held, \{obj\})}
} & \verb|What government position does {subject} hold?| \\
& \verb|Does {subject} hold government position {object}?| \\
\multirow{2}{6cm}{
    \texttt{(\{subj\}, member of sports team, \{obj\})}
} & \verb|What sports team is {subject} a member of?| \\
& \verb|Is {subject} a member of {object}?| \\
\multirow{3}{6cm}{
    \texttt{(\{subj\}, unmarried partner, \{obj\})}
} & \verb|Who is the unmarried partner of {subject}?| \\
& \verb|Who is the unmarried partner of {object}?| \\
& \verb|Is {object} the unmarried partner of {subject}?| \\
\multirow{2}{6cm}{
    \texttt{(\{subj\}, residence, \{obj\})}
} & \verb|Where does {subject} reside?| \\
& \verb|Does {subject} reside in {object}?| \\
\multirow{2}{6cm}{
    \texttt{(\{subj\}, headquarters location, \{obj\})}
} & \verb|Where is the headquarters location of {subject}?| \\
& \verb|Is the headquarters location of {subject} in {object}?| \\
\multirow{2}{6cm}{
    \texttt{(\{subj\}, P463, \{obj\})}
} & \verb|What organization is {subject} a member of?| \\
& \verb|Is {subject} a member of {object}?| \\
\multirow{2}{6cm}{
    \texttt{(\{subj\}, member of political party, \{obj\})}
} & \verb|What political party is {subject} a member of?| \\
& \verb|Is {subject} a member of {object}?| \\
    \end{tabular}
    \caption{Question-answer templates in the News domain}
    \label{tab:wiki_qs_templates}
\end{table*}


\paragraph{Prolific Details}


We recruited a total of 680 English-speaking prolific annotators from the United States, with each annotator spending an average of 16:50 minutes on the task ($\sim$ 7 minutes to read and understand instructions). We paid annotators an average of $\$14.20$ per hour.
This task was deemed exempt from IRB review.
No personally-identifiable information was collected or stored, and all prolific annotators were associated with an anonymous prolific ID.


\subsection{Synthetic Conversations}
\label{app:convos}
We also construct this dataset in three stages:
\paragraph{Generating World States $W$.} We model the underlying world and its transformations as a Markov chain with states $S$ and a set of allowable transitions $T(S)$ determined by $S$. At each timestep, we randomly sample a transition from $T(S)$ uniformly at random.
States $S$ are described by a set of relations \verb|(subj, rel, obj)|. 
The full list of entities types and relations for each entity type can be found in~\Cref{tab:convo_states}. To construct each world, we subsample 10 people and 5 companies, and randomly initialize their kinship and employment relations.
Transitions $t\in T(S)$ change one or more relation in the state. To be able to test the limits of our propagation, the set of transitions we define in this domain all change more than one relation: for example, ``\textit{Bob changed jobs to work at Google}'' changes the \textit{employees} of Google, the set of \textit{coworkers} of Bob, the set of \textit{coworkers} of all Google employees, and the set of \textit{coworkers} of all employees of Bob's former company, etc. The full list of transitions and their downstream effects can be found in~\Cref{tab:convo_transitions}.

\paragraph{Generating Conversations $L$.} We generate conversations by sampling two people in the world $p_1$ and $p_2$ and prompting two LLMs with their corresponding personas and initial facts.
We then generate twelve conversation ``chunks'' as follows:
We begin by sampling the next transition we want to make in the world.
The transition corresponds to a natural language string that corresponds to only a single relation. However, we know that each transition is associated with multiple changing relations. To be able to infer the \textit{downstream} changes of a single relation changing, we need to know auxiliary facts related to the \textit{object} of the changed relation.
In the multi-hop subset of this dataset, we mention auxiliary facts in the \textit{prior} conversation chunks, while only mentioning the immediate transition (on a single relation) in the current chunk (\textit{without} mentioning any downstream changes). Thus, to make the correct downstream inferences on this subset, the system must retrieve and reason across facts from prior conversation chunks.

For the singlehop subet, we mention \textit{all downstream effects} in the same conversation chunk that a transition is made.

\begin{table*}[]
    \centering
    \begin{tabular}{p{2cm}p{13cm}}
    \toprule
        Entity Type & Possible Relations \\
        \midrule
        Person & spouse, parents, children, job, company, hobbies, coworkers, work location, boss, salary, industry, is-employed-full-time, work hours, workplace, siblings, parents-in-law, children-in-law, step-parents, step-children, equipment necessary for hobbies \\
        Company & employees, jobs, head, location, industry, workplace type \\
        Job & company, salary, is-full-time, work hours \\
        Hobby & equipment necessary for hobby \\
        \bottomrule
    \end{tabular}
    \caption{Full list of entities and relations defining each world state in the Conversation domain.}
    \label{tab:convo_states}
\end{table*}

\begin{table*}[]
    \centering
    \begin{tabular}{p{4cm}p{10cm}}
    \toprule
        Transition type & Downstream effects \\ \midrule
        \texttt{person.job} changes from \texttt{job1} to \texttt{job2} & 
        person.company, person.coworkers, person.work-location, person.boss, person.salary, person.industry, person.is-employed-full-time, person.work-hours, person.workplace, job1.company.employees, job2.company.employees \\ \midrule
        \texttt{person.spouse} changes from \texttt{person1} to \texttt{person2} & person.parents-in-law, person.parents.children-in-law, person.children.step-parents, person.step-children, person1.spouse, person1.parents-in-law, person1.parents.children-in-law, person2.spouse, person2.parents-in-law, person2.parents.children-in-law, person2.children.step-parents, person2.step-children \\ \midrule
        \texttt{person} adopts \texttt{child} & person.children, child.parents, child.siblings, child.spouse.parents-in-law, person.children-in-law, child.step-parents, person.spouse.step-children, person.children.siblings \\ \midrule
        \texttt{person} gets a new hobby \texttt{hobby} & person.equipment-necessary-for-hobbies \\ \midrule
        \texttt{job.salary} changes & for all people that have that job: person.salary \\ \midrule
        \texttt{job.work-hours} changes & for all people that have that job: person.work-hours \\
        \bottomrule
    \end{tabular}
    \caption{Full list of possible state transitions in the Conversation domain. Note the set of available transitions may vary depending on the underlying state.}
    \label{tab:convo_transitions}
\end{table*}



\paragraph{Generating Question-Answers Pairs $(q,\{a\})$.} Given a world state at time $t$, we query \textit{all} facts about the world. Similar to the news setting, we automate generation of questions and answers through templates. Templates in this setting can be found in~\Cref{tab:convo_qs_templates}.


\begin{table*}[]
\small
    \centering
    \begin{tabular}{p{8cm}p{7cm}}
\multirow{2}{8cm}{
    \texttt{(\{subj\}, spouse, \{obj\})}
} & \verb|Who is the spouse of {subj}?| \\
& \verb|Who is the spouse of {obj}?| \\
\multirow{1}{8cm}{
    \texttt{(\{subj\}, job, \{obj\})}
} & \verb|What is the job of {subj}?| \\
\multirow{1}{8cm}{
    \texttt{(\{subj\}, company, \{obj\})}
} & \verb|Which company does {subj} work at?| \\
\multirow{1}{8cm}{
    \texttt{(\{subj\}, hobbies, \{obj\})}
} & \verb|List all known hobbies of {subj}.| \\
\multirow{1}{8cm}{
    \texttt{(\{subj\}, coworkers, \{obj\})}
} & \verb|List all known coworkers of {subj}.| \\
\multirow{1}{8cm}{
    \texttt{(\{subj\}, work location, \{obj\})}
} & \verb|In which city does {subj} work?| \\
\multirow{1}{8cm}{
    \texttt{(\{subj\}, boss, \{obj\})}
} & \verb|Who is the head of {subj}'s workplace?| \\
\multirow{1}{8cm}{
    \texttt{(\{subj\}, salary, \{obj\})}
} & \verb|What is the salary of {subj}?| \\
\multirow{1}{8cm}{
    \texttt{(\{subj\}, industry, \{obj\})}
} & \verb|What industry does {subj} work in?| \\
\multirow{1}{8cm}{
    \texttt{(\{subj\}, is-employed-full-time, \{obj\})}
} & \verb|Does {subj} work full-time or part-time?| \\
\multirow{1}{8cm}{
    \texttt{(\{subj\}, work-hours, \{obj\})}
} & \verb|What are the work hours of {subj}?| \\
\multirow{1}{8cm}{
    \texttt{(\{subj\}, workplace, \{obj\})}
} & \verb|What type of workplace does {subj} work out of?| \\
\multirow{1}{8cm}{
    \texttt{(\{subj\}, parents, \{obj\})}
} & \verb|List all parents of {subj}.| \\
\multirow{1}{8cm}{
    \texttt{(\{subj\}, children, \{obj\})}
} & \verb|List all children of {subj}.| \\
\multirow{1}{8cm}{
    \texttt{(\{subj\}, siblings, \{obj\})}
} & \verb|List all siblings of {subj}.| \\
\multirow{1}{8cm}{
    \texttt{(\{subj\}, parents-in-law, \{obj\})}
} & \verb|List all parents-in-law of {subj}.| \\
\multirow{1}{8cm}{
    \texttt{(\{subj\}, children-in-law, \{obj\})}
} & \verb|List all children-in-law of {subj}.| \\
\multirow{1}{8cm}{
    \texttt{(\{subj\}, step-parents, \{obj\})}
} & \verb|List all step-parents of {subj}.| \\
\multirow{1}{8cm}{
    \texttt{(\{subj\}, step-children, \{obj\})}
} & \verb|List all step-children of {subj}.| \\
\multirow{1}{8cm}{
    \texttt{(\{subj\}, necessary equipment for hobby, \{obj\})}
} & \verb|List all equipment {subj} needs for their hobbies.| \\
    \end{tabular}
    \caption{Question-answer templates in the Conversation domain}
    \label{tab:convo_qs_templates}
\end{table*}


\begin{table}[!t]
    \centering
    \footnotesize
    \begin{tabular}{lcc}
    \toprule
       Relation type  & \# \texttt{(s, r)} & \# \texttt{(s, r, o)} \\
       \midrule
       Member of sports team & 284 & 382 \\
       Position held & 164 & 382 \\
       Employer & 38 & 77 \\
       Chairperson & 20 & 42 \\
       Head of state & 9 & 18 \\
       CEO & 6 & 13 \\
       Unmarried partner & 5 & 12 \\
       Residence & 4 & 8 \\
       Headquarters & 2 & 4 \\
       Member of political party & 1 & 2 \\
       \midrule
       Total & 533 & 1174  \\
       \bottomrule
    \end{tabular}
    \caption{Breakdown of changed relation types in the News domain, categorized by number of unique \texttt{(subj, rel)} pairs and \texttt{(subj, rel, obj)} triples.}
    \label{tab:news_relations}
\end{table}


\section{Dataset Statistics}
\label{app:dataset_stats}
The breakdown of changes in each of our datasets can be found in~\Cref{tab:news_relations} for news articles and~\Cref{fig:convos_stats} for conversations. The breakdown of questions for conversations can be found in~\Cref{tab:convos_questions}.


\begin{table}[!t]
    \centering
    \resizebox{\columnwidth}{!}{
    \footnotesize
    \begin{tabular}{lccc}
    \toprule
       Question Topic & \# Yes/No & \# Multiple Choice &  \# MC Choices \\
       \midrule
       Boss & 140 & 74 & 26\\
       Coworkers & 481 & - & -\\
       Industry & - & 74 & 26 \\
       Is employed full-time & 82 & - & -\\
       Salary & 158 & 80 & 11 \\
       Work hours & 110 & 64 & 10 \\
       Work Location & 274 & 72 & 20 \\
       Workplace & 140 & 74 & 26 \\
       \midrule
       Total & 1385 & 438  \\
       \bottomrule
    \end{tabular}
    }
    \caption{Distribution of generated questions in the Synthetic Conversation domain, categorized by question topic and type.}
    \label{tab:convos_questions}
\end{table}


\begin{figure*}
        \centering
        \includegraphics[width=\linewidth]{figures/transitions_overview.pdf}
        \caption{Distribution of changed relation types in the (A) News and (B) Conversation domains. (A) depicts the number of transitions per year of each month, while (B) depicts the number of relation types that underwent that number of changes across all conversations.}
        \label{fig:convos_stats}
\end{figure*}

\section{Qualitative Analysis}


\subsection{Error Analysis: Conversations (Multihop)}
\label{sec:multihop_errors}
Prototypical examples of multihop edit errors can be found below:

\paragraph{Retrieval Errors} Sometimes, the full list of facts that need to be updated are not retrieved. For example:

Input conversation chunk:
\begin{quote}
\it{
2023-11-01

Katie: Hey Olivia! How have you been? Guess what? I've changed my job to Library Assistant at Central Public Library! What's new with you?

Olivia: Hey Katie! That's amazing news! Congrats on the new job as a Library Assistant at Central Public Library!

...
}
\end{quote}

Retrieved facts:

\begin{itemize}
\it{
\item The role of General Practitioner at HealthFirst Medical Clinic is a full-time job.
\item The role of Library Assistant at Central Public Library is a full-time job.
\item Rachel has a full-time job as a Medical Assistant at HealthFirst Medical Clinic.
\item Rachel works at HealthFirst Medical Clinic as a Medical Assistant.
\item The work hours of a Library Assistant at Central Public Library are from 9 to 17.
\item Katie works full-time at the Urban Development Project.
\item Olivia works full-time at the airport.
\item The salary for a Library Assistant at Central Public Library is \$80,000.
\item Peter works from 9 to 17 at the Central Public Library.
\item Diana and Liam both have full-time jobs at the Urban Development Project.
\item The salary of an Archivist at Central Public Library is \$130,000.
}
\end{itemize}

The retriever at step 1 failed to retrieve the following facts to update. At inference time, these facts were retrieved and affected the LM's answer on questions:
\begin{itemize}
{\color{red}
\it{
\item Katie's working hours are from 10 to 15.
\item Katie's salary is \$100000.
\item Diana and Liam work with Katie.
}
}
\end{itemize}



\paragraph{Updating Errors}
Sometimes, facts that needed to be updated were retrieved, but were not updated by the LM. For example,

Input conversation chunk:
\begin{quote}
\it{
2023-11-26

Katie: Hey Olivia! How have you been? I heard you changed your job to Safety Officer at Urban Development Project! How's the new role treating you?

Olivia: Hey Katie! Yes, I did change my job to Safety Officer at Urban Development Project! The new role has been quite a change, but I'm really enjoying it so far. How have you been?

...
}
\end{quote}

Retrieved fact:

\begin{itemize}
\it{
\item Olivia works from 9 to 17 at her job.
}
\end{itemize}

Model-outputted update (and reasoning):
\begin{quote}
\begin{lstlisting}
Summary of changes:


    Olivia changed her job to Safety Officer at Urban Development Project.

    No information is provided about Olivia's work schedule.


Reasoning:

The input does not provide any information that would suggest Olivia's work schedule has changed. The conversation is about Olivia's new job, Katie's hobbies, and their plans for social activities. There is no mention of Olivia's work schedule or any changes that would affect it. Therefore, the fact "Olivia works from 9 to 17 at her job" is likely still true as of 2023-11-01.

Answer: No Change
\end{lstlisting}
\end{quote}

Note that this fact should've been flagged for potential updating and forwarded to the rewrite step (where the LM has an opportunity to retrieve more facts to confirm or deny this fact). 
However, the LM does not classify this fact as \textit{make false} because it is not directly contradicted by the input.



\section{Use of AI Assistants}
Code was written with Co-pilot turned on. GPT* models were also consulted for creating acronyms for the method and dataset names.

\end{document}
