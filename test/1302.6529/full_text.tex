\input  amstex
\input amsppt.sty
\magnification1200
\vsize=23.5truecm
\hsize=16.5truecm
%\vcorrection{-10truemm}
\NoBlackBoxes
\def\d{d\!@!@!@!@!@!{}^{@!@!\text{\rm--}}\!}
\def\grad{\operatorname {grad}}
\def\divg{\operatorname {div}}
\def\supp{\operatorname{supp}}
\def\Cnul{C_0^{\infty}}
\def\C{C^{\infty}}
\def\Re{\operatorname{Re }}
\def\Im{\operatorname{Im }}
\def\pj{\operatorname {pr}_J}
\def\pjo{\operatorname{pr}_{J_0}}
\def\pjk{\operatorname{pr}_{J_k}}
\def\prk{\operatorname{pr}_k}
\def\pr{\operatorname{pr}}
\def\Hloc#1{H^{\operatorname{loc}}_{(#1)}} 
\def\vn{{\vec n}}
\def\pr{\operatorname{pr}}
\def\leg{\;\dot{\le}\;}
\def\geg{\;\dot{\ge}\;}
\def\eg{\;\dot {=}\;}
\def\crp{\overline{\Bbb R}_+}
\def\crm{\overline{\Bbb R}_-}
\def\crpm{\overline{\Bbb R}_\pm}
\def\rn{{\Bbb R}^n}
\def\rnp{{\Bbb R}^n_+}
\def\rnm{\Bbb R^n_-}
\def\rnpm{\Bbb R^n_\pm}
\def\crnp{\overline{\Bbb R}^n_+}
\def\crnm{\overline{\Bbb R}^n_-}
\def\crnpm{\overline{\Bbb R}^n_\pm}
\def\comega{\overline\Omega }
\def\srplus{\Cal S(\overline{\Bbb R}_+)}
\def\srnp{\Cal S(\overline{\Bbb R}^n_+)}
\def\srnpm{\Cal S(\overline{\Bbb R}^n_\pm)}
\def\ch{\Cal H}
\def\chp{\Cal H^+}
\def\chm{\Cal H^-}
\def\chpm{\Cal H^\pm}
\def\Rn{\Bbb R^n}
\def\ang#1{\langle {#1} \rangle}
\def\lag{\Bbb R^{n-1}\times\Bbb R}
\def\H{\overline H}
\def\Op{\operatorname{Op}}
\def\rnprod{\Bbb R^{n-1}\!\times\!\overline{\Bbb R}^n_+}
\def\inv{^{-1}}
\def\simto{\overset\sim\to\rightarrow}
\def\ttilde{\overset{\,\approx}\to}
\def\set#1{\{{#1}\}}
\def\Cnoll{\Cnul}
\def\scalar#1#2{\langle#1,#2\rangle}
\def\domain#1{\{\,#1\,\}}
\def \im{\text{\rm i}}
\def\Zfrac{\tsize\frac1{\raise 1pt\hbox{$\scriptstyle z$}}}
\def\zfrac{\frac1{\raise 1pt\hbox{$\scriptscriptstyle z$}}}
%\def\d{d\!\!\!/ }
\def\crpp{\overline {\Bbb R}^2_{++}}
\def\rpp{ {\Bbb R}^2_{++}}
\def\rp{ \Bbb R_+}
\def\rmi{ \Bbb R_-}
\define\tr{\operatorname{tr}}
\define\op{\operatorname{OP}}
\define\Tr{\operatorname{Tr}}
\define\srplusp{\Cal S(\crpp)}
\define\srp{\Cal S_+}
\define\srpp{\Cal S_{++}}
\define\stimes{\!\times\!}




\document

\medskip

\topmatter
\title
Heat kernel estimates for pseudodifferential operators, fractional
Laplacians and Dirichlet-to-Neumann operators
\endtitle
\author Heiko Gimperlein and Gerd Grubb \endauthor
\affil
{Department of Mathematical Sciences, Copenhagen University,
Universitetsparken 5, DK-2100 Copenhagen, Denmark.}
%E-mail {\tt grubb\@math.ku.dk}
\endaffil
\rightheadtext{Heat kernel estimates}
\abstract
The purpose of this article is to establish
 upper and lower estimates for the integral kernel of the
semigroup $\exp(-tP)$ associated to a classical, strongly elliptic
pseudodifferential operator $P$ of positive order on a closed manifold. The Poissonian bounds
generalize those obtained for perturbations of fractional powers of the
Laplacian. In the selfadjoint case, extensions to $t\in{\Bbb C}_+$ are
studied. In particular, our results apply to the Dirichlet-to-Neumann
semigroup.
\endabstract
\subjclass
35K08, 58J35, 58J40, 47D06
\endsubjclass


\endtopmatter

\head Introduction \endhead


Let $M$ be a compact $n$-dimensional Riemannian manifold and $P$ a
classical, strongly elliptic pseudodifferential operator ($\psi $do) on
$M$ of order $d>0$. We consider upper and lower estimates for the integral
kernel $\Cal K_{V}(x,y,t)$ of the generalized heat semigroup
$V(t)=e^{-tP}$. Semigroups generated by such nonlocal operators have been
of recent interest in
different settings.

1) For a Riemannian manifold $\widetilde M$ with boundary $M$, the
Dirichlet-to-Neumann operator is a first-order pseudodifferential operator
on $M$ with principal symbol $|\xi|$.
Arendt and Mazzeo [AM07], [AM12], initiated the study of the associated
semigroup and its relation to eigenvalue inequalities, motivating later
studies e.g.\ by Gesztesy and
Mitrea [GM09] and Safarov [S08].

2) The heat kernel generated by fractional powers of the Laplacian
$\Delta ^{d/2}$ and their perturbations provides another example. Sharp
estimates for $e^{-t\Delta^{d/2}}$, $0<d<2$, can be obtained from those
for $e^{-t\Delta}$ by subordination formulas. For perturbations on bounded
domains in $\Bbb{R}^n$, recent work on estimates includes Chen, Kim and
Song [CKS12] and other works by these authors, and Bogdan et al.\ [BGR10].

In this article we generalize the Poissonian estimates obtained in the
second case to parameter-elliptic operators $P$ on closed manifolds,
by pseudodifferential methods. In
particular, we allow nonselfadjoint operators. A main result is:
% the following estimate:

\proclaim{Theorem} The kernel of the semigroup satisfies
$$
|\Cal K_V(x,y,t)|\le C e^{-c_1t}t\,(d(x,y)+t^{1/d})^{-n-d}, \text{ for }x,y\in
M, t\ge 0,\tag$*$
$$
for any $c_1$ smaller than the infimum $\gamma (P)$ of the
real part of the spectrum of $P$. 

If $P$ is selfadjoint $\ge 0$, the
estimates extend to complex $t=e^{i\theta }|t|$ for $|\theta
|<\frac\pi  2$, with uniform estimates
$$
|\Cal K_{V}(x,y,t)|\le C (\cos \theta  )^{-N}
e^{-\gamma(P)\operatorname{Re}t}\,
\frac{|t|}{(d(x,y)+|t|^{1/d})^{d}}((d(x,y)+|t|^{1/d})^{-n}+1),\tag $**$
$$
where $N=\operatorname{max} \{\tfrac n d , \tfrac{7n}{2} + 4 d + 7 \}$.
\endproclaim

Here $d(x,y)$  denotes the distance between $x$ and $y$. If $P$ is a system, it suffices that $P-\lambda $ is parameter-elliptic on
the rays in a sector containing $\{\operatorname{Re}\lambda \le 0\}$. Extending
$(*)$, also derivatives of the kernel, and, if further spectral information is available, a refined
description of the long-time behavior, are obtained in the paper.

For the Dirichlet-to-Neumann operator, as well as for the perturbations of
fractional powers of the Laplacian of orders $0<d<2$, we get not only
upper estimates but also similar lower estimates at small distances.

The estimate ($*$) exhibits a large class of operators which satisfy
upper estimates closely related to those studied abstractly in Duong
and Robinson [DR96].
As a simple application of ($**$) and H\"o{}lder's inequality, one can for
example obtain ultracontractive estimates
$$
\|e^{-tP}\|_{\Cal L(L_p,L_q)}
\le C (\cos \theta )^{-N}\big\|\frac{|t|}{(d(x,y)+|t|^{1/d})^{d}}((d(x,y)+|t|^{1/d})^{-n}+1)\big\|_{L_{q,y}L_{p',x}},$$
 uniformly for $t\in{\Bbb C}_+$. In the case of operators with
Gaussian heat kernel estimates, a rich spectral theory has been
developed  (see e.g.\ Arendt [A04], Ouhabaz [O05]).


With the help of comparison principles, our result implies Poissonian
estimates e.g.\ for boundary problems in an open subset $\Omega$ of $M$: If $P$ is the
variational operator associated to a Dirichlet form $a$ with domain $\Cal
D\subset L_2(M)$, we consider the abstract Dirichlet realization
$P_\Omega$ associated to the closure of $a|_{\Cal D \cap C_0(\Omega)}$. In
the case where $a$ is Markovian, one obtains $0\leq \Cal
K_{e^{-tP_\Omega}} \leq \Cal K_{e^{-tP}}$ on $\Omega$. See Grigor'yan and
Hu [GH08] for more refined comparison principles.



\medskip

\noindent {\it Outline.} Section 1 collects some known facts. In Section 2 we
treat semigroups generated by nonselfadjoint $P$ for $t\ge 0$. Section
3 extends the estimates to complex $t$ for selfadjoint $P$. Section 4
includes lower estimates for perturbations of fractional powers of the
Laplacian and for the Dirichlet-to-Neumann operator.


\head 1. Preliminaries \endhead

{\it Notation:} $\ang{\xi }=\sqrt{\xi ^2+1}$. The indication $\leg$ means ``$\le$ a
constant times'', $\geg$ means  ``$\ge$ a
constant times'', and $\eg$ means that both hold.

\medskip
Let $P$ be a classical $\psi
$do of order $d\in{\Bbb R}_+$, acting in a Hermitian $N$-dimensional
 $C^\infty $
vector bundle $E$ over a closed, compact
Riemannian $n$-dimensional manifold $M$.
We assume that $P-\lambda $ is
parameter-elliptic on all rays with argument in $\,]\frac\pi 2-\varphi
_0, \frac{3\pi }2+\varphi _0 [\,$ for some $\varphi _0\in
\,]0,\frac\pi 2[\,$. (In the notation of Seeley [S67], these rays are
``rays of minimal growth''.)
From $P$ one can define the generalized heat operator $V(t)=e^{-tP}$,
$t\ge 0$, a holomorphic semigroup generated by $P$, as explained in
detail e.g.\ in [G96], Sect.\ 4.2.
The kernel $\Cal
K_{V}(x,y,t)$ ($C^\infty $ for $t>0$)
was analyzed there in its dependence on $t$, but mainly with a view
to sup-norm estimates over all $x,y$, allowing an analysis of the
diagonal behavior, that of $\Cal K_{V}(x,x,t)$. We shall expand the
analysis here to give more information on $\Cal K_{V}(x,y,t)$.
\medskip

For convenience of the reader we recall the definitions of symbol spaces that are used.
For $d\in{\Bbb R}$, the symbol space $S^d_{1,0}({\Bbb R}^n\times{\Bbb
R}^n)$ consists of the $C^\infty $-functions $a(x,\xi )$ ($x,\xi
\in{\Bbb R}^n$) such that for all $\alpha ,\beta \in{\Bbb N}_0^n$,
$$
|D_x^\beta D_\xi ^\alpha a(x,\xi )|\leg \ang\xi ^{d-|\alpha |};\tag1.1
$$
it is a Fr\'echet space provided with the seminorms $\sup_{x,\xi
}|\ang\xi ^{-d+|\alpha |}D_x^\beta D_\xi ^\alpha a|$. The symbols
define operators $A=\operatorname{Op}(a(x,\xi ))$ of order $d$ by
$$
\operatorname{Op}(a(x,\xi ))u= \Cal F ^{-1}
a(x,\xi)\Cal Fu=\int_{\Bbb
R^n}e^{ix\cdot\xi }a(x,\xi )\hat u(\xi )\, \d\xi,
$$
where $\Cal F u=\hat u$ denotes the Fourier transform and $\d\xi =(2\pi )^{-n}d\xi $. The operator maps from
$\Cal S({\Bbb R}^n)$ to $\Cal S({\Bbb R}^n)$, extending to suitable
spaces of distributions and Sobolev spaces, and obeying various
composition rules.

The space of {\it classical symbols} of order $d$, $S^d({\Bbb R}^n\times{\Bbb
R}^n)$, is the subset of $S^d_{1,0}({\Bbb R}^n\times{\Bbb
R}^n)$ where $a(x,\xi )$ moreover has an asymptotic expansion
$a\sim \sum_{l\in{\Bbb N}_0}a_{d-l}$ in terms $a_{d-l}(x,\xi )$
homogeneous in $\xi $ of degree $d-l$ for $|\xi |\ge 1$, such that
$a'_M=a-\sum_{l<M}a_{d-l}\in S^{d-M}_{1,0}$ for all $M\in {\Bbb
N}_0$. The principal symbol $a_d$ is often denoted $a^0$.

It should be noted that we here use the globally estimated symbols of H\"ormander [H83],
Section 18.1, which have the
advantage that remainders are kept inside the calculus.

Operators on manifolds are defined by use of local coordinates and
rules for change of variables, composition with cut-off functions etc.; we refer to the quoted
works for details.


The book [G96] moreover includes parameter-dependent symbols $a(x,\xi
,\lambda )$ for $\lambda $ in a sector of ${\Bbb C}$, with special
symbol estimates involving the parameter (also operators on manifolds
with boundary are treated there).
\medskip

Consider a localized situation where the symbol $p(x,\xi)$ of $P$ is defined in a bounded open
subset of ${\Bbb R}^n$ --- we can assume it is extended to
${\Bbb R}^n$, with symbol estimates valid
uniformly in $x$. The abovementioned hypothesis of parameter-ellipticity means that
the spectrum of the principal symbol $p^0(x,\xi)$ (an $N\times N$-matrix) is contained in the
sector $\{\lambda \mid |\arg\lambda |\le \theta _0\}$, $\theta _0=\frac\pi 2-\varphi _0$, when $|\xi |\ge
1$. This holds in particular when $P$ is strongly elliptic, for then
$$
\operatorname{Re}(p^0(x,\xi)v,v)\ge c|\xi|^d|v|^2, \text{ for
}|\xi|\ge 1,\, v\in{\Bbb C}^N,
\text{ with }c>0, \tag1.2
$$
and hence since
$$
|\operatorname{Im}(p^0v,v)|\le |(p^0v,v)|\le C|\xi|^d|v|^2\le
c^{-1}C\operatorname{Re}(p^0v,v),\text{ for }|\xi|\ge 1,\, v\in{\Bbb C}^N,\tag1.3
$$
$P$ satisfies the condition of  parameter-ellipticity with $\varphi _0=\frac\pi 2-\theta _0$,
where  $\theta _0=\arctan (c^{-1}C)$ \linebreak$\in \,]0,\frac\pi 2[\,$. When $P$
is scalar, the two ellipticity properties are equivalent, but for systems, strong
ellipticity is more restrictive than the mentioned parameter-ellipticity (also
called parabolicity of $\partial_t+P$).

The spectrum $\sigma (P)$ of $P$ lies in a right half-plane and has a finite lower
bound
$\gamma (P)=\inf\{\operatorname{Re}\lambda \mid \lambda \in \sigma (P)\}$.
We can modify $p^0$ for small $\xi$ such that $\sigma (p^0(x,\xi))$ has a positive lower bound
throughout and lies in $
\{\lambda =re^{i\theta }\mid r>0,\, |\theta |\le\theta _0\}$.

The information in the following is taken from [G96], Section 3.3.

The resolvent $Q_\lambda =(P-\lambda )^{-1}$ exists and is holomorphic in
$\lambda $ on a neighborhood of a set
$$
W_{r_0,\varepsilon }=\{\lambda \in{\Bbb C}\mid |\lambda |\ge r_0,\,
\arg \lambda \in [\theta _0+\varepsilon , \pi -\theta _0-\varepsilon
],\, \operatorname{Re}\lambda \le \gamma (P)-\varepsilon \}.\tag1.4
$$
(with  $\varepsilon >0$).
There
exists a parametrix $Q'_\lambda $
on a neighborhood of a possibly larger set (with $\delta >0,\varepsilon >0$)
$$
V_{\delta ,\varepsilon }=\{\lambda \in{\Bbb C}\mid |\lambda |\ge
\delta \text{ or }
\arg \lambda \in [\theta _0+\varepsilon , \pi -\theta _0-\varepsilon
] \};
$$
such that this parametrix coincides with $(P-\lambda )^{-1}$ on
the intersection. Its symbol $q(x,\xi ,\lambda )$ in local coordinates
is holomorphic in $\lambda $ there and has the form
$$
%\aligned
q(x,\xi,\lambda )\sim \sum_{l\ge 0}q_{-d-l}(x,\xi,\lambda )
,\text{ where }
q_{-d}=({p^0(x,\xi)-\lambda })^{-1}.\tag1.5
$$
Here when $P$ is scalar,
$$
 q_{-d-1}={b_{1,1}(x,\xi
)}q_{-d}^{2},\; \dots,\;
q_{-d-l}=\sum_{k=1}^{2l}{b_{l,k}(x,\xi)}{q_{-d}^{k+1}},\; \dots\; ;
\tag1.6$$
with symbols $b_{l,k}$  independent of $\lambda $ and homogeneous of
degree   $dk-l$ in $\xi$ for $|\xi|\ge 1$. When $P$ is a system, each
 $q_{-d-l}$ is for $l\ge 1$ a finite sum of terms with the structure
$$ r(x,\xi ,\lambda )=b_1q_{-d}^{\nu _1}b_2q_{-d}^{\nu _2}\cdots b_Mq_{-d}^{\nu _M}b_{M+1},\tag1.7
$$
where the $b_k$ are homogeneous $\psi $do symbols of order $s_k$ independent of
$\lambda $, the $\nu _k$ are positive integers with sum $\ge 2$, and
$s_1+\dots +s_{M+1}-d(\nu _1+\dots+\nu _M)=-d-l$. (Further information
and references in Remark 3.3.7.)
Moreover, the remainder
$q'_M=q-\sum_{l<M}q_{-d-l}$ satisfies for $\lambda $ with $|\pi
-\arg\lambda |\le\frac\pi 2+\varphi $, any $|\varphi |<\varphi _0$,
$$
|D_{x}^\beta D_{\xi}^\alpha q'_M(x,\xi,\lambda )|\leg \ang{\xi
}^{d-|\alpha |-M}(1+|\xi|+|\lambda |^{1/d})^{-2d}, \text{ when }M+|\alpha |>d
.\tag1.8$$
(Cf.\ Theorems 3.3.2, and 3.3.5, applied to the rays with arguments in
$\,]\frac\pi 2-\varphi _0,\frac{3\pi }2+\varphi _0[\,$.)

\head 2. Semigroups generated by parameter-elliptic pseudodifferential
operators \endhead


 As explained in [G96], Section 4.2, the semigroup $V(t)=e^{-tP }$ can be defined from $P $ by the
 Cauchy integral formula
$$
V(t)=\tfrac i {2\pi }\int_{\Cal C}e^{-t\lambda }(P -\lambda )^{-1}\,d\lambda ,\tag2.1
$$
where $\Cal C$ is a suitable curve going in the positive direction
around the spectrum of
$P $; it can be taken as the boundary of $W_{r_0,\varepsilon }$ for a
small $\varepsilon $.
 In the local coordinate patch
 the symbol is (for any $M\in{\Bbb N}_0$)
$$
\aligned
v(x,\xi,t)&=
v_{-d}+\dots+v_{-d-M+1}+v'_M\sim \sum_{l\ge 0}v_{-d-l}(x,\xi
,t),\text{ where }\\
v_{-d-l}&=\tfrac i {2\pi }\int_{\Cal C }e^{-t\lambda }q_{-d-l}(x,\xi
,\lambda )\,d\lambda,\quad v'_{M}=\tfrac i {2\pi }\int_{\Cal C }e^{-t\lambda }q'_{M}\,d\lambda.
\endaligned
 \tag2.2
$$

A prominent example is $e^{-t\sqrt\Delta \,}$ where $\Delta $
denotes the (nonnegative) Laplace-Beltrami operator on $M$. This is a  Poisson operator
from $M$ to $M\times\crp$ as defined  in the Boutet de Monvel calculus
([B71], cf.\ also [G96]), when $t$ is identified with $x_{n+1}$. When $M$ is replaced by ${\Bbb R}^n$, its kernel is
the well-known Poisson kernel$$
{\Cal K}(x,y,t)= c_n\frac t{|(x-y,t)|^{n+1}}\tag2.3
$$
for the operator solving the Dirichlet problem
for $\Delta $ on ${\Bbb R}^{n+1}_+$.

Also more general operator families $V(t)=e^{-tP}$ with $P$ of order 1 are sometimes spoken of as Poisson operators (e.g.\ by
Taylor [T81]), and indeed we can show that for $P$ of any order
$d\in\rp$, $V(t)$ identifies with a Poisson operator in the Boutet de
Monvel calculus.
This will be accounted for in detail elsewhere. In order to match the conventions for Poisson
symbol-kernels, the indexation in (2.2) is chosen slightly differently from that in [G96],
Section 4.2, where $v_{-d-l}$ would be denoted $v_{-l}$.
%
We define $V_{-d-l}(t)$ and $V'_M(t)$ in local coordinates to be the
$\psi $do's with symbol $v_{-d-l}(x,\xi ,t)$ resp.\ $v'_{M}(x,\xi ,t)$.
The
kernel $\Cal K_{V}(x,y,t)$ is in local coordinates expanded according
to the symbol expansion:
$$
\Cal K_V(x,y,t)= \sum_{0\le l<M}\Cal K_{V_{-d-l}}(x,y,t)+\Cal K_{V'_M}(x,y,t).
\tag2.4
$$

The following result follows from [G96].


\proclaim{Theorem 2.1}
$1^\circ$ In local coordinates,
the kernel terms satisfy for some $c'>0$:
$$
|\Cal K_{V_{-d-l}}(x,y,t)|\leg e^{-c't}\cases t^{(l-n)/d}\text{ if
}d-l>- n, \\ t\,(
|\log t|+1)\text{ if
}d-l= -n,\\
t\text{ if
}d-l<- n.
\endcases\tag2.5
$$
For a given $c_0>0$ we can modify $p^0$ to satisfy
$\inf_{x,\xi }\gamma (p^0(x,\xi ))\ge c_0$; then $c'$ can be any
number in
$\,]0,c_0[\,$.

$2^\circ$ Moreover, with the modification in $1^\circ$ used with
$c_0=\gamma (P)$ if $\gamma (P)>0$, the remainder satisfies
$$
|\Cal K_{V'_{M}}(x,y,t)|\leg e^{-c_1t}\cases t^{(M-n)/d} \text{ if
}d-M>- n, \\ t\,(|
\log t|+1)\text{ if
}d-M= -n,
\\t\text{ if
}d-M<- n,
\endcases\tag2.6
$$
for any $c_1<\gamma (P)$.
In particular,
$$
|\Cal K_{V}(x,y,t)|\leg e^{-c_1t}t^{-n/d}.\tag2.7
$$
\endproclaim

\demo{Proof} The theorem was shown with slightly less precision on the
constants $c',c_1$ in [G96], Theorems 4.2.2 and 4.2.5. It was there aimed
towards applications where $d$ is integer. The estimates of resolvent
symbols in Section 3.3
are still valid when $d\in\rp$, but the replacement of $P$ by
$P+a$ ($a\in{\Bbb R}$) in the beginning of Section 4.2 on heat operators only gives a
classical $\psi $do when $d$ is integer, so we need another device to
take the value of $\gamma (P)$ into account for general $d\in\rp$. We shall now
explain the needed modifications, with reference to [G96].

For $1^\circ$, the proof in Theorem 4.2.2 shows the validity of (2.5)
with a small positive $c'<\inf_{x,\xi }\gamma (p^0(x,\xi ))$. For a given $c_0>0$, the proof goes
through to allow any $c'<c_0$, when $p^0(x,\xi ) $ is modified for $|\xi |\le R$ (for a
possibly large $R$) to satisfy $\inf \gamma (p^0(x,\xi ))\ge
c_0$.

For $2^\circ$, the remainder symbol $q'_M$ is holomorphic on
$W_{r_0,\varepsilon }$; here if $\gamma (P)>0$ we define the terms $q_{-d-l}$ as under
$1^\circ$, with $c_0=\gamma (P)$. For large $M $, $q'_M$ is $\leg \ang\lambda
^{-2}$. The proof of Th.\ 4.2.2 gives an estimate of $\Cal K_{V'_M}$ by
$e^{-c_{1}t}t\,(1+|\log t|)$, and the proof of Theorem
4.2.5 shows how to remove the logarithm. The estimates of $\Cal K_{V'_M}$ for lower values of $M$ follow by addition of the estimates of finitely many $\Cal K_{V_{-d-l}}$-terms.\qed
\enddemo

We shall improve this to give information on the dependence on $|x-y|$
also. This will rely on the following result on kernels of
$S^r_{1,0}$-$\psi $do's, found e.g.\ in Taylor [T81], Lemma XII 3.1, or
[T96], Proposition VII 2.2.


\proclaim{Proposition 2.2} Let $a \in C^\infty(\Bbb R^n \times \Bbb R^n)$
be such that for some $r \in \Bbb R$, $N \in {\Bbb N}_{0}$ with $N > n+r$,
and all $0 \leq |\alpha|
\leq N$,
$$
\operatorname{sup}_{x,\xi } \ang{\xi}^{-r+|\alpha|} |D_\xi^\alpha a(x, \xi)| <
\infty .\tag2.8
$$
Then the inverse Fourier transform $\Cal K_A(x,y)=\Cal F^{-1}_{\xi \to
z}a(x,\xi )|_{z=x-y}$  is
$O(|x-y|^{-N})$ for $|x-y|\to\infty $, and satisfies
for $|x-y|>0$:
$$
|\Cal K_A(x,y)|\leg \cases  |x-y|^{-r-n}\text{ if }r> -n,\\ |\log
|x-y||+1\text{ if }r=-n,\\
1 \text{ if } r<-n.\endcases \tag2.9
$$
In particular, if $a\in S^r_{1,0}({\Bbb R}^n\times{\Bbb R}^n)$ defining the $\psi $do $A$, the estimates hold for its kernel
$\Cal K_A(x,y)$ for all $N>n+r$, each estimate depending only on the
listed symbol seminorms.

\endproclaim

The dependence on $N$ follows
from an inspection of the proof.


´
In the scalar case the kernel study can be based on nice
explicit formulas, that we think are worth explaining.
Consider the contribution from one of the terms in (1.6). As
integration curve we can here use $C_{\theta }$ consisting of the two rays
$re^{i\theta }$ and $re^{-i\theta }$, $\theta =\theta _0+\varepsilon
$. For $t>0$, a replacement of $t\lambda $ by $\varrho $ gives:
$$\aligned
w_{l,k}(x,\xi,t)&=\tfrac i {2\pi }\int_{C_\theta  }e^{-t\lambda
}\frac{b_{l,k}(x,\xi)}{(p^0(x,\xi)-\lambda )^{k+1}}\,d\lambda=\tfrac i {2\pi
}\int_{C_\theta }e^{-\varrho  }\frac{t^{k}b_{l,k}}{(tp^0-\varrho)^{k+1} }\,d\varrho \\
&=\tfrac i {2\pi }t^{k}b_{l,k}\int_{C_{\theta ,R} }\frac{e^{-\varrho
}}{(tp^0-\varrho  )^{k+1}}\,d\varrho =\tfrac 1{k!}t^{k}b_{l,k}e^{-tp^0};
\endaligned\tag2.10$$
here we have replaced the integration curve by
a closed curve $C_{\theta ,R}$ connecting the two rays by a circular piece in the
right half-plane  with
radius $R\ge 2t|p^0(x,\xi)|$, and applied the Cauchy integral formula for
derivatives of holomorphic functions. This shows:
$$
v_{-d}=e^{-tp^0},\quad v_{-d-l}(x,\xi,t)=\sum_{k=1}^{2l}\tfrac1{k!}t^k b_{l,k}(x,\xi
)e^{-tp^0(x,\xi)} \text{ for }l\ge 1.\tag2.11
$$
Then the kernels of the
$V_{-d-l}(t)$ can be estimated by the following observations.

\proclaim{Proposition 2.3} Let $p^0(x,\xi )$ be the principal symbol of a
classical scalar strongly elliptic $\psi $do $P$ on ${\Bbb R}^n$ of order
$d\in\rp$, chosen such that $\operatorname{Re}p^0(x,\xi )\ge c_0>0$.

$1^\circ$ Let $c'\in[0,c_0[\,$. For any $j\in{\Bbb N}_0$, $(t(p^0(x,\xi )-c'))^je^{-t(p^0(x,\xi )-c')}$ is in
$S^0_{1,0}({\Bbb R}^n\times{\Bbb R}^n)$ uniformly in $t\ge 0$.

$2^\circ$
Let
$$
w(x,\xi ,t)=\tfrac i {2\pi }\int_{C_\theta }e^{-t\lambda }\frac{b(x,\xi )}{(p^0(x,\xi )-\lambda )^{k+1}}\,d\lambda, \tag2.12
$$
where $k\ge 1$ and $b\in S^{dk-l}_{1,0}({\Bbb R}^n\times{\Bbb R}^n)$. Then
$$
w(x,\xi ,t)=\tfrac 1{k!}t^{k}b(x,\xi )e^{-tp^0(x,\xi )}=e^{-c't}t\,w'(x,\xi ,t), \tag2.13
$$
where %$be^{-tp^0}\in S^{dk-l}_{1,0}({\Bbb R}^n\times{\Bbb R}^n)$,
$w'(x,\xi ,t)
\in S^{d-l}_{1,0}({\Bbb R}^n\times{\Bbb R}^n)$, uniformly  for $t\ge 0$.

Moreover,
$\tilde w(x,z,t)=\Cal F^{-1}_{\xi \to z}w$
satisfies for any $c'\in \,]0,c_0[\,$:
$$
|\tilde w(x,z,t)|\leg e^{-c't}\cases t\,|z|^{l-d-n}\text{ if }d-l>- n,\\
t\,(|\log |z||+1)\text{ if }d-l= -n ,\\
t\text{ if }d-l< -n .\endcases
\tag 2.14
$$
It follows that for $l\ge 1$, $\Cal
K_{V_{-d-l}}(x,y,t)=\Cal F^{-1}_{\xi \to z}v_{-d-l}(x,\xi
,t)|_{z=x-y}$ satisfies the estimates
$$
|\Cal K_{V_{-d-l}}(x,y,t)|\leg e^{-c't}\cases t\,|x-y|^{l-d-n}\text{ if }d-l>- n,\\
t\,(|\log |x-y||+1)\text{ if }d-l= -n ,\\
t\text{ if }d-l< -n .\endcases
\tag 2.15
$$
\endproclaim

\demo{Proof}
$1^\circ$. For each fixed $t>0$, $e^{-tp^0(x,\xi )}$ is rapidly decreasing in
$\xi $, hence is in  $S^{-\infty }_{1,0}$. But for our purposes we
need estimates that hold uniformly in $t$ for $t\to 0$. Let
$$
M_{j,k,l}=\sup_{s\ge 0 }s^l\partial_s^k(s^je^{-s}).
$$
Then for $t\ge 0$, $\xi \in{\Bbb R}^n$,
$$
\aligned
&|(tp^0(x,\xi ))^je^{-tp^0(x,\xi )}|\le M_{j,0,0},\\
& |\partial_{\xi _i}\bigl((tp^0 )^je^{-tp^0}
\bigr)|=|\partial_s(s^je^{-s})|_{s=tp^0 }t\partial_{\xi
_i}p^0 |\le M_{j,k,1}|(p^0) ^{-1}\partial_{\xi
_i}p^0 |\leg \ang{\xi }^{-1},\; \dots\\
&|\partial_{\xi }^\alpha \bigl((tp^0 )^je^{-tp^0
}\bigr)|\leg \ang{\xi }^{-|\alpha |},\dots
\endaligned\tag2.16
$$
showing the assertion for $c'=0$. (2.16) holds also if $p^0$ is
replaced by $p^0-c'$ throughout, when $c'\in \,]0,c_0[\,$.

$2^\circ$.
The first identity in
(2.13) was shown in (2.10).
We can also write
$$
w(x,\xi ,t)=\tfrac 1{k!}t\,b(p^0-c')^{1-k}(t(p^0-c'))^{k-1}e^{-c't}e^{-t(p^0-c')}=e^{-c't}t\,w'(x,\xi ,t).
$$
Here $b(p^0-c')^{1-k}$ is in $S^{d-l}_{1,0}$, independent of $t$, and by
$1^\circ$, $(t(p^0-c'))^{k-1}e^{-t(p^0-c' )}$ is uniformly in
$S^0_{1,0}$, so it follows that
$w'$ is uniformly in $S^{d-l}_{1,0}$. We can now apply Proposition 2.2
to draw the conclusion (2.14).

Since $v_{-d-l}(x,\xi ,t)$ is a sum of such terms when $l\ge 1$, the estimates
(2.15) follow.
\qed

\enddemo



For systems $P$ we can use systematic estimates from [G96]. We find for
general $P$:



\proclaim{Theorem 2.4}
$1^\circ$ In local coordinates,
$\Cal K_{V_{-d}}$ satisfies for some $c'>0$:
$$
|\Cal K_{V_{-d}}(x,y,t)|\leg e^{-c't} t\, |x-y|^{-d-n}.\tag2.17
$$
For $l\ge 1$, the kernels $\Cal K_{V_{-d-l}}$ satisfy {\rm (2.15)}.
If $\gamma (P)>0$,  we  modify $p^0$ to satisfy
$\inf_{x,\xi }\gamma (p^0(x,\xi ))\ge \gamma (P)$, then $c'$ can be any
number in $\,]0,\gamma (P)[\,$.


$2^\circ$ Moreover, with $p^0$ chosen as in $1^\circ$,
$$
|\Cal K_{V'_{M}}(x,y,t)|\leg e^{-c_1t}\cases t\, |x-y|^{M-d-n}\text{ if
}d-M>- n, \\ t\,(|
\log |x-y||+1)\text{ if
}d-M= -n,\\
t\text{ if
}d-M<- n,
\endcases\tag2.18
$$
for any $c_1<\gamma (P)$.
In particular,
$$
|\Cal K_{V}(x,y,t)|\leg e^{-c_1t}t\, |x-y|^{-d-n}.\tag2.19
$$
\endproclaim

\demo{Proof} $1^\circ$.
When $P$ is scalar, the estimates in (2.15) for $l\ge 1$ are shown in
Proposition 2.3, when we take $c_0=\gamma (P)$ if $\gamma (P)>0$.
For general systems $P$, the symbols $q_{-d-l}$ are sums of symbols as
in (1.7), and we apply [G96], Lemma 4.2.3. Here (4.2.35) with
$k=-d-l$ shows that
$$
|D_x^\beta D_\xi ^\alpha v_{-d-l}(x,\xi ,t)|\leg \ang\xi ^{d-l-|\alpha |}te^{-c't},
$$
for all $\alpha ,\beta $. Actually, the estimate (4.2.35) has $e^{-ct\ang\xi
^d}$ with a positive $c$ as the last factor, but an inspection of the
proof (the location of integral contours) shows that $e^{-ct\ang\xi
^d}$ can be replaced by $e^{-c't}$,
if $c'<\inf\gamma
(p^0(x,\xi ))$.
This shows that $e^{c't}t^{-1}v_{-d-l}$ is in $S^{d-l}_{1,0}$ uniformly in
$t$, so the estimates of the $\Cal K_{V_{-d-l}}$ follow by use of
Proposition 2.2.

 For $l=0$, we can
argue as follows in the scalar case: For each $j=1,\dots,n$,
$$
\partial_{\xi _j}v_{-d}=\partial_{\xi _j}e^{-tp^0}=-t(\partial_{\xi _j}p^0)e^{-tp^0},$$
where $\partial_{\xi _j}p^0\in S^{d-1}_{1,0}$. Now as in Proposition 2.3,
$e^{-c't}\partial_{\xi _j}p^0e^{-t(p^0-c')}$ is in $S^{d-1}_{1,0}$ uniformly in
$t$, and hence $\tilde v_{-d}=\Cal F^{-1}_{\xi \to z}v_{-d}$
satisfies, since $d-1>-n$,
$$
|z_j\tilde v_{-d}|\leg  e^{-c't}t\,|z|^{-d+1-n}.\tag2.20
$$
Taking the square root of the sum of squares for $j=1,\dots,n$, we find
after division by $|z|$ that
$$
|\tilde v_{-d}|\leg e^{-c't}t\,|z|^{-d-n}.\tag2.21
$$

In the systems case we note that $$
\partial_{\xi
_j}q_{-d}=-q_{-d}(\partial_{\xi _j}p^0)q_{-d},\tag2.22
$$ since
$\partial_{\xi
_j}[(p^0-\lambda )(p^0-\lambda )^{-1}]$
$=0$. Lemma 4.2.3 applies to
this in the same way as above, showing that
$$
|D_x^\beta D_\xi ^\alpha \partial_{\xi _j}v_{-d}(x,\xi ,t)|\leg \ang\xi ^{d-1-|\alpha |}te^{-c't},
$$
so $e^{c't}t\partial_{\xi _j}v_{-d}$ is
is uniformly in $S^{d-1}_{1,0}$. We conclude (2.20), from which (2.21)
follows, implying (2.17).


$2^\circ$. Here the estimate in (2.18) has already been shown for large $M$ in
Theorem 2.1.
For lower values of $M$, we can add the
estimates of the entering homogeneous terms $\Cal K_{V_{-d-l}}$ with
$l\ge M$;
the top term gives the weakest estimate. (It is used that $x$ and $y$ need only run in a
bounded set, for the contribution from the localized piece.)
\qed


\enddemo


Theorems 2.1 and 2.4 together lead to Poisson-like kernel estimates:

\proclaim{Theorem 2.5} %Assumptions as in Theorem {\rm 2.4}.
$1^\circ$ One has
in local coordinates:
$$
|\Cal K_{V_{-d-l}}(x,y,t)|\leg e^{-c't}\cases t\,( |x-y|+ t^{1/d})^{l-d-n}\text{ if
}d-l>- n, \\ t\,(|
\log (|x-y|+t^{1/d})|+1)\text{ if
}d-l= -n,\\
t\text{ if
}d-l<- n,
\endcases\tag2.23
$$
for some  $c'>0$. If $\gamma (P)>0$, we  modify $p^0$ to satisfy
$\inf_{x,\xi }\gamma (p^0(x,\xi ))\ge \gamma (P)$; then $c'$ can be any number in $\,]0,\gamma (P)[\,$.

$2^\circ$ Moreover, with $p^0$ chosen as in $1^\circ$,
$$
|\Cal K_{V'_{M}}(x,y,t)|\leg e^{-c_1t}\cases t\,( |x-y|+ t^{1/d})^{M-d-n}\text{ if
}d-M>- n, \\ t\,(|
\log(|x-y|+t^{1/d})|+1)\text{ if
}d-M=  -n,\\
 t\,\text{ if }d-M<  -n,
\endcases\tag2.24
$$
for any $c_1<\gamma (P)$.
In particular,
$$
\aligned
|\Cal K_{V}(x,y,t)|&\leg e^{-c_1t}t\,(|x-y|+t^{1/d})^{-d-n},\\
|\Cal K_{V'_1}(x,y,t)|&\leg e^{-c_1t}t\,(|x-y|+t^{1/d})^{1-d-n}.
\endaligned
\tag2.25
$$

$3^\circ$ For the operators defined on $M$, one has (with $d(x,y)$
denoting the distance between $x$ and $y$)
$$
|\Cal K_{V}(x,y,t)|\leg e^{-c_1t}t\,(d(x,y)+t^{1/d})^{-d-n},\tag2.26
$$
for any $c_1<\gamma (P)$.
\endproclaim

\demo{Proof}  $1^\circ$--$2^\circ$. In the region where $|x-y|\ge t^{1/d}$,
$$
|x-y|\le |x-y|+t^{1/d}\le 2|x-y|,
$$
in other words, $|x-y|\eg |x-y|+t^{1/d}$. Then the estimates in
Theorem 2.4 imply the validity of the above estimates on this region.

In the region where $|x-y|\le t^{1/d}$, we have instead that $t^{1/d}\eg
|x-y|+t^{1/d}$.
Then the estimates in Theorem 2.1 imply the above estimates on
that region; for example
$$
t^{-n/d}=t\, (t^{1/d})^{-d-n}\eg t\, (|x-y|+t^{1/d})^{-d-n}
$$
there. For the two regions together, this shows (2.23)--(2.25).

$3^\circ$. This follows from the estimates in local coordinates.
\qed
\enddemo

When the eigenvalues of $P$ with real part equal to $\gamma (P)$
(necessarily finitely many) are semisimple (i.e., the algebraic
multiplicity equals the geometric multiplicity), we can sharpen the
information on the behavior for $t\to\infty $:


\proclaim{ Corollary 2.6} Assume that all eigenvalues of $P$ with real
part $\gamma(P)$ are semisimple (it holds in particular when $P$ is selfadjoint). Then
$$
|\Cal K_{e^{-tP}}(x,y,t)|\leg e^{-\gamma(P)t}
\frac{t}{(d(x,y)+t^{1/d})^d}\left(
(d(x,y)+t^{1/d})^{-n}+1\right) .\tag 2.27
$$
\endproclaim

\demo{Proof}
The spectral projections $\Pi_{j} = \frac{i}{2\pi}\int_{{\Cal C}_j}
(P-\lambda)^{-1} d\lambda$ onto the eigenspaces $X_{j}$ for the
eigenvalues
$\{\lambda _1,\dots,\lambda _k\}$ with real part
$\gamma(P)$ (where $\Cal C_j$ is a small circle around the eigenvalue),
are pseudodifferential operators of order $-\infty$, and their kernels
$\Cal
K_{\Pi _{j}}(x,y)$ are
bounded. If $\varepsilon>0$, the operator $P' = P + \varepsilon
\sum_{j=1}^k\Pi_{j}$ satisfies $\gamma(P')>\gamma(P)$. By
Theorem 2.5 applied to $P'$,
$$
|\Cal K_{e^{-tP'}}(x,y,t)|\leg e^{-\gamma(P)t} t (d(x,y)+t^{1/d})^{-d-n} .
$$
On the other hand, $V(t)=e^{-tP'}+(1- e^{-\varepsilon
t})\sum_{j=1}^ke^{-t\lambda _j}\Pi_{j }$, so
$$
\Cal K_{e^{-tP}}(x,y,t)=\Cal
K_{e^{-tP'}}(x,y,t) +(1- e^{-\varepsilon t})\sum_{j=1}^ke^{-t\lambda _j}
{\Cal K}_{\Pi_{j}}(x,y).
$$
From
$$
1-e^{-\varepsilon t}\leq \min\{1, \varepsilon t\} \leg
\frac{t}{(\operatorname{diam}(M) +t^{1/d})^d} \leq \frac{t}{(d(x,y)
+t^{1/d})^d} ,
$$
we conclude that $(1- e^{-\varepsilon t})|{\Cal K}_{\Pi_j}(x,y)|\leg
\frac{t}{(d(x,y) +t^{1/d})^d}$, and (2.27) follows since
$|e^{-t\lambda _j}|=e^{-t\gamma (P)}$ for each $j$.\qed
\enddemo


\example{Remark 2.7}
The proof of Corollary 2.6 allows to sharpen the estimates in Theorem
2.5 and Theorem 2.9 below
also in the general case where the eigenvalues with real part
$\gamma(P)$ are not all semisimple.
Denote by $r$ the dimension of the largest irreducible $P$-invariant
subspace of any eigenspace $X_{j}$ associated to an eigenvalue with real
part $\gamma(P)$. Then in Theorems 2.5 and 2.9 we may replace the upper bound
$e^{-c't} t (d(x,y)+t^{1/d})^{-d-n-k}$ by
$$e^{-\gamma(P)t}
(1+t^{r-1})\frac{t}{(d(x,y)+t^{1/d})^d}\bigl(
(d(x,y)+t^{1/d})^{-n-k}+1\bigr).\tag2.28
$$
\endexample



It is not hard to extend the estimates to complex $t$ in a sector
around $\rp$. Namely, since $p^0$ has its spectrum in the sector
$\{|\arg \lambda |\le \theta _0\}$,
$e^{i\varphi }P$ satisfies the parameter-ellipticity condition when $|\varphi |<\varphi _0=\frac\pi 2-\theta _0$.
For each $\varphi $ it generates a semigroup $e^{-te^{i\varphi }P}$,
and these operator families coincide with the holomorphic extension of
$V(t)$ to the rays $\{re^{i\varphi }\}$ in the sector  $V_{\varphi _0}=\{t\in{\Bbb C}\mid
|\operatorname{arg}t|<\varphi _0\}$. On each ray we have the estimates
in Theorem 2.5, they hold uniformly in closed subsectors of $V_{\varphi
_0}
$. We have hereby obtained:


\proclaim{Theorem 2.8} With $\varphi _0$ and $\theta _0$ defined as in
the beginning of  Section {\rm 1},
the semigroup
generated by $P$ extends holomorphically to the sector $\{|\arg t|<\varphi _0\}$, and the
estimates in Theorem {\rm 2.5} hold in terms of  $|t|$ on any closed sector $\{|\operatorname{arg}t|\le\varphi \}$ with
$0<\varphi <\varphi _0$, taking $c_1<\min_{|\varphi '|\le \varphi }\gamma (e^{i\varphi '}P)$.
\endproclaim

More information in the case where $P$ is selfadjoint will be given in
Section 3 below.

Also the derivatives
of the kernels can be estimated by use of the symbol estimates in [G96].



%; this gives estimates with $d$ replaced by
%$d+|\beta  |$, as in [G96], (4.2.60).


\proclaim{Theorem 2.9} %Assumptions as in Theorem {\rm 2.4}.
$1^\circ$ One has
in local coordinates:
$$
\multline
|D_x^\beta D_{y}^\gamma D_t^j\Cal K_{V_{-d-l}}(x,y,t)|\leg \\
e^{-c't}\cases t\,( |x-y|+ t^{1/d})^{l-(1+j)d-|\gamma |-n}\text{ if
}(j+1)d+|\gamma | -l>- n, \\ t\,(|
\log (|x-y|+t^{1/d})|+1)\text{ if
}(j+1)d+|\gamma |-l= -n,\\
t\text{ if
}(j+1)d+|\gamma |-l<- n,
\endcases
\endmultline
\tag2.29
$$
for some  $c'>0$. If $\gamma (P)>0$, we  modify $p^0$ to satisfy
$\inf_{x,\xi }\gamma (p^0(x,\xi ))\ge \gamma (P)$; then $c'$ can be any number in $\,]0,\gamma (P)[\,$.

$2^\circ$ Moreover, with $p^0$ chosen as in $1^\circ$,
$$
\multline
|D_x^\beta D_{y}^\gamma D_t^j\Cal K_{V'_{M}}(x,y,t)|\leg \\
e^{-c_1t}\cases t\,( |x-y|+ t^{1/d})^{M-(j+1)d-|\gamma |-n}\text{ if
}(j+1)d+|\gamma |-M>- n, \\ t\,(|
\log(|x-y|+t^{1/d})|+1)\text{ if
}(j+1)d+|\gamma |-M=  -n,\\
 t\,\text{ if }(j+1)d+|\gamma |-M<  -n,
\endcases
\endmultline
\tag2.30
$$
for any $c_1<\gamma (P)$.

$3^\circ$ The estimates of derivatives of $\Cal K_V$ hold for the operator defined on $M$ with
$|x-y|$ replaced by $d(x,y)$.



\endproclaim

\demo{Proof} As in Theorem 2.7, the estimates are pieced together from
estimates generalizing those in Theorem 2.1 resp.\ Theorem 2.4 to
include derivatives. We use that
$$
\multline
|D_x^\beta D_y^\gamma D_t^j\Cal K_{V_{-d-l}}(x,y,t)|=|D_x^\beta
D_z
^\gamma  D_t^j\tilde v_{-d-l}(x,z ,t)\big|_{z=x-y}|\\
=|\Cal F^{-1}_{\xi
\to z}(\xi ^\gamma D_x^\beta D_t^j v_{-d-l}(x,\xi ,t))\big|_{z=x-y}|.
\endmultline
$$

To generalize Theorem 2.1 to allow
$x$- and $y$-derivatives we just have to apply the arguments of
[G96] Theorems 4.2.2 and 4.2.5 to the modified symbols $\xi ^\gamma
D_x^\beta  v_{-d-l}$, to get the  estimates (2.29)
with  $|x-y|$ replaced by 0. Derivatives with respect to $t$ alone are
explained in Theorem 4.2.5; finally this is combined with $x$- and
$y$-derivatives in a straightforward way.
Similar considerations work for remainders; here we can in fact refer
directly to (4.2.60) for large $M$, and the statements for lower $M$ follow by
addition of the appropriate set of estimates of $\Cal K_{V_{-d-l}}$-terms. This gives the expected generalization of
Theorem 2.1, namely (2.29)--(2.30) with $|x-y|$ replaced by 0.

For the generalization of Theorem 2.4 we note that estimates
$$
|\xi ^\gamma D_x^\beta D_\xi ^\alpha D_t^jv_{-d-l}(x,\xi ,t)|\leg
\ang\xi ^{(j+1)d+|\gamma |-|\alpha |-l}t e^{-c't}
$$
for $|\alpha  |+l>0$, all $\beta $, $j$, follow
from [G96] Lemma 4.2.3
(see the remarks around (4.2.40) for how to include
$t$-derivatives, as done also in Theorem 4.2.5). Thus
$e^{c't}t^{-1}\xi ^\gamma D_x^\beta D_t^jv_{-d-l}$ is in
$S^{(j+1)d+|\gamma |-l}_{1,0}$ uniformly in $t$, and it follows by Proposition
2.2 that
$$
|D_z ^\gamma D_x^\beta D_t^j\tilde v_{-d-l}(x,z ,t)|\leg
 e^{-c't}\cases t|z|^{-(j+1)d-|\gamma |+l-n},\text{ if
 }(j+1)d+|\gamma | -l>- n, \\
t\,(|
\log (|z|+t^{1/d})|+1)\text{ if
}(j+1)d+|\gamma |-l= -n,\\
t\text{ if
}(j+1)d+|\gamma |-l<- n.\endcases
$$
This implies
estimates as in (2.29) with $|x-y|+t^{1/d}$ replaced by $|x-y|$.
The conclusion is immediate for $l\ge 1$, and for $l=0$, we use the
estimates of $D_{\xi _j}v$ as in the proof of Theorem 2.4.
Again for remainder estimates, we can appeal to (4.2.60) for
large $M$.

The proof is now completed as in Theorem 2.7.
\qed
\enddemo

\head 3. Estimates in the complex plane for selfadjoint operators \endhead

In this section we shall derive some uniform estimates for the
extension of the semigroup into the region ${\Bbb C}_+=\{t
\in{\Bbb C}\mid \operatorname{Re}t >0\}$, when $P$ is
selfadjoint. To do so, we need to account for how the estimates of
symbols and remainders like (1.6--8) depend  on $\arg
\lambda $. We assume for simplicity that $P$ is $\ge 0$.

As already observed in Theorem 2.7, $V(t)$ exists for $t\in {\Bbb
C}_+$. There are uniform estimates on closed subsectors, but to
describe the behavior for rays
near the imaginary axis we need estimates of $q(x,\xi ,\lambda )$ for
$\lambda $ near $\rp$.

As in [G96], we denote $|\lambda |^{1/d}=\mu $, and write $\ang{(\xi ,\mu )}=(1+|\xi
|^2+\mu ^2)^{1/2}$ for short as $\ang{\xi ,\mu }$;
it is $\eg (1+|\xi |+|\lambda |^{1/d})$.

\proclaim{Proposition 3.1} Let $P$ be selfadjoint $\ge 0$ and let
$\lambda \in{\Bbb C}$ with $\arg \lambda =\varphi \in \,]0,\frac\pi
2[\,$. Then
$$
\aligned
|q_{-d}(x,\xi ,\lambda )|&=|(p^0(x,\xi )-\lambda )^{-1}|\leg (\sin \varphi )^{-1}\ang{\xi ,\mu }^{-d},\\
|D_{x}^\beta D_{\xi}^\alpha q_{-d}(x,\xi,\lambda )|&\leg (\sin \varphi
)^{-1-|\alpha |-|\beta |}\ang\xi ^{d-l-|\alpha |}\ang{\xi ,\mu }^{-2d}
,\text{ when }|\alpha |+|\beta |>0.
\endaligned\tag3.1
$$
For all $l,\alpha ,\beta $ with $l>0$,
$$
|D_{x}^\beta D_{\xi}^\alpha q_{-d-l}(x,\xi,\lambda )|\leg (\sin \varphi )^{-2l-|\alpha |-|\beta |}\ang\xi ^{d-l-|\alpha |}\ang{\xi ,\mu }^{-2d}.
\tag3.2
$$

\endproclaim

\demo{Proof}
We have for  $\lambda =e^{i\varphi }|\lambda |$ with $\varphi \in
\,]0,\frac\pi 2[\,$,  $v\in{\Bbb C}^N$, since $p^0(x,\xi )$ is
symmetric with lower bound $\ge c\ang\xi ^d$:
$$\aligned
|(p^0v,v)-\lambda |v|^2|&\ge |\operatorname{Im}((p^0v,v)-|\lambda
|e^{i\varphi }|v|^2)|=|\lambda |\sin\varphi |v|^2,\\
|(p^0v,v)-\lambda |v|^2|&=|e^{-i\varphi }(p^0v,v)-|\lambda ||v|^2|\ge
|\operatorname{Im}e^{-i\varphi }(p^0v,v)|\\&=\sin\varphi (p^0v,v)\ge
\sin\varphi \,c\ang \xi ^d|v|^2,
\endaligned
$$
from which follows
$$
|(p^0-\lambda )v||v|\ge |((p^0-\lambda )v,v)|\geg \sin\varphi (|\lambda |+\ang\xi ^d)|v|^2.
$$
 This implies that $|(p^0-\lambda )^{-1}|\leg (\sin\varphi
 )^{-1}(\ang\xi ^d+|\lambda |)^{-1}\eg (\sin\varphi
 )^{-1}\ang{\xi ,\mu }^{-d}$, showing (3.1).

The other estimates follow as in [G96] from the structure of the
terms in the parametrix, using (3.1): $q_{-d-l}$ is for $l\ge 1$ a
finite sum of terms,
where $\nu _1+\dots+\nu _M\ge 2$ takes
values up to $2l$,
$$
r(x,\xi ,\lambda )=b_1q_{-d}^{\nu _1}b_2q_{-d}^{\nu _2}\cdots b_Mq_{-d}^{\nu _M}b_{M+1},
$$
cf.\  (1.7). Each $q_{-d}$ contributes with a factor $(\sin\varphi )^{-1}$,
and there are up to $2l$ such factors; this shows (3.2) for $\alpha
=\beta =0$. Each differentiation may hit a factor $q_{-d}$ giving an
extra $(\sin\varphi )^{-1}$ in view of (2.22); this leads to the
estimates (3.2) by the Leibniz formula.
\qed
\enddemo

Estimates of remainders $Q'_M$ and their symbols $q'_M$ are more
difficult to
work out, since they depend on the interplay between the exact
resolvent $Q_\lambda $ and the homogeneous symbol terms, and they will
be more
costly in powers of $(\sin\varphi )^{-1}$ the larger $M$ is taken. We
shall here go directly to remainder {\it kernel} estimates.

More precisely, we consider the kernel of the
operator $Q'_M=Q_\lambda -\sum_{l<M}Q_{-d-l}$, where each $Q_{-d-l}$ is an
operator on the manifold $M$ constructed from the symbols $q_{-d-l}$
in local coordinates.

For $\lambda $ as in Lemma 3.1,
$$
\|(P-\lambda )u\| \|u\|\ge |((P-\lambda )u,u)|\ge \operatorname{Im}\lambda \|u\|^2=\sin\varphi \,|\lambda |\,\|u\|^2;
$$
hence the resolvent $Q_\lambda =(P-\lambda )^{-1}$ has operator norm
$\le (\sin\varphi \,|\lambda |)^{-1}$ in $L_2(M)$. The operator
$Q'_M=Q_\lambda -\sum_{l<M}Q_{-d-l}$ is a $\psi $do of order $-d-M$
(for each $\lambda $). Moreover,
$$
\aligned
Q'_M&=Q'_M(P-\lambda )Q_\lambda =\widetilde R_MQ_\lambda ,\text{ where
}\\
\widetilde R_M&=Q'_M(P-\lambda )=1-{\sum}_{l<M}Q_{-d-l}(P-\lambda )
\endaligned\tag3.3
$$
is a $\psi $do of order $-M$. Now
$$
\|\widetilde R_MQ_\lambda \|_{\Cal L(L_2, H^s)}\le (\sin\varphi \,|\lambda |)^{-1}\|\widetilde R_M\|_{\Cal L(L_2, H^s)}.
$$
For $s>n$, it is known that a $\psi $do continuous from $L_2(M)$ to
$H^s(M)$  has a continuous kernel estimated by the
operator norm; then
 the kernel of $Q'_M=\widetilde R_MQ_\lambda $ is continuous
and  is  estimated by
$$
|\Cal K_{Q'_M}(x,y,\lambda )|\leg \|Q'_M \|_{\Cal L(L_2, H^s)}\leg (\sin\varphi |\lambda |)^{-1}\|\widetilde R_M \|_{\Cal L(L_2, H^s)}.\tag3.4
$$

In preparation for the study of $\widetilde R_M$ and its dependence on
$\lambda $ we prove a lemma on a
typical composition
formula, treated by basic methods in the $\psi $do theory.



\proclaim{Lemma 3.2}
Let
$b(x,\xi ) \in
S^{d_2}_{1,0}({\Bbb R}^n\times{\Bbb R}^n)$, and let $a(x,\xi ,\lambda ) \in S^{d_1}_{1,0}({\Bbb R}^n\times{\Bbb R}^n)$
with respect to $(x,\xi )$, with $\lambda $ as in Proposition {\rm
3.1}, such that for some $d'\ge 0$, $N \in \Bbb R$ one has for all $\alpha ,\beta \in{\Bbb N}_0^n$,
$$
|D_x^\beta D_\xi ^\alpha a(x,\xi ,\lambda )|\leg
(\sin\varphi)^{-N-|\alpha|-|\beta|}\ang\xi^{d' + d_1-|\alpha|}
\ang{\xi ,\mu }^{-d'} \ .\tag3.5
$$

$1^\circ$ There exists $c (x,\xi ,\lambda ) \in S^{d_1+d_2}_{1,0}({\Bbb R}^n\times{\Bbb R}^n)$
such that $\operatorname{Op}(a) \operatorname{Op}(b) =
\operatorname{Op}(c)$, and for every $M \in{\Bbb N}_0$, %with $M\geq d_1+d$
$$
c(x,\xi ,\lambda ) = {\sum}_{|\alpha|<M} \tfrac{1}{\alpha!}\ D_\xi^\alpha
a(x,\xi ,\lambda )
\partial_x^\alpha b(x,\xi) + r_M(a,b) \ ,\tag3.6
$$
where
$$
|D_x^\beta D_\xi ^\alpha r_M(a,b)| \leg
(\sin\varphi)^{-N-M-|\alpha|-|\beta|}\ang\xi^{d'+d_1+d_2-M-|\alpha|}
\ang{\xi ,\mu }^{-d'}.\tag3.7
$$

$2^\circ$ If {\rm (3.5)} for $\alpha =\beta =0$ is replaced by
$$
|a(x,\xi ,\lambda )|\leg
(\sin\varphi)^{-N}\ang\xi^{d + d_1}
\ang{\xi ,\mu }^{-d} \ .\tag3.8
$$
for some $0\le d\le d'$, then {\rm (3.6)} holds with
{\rm (3.7)} valid for $M\ge 1$ and the
estimates of $r_0$  replaced by
$$
|D_x^\beta D_\xi ^\alpha r_0(a,b)| \leg
(\sin\varphi)^{-N-|\alpha|-|\beta|}\ang\xi^{d+d_1+d_2-|\alpha|}
\ang{\xi ,\mu }^{-d}.
\tag3.9
$$

$3^\circ$ For $\gamma \in{\Bbb N}_0^n$, $D^\gamma \Op(a)=\Op(a^\gamma)
$, where
$$
|D_x^\beta D_\xi ^\alpha a^\gamma (x,\xi ,\lambda )|\leg
{\sum}_{k\le |\gamma |}(\sin\varphi)^{-N-k-|\alpha|-|\beta|}\ang\xi^{d' + d_1+|\gamma |-k-|\alpha|}
\ang{\xi ,\mu }^{-d'} \ .\tag3.10
$$
%%GG

\endproclaim

\demo{Proof} $1^\circ$. Let $\chi (x,\xi )$ denote a $C^\infty
$-function that is 1 for $|x|^2+|\xi |^2\le 1$ and vanishes for
$|x|^2+|\xi |^2\ge 2$, then we can replace the given symbols by
their products with $\chi (\varepsilon x,\varepsilon \xi )$, which
makes all integrals calculated below convergent.
It is known in the theory (by the technique of oscillatory integrals, cf.\ [H83],
Sect.\ 7.8), that the resulting symbols converge to the given symbols
for $\varepsilon \to 0$
in all the seminorms that are involved. The modified symbols will again
be denoted $a$, $b$.
We can also assume that $b$ has compact support in $x$
(in a set containing the $x$ for which we need the formula). Then
$
\hat b(\eta ,\xi )=\Cal F_{x\to \eta }b(x,\xi  )
$
 satisfies
$$
|D_\xi ^\alpha \hat b(\eta ,\xi )|\leg \ang\eta ^{-N'}\ang\xi ^{d_2-|\alpha |},\tag3.11
$$
for all $\alpha , N'$.
It follows from the $\psi $do
defining  formula
that $\operatorname{Op}(a)\Op(b)=\Op (c)$, where
$$
\aligned
c(x,\xi ,\lambda )&=\int_{{\Bbb R}^{4n}}a(x,\eta ,\lambda
)b(y,\xi )e^{i(x-y)\cdot\eta }e^{i(y-z)\cdot\xi }\,dz\d\xi dy\d\eta \\
&=\int_{{\Bbb R}^n}a(x,\xi +\eta ,\lambda)\hat b(\eta ,\xi
)e^{ix\cdot\eta }\,\d\eta .
\endaligned\tag3.12
$$
If $M>0$, we insert the Taylor expansion of $a$ in $\xi $ up to order $M$,
$$
\multline
a(x,\xi +\eta ,\lambda )= {\sum}_{|\alpha |<M}\tfrac1{\alpha !}\eta
^\alpha \partial_{\xi }^\alpha a(x,\xi ,\lambda )\\
+{\sum}_{|\alpha
|=M}\tfrac M{\alpha !}\eta ^\alpha \int_0^1(1-h)^{M-1}\partial_\xi ^\alpha a(x,\xi +h\eta ,\lambda )\,dh,
\endmultline
$$
obtaining that $c=c_{<M}+r_M$, where
$$
\aligned
c_{<M}&=(2\pi )^{-n}\int_{{\Bbb R}^n}{\sum}_{|\alpha |<M}\tfrac1{\alpha
!}\partial_\xi ^\alpha a(x,\xi ,\lambda )\eta ^\alpha \hat b(\eta ,\xi
)e^{ix\cdot\eta }\,\d\eta\\
&={\sum}_{|\alpha|<M} \tfrac{1}{\alpha!}\partial_\xi^\alpha
a(x,\xi ,\lambda )
D_x^\alpha b(x,\xi)={\sum}_{|\alpha|<M} \tfrac{1}{\alpha!}D_\xi^\alpha
a(x,\xi ,\lambda )
\partial_x^\alpha b(x,\xi), \\
r_M&=(2\pi )^{-n}\int_{{\Bbb R}^n}{\sum}_{|\alpha |=M}\tfrac M{\alpha
!}\int_0^1(1-h)^{M-1}\partial_\xi ^\alpha a(x,\xi +h\eta ,\lambda )\,dh \,\eta ^\alpha \hat b(\eta ,\xi
)e^{ix\cdot\eta }\,\d\eta.
\endaligned
$$
The sum over $|\alpha |<M$ equals the sum in (3.6). For the
last integral we use that
$$
\aligned
|\partial_\xi ^\alpha a(x,\xi +h\eta ,\lambda )|&\leg (\sin\varphi
)^{-N-M}\ang{\xi +h\eta }^{d'+d_1-M}\ang{\xi +h\eta ,\mu }^{-d'}\\ &\leg
(\sin\varphi
)^{-N-M}\ang{\xi }^{d'+d_1-M}\ang{\xi ,\mu }^{-d'}\ang\eta ^{|d'+d_1-M|+d'},
\endaligned
$$
by the Peetre inequality. Taking this together with the estimates
(3.11) of $\hat b$ (with a large $N'$), we can conclude
that
$$
|r_M|\leg
(\sin\varphi
)^{-N-M}\ang{\xi }^{d'+d_1+d_2-M}\ang{\xi ,\mu }^{-d'}.
$$

For $M=0$, we apply such considerations directly to $r_0=c(x,\xi ,\lambda
)$ in (3.12):
$$
|r_0|\leg (\sin\varphi )^{-N}\int \ang{\xi +\eta }^{d'+d_1}\ang{\xi
+\eta ,\mu }^{-d'}\ang\eta ^{-N'}\ang\xi ^{d_2}\,\d\eta \leg
(\sin\varphi )^{-N} \ang{\xi  }^{d'+d_1+d_2}\ang{\xi
 ,\mu }^{-d'}.
$$

Derivatives of $r_M$ in $x$ and $\xi $ are treated in a similar way.

In the case $2^\circ$ the proof goes through in a similar way, except
that $d'$ is replaced by $d$ in expressions containing
undifferentiated factors $a$.

In $3^\circ$,  the $\lambda $-independent factor is to the left, and
(3.12) holds with integrand \linebreak$(\xi +\eta )^\gamma \Cal F_{z\to\eta }a(z,\xi
,\lambda )e^{ix\cdot\eta }$. The Taylor expansion of $(\xi +\eta )^\gamma $ is
a finite binomial expansion ${\sum}_{\kappa \le \gamma }\tbinom
\gamma  \kappa \xi ^{\gamma -\kappa }\eta ^\kappa $ and leads to  a finite
composition formula where the estimates (3.10) of the terms can be read off directly.
\qed


\enddemo

The composed symbol $c=r_0(a,b)$ is also denoted $a\circ b$ (used in
[G96]) or $a\# b$.


The lemma will be used in the following investigation of the symbol
%and norm
of $\widetilde R_M$. We denote $P-\lambda =\widetilde P$, with the
parameter-dependent symbol $\tilde p(x,\xi ,\lambda )=p(x,\xi
)-\lambda $ in local coordinates; here for any $M\in {\Bbb N}_0$,
$$
\aligned
P&=\sum_{k<M}p_{d-k}+p'_M,\quad \tilde P=\sum_{k<M}\tilde
p_{d-k}+\tilde p'_M,\text{ with}\\
\tilde p_d&=p-\lambda ,\quad \tilde p_{d-k}=p_{d-k}\text{ for
}k>0,\quad \tilde p'_M=p'_M\text{ for }M>0.
\endaligned\tag3.13
$$
$p_d$ is also denoted $p^0$.
The $p_{d-k}$ are homogeneous in $|\xi |$ of degree $d-k$ for $|\xi
|\ge 1$, and $p'_M\in S^{d-M}_{1,0}({\Bbb R}^n\times{\Bbb R}^n)$.


\proclaim{Proposition 3.3} Let $M\ge 1$. The symbol $\widetilde r_M(x,\xi ,\lambda )$ of
$\widetilde R_M$ (cf.\ {\rm (3.3)}) satisfies in local coordinates:
$$
|D_x^\beta D_\xi ^\alpha \widetilde r_M(x,\xi ,\lambda )|\leg (\sin\varphi )^{-2M+1-|\alpha |-|\beta |}\ang\xi ^{d-M-|\alpha |}\ang{\xi ,\mu }^{-d}.\tag3.14
$$
\endproclaim

\demo{Proof}
We have that
$$
\widetilde r_M=1-{\sum}_{k<M}{\sum}_{l<M}q_{-d-l}\circ
\tilde p_{d-k}- {\sum}_{l<M}q_{-d-l}\circ
\tilde p'_{M}.
$$


The terms in the parametrix symbol ${\sum}_{l\geq
0} q_{-d-l}$ are constructed as  solutions to the successive equations
for $m\in{\Bbb N}_0$:
$$
{\sum}_{|\alpha| + k + l = m}
\tfrac{1}{\alpha!}\ D_\xi^\alpha q_{-d-l} \partial_x^\alpha \tilde p_{d-k} =\cases
 1 \text{ for }m=0,\\
0\text{ for }m =1,2,\dots,
\endcases \tag 3.15
$$
cf.\ e.g.\ Seeley [S67], (1). We can use the truncated composition formula in Lemma 3.2
to compute the
symbol $\widetilde{r}_{  M}\ $ of $\widetilde{R}_{  M}\
$ with expansions in up to  ${M}$ homogeneous terms:
$$
\aligned
 \widetilde{r}_{  M} &=
1-{\sum}_{k<M}{\sum}_{l<M}
\Bigl\{{\sum}_{k+l+|\alpha| < {M}} \tfrac{1}{\alpha!}\ D_\xi^\alpha
q_{-d-l} \partial_x^\alpha  \tilde p_{d-k} + r_{{M}-k-l}(q_{-d-l},
\tilde p_{d-k})\Bigr\} \\
 & \qquad - r_0\Bigl({\sum}_{l<M}q_{-d-l},
\tilde p'_{M}\Bigr) \ .
\endaligned
$$
By (3.15),
$$
{\sum}_{k<M}{\sum}_{l<M}
{\sum}_{k+l+|\alpha| <M} \tfrac{1}{\alpha!}\ D_\xi^\alpha q_{-d-l}
\partial_x^\alpha \tilde p_{d-k} = 1.
$$
Thus $\widetilde r_M$ consists of the following terms:
$$
\widetilde{r}_{  M}  =
-{\sum}_{k<M}{\sum}_{l<M}
r_{{M}-k-l}(q_{-d-l}, \tilde p_{d-k}) - r_0\Bigl({\sum}_{l<M}q_{-d-l}, \tilde p'_{M}\Bigr) \ .
$$

Using the estimates (3.1)  and (3.2) together with
$|D_x^\beta D_\xi^\alpha \tilde p_{d-k}(x,\xi)| \leg \ang\xi^{d-k-|\alpha|}$,
% HG: d'
we obtain from Lemma 3.2 with $d'=2d$, $d_1 = -d-l$ and $d_2 = d-k$ that for $l \geq 1$:
$$
\aligned
|D_x^\beta &D_\xi^\alpha r_{{M}-k-l}(q_{-d-l}, \tilde p_{d-k})|\\
&\leg
% HG: \ang{\xi ,\mu }^{-2d}
(\sin\varphi)^{-{M}+k+l-2l-|\alpha|-|\beta|}\ang\xi^{2d-d-l+d-k-({M}-k-l)-|\alpha|}
\ang{\xi ,\mu }^{-2d}\\
& \leq % HG: \ang{\xi ,\mu }^{-2d}
(\sin\varphi)^{- 2M +
1-|\alpha|-|\beta|}\ang\xi^{2d-{M}-|\alpha|}\ang{\xi ,\mu }^{-2d},
%%GG removed a d in sine power and changed d to 2d
\endaligned\tag 3.16
$$
since $k-l\ge -M+1$.

For $l=0$ we find in view of Lemma 3.2 $2^\circ$, since $k<M$,
%%GG 2d also here
$$
\aligned
|D_x^\beta &D_\xi^\alpha r_{{M}-k}(q_{-d}, \tilde p_{d-k})|\\
 &\leg
(\sin\varphi)^{-1-{(M-k)}-|\alpha|-|\beta|}\ang\xi^{2d-d+d-k-{(M-k)}-|\alpha|}\ang{\xi
,\mu }^{-2d}\\
&\leg
(\sin\varphi)^{-{M}-1-|\alpha|-|\beta|}\ang\xi^{2d-{M}-|\alpha|}\ang{\xi ,\mu }^{-2d}
\ .
\endaligned\tag3.17
$$
Finally,
$$
\aligned
|D_x^\beta &D_\xi^\alpha  r_0\Bigl({\sum}_{l<M}q_{-d-l},
\tilde p'_{M}\Bigr)|\\
 &\leg (\sin\varphi)^{-1-|\alpha|-|\beta|}\ang\xi^{d-
M-|\alpha|}\ang{\xi ,\mu }^{-d}\\
% HG: typo l<=M replaced by l<M, some d's replaced by 2d's
&\quad +{\sum}_{l<M}(\sin\varphi)^{-2l
-|\alpha|-|\beta|}\ang\xi^{2d-d-l+(d-M)
-|\alpha|}\ang{\xi ,\mu }^{-2d} \\
& \leg (\sin\varphi)^{-2  M
+1-|\alpha|-|\beta|}\ang\xi^{d-
M-|\alpha|}\ang{\xi ,\mu }^{-d} \ .
\endaligned \tag3.18
$$
An addition of the contributions (using that $\ang\xi /\ang{\xi ,\mu
}\le 1$) gives (3.14).
\qed
\enddemo

Now let us show how the kernel of $V'_M(t)$ is estimated. Similarly to
Corollary 2.6, it will be convenient to write $P=P^\varepsilon
-\varepsilon \Pi _0$, where $\Pi _0$ is the orthogonal projection onto
the zero eigenspace of $P$, and $\varepsilon >0$ is chosen $\le $ the
lowest positive eigenvalue, whereby $P^\varepsilon =P+\varepsilon
\Pi _0$ is $\ge \varepsilon $. Here $\Pi _0$ is the $\psi $do of order 0 with kernel ${\sum}_{j=1}^\nu
\varphi _j(x)\varphi _j(y)^*$, for an orthonormal basis $\varphi
_1,\dots,\varphi _\nu $ of the zero eigenspace.
Then $$
V(t)=V^\varepsilon (t)+(1-e^{-\varepsilon t})\Pi _0,\text{ where }V^\varepsilon (t)=e^{-tP^\varepsilon };
$$
and it is the latter operator that needs investigation. $V^\varepsilon (t)$ is defined from the resolvent
$Q^\varepsilon _\lambda =Q_\lambda -(\varepsilon -\lambda )^{-1}\Pi
_0=(P^\varepsilon -\lambda )^{-1}$ by
$$
V^\varepsilon (t)=\tfrac i {2\pi }\int_{\Cal C}e^{-t\lambda }Q^\varepsilon _\lambda \,d\lambda .\tag3.19
$$
When
$t\in {\Bbb C}_+$ with argument $\arg t=\theta
\in [0,\frac\pi 2[$,  say, we must assure that
$\operatorname{Re}\lambda t\to -\infty $ on the integral curve $\Cal C$. With $\frac\pi 2-\theta $ denoted $2\varphi $, this is assured
if $\lambda $ runs on a contour formed of the rays $\lambda =re^{\pm i\varphi }$,
connected near 0 by a circle of radius $\varepsilon '<\varepsilon $
passing {\it to the
right of} 0. Denote $\inf _{\lambda \in\Cal C}\operatorname{Re}\lambda
=c_1>0$.

For simplicity of notation we drop the $\varepsilon $-index in the
next calculations.

\example {Remark 3.4} The argument $\theta =\frac\pi 2-2\varphi $ of
$t$ runs in $[0,\frac\pi  2[\,$, when $\varphi $ runs in
$\,]0,\frac\pi 4]$. On this interval, $\sin\varphi \eg \sin(2\varphi
)=\cos\theta $, so they can be used interchangeably in our estimates.
\endexample

To find ${V'_M}(t)$ we can plug $Q'_M$ into the integral (3.19)  and study
the kernels, trying to get an estimate in terms of $|t|$. This cannot
quite be achieved with (3.19), but better estimates
are obtained if we first make use of the resolvent formula
$$
Q_\lambda = -\lambda ^{-1}+\lambda ^{-1}Q_\lambda P.
$$
This gives
$$
\aligned
V(t)&=\tfrac i {2\pi }\int_{\Cal C}e^{-t\lambda }(-\lambda
^{-1}+\lambda ^{-1}Q _\lambda P)\,d\lambda =\tfrac i {2\pi }\int_{\Cal
C}e^{-t\lambda }\lambda ^{-1}Q _\lambda P\,d\lambda,\\
\partial_tV(t)&=-\tfrac i {2\pi }\int_{\Cal C}e^{-t\lambda }Q _\lambda P\,d\lambda.
\endaligned\tag3.20
$$
Thus $\partial_t\Cal K_{V'_M}$ is the kernel of the integral of the  $M$-th
remainder $-(Q_\lambda P)'_M$ of $-Q _\lambda P$. We know that $\Cal K_{V'_M}$
vanishes at $t=0$, and want to show boundedness of the last integral
applied to the kernel of the $M$-th remainder.
Here (cf.\ also (3.3))
$$
Q_\lambda P=({\sum_{l<M}}Q_{-d-l}+Q'_M)P=
{\sum}_{l<M}Q_{-d-l}P+ \widetilde R_MQ _\lambda P={\sum}_{l<M}Q_{-d-l}P+ \widetilde R_MPQ _\lambda .
$$

The term $\widetilde R_MPQ _\lambda $ already belongs to the $M$-th
remainder $(Q_\lambda P)'_M$, and its kernel
is estimated by
$$
\|\widetilde R_MP\|_{\Cal L(L_2,H^{n+1})}(\sin\varphi |\lambda |)^{-1}
$$
where $\widetilde R_M P$ can be treated by another application of
Lemma 3.2.

We shall here need an estimate of
$L_2$-bounds in terms of symbol seminorms. Many variants are known,
and we use the following, found in Marschall [M87], Theorem 2.1.


\proclaim{Theorem 3.5}
Let $a \in S^0_{1,0}(\Bbb R^n \times \Bbb R^n)$ be such that for some
$C>0$, $N \in {\Bbb N}_{0}$ with $N>\frac{n}{2}$, and all $\alpha, \beta
\in {\Bbb N}_{0}^n$ with $0 \leq |\alpha|
\leq N$, $0 \leq |\beta|\leq 1$,
$$
\operatorname{sup}_{x,\xi } \ang{\xi}^{-|\alpha|} |D_x^\beta D_\xi^\alpha
a(x, \xi)| \leq C <
\infty .\tag3.21
$$
Then the associated operator $A = \operatorname{Op}(a)$ is bounded on
$L_2(\Bbb R^n)$, and $\|A\|_{{\Cal L}(L_2(\Bbb R^n))} \leg C$.
\endproclaim

The dependence of the operator norm on $C$ follows from an inspection
of the proof.


\proclaim{Proposition 3.6}
The symbol $\ttilde r=r_0(\widetilde
r_M,p)$ of $\ttilde R=\widetilde R_MP$ satisfies
$$
|D_x^\beta D_\xi ^\alpha \ttilde r| \leg
(\sin\varphi)^{-2M+1-|\alpha|-|\beta|}\ang\xi^{2d-M-|\alpha|}
\ang{\xi ,\mu }^{-d}.\tag3.22
$$
Moreover, for $M\ge n+1+2d$,
$$
\aligned
\|\ttilde R \|_{\Cal L(L_2, H^{n+1})}&\leg (\sin\varphi )^{-2M-3n/2-2
}\ang\lambda ^{-1},\text{ and hence}\\
\|\widetilde R_MPQ_\lambda  \|_{\Cal L(L_2, H^{n+1})}&\leg (\sin\varphi )^{-2M-3n/2-3
}\ang\lambda ^{-1}|\lambda |^{-1}.
\endaligned\tag3.23
$$

\endproclaim

\demo{Proof}
An application of  Lemma 3.2 $1^\circ$ to the composition $\ttilde
R=\widetilde R_MP$ shows (3.22).
 Next,
$$
\|\ttilde R\|_{\Cal L(L_2, H^{n+1})}=\Bigl({\sum}_{|\gamma |\le n+1}\|D^\gamma \circ\ttilde R \|^2_{\Cal
L(L_2)}\Bigr)^\frac12=\Bigl({\sum}_{|\gamma |\le n+1}\|\ttilde R^\gamma  \|^2_{\Cal
L(L_2)}\Bigr)^\frac12,
$$
where an application of Lemma 3.2 $3^\circ$
shows  that for $|\gamma |\le n+1$,
$$
|D_x^\beta D_\xi ^\alpha \ttilde r^\gamma (x,\xi ,\lambda )|\leg
(\sin\varphi )^{-2M+1-(n+1)-|\beta |-|\alpha |}\ang\xi ^{2d-M+(n+1)-|\alpha |}\ang\lambda ^{-1}.
$$
According to Theorem 3.5, $\ttilde R^\gamma $ is bounded in $L_2$
when $M\ge 2d+n+1$, with operator norm estimated by the seminorms of the
symbol derivatives with $|\alpha |\le N$, $|\beta |\le 1$ for some  $N>
n/2$. Hence the norm is estimated by
$$
(\sin\varphi )^{-2M-n-1-(n/2+\delta ) }\ang\lambda ^{-1}\le(\sin\varphi )^{-2M-3n/2-2 }\ang\lambda ^{-1},
$$
where $\delta $ equals $\frac12$ or 1. Summing over $|\gamma |\le
n+1$, we obtain the first line in (3.23), and the second follows,
since $\|Q_\lambda \|_{\Cal L(L_2)}\le (\sin\varphi \,|\lambda
|)^{-1}$.
\qed
\enddemo




For the term ${\sum}_{l<M}Q_{-d-l}P$ we can apply Lemma 3.2 to the
compositions, much as in Proposition 3.3.

\proclaim{Proposition 3.7} For the $M$-th remainder $\widetilde
R'_M=\bigl({\sum}_{l<M}Q_{-d-l}P\bigr)'_M$ of ${\sum}_{l<M}Q_{-d-l}P$,
the symbol of ${\sum}_{l<M}Q_{-d-l}P$ is in
local coordinates
% HG eliminated a second "is"
a sum of homogeneous terms of degree
$0,-1,\dots,-M+1$ plus a remainder $\widetilde r'_M$ of the form
$$
\widetilde r'_M=q_{-d}p'_M+\ttilde r_M,\tag3.24
$$
where
$$
|D_x^\beta D_\xi^\alpha \ttilde r_{{M}}(x,\xi ,\lambda )|\leg
(\sin\varphi)^{- 2M +
1-|\alpha|-|\beta|}\ang\xi^{2d-{M}-|\alpha|}\ang{\xi ,\mu }^{-2d},
\tag3.25
$$
\endproclaim

\demo{Proof} We shall use estimates from the proof of Proposition
Y. Recall from (3.13) that $\tilde p$ only differs from $p$ in the
principal term. We have
$$
q_{-d-l}\circ p_{k-d}={\sum}_{k+l+|\alpha| < {M}} \tfrac{1}{\alpha!}\ D_\xi^\alpha
q_{-d-l} \partial_x^\alpha  p_{d-k} + r_{{M}-k-l}(q_{-d-l},
 p_{d-k}),
$$
where the terms in the sum are homogeneous of degrees $>-M$, and the
remainders are estimated as in (3.16) for $l>0$, (3.17) for $l=0$,
contributing to (3.25). It
remains to consider
$$
{\sum}_{l<M}q_{-d-l}\circ p'_M={\sum}_{l<M}r_0(q_{-d-l}, p'_M).\tag 3.26
$$
For $l>0$, $r_0(q_{-d-l}, p'_M)$ satisfies estimates as in
$(3.25)$. For $l=0$,
$$
r_0(q_{-d}, p'_M)=q_{-d}p'_M+r_1(q_{-d}, p'_M),
$$
where $r_1(q_{-d}, p'_M)$ satisfies estimates as in (3.25) by Lemma 3.2
$2^\circ$. We collect the terms in
$$
\ttilde r_M={\sum}_{k+l<M}r_{{M}-k-l}(q_{-d-l}, p_{d-k})+{\sum}_{0<l<M}r_0(q_{-d-l}, p'_M)+
r_1(q_{-d}, p'_M),
$$
then (3.24) holds with (3.25).\qed
\enddemo

Then we can finally show the estimate of the remainder kernel:

\proclaim{Theorem 3.8} Let $P$ be selfadjoint strongly elliptic of order
$d>0$ on $M$, with $\gamma (P)\ge 0$. The remainder kernel $\Cal
K_{V'_M}$ satisfies for $M\ge n+1+2d$, $t\in {\Bbb C}_+$ with $\arg
t=\frac\pi 2-2\varphi \in [0,\frac\pi 2[\,$,
$$
|\Cal K_{V'_M}(x,y,t)|\leg (\sin\varphi
)^{-2M-3n/2-3}e^{-c'\operatorname{Re}t}\operatorname{min} \{|t|, 1\}
%HG: inserted "min" according to modification in proof
,\tag 3.26
$$
where $c'>0$ if $\gamma (P)>0$,  $c'=0$ if $\gamma (P)=0$.
\endproclaim

\demo{Proof} If $\gamma (P)>0$ we use the preceding estimates directly
to analyse
$$
\partial_t\Cal K_{V'_M}=-\tfrac i {2\pi }\int_{\Cal C}e^{-t\lambda
}\Cal K_{(Q _\lambda P)'_M}\,d\lambda.
$$
Here $(Q_\lambda P)'_M=\widetilde R_MPQ_\lambda +\widetilde R'_M$, cf.\
Propositions Z and V. They contribute as follows:

From $\widetilde R_MPQ_\lambda$ we get using (3.23)
$$
\aligned
|\tfrac i {2\pi }&\int_{\Cal C}e^{-t\lambda
}\Cal K_{(\widetilde R_MPQ _\lambda )'_M}\,d\lambda|\\
&\leg (\sin\varphi
)^{-2M-3n/2-3}(\int_\varepsilon ^\infty e^{-\sin\varphi |t|r}\ang
r^{-1}r^{-1}\,dr +\int_{|\varphi '|\le \varphi
}e^{-c_1\operatorname{Re}t}\ang\varepsilon ^{-1}|\varepsilon
|^{-1}\,d\varphi ')\\
&\leg (\sin\varphi
)^{-2M-3n/2-3}e^{-c'\operatorname{Re}t},
\endaligned\tag3.28$$
 for some $c'>0$.

From $\tilde R'_M$ we get two terms. One is the operator with symbol
$\ttilde r_M$, which satisfies the estimates (3.25). Since the kernel
equals $\Cal F^{-1}_{\xi \to z}\ttilde r_M|_{z=x-y}$, we find for
$M\ge n+1+2d$ that
$$
|\Cal K_{\ttilde R_M}(x,y,\lambda )|\leg (\sin\varphi )^{-2M+1}\ang\lambda ^{-2}.
$$
Insertion in the integral as in (3.28) gives a bound $(\sin\varphi
)^{-2M+1}e^{-c'\operatorname{Re}t}$.


To find the contribution from  $q_{-d}p'_M$, we first perform the integration:
$$
\tfrac i {2\pi }\int_{\Cal C}e^{-t\lambda
}(p^0-\lambda )^{-1} p'_M\,d\lambda
=e^{-tp^0}p'_M.
$$
% HG inserted a missing "p'_M" on the left hand side
Here we can borrow an estimate from Lemma 3.9 below. By (3.33),
$$
\aligned
|D_x^\beta D_\xi ^\alpha (e^{-tp^0}p'_M)|&\leg (\sin \varphi)^{-M
-|\alpha|-|\beta|}
\ang\xi ^{-M-|\alpha|}e^{-c\operatorname{Re}t \ang \xi ^d}\\
&\leg (\sin \varphi)^{-M
-|\alpha|-|\beta|}
\ang\xi ^{-M-|\alpha|}e^{-c'\operatorname{Re}t }, \text{ hence}\\
|\Cal K_{\Op(e^{-p^0t}p'_M)}|&\leg(\sin \varphi)^{-M
}e^{-c'\operatorname{Re}t}\text{ when }M\ge n+1
.\endaligned
$$


Since the latter estimates are dominated
by that from $\widetilde R_MPQ_\lambda $, we conclude
that
$|\partial_t\Cal K_{V'_M}(x,y,t)|$ is estimated as in (3.28). Then an
integration with respect to $t$ using that $\Cal K_{V'_M}(x,y,0)=0$
shows (3.26).

In the case where $\gamma (P)=0$, the above considerations will be
valid for $V^\varepsilon (t)$ as in (3.19). We then have to add
$(1-e^{-\varepsilon t})\Pi _0$, which has a smooth kernel bounded by
$\operatorname{min} \{|t|, 1\}$
%HG: taken care of [times what?]
and we reach the conclusion in the theorem.
\qed
\enddemo

Now we turn to the contribution from the homogeneous terms.

\proclaim{Lemma 3.9}
 Let $M\in\Bbb N$, let
$\sigma_1,\dots,\sigma_{M}$ be nonnegative integers with
 $$
\sigma=\sigma_1+\cdots+\sigma_{M}\ge 1,\tag3.29
 $$
 and let $f(x,\xi,\lambda)$ be a (matrix-formed) symbol
%of order $k\in\Bbb R$
of the form
 $$
f(x,\xi,\lambda)=f_1(p_d-\lambda)^{-\sigma_1}f_2(p_d-\lambda)^{-\sigma_2}
\cdots(p_d-\lambda)^{-\sigma_M}f_{M+1},\tag3.30
 $$
 where the $f_j(x,\xi)$ are $\psi $do symbols of order $s_j\in\Bbb
R$, homogeneous for $|\xi|\ge1$.
% and uniformly estimated in $x$.
 Denote $s_1+\cdots+s_{M+1}=s$, then the order of $f$ is $k=s-\sigma d$. Let
$F_\lambda=\Op(f(x,\xi,\lambda))$ on $\Bbb
R^n$, and let $E(t)$ be the operator family defined from $F_\lambda$
for $\operatorname{Re}t>0$ by
 $$
E(t)=\tfrac i{2\pi}\int_{\Cal C}e^{-t\lambda}F_\lambda
\,d\lambda,\tag3.31
 $$
when $t=e^{i(\frac\pi 2-2\varphi )}|t|$.
 Then $E(t)=\Op(e(x,t,\xi))$, where the symbol
 $$
e(x,t,\xi)=\tfrac
i{2\pi}\int_{\Cal C}e^{-t\lambda}f(x,\xi,\lambda)\,d\lambda\tag3.32
 $$
satisfies:
%For $t$
% it satisfies %the quasi-homogeneity condition and
%the estimates
 $$
\alignedat2
&\text{\rm (i)} &&\quad
e(x,s^{-d}t,s\xi)=s^{d+k}e(x,t,\xi)\text{ for }
|\xi|\ge 1,\;s\ge 1,\\
& \text{\rm (ii)} &&\quad
|D_x^\beta D_\xi^\alpha e(x,t,\xi)|\leg (\sin \varphi)^{-\sigma
-|\alpha|-|\beta|}
\ang\xi ^{d+k-|\alpha|}e^{-c\operatorname{Re}t \ang \xi ^d}.
\endalignedat\tag 3.33
 $$
% The optimal estimate would be (\sin \varphi)^{-\sigma
%-|\alpha|-|\beta|+1}, but I do not know how to show this for systems.
%with a fixed $c>0$.
The kernel of $E(t)$ satisfies for $d+k>- n$
$$|\Cal K_E(x,y,t)| \leg (\sin \varphi)^{-\sigma-(d+k+n)/d}
e^{-c'\operatorname{Re}t} |t|^{-(d+k+n)/d} ,\tag 3.34$$
with $c'>0$.

If $\sigma \geq 2$,
$$|D_x^\beta D_\xi^\alpha e(x,t,\xi)| \leg  (\sin
\varphi)^{-\sigma-|\alpha|-|\beta|} |t| \ang\xi
^{2d+k-|\alpha|}e^{-c'\operatorname{Re}t \ang \xi ^d} \ . \tag 3.35$$
In this case, the kernel satisfies for $d+k\leq - n$
$$|\Cal K_E(x,y,t)| \leg (\sin \varphi)^{-\sigma}
e^{-c'\operatorname{Re}t} \cases
|t|\,(
|\log \operatorname{Re}t |+1)\text{ if
}d+k= -n,\\
 |t|\text{ if
}d+k<- n.
\endcases \tag 3.36$$

\endproclaim

\demo{Proof}
As in [G96], Lemma 4.2.3, we can pass the operator definition through
the integral.
To estimate $e$, we first consider $|\xi| \leq 1$. We use the residue
theorem and that $p^0$ is selfadjoint to obtain
$$|e(t,x,\xi)| = |\tfrac i{2\pi}\int_{\Cal C}e^{-t\lambda}f(x,\xi,\lambda)\,d\lambda| \leg (1+|t|^{\sigma-1})
e^{-c\operatorname{Re}t}\ .$$
Here, $c = \gamma(p^0(x,\xi))$.

For $|\xi| \geq 1$, we replace $\Cal C$ by a closed,
homogeneous curve $\Cal C_{c,C}$ around the spectrum of
$p^0(x,\xi)$. $\Cal C_{c,C}$ coincides with
$\Cal C$ on a annulus of inner radius $c |\xi|^{d}$ and
outer radius $C |\xi|^{d}$, is closed by the segments of the boundary of
this annulus which lie to the right of $\Cal C$. Then by
homogeneity
$$|e(t,x,\xi)|= |\tfrac i{2\pi}\int_{\Cal C_{c,C}}e^{-t\lambda}f(x,\xi,\lambda)\,d\lambda| \leg (\sin
\varphi)^{-\sigma} \ang \xi ^{d} \ang \xi ^{k} e^{-\tfrac c 2
\operatorname{Re}t |\xi|^d}\ .$$
Combining the two estimates, we conclude (3.33) for $\alpha = \beta = 0$.
The derivatives $D_x^\beta D_\xi^\alpha e(x,t,\xi)$ are of a similar form,
with $k$ and $\sigma$ replaced by $k-|\alpha|$
resp.~$\sigma+|\alpha|+|\beta|$.

To show (3.34) for $d+k>- n$, we estimate $\Cal K_E$ by comparing $e$ with its
homogeneous extension $e^h$:
$$\Cal K_E(x,y,t) = \int_{\Bbb R^n} e^{i (x -y)\cdot \xi} e^h(x,t,\xi)\,
\d\xi + \int_{|\xi|\leq 1} e^{i (x -y)\cdot \xi} (e-e^h)\, \d\xi \ .$$
Using (3.33) and a homogeneous variant,
$$
\multline
|\Cal K_E(x,y,t)| \leg (\sin \varphi)^{-\sigma} e^{-c_1
\operatorname{Re}t} \int_{\Bbb R^n} e^{-c_2 \operatorname{Re}t
|\xi|^{d}} |\xi|^{d+k}\, \d\xi \\
+ (\sin \varphi)^{-\sigma} e^{-c_1
\operatorname{Re}t} \int_{|\xi|\leq 1} e^{-c_2 \operatorname{Re}t
|\xi| ^{d}}(\ang \xi ^{d+k} + |\xi|^{d+k})\, \d\xi \ .
\endmultline
$$
The first integral is $\eg (\operatorname{Re}t)^{-(d+k+n)/d} \eg (\sin \varphi)^{-(d+k+n)/d} |t|^{-(d+k+n)/d}$, while the second remains
bounded as $|t|\to 0$.

Now consider the case where $\sigma \geq 2$. As $|f|\leg \ang\lambda
^{-2}$ away from $\Bbb R_+$, the integral converges uniformly in $t\geq
0$. We may deform $\Cal C$ to a closed curve in the left
half-plane, where $f$ is holomorphic, to conclude $e(x,0,\xi) = 0$. Also,
using $(-\lambda)(p_d-\lambda)^{-1} = 1-p_d(p_d-\lambda)^{-1}$,
$$
\partial_t e (x,t,\xi)= \tfrac
i{2\pi}\int_{\Cal C}e^{-t\lambda}(-\lambda)
f(x,\xi,\lambda)\,d\lambda
$$
can be expressed in terms of $e$ and a second term of the same form, with
one of the $s_j$ replaced by $s_j+d$. By (3.33)
$$
|\partial_t e(x,t,\xi)| \leg (\sin \varphi)^{-\sigma} \ang\xi
^{2d+k}e^{-c\operatorname{Re}t \ang \xi ^d}
$$
and hence, since the value at $t=0$ is 0,
$$|e(x,t,\xi)| \leg  (\sin \varphi)^{-\sigma} |t| \ang\xi
^{2d+k}e^{-c\operatorname{Re}t \ang \xi ^d} \ .$$
This shows (3.35) for $\alpha = \beta = 0$. The proof for $D_x^\beta
D_\xi^\alpha e(x,t,\xi)$ is analogous. The estimate (3.36) is obtained similarly to (3.34), using (3.35) instead of (3.33).
\qed
\enddemo


\proclaim{Theorem 3.10}
$1^\circ$ In local coordinates the kernel terms satisfy for some $c'>0$:
$$
|\Cal K_{V_{-d-l}}(x,y,t)|\leg (\sin \varphi)^{-2l}
e^{-c'\operatorname{Re}t}\cases (\sin\varphi)^{(l-n)/d}|t|^{(l-n)/d}\text{ if
}d-l>- n, \\  |t|\,(
|\log \operatorname{Re}t |+1)\text{ if
}d-l= -n,\\
 |t|\text{ if
}d-l<- n.
\endcases\tag3.37
$$
$2^\circ$ Let $N = \operatorname{max} \{\tfrac n d , \tfrac{7n}{2} + 4 d + 7 \}$, Then the full kernel satisfies
$$
|\Cal K_{V}(x,y,t)|\leg \sin(\varphi)^{-N}
e^{-c'\operatorname{Re}t} |t|^{-n/d} \ .
$$
For a given $c_0>0$ we can modify $p^0$ to satisfy
$\inf_{x,\xi }\gamma (p^0(x,\xi ))\ge c_0$; then $c'$ can be any
number in
$\,]0,c_0[\,$.
\endproclaim
\demo{Proof}
$1^\circ$. For $l \geq 1$, the assertion follows from Lemma 3.9,
(3.34) resp.~(3.36).

On the other hand, we explicitly compute for $l=0$
$$
\aligned
|\Cal K_{V_{-d}}&(x,y,t)| = (2 \pi)^{-n}|\int_{\Bbb R^n} e^{i (x -y)\cdot \xi}
e^{-tp^0(x,\xi)}\, \d\xi |\\
& \leg e^{-c_1 \operatorname{Re}t} \big(\int_{\Bbb R^n} e^{-c_2
\operatorname{Re}t |p^0_h(x,\xi)|}\, \d\xi + \int_{|\xi|\leq 1} (e^{-c_2 \operatorname{Re}t |p^0(x,\xi)|}-e^{-c_2
\operatorname{Re}t |p^0_h(x,\xi)|})\, \d\xi\big)\\
& \leg e^{-c_1 \operatorname{Re}t} \big(\int_{\Bbb R^n} e^{-\operatorname{Re}t |\xi|^d}\, \d\xi
+ 1\big) \leg e^{-c' \operatorname{Re}t} (\operatorname{Re}t)^{-n/d} \ .
\endaligned$$
The assertion follows, since $\operatorname{Re}t \eg |t|\sin \varphi$.

$2^\circ$. We choose $M=n+1+2d$ in Theorem 3.8 and add $\Cal K_{V_{-d-l}}$ for $0\leq l<M$. The most singular terms dominate.
%[Can we do better using different estimates in $|t|>(\sin\varphi)^\mu$ and $|t|<(\sin\varphi)^\mu$?]

See Section 2 for how to obtain the allowed range of $c'$.
\qed
\enddemo



\proclaim{Theorem 3.11}
$1^\circ$ In local coordinates,
$\Cal K_{V_{-d}}$ satisfies for some $c'>0$:
$$
|\Cal K_{V_{-d}}(x,y,t)|\leg (\sin\varphi)^{-n-1} e^{-c'\operatorname{Re}t}
|t|\, |x-y|^{-d-n}.\tag 3.38
$$
For $l\ge 1$, the kernels $\Cal K_{V_{-d-l}}$ satisfy
$$
\multline
|\Cal K_{V_{-d-l}}(x,y,t)|\leg \\
(\sin\varphi)^{-[n-l+1]_+-2l}
e^{-c'\operatorname{Re}t}\cases |t|\,|x-y|^{l-d-n}\text{ if }d-l>- n,\\
 |t|\,(|\log |x-y||+1)\text{ if }d-l= -n ,\\
 |t|\text{ if }d-l< -n .\endcases
\endmultline
\tag 3.39
$$
$2^\circ$ Let $N = \tfrac{7n}{2} + 4 d + 7$. Then the full kernel satisfies
$$
|\Cal K_{V}(x,y,t)|\leg \sin(\varphi)^{-N}
e^{-c'\operatorname{Re}t} |t|\,|x-y|^{-d-n} \ .
$$
If $\gamma (P)>0$,  we  modify $p^0$ to satisfy
$\inf_{x,\xi }\gamma (p^0(x,\xi ))\ge \gamma (P)$, then $c'$ can be any
number in $\,]0,\gamma (P)[\,$.
\endproclaim
\demo{Proof}
$1^\circ$. For $l\geq 1$, we obtain from Lemma 3.9, (3.35), that
$$|D_\xi^\alpha v_{-d-l}(x,t,\xi)| \leg (\sin\varphi)^{-2l-|\alpha|} \ang
\xi ^{d-l-|\alpha|} |t| e^{-c'\operatorname{Re}t}\ .$$
The estimate (3.39) then follows from the kernel estimates in Proposition 2.2.

For $l=0$, $v_{-d}(x,t,\xi) = e^{-tp^0(x,\xi)}$, and we obtain (3.38) as
in the proof of Theorem 2.4.

$2^\circ$. We choose $M=n+1+2d$ in Theorem 3.8 and add $\Cal K_{V_{-d-l}}$ for $0\leq l<M$. The most singular terms dominate.
\qed
\enddemo

The estimates hold also when $\varphi \in [0,\frac\pi 4[\, $ is
replaced by $\varphi \in \,]-\frac\pi 4,0]$ and $\sin\varphi $ is
replaced by $|\sin\varphi |$. Note that for $t$ with $\arg t=\theta \in \,]-\frac\pi 2,\frac\pi
2[\,$, $\cos \theta =|\sin 2\varphi |\eg |\sin\varphi |$, when $\theta
=\pm(\frac\pi  2-2|\varphi |)$ (cf.\ Remark 3.4).
Then we can formulate the final result obtained by
combining Theorem 3.10 and Theorem 3.11 as in the proof of
Theorem 2.5, as follows:

\proclaim{Theorem 3.12}
Let $P$ be selfadjoint strongly elliptic of order
$d>0$ on $M$, with $\gamma (P)\ge 0$. The heat kernel $\Cal
K_{V}$ satisfies for $t\in {\Bbb C}_+$ with $\arg
t=\theta  \in \,]-\frac\pi 2,\frac\pi 2[\,$ the Poisson estimate,
where $N=\operatorname{max} \{\tfrac n d , \tfrac{7n}{2} + 4 d + 7
\}$:
$$
|\Cal K_{V}(x,y,t)|\leg (\cos \theta  )^{-N}
e^{-\gamma(P)\operatorname{Re}t}\,\frac {|t|}{(d(x,y)+|t|^{1/d})^{d}}((d(x,y)+|t|^{1/d})^{-n}+1).\tag3.40
$$

\endproclaim

\demo{Proof}  In the region where $|t|^{1/d} \leq d(x,y)$, $d(x,y) \eg
d(x,y)+|t|^{1/d}$, and the asserted estimate follows from Theorem 3.11.
In the region where $|t|^{1/d} \geq d(x,y)$, we use $|t|^{1/d} \eg
d(x,y)+|t|^{1/d}$ and Theorem 3.10.\qed
\enddemo

\example{Remark 3.13}  The $+7$ in $N$ partly stems from repeated rounding up to the nearest
integer. It may be reduced by choosing $M+2d$ closer to $n$ and using
a version of Theorem 3.5 with a non-integer number of derivatives, or
sharper versions for negative orders.

Our result applies in particular to the Dirichlet-to-Neumann operator.
For this operator ter Elst and Ouhabaz [EO13] have estimates in terms
of $-N$-th powers of  $\cos\theta $, where the dimension $n$ enters nonlinearly in $N$.
\endexample

\head 4. Kernels of heat semigroups for
perturbations of
fractional Laplacians and the Dirichlet-to-Neumann operator \endhead

This section complements the general upper bounds from Section 2
with lower estimates in the case of fractional powers of the Laplacian and the
Dirichlet-to-Neumann operator.

Let $\Delta $ be the (nonnegative) Laplace-Beltrami operator on the closed,
compact Riemannian $n$-dimensional manifold $M$; it defines a
selfadjoint nonnegative operator on $L_2(M)$, also denoted $\Delta$. In
this case,
$\Delta^{d/2} \,$ is an elliptic pseudodifferential operator of order $d$
on $M$, with positive principal symbol $|\xi|^d$, defining a selfadjoint
nonnegative
operator on $L_2(M)$; it generates a holomorphic semigroup $V^d(t) =
e^{-t\Delta^{d/2}
\,}$ with $C^\infty $-kernel for $t>0$,
$$
{\Cal K}_{V^d}(x,y,t)=\ang{\delta _x,V^d(t)\delta _y}.
$$



The semigroups $e^{-t\Delta }$ and $V^d(t)$ are related by subordination
formulas. For $d=1$, they assume a simple form:

\proclaim{Lemma 4.1}
Let $\lambda \ge 0$. One has for $t\ge 0$:
$$
 e^{-t\sqrt{\lambda }} = \frac{1}{2\sqrt{\pi }}\int_0^\infty e^{-s \lambda} t e^{-\frac{t^2}{4s}}
s^{-\frac32}ds\,.\tag4.1
$$
\endproclaim

\demo{Proof}  Let $\alpha =t\sqrt{\lambda }\,/2$ and let
$x=\frac{t}{2} s^{-\frac12}$; then $dx=-\frac{t}{4} s^{-\frac32}ds$, and
equation (4.1)
is turned into
$$
\sqrt{\pi }\,e^{-2\alpha }=\int_0^\infty e^{-x^2-\frac{\alpha ^2}{x^2}}2\,dx
.\tag4.2$$
To show this, note that the left-hand side $I(\alpha )$ satisfies
$
I(\alpha )\in C^1(\rp)$, $ \lim_{\alpha \to 0+}I(\alpha )=\sqrt{\pi }$,
and for $\alpha >0$ (with $y=\alpha x^{-1}$, $dy=-\alpha x^{-2}dx$):
$$
\partial_\alpha I(\alpha )=\int_0^\infty e^{-x^2-\frac{\alpha
^2}{x^2}}(-4\alpha  )x^{-2}\,dx=-2\int_0^\infty e^{-\frac{\alpha
^2}{y^2}-y^2}2\,dy
=-2I(\alpha ).
$$
Thus $I(\alpha )=ce^{-2\alpha }$ with $c=\sqrt{\pi }\,$.
\qed
\enddemo

By Zolotarev [Z86] (see also Grigor'yan [G03]),  there
exists for any $0<d<2$ a
non-negative function
$\eta_t^d(s)$ such that
$$
e^{-t\lambda^{d/2}} = \int_0^\infty e^{-s\lambda}\ \eta_t^d(s)\, ds  .\tag4.3
$$
Here $\eta_t^d$ has the following properties
$$
\align
\eta_t^d(s) &= t^{-2/d}\eta_1^d(\frac{s}{t^{2/d}}) \qquad (s,t>0)\ ,
\tag4.4
%\label{etaprop1}
\\
\eta_t^d(s) &\leg t s^{-1-\frac{d}{2}} \qquad (s,t>0)\ ,\tag4.5
%\label{etaprop2}
\\
\eta_t^d(s) & \eg t s^{-1-\frac{d}{2}} \qquad (s\geq t^{2/d}>0) \
.\tag4.6
%\label{etaprop3}
\endalign
$$
By an application of the spectral theorem, we obtain for all $t>0$,
$$
V^d(t)f = e^{-t\Delta^{d/2}} f = \int_0^\infty e^{-\tau\Delta}f \
\eta_t^d(\tau)\ d\tau \ , \text{ for
all }f\in H^s(M).\tag4.7
$$
In view of (4.7), it holds that
$$%\aligned
\ang{\delta _x, V^d(t)\delta _y}=
\ang{\delta _x,\int_0^\infty e^{-\tau \Delta} \delta_y \, \eta_t^d(\tau)\, d\tau}
=\int_0^\infty \ang{\delta _x,e^{-\tau\Delta}\delta _y}\, \eta_t^d(\tau) \, d\tau,
%\endaligned
$$
resulting in an identity for
the kernels: For all $t>0$,
$$
{\Cal K}_{V^d \,}(x,y,t) = \int_0^\infty {\Cal K}_{e^{-\tau\Delta
}}(x,y)\ \eta_t^d(\tau) \,d\tau \ , \text{ for }(x,y)\in M\times M.\tag4.8
$$
Using this formula, we can deduce upper and lower estimates for ${\Cal
K}_{V^d\,}$ from those known for ${\Cal K}_{e^{-\tau
{\Delta  }}}$.
The following upper and lower estimates are well-known (see e.g.\ L.\
Saloff-Coste [S10]):
$$
\frac{c_1}{{\Cal V}(x,\sqrt\tau \,)}e^{-C_1\frac{d(x,y)^2}\tau }
\le {\Cal K}_{e^{-\tau \Delta }}(x,y)\le
\frac{c_2}{\Cal{V}(x,\sqrt\tau \,)}e^{-C_2\frac{d(x,y)^2}\tau }.\tag4.9
$$
Here $\Cal V(x,r)$ denotes the volume of a ball of radius $r$ around
$x$. For a closed compact $n$-dimensional manifold $M$, $\Cal V(x,r)\eg r^n$
for small $r$, and $\Cal V(x,r)$ equals the volume of the connected
component containing $x$ when $r\ge \operatorname{diam} M$. Hence
$$
\Cal V(x,\sqrt\tau \,)^{-1}\eg (\tau ^{n/2})^{-1}+1.\tag4.10
$$
%\enddemo

\proclaim{Theorem 4.2}
 Let $0<d<2$. The kernel of the semigroup
$V^d(t)=e^{-t\Delta^{d/2}}$ satisfies for $t\ge 0$:
$$
\Cal{K}_{e^{-t\Delta^{d/2}}}(x,y) \eg
\frac{t}{(d(x,y)+t^{1/d})^d}\left((d(x,y)+t^{1/d})^{-n}+1\right)\ . \tag4.11
$$
\endproclaim

\demo{Proof} The upper estimate follows already from Corollary
1.6. The following proof moreover extends to give the lower estimate.
Inserting the
heat kernel bounds (4.9),
(4.10) into (4.8) and using (4.5), we find
$$
\align
\Cal{K}_{V^d}(x,y,t) & \leg \int_0^\infty (\tau^{-n/2} + 1)\
\eta_t^d(\tau) \ e^{-C\frac{d(x,y)^2}{\tau}} d\tau \\
& \leg t \int_0^\infty \tau^{-n/2} \ \tau^{-1-\frac{d}{2}}\
e^{-C\frac{d(x,y)^2}{\tau}} d\tau + t \int_0^\infty  \tau^{-1-\frac{d}{2}}\ e^{-C
\frac{d(x,y)^2}{\tau}} d\tau\ .
\endalign
$$
By a change of variables $\tau \mapsto C d(x,y)^2 \tau$, the first term equals
$$t (Cd(x,y)^2)^{-\frac{d+n}{2}} \int_0^\infty \tau^{-\frac{n+d}{2}-1}\
e^{-1/\tau}\ d\tau \eg \frac{t}{d(x,y)^{n+d}}\ .\tag4.12$$
Similarly, the second term is
$$t (Cd(x,y)^2)^{-\frac{d}{2}} \int_0^\infty \tau^{-\frac{d}{2}-1}\ e^{-1/\tau}\
d\tau \eg \frac{t}{d(x,y)^{d}}\ ,$$
and altogether,
$${\Cal K}_{V^d}(x,y,t)  \leg
\frac{t}{d(x,y)^{d}}\left(d(x,y)^{-n}+1\right) \ .$$
On the other hand, using the uniform bound ${\Cal K}_{e^{-\tau\Delta}}(x,y) \leg
\tau^{-n/2}+1$ and (4.4),
we obtain
$$
\align
{\Cal K}_{V^d}(x,y,t) & \leg \int_0^\infty (\tau^{-n/2} + 1)\
\eta_t^d(\tau)\ d\tau
 = \int_0^\infty (\tau^{-n/2}+1) \ \eta_1^d\left(\frac{\tau}{t^{2/d}}\right)\
t^{-2/d} \ d\tau \\
& = \int_0^\infty (t^{-n/d}\tau^{-n/2}+1) \ \eta_1^d(\tau)\ d\tau
 \eg t^{-n/d}+1 \ .
\endalign
$$
Thus
$${\Cal K}_{V^d}(x,y,t) \leg \min\Bigl\{t^{-n/d}+1,
\frac{t}{d(x,y)^{d}}\left(d(x,y)^{-n}+1\right)\Bigr\} \ .$$
If $t^{1/d}\geq d(x,y)$,
$$t^{-n/d}\leg t^{-n/d}\Bigl(\frac{d(x,y)}{t^{1/d}}+1\Bigr)^{-n-d} =
t(d(x,y)+t^{1/d})^{-n-d}$$
and
$$1 \leg \Bigl(\frac{d(x,y)}{t^{1/d}}+1\Bigr)^{-d} =
t(d(x,y)+t^{1/d})^{-d} \ .$$
On the other hand, for $t^{1/d}\leq d(x,y)$ we have $d(x,y) \eg
d(x,y)+t^{1/d}$ and hence
$$\frac{t}{d(x,y)^{d}}\bigl(d(x,y)^{-n}+1\bigr)\leg
\frac{t}{(d(x,y)+t^{1/d})^{d}}\left((d(x,y)+t^{1/d})^{-n}+1\right) \ .$$
This shows ``$\leg$'' in (4.11).

To show the opposite inequality in (4.11), note that the integrand in (4.8)
%\eqref{subord}
is
non-negative, and (4.9), (4.10) imply
$${\Cal K}_{V^d}(x,y,t) = \int_0^\infty
{\Cal K}_{e^{-\tau\Delta}}(x,y)\ \eta_t^d(\tau)\ d\tau \geg \int_\alpha^\infty (\tau^{-n/2} +
1)\ \eta_t^d(\tau) \ e^{-C\frac{d(x,y)^2}{\tau}} d\tau $$
for $\alpha = \max\{ t^{2/d}, d(x,y)^2\}$. Now, for $\tau \geq d(x,y)^2$,
$e^{-C\frac{d(x,y)^2}{\tau}} \geq e^{-C}$. Then by
(4.6),
%\eqref{etaprop3}
$$
\align
{\Cal K}_{V^d}(x,y,t) & \geg \int_\alpha^\infty (\tau^{-n/2} + 1)\ t
\tau^{1-\frac{1}{2}} \ d\tau
  \eg t \bigl(\alpha^{-\frac{n+d}{2}}+\alpha^{-\frac{d}{2}}\bigr) \\
&  = \min\bigl\{t^{-n/d}, t d(x,y)^{-n-d}\bigr\} + \min\bigl\{1, td(x,y)^{-d}\bigr\}\\
& \geq t (d(x,y)+t^{1/d})^{-n-d}+t(d(x,y)+t^{1/d})^{-d} \ .\quad\square
\endalign
$$

\enddemo

For $d=1$ this complies well with the explicit kernel
formula (2.3) for the Poisson operator solving the Dirichlet problem
for the Laplacian on ${\Bbb R}^{n+1}_+$.


We also consider the case where $M$ is the boundary of  a compact
$(n+1)$-dimensional Riemannian manifold $\widetilde M$
with boundary. With $\Delta  $ denoting the nonnegative Laplace-Beltrami
operator on $M$, we shall compare  ${\Cal K}_{e^{-t\sqrt{\Delta  }\,}}$
with the kernel of the semigroup generated by the (nonnegative)
Dirichlet-to-Neumann
operator $P_{DN}$ on $M$. $P_{DN}$ is the operator mapping $u$ to the
normal derivative
$\partial_\nu \widetilde u$, where $\widetilde u$ is the harmonic
function on $\widetilde M$ with boundary value $u$.
It is known (cf.\ [G71]) that $P_{DN}$ is an elliptic
pseudodifferential operator of order 1 on $M$ with the same principal
symbol as $\sqrt{\Delta  }\,$.



Since $\Delta ^{d/2}$ is a classical strongly elliptic $\psi $do of
order $d$,
%Theorem 2.5 and its corollary apply
%to it to show an upper estimate similar
%to that in (4.11). Moreover,
Theorem 2.5 applies to all operators of
the form $P=\Delta ^{d/2}+P'$ with $P'$ classical of order
$d-1$, giving upper estimates of the absolute value of the
kernels; note that no selfadjointness is required. For such operators
we can also show lower estimates.


\proclaim{Theorem 4.3} Let $d\in \,]0,2[\,$ and let $P$ be a classical $\psi
$do of order $d$ with the same principal
symbol as $\Delta^{d/2}$.
Then the kernel of $V(t)=e^{-tP}$ satisfies for $t\ge 0$:
$$
|\Cal K_{V}(x,y,t)|\leg  t\,\bigl( (d(x,y)+t^{1/d})^{-n-d}+(d(x,y)+t^{1/d})^{-d}\bigr)
+e^{-c_1t}t\,(d(x,y)+t^{1/d})^{1-n-d},\tag4.13
$$
for any $c_1<\gamma (P)$ ($c_1=\gamma (P)$ if Corollary {\rm 2.6}
applies). Moreover, there is an $r>0$ such that
$$
|\Cal K_{V}(x,y,t)|\geg
 t\,(d(x,y)+t^{1/d})^{-d-n} ,\text{ for }d(x,y)+t^{1/d}\le r.\tag4.14
$$
\endproclaim

\demo{Proof} As $P$ and $\Delta^{d/2}$ have the same principal symbol,
$$
V(t)=V^d(t)+V' ,
$$
where $V'$ is of lower order, more precisely $V'$
is the difference between the first remainders for $V(t)=e^{-tP}$ and
$V^d(t)=e^{-t\Delta^{d/2}}$, as in the second line of (2.25). Hence
$$
|\Cal K_{V'}(x,y,t)|\leg  e^{-c_1 t}t\, (d(x,y)+t^{1/d})^{1-n-d}.
\tag4.15
$$
Now (4.11) and (4.15) together imply (4.13).

To obtain the lower estimate (4.14), we note that $$
cs^{-n-d}-c's^{1-n-d}=cs^{-n-d}(1-c'c^{-1}s)\ge 2^{-1}cs^{-n-d},\text{
when }s\le c/(2c'),\tag4.16
$$
so for $t$ in a bounded
set where $e^{-c_1t}\le c'$, the lower estimate in
(4.11) implies that (4.14) holds for small $d(x,y)+t^{1/d}$.
\qed

\enddemo

We can also obtain upper and lower estimates for the Dirichlet-to-Neumann
 operator.


\proclaim{Theorem 4.4}
The kernel of $e^{-tP_{DN}}$ satisfies for $t\ge 0$:
$$
\Cal K_{e^{-tP_{DN}}}(x,y,t)\leg
\frac{t}{d(x,y)+t}\,\left((d(x,y)+t)^{-n}+1\right),\tag4.17
$$
and there is an $r>0$ such that it satisfies
$$
\Cal K_{e^{-tP_{DN}}}(x,y,t)\geg t\,(d(x,y)+t)^{-1-n},\text{ for
}d(x,y)+t\le r.\tag4.18
$$
\endproclaim

\demo{Proof} Here $P_{DN}$ is known to be selfadjoint nonnegative, and
the semigroup has real,
nonnegative kernel ([AM07], [AM12]), so that we may omit absolute values.
The upper estimate (4.17) follows from Corollary 2.6. The lower
estimate (4.18) follows from Theorem 4.3 since $P_{DN}$ differs from $\Delta
^{1/2}$ by a
classical $\psi $do of order 0.
\qed
\enddemo




\example{Remark 4.5} This work was inspired from a conversation of the
second author with W.\ Arendt and A.\ ter Elst in August 2012, where
we suggested the applicability of pseudodifferential methods as in
[G96] to the Dirichlet-to-Neumann semigroup. We have very recently learned of the efforts of ter Elst and Ouhabaz
in [EO13], giving an analysis of the Dirichlet-to-Neumann semigroup by
somewhat different methods, and obtaining some of the same results as
those presented here.
\endexample












\Refs
\widestnumber\key{[BGR10]}


\ref \no[A04] \by W. Arendt \paper Semigroups and evolution equations:
Functional calculus, regularity and kernel estimates \inbook
Handbook of Differential Equations, Evolutionary Equations \vol 1 \publ
North-Holland \publaddr Amsterdam
\yr2004\pages 1--85
\endref


\ref \no[AM07]\by W. Arendt and R. Mazzeo \paper Spectral properties of
the Dirichlet-to-Neumann operator on Lipschitz domains \jour
Ulmer Seminare \vol Heft 12 \yr2007\pages  28--38
\endref


\ref \no[AM12]\by W. Arendt and R. Mazzeo \paper Friedlander's
eigenvalue inequalities and the Dirichlet-to-Neumann semigroup \jour
Commun. Pure Appl. Anal. \vol 11 \yr2012\pages  2201--2212
\endref

\comment
\ref \key[B95]\by A. Bendikov \paper Symmetric stable semigroups on the
infinite dimensional torus \jour Expo. Math. \vol 13 \yr 1995 \pages
39--80
\endref
\endcomment

\ref \no[BGR10]\by K. Bogdan, T. Grzywny and M. Ryznar \paper Heat kernel
estimates for the fractional
Laplacian with Dirichlet conditions \jour
Ann. Probab. \vol 38 \yr2010\pages  1901--1923
\endref


\ref\no[B71]\by
  L.~Boutet de Monvel  \paper Boundary problems for pseudodifferential
operators\jour
 {Acta Math.} \vol126\pages  11--51 \yr 1971\endref


\ref \no[CKS12] \by Z.-Q. Chen, P. Kim and R. Song \paper Dirichlet heat
kernel estimates for fractional Laplacian with gradient perturbation \jour
Ann. Prob. \vol 40 \yr 2012 \pages 2483--2538\endref

\ref \no[DR96]\by X. T. Duong and D. W. Robinson \paper Semigroup kernels,
Poisson bounds, and holomorphic functional calculus \jour
J. Funct. Anal. \vol 142 \yr1996\pages  89--128
\endref

\ref \no[EO13] \by A. F. M. ter Elst and E. M. Ouhabaz
\paper Analysis of the heat kernel of the Dirichlet-to-Neumann operator
\finalinfo arXiv:1302.4199
\endref

\ref \no[GM09] \by F. Gesztesy and M. Mitrea \paper Nonlocal Robin
Laplacians and some remarks on a paper by
Filonov on eigenvalue inequalities \jour J. Diff. Eq. \vol 247 \yr 2009
\pages 2871--2896\endref

\ref \no[G03] \by A. Grigor'yan \paper Heat kernels and function theory on
metric measure spaces \inbook
Heat kernels and analysis on manifolds, graphs, and metric spaces (Paris,
2002), Contemp. Math. \vol 338 \publ Amer. Math. Soc. \publaddr
Providence, RI
\yr2003\pages 143--172
\endref

\ref \no[GH08]\by A. Grigor'yan and  J. Hu
\paper Off-diagonal upper estimates for the heat kernel
of the Dirichlet forms on metric spaces \jour  Invent. Math.\vol 174
\pages  81--126 \yr 2008
\endref

\ref\no [G71] \by G. Grubb \paper On coerciveness and semiboundedness
of general boundary problems \yr 1971 \vol 10 \pages 32--95
\jour Israel J.
\endref


 \ref\no[G96]\by
{G.~Grubb}\book Functional Calculus of Pseudodifferential
     Boundary Problems.
 Pro\-gress in Math.\ vol.\ 65, Second Edition \publ  Birkh\"auser
\publaddr  Boston \yr 1996\finalinfo first edition issued 1986\endref

\comment
\ref\no[G09]\by G. Grubb\book Distributions and Operators. Graduate
Texts in Mathematics, 252 \publ Springer \publaddr New York\yr 2009
 \endref
\endcomment

\ref\no[H83]\by L. H\"ormander \book The Analysis of Linear Partial
Differential Operators, I--IV, \yr 1983--85 \publ Springer Verlag
\publaddr Berlin, Heidelberg
\endref


\ref\no[O05]\by E. M. Ouhabaz \book Analysis of Heat Equations on
Domains, London Math. Soc. Monograph Series \vol 31 \yr 2005 \publ
Princeton University Press \publaddr Princeton, NJ\endref



\ref \no[S08] \by Y. Safarov \paper On the comparison of the Dirichlet and
Neumann counting functions \inbook
Spectral Theory of Differential Operators: M.Sh. Birman 80th Anniversary
Collection, Amer. Math. Soc. Transl. Ser. 2 \vol 225 \publ Amer. Math.
Soc. \publaddr Providence, RI
\yr2008\pages 191-204
\endref



\ref\no[S10] \by L. Saloff-Coste \paper The heat kernel and its
estimates \inbook
 Probabilistic approach to geometry.
Adv. Stud. Pure Math. \vol 57\publ  Math. Soc. Japan \publaddr Tokyo
\yr2010\pages 405--436
\endref


\comment
\ref\key[S64]\by R. T. Seeley \paper Extension of $C^\infty $
functions defined in a half space\jour Proceedings Amer. Math. Soc. \vol 15\yr 1964 \pages 625--626
\endref
\endcomment

\ref\key[S67]\by R. T. Seeley \paper Complex powers of an elliptic
operator\jour Amer. Math. Soc. Proceedings Symposia Pure Math. \vol
10\yr 1967\pages 288--307
\endref

\comment
\ref \key[S69]\by R. T. Seeley\paper The resolvent of an elliptic
boundary problem \jour Amer. J. Math. \vol 91 \yr 1969 \pages 889--920
\endref
\endcomment

\ref\key[T81]\by M.~E. Taylor\book
 Pseudodifferential Operators \publ
Princeton University Press \publaddr Princeton, NJ \yr1981
\endref

\ref\key[T96]\by M.~E. Taylor\book Partial Differential
Equations II: Qualitative Studies of Linear Equations. Applied
Mathematical Sciences, 116 \publ Springer \publaddr New York\yr 1996
\endref

\ref\key[Z86]\by V. M. Zolotarev \book
 One-dimensional Stable Distributions. Transl. Math. Monographs \vol 65
\publ Amer. Math. Soc. \publaddr Providence, RI \yr1986
\endref





\endRefs



\enddocument

