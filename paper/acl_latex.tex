\pdfoutput=1

\documentclass[11pt]{article}

\usepackage[preprint]{acl}

\usepackage{times}
\usepackage{latexsym}

\usepackage[T1]{fontenc}

\usepackage[utf8]{inputenc}

\usepackage{microtype}

\usepackage{inconsolata}

\usepackage{times}
\usepackage{latexsym,amssymb}
\usepackage{booktabs}
\usepackage{amsmath}
\usepackage{comment}
\usepackage{graphicx}
\usepackage{stmaryrd}
\usepackage{bm}
\usepackage{dsfont}
\usepackage{xspace}
\usepackage{longtable}
\usepackage{algorithm2e}
\usepackage{multirow,multicol}
\usepackage{lscape}
\usepackage{tabularx}
\usepackage[capitalize]{cleveref}
\usepackage{booktabs}
\usepackage{arydshln}
	
\usepackage{soul}
\usepackage{pdflscape}
\usepackage{afterpage}
\usepackage[utf8]{inputenc}

\usepackage{graphicx}
\usepackage{color-edits}

\usepackage{soul}
\usepackage{listings}
\lstset{
basicstyle=\ttfamily,
numbers=left,
columns=flexible,
breaklines=true,
breakautoindent=false,
breakindent=0ex,
}
\AtBeginEnvironment{quote}{\texttt\small}

\addauthor{el}{purple}


\title{Language Modeling with Editable External Knowledge}



\author{
 \textbf{Belinda Z. Li\textsuperscript{1}},
 \textbf{Emmy Liu\textsuperscript{2}},
 \textbf{Alexis Ross\textsuperscript{1}},
 \textbf{Abbas Zeitoun\textsuperscript{1}},
\\
 \textbf{Graham Neubig\textsuperscript{2}},
 \textbf{Jacob Andreas\textsuperscript{1}}
\\
\texttt{\{bzl, alexisro, zeitoun, jda\}@mit.edu} \\
\texttt{\{mengyan3, gneubig\}@cs.cmu.edu}
\\
\textsuperscript{1} Massachusetts Institute of Technology, CSAIL \\
\textsuperscript{2} Carnegie Mellon University, Language Technologies Institute
}

\usepackage[disable]{todonotes}
\newcommand{\ourmethod}{\textsc{erase}\xspace}
\newcommand{\ourmethodlong}{\textbf{E}nhancing \textbf{R}etrieval \textbf{A}ugmentation with \textbf{S}elf-consistent \textbf{E}diting\xspace}
\newcommand{\ourdataset}{\textsc{clark}\xspace}
\newcommand{\ourdatasetfull}{\textbf{C}ontinual \textbf{L}earning And \textbf{R}evising \textbf{K}nowledge\xspace}
\newcommand{\bzl}[2][]{\todo[color=red!25, #1]{\textbf{bzl}: #2}}
\newcommand{\azcom}[2][]{\todo[color=purple!25, #1]{\textbf{az}: #2}}
\newcommand{\elcom}[2][]{\todo[color=blue!25, #1]{\textbf{el}: #2}}
\newcommand{\gncom}[2][]{\todo[color=green!25, #1]{\textbf{gn}: #2}}

\newcommand{\doc}{d}
\newcommand{\kb}{\mathcal{K}}
\newcommand{\fact}{f}
\newcommand{\facthist}{H}
\newcommand{\plm}{p_\textsubscript{LM}}
\newcommand{\timestamp}{\tau}
\newcommand{\truthval}{v}
\newcommand{\embed}{\mathcal{E}}
\DeclareMathOperator*{\argtopk}{\arg\,\mathrm{top-k}}



\newcommand{\note}[2]{{\color{#1}{#2}}}
\newcommand{\ajrsticky}[2][]{\todo[color=orange!25, #1]{\textbf{ajr}: #2}}
\newcommand{\ajr}[1]{\note{blue}{\textbf{ajr:} [#1]}}
\newcommand{\ajrtodo}[1]{\note{blue}{[\textbf{todo: #1}]}}
\newcommand{\bzli}[1]{\note{red}{\textbf{bzl:} [#1]}}
\newcommand{\abbas}[1]{\note{purple}{\textbf{abbas:} [#1]}}
\newcommand{\jda}[1]{{\color{magenta}\textbf{jda}: #1}}
\newcommand{\el}[1]{{\color{purple}\textbf{el}: #1}}
\newcommand{\gn}[1]{\note{green}{\textbf{gn:} [#1]}}


\newcommand{\eg}{\emph{e.g.,}\xspace}
\newcommand{\ie}{\emph{i.e.,}\xspace}
\newcommand{\quotetext}[1]{\emph{#1}}
\newcommand{\sect}[1]{\S\ref{#1}}

\begin{document}
\maketitle
\begin{abstract}
When the world changes, so does the text that humans write about it. How do we build language models that can be easily updated to reflect these changes? One popular approach is retrieval-augmented generation, in which new documents are inserted into a knowledge base and retrieved during prediction for downstream tasks.
Most prior work on these systems have focused on improving behavior during \textit{prediction} through better retrieval or reasoning.
This paper introduces \ourmethod, which instead improves model behavior \emph{when new documents are acquired}, by incrementally deleting or rewriting other entries in the knowledge base each time a document is added.
In two new benchmark datasets evaluating models' ability to answer questions about a stream of news articles or conversations, \ourmethod
improves accuracy relative to conventional retrieval-augmented generation by 7--13\% (Mixtral-8x7B) and 6--10\% (Llama-3-8B) absolute.%
\footnote{Code and data are available at \url{https://github.com/belindal/ERASE}}%
\end{abstract}

\begin{figure}[t!]
    \centering \includegraphics[width=0.9\columnwidth]{figures/teaser4.pdf}
    \caption{
    In standard retrieval augmented generation (RAG), new facts are simply added to an existing knowledge base $\kb$. This can lead to stale facts in $\kb$, which can in turn lead to incorrect predictions at inference time. In contrast, when \ourmethod reads a new input article, it not only adds new facts to $\kb$, but also \emph{updates} it. \ourmethod can edit or delete (not pictured) existing facts to keep $\kb$ up to date, thereby enabling correct predictions at inference time. The same LM is used to update the memory and make predictions.
    }
    \label{fig:teaser}
\end{figure}
\section{Introduction}
\label{sec:intro}
The world---and the language we used to describe it---are constantly changing.
Consider the example shown in \autoref{fig:teaser}. After reading the article \quotetext{After Queen Elizabeth II died, the Queen's oldest son Charles has now become King Charles III,} a knowledgeable reader might update an entire system of related beliefs, \eg that King Charles III is now also the new head of Scotland. 
How can we train language models and other software systems to reflect these changes?

Continual learning methods tackle the problem of a changing world by incrementally \emph{training} on new information \cite{nell, wang2024comprehensive}. 
But in language models, a simple (and often extremely effective) approach simply presents new information in models' inputs by leveraging either long-context methods
\citep{efficientTansformers} or retrieval augmented generation (RAG; \citealp{RAG}).
which appends new documents to a knowledge base and retrieves a subset of relevant documents to condition on at prediction time \citep{guu2020, lewis2020}.

An important limitation of current RAG approaches is that they sometimes retrieve \emph{stale} documents that have been invalidated by new information. In \cref{fig:teaser}, the article \quotetext{After Queen Elizabeth II died...} would be appended to the existing knowledge base, which includes a fact about Queen Elizabeth's reign when she was alive, \eg \quotetext{Queen Elizabeth II is head of state of...Scotland.} When answering questions about the Scottish head of state, this document might be retrieved, leading the LLM to produce incorrect answers. Past attempts to address this issue have focused on improved \emph{retrieval} methods, but not on ensuring accuracy and consistency of the document collection itself.

This paper describes a method for retrieval-augmented generation that attempts to ensure that the external knowledge base always represents the {current} state of the world. This method, which we call \ourmethod (\ourmethodlong; \sect{s:method}), enables accurate language modeling by updating the knowledge base at \emph{document insertion} time---\ie when new documents are read and added to the knowledge base---rather than at prediction time. Every time a new document is acquired, \ourmethod identifies related documents in the knowledge base and decides whether to keep, edit, or delete them. These operations allow new information to be propagated and prevent stale information from being used for inference. In Figure~\ref{fig:teaser}, \ourmethod not only adds the new article to the knowledge base, but also \emph{edits} the existing fact \quotetext{\st{Queen Elizabeth II} 
$\rightarrow$ \textbf{King Charles III} is head of...Scotland}, thereby enabling correct prediction when this document is retrieved.

We evaluate \ourmethod's performance on question-answering (QA) tasks about a set of continually changing facts described by a stream of text. 
To do so, we introduce a new benchmark dataset, \ourdataset (\ourdatasetfull; \sect{s:dataset}), which contains two domains: (1) \ourdataset-\textsc{News}, a factual QA domain consisting of a set of timestamped news articles paired with questions and timestamped answers; (2) \ourdataset-\textsc{Conversations}, a long-conversation domain where facts about conversation participants evolve over the course of the conversation. The conversation domain contains both single-hop 
and multi-hop edits, the latter of which requires multi-hop inferences at the memory updating stage. 

On this benchmark, \ourmethod outperforms standard RAG baselines and long-context models, giving 
7--13\%
(Mixtral-8x7B) and 
6--10\%
(Llama-3-8B) absolute improvements in accuracy %
compared to standard RAG on the factual QA domain and single-hop section of the conversation domain. On the multi-hop subset, we find that \ourmethod performs comparably to baselines, suggesting there is room for future work to improve multi-hop memory editing.






























\section{Background and Related Work}

\ourmethod belongs to a growing body of work aimed at developing LM-based systems that can be updated after training. \ourmethod builds specifically on approaches that update LMs by modifying \emph{inputs} rather than parameters---as discussed below, such methods are more flexible, and often more robust, than alternatives. 

\paragraph{Long-context and retrieval-augmented generation: updating LMs via conditioning} 
One simple and effective way to update LMs is simply to include new information in their context window before inputs to the task of interest (e.g.\ by prepending a question about current events with a sequence of news articles). 
But this approach begins to face challenges when text containing new information is extremely long (e.g.\ comprising thousands of news articles). In these cases, it is neccessary either to use LMs specialized for very long input sequences, or to select a subset of inputs to condition on for each new query to the model (sometimes referred to as retrieval-augmented generation, or RAG). 

Long-context models \citep{wang2020linformer,kitaev2020reformer,press2021train,su2024roformer} focus on modifying LM architectures to allow long sequences to be processed efficiently, or to extrapolate to long inputs.
RAG methods, by contrast, dynamically 
construct relevant contexts tailored to individual queries \citep{guu2020,lewis2020}.
Previous work has explored auxiliary models that selectively choose when to perform retrieval \citep{mitchell2022memory}, or abstain from answering questions when retrieved sources present conflicting or outdated information \citep{chen-etal-2022-rich,zhang-choi-2023-mitigating}.
Other work has examined augmenting LMs with \textit{knowledge graphs} \citep{ijcai2023p0734,modarressi2024memllm}, structured relational knowledge bases that may be timestamped and whose nodes and edges may be updated.
However, such structure can be difficult to construct and risks throwing away essential information; these methods are generally less used than unstructured knowledge bases.

\paragraph{Continual learning: updating LMs via fine-tuning} A broader class of methods, applicable to a much broader class of machine learning models, study the problem of robustly performing \textbf{continual learning} under a non-stationary data distribution \cite{nell, wang2024comprehensive} via training objectives that ensure that new information is retained but old information is not forgotten \cite{jang2022continual,mehta2023dsi, jang2023temporalwiki}. Previous work on LMs has explored the use of continual pretraining \citep{jin-etal-2022-lifelong}, modified pretraining objectives \citep{xu-etal-2023-kilm}, and synthetic data generation \citep{padmanabhan_2023_distill,akyurek2024deductive}.
Continual learning methods are computationally intensive and less widely used than RAG and related methods in language models.

\paragraph{Model editing: updating LMs with targeted interventions} A final category of methods alter LM behavior by making targeted interventions to their parameters, either using specialized secondary ``editing'' models \citep{decao2021editing, mitchell2022fast}
or performing closed-form updates
\citep{meng2022locating, meng2023massediting}. Current methods reliably 
update facts but not all their implications \cite{onoe-etal-2023-lms,hua2024propagation}, and are generally outperformed by retrieval- or fine-tuning-based methods.
 



\paragraph{Evaluating updates}
Few resources are currently available for evaluating models' ability to generate text about \emph{changing} features of the world while attributing these changes to known source of information.
The Entity Cloze by Date (ECBD) dataset contains entities from Wikidata along with cloze-style sentences \citep{onoe-etal-2022-entity}, and the LoCoMo dataset contains long conversations to measure long-term memory in models \citep{maharana2024lococmo}; unlike \ourdataset, these datasets do not isolate entities whose properties \emph{change} over time.
Many datasets~\citep{zhang-choi-2021-situatedqa,timeqa,meem2024patquestions,dhingra-etal-2022-time,kasai2023realtime,vu2023freshllms} have been released studying temporally-situated question answering; 
however, contexts in these datasets consist only of dates and not source documents.
This makes it difficult to compare results across implementations: were improvements due to a better system, or simply due to a more complete set of documents in the knowledge base?
In \ourdataset, we release both our questions and attributable source documents for those questions.







\begin{figure*}[ht!]
    \centering
    \includegraphics[width=0.9\textwidth]{figures/method3.pdf}
    \caption{Overview of \ourmethod. We begin by retrieving existing facts relevant to input and prompting a LM to update them. We also extract facts from the input to add to our knowledge base.}
    \label{fig:method}
\end{figure*}

\section{\ourmethod Method}
\label{s:method}

We seek to develop a system that can generate text (e.g.\ for the question answering task depicted in \cref{fig:teaser}) while updating its behavior in response to a continuous stream of documents describing a changing state of the world (e.g.\ the article about the death of Queen Elizabeth II, shown with a yellow background in \cref{fig:method}).
Informally, \ourmethod uses these documents to populate and edit a knowledge base that stores a collection of facts extracted from documents and represented as natural language strings (e.g.\ the identity of the new king, and the duration of Elizabeth II's reign, shown with gray backgrounds in \cref{fig:method}). Importantly, the knowledge base records not just the content of each fact, but when it was first added, and (if relevant) when it ceased to be true. As new documents arrive, \ourmethod attempts to maintain the knowledge base in a \emph{consistent} state---containing only facts that are currently true---by rewriting facts or marking them as false when contradictory facts are introduced by new documents (e.g.\ deleting facts about Elizabeth II's health and updating other references to the UK monarchy). During prediction, \ourmethod then operates like a normal RAG appach: retrieving true facts that are relevant to a given query.








More formally, we begin with a \textbf{language model} encoding a conditional distribution over strings $\plm(\textrm{prediction} \mid \textrm{context})$. 
When a 
new \textbf{document} $\doc_i$ is received with some \textbf{timestamp} $\tau_i$, 
we update the \textbf{knowledge base} $\kb$---each entry in $\kb$ consists of both a \textbf{fact} $\fact_j$ and a \textbf{fact history} $\facthist_j = [(\timestamp_{j0}, \truthval_{j0}), (\timestamp_{j1}, \truthval_{j1}), \ldots]$, where each $\timestamp_{jk}$ is a timestamp and $\truthval_{jk}$ is a \textbf{truth value} indicating whether $\fact_j$ was known to be true or false at time $\timestamp_{jk}$.
We then 
parse the new document into a sequence of facts $\fact_j$ using the LM.\gncom{Separately, is this parsing process described anywhere? It would be nice to explain a little more how we get the facts.} \bzl{yes in the appendix}

Unlike standard RAG methods, it is not in general necessary for facts extracted from documents to correspond one-to-one with facts in the knowledge base: knowledge base entries may also arise by editing old facts in response to new articles. To accomplish this, \ourmethod incorporates new documents into the knowledge base in three steps: \textbf{retrieval}, \textbf{updating}, and \textbf{adding}.


\paragraph{Step 1: Retrieve facts to edit.} 
\begin{equation}
    R \leftarrow \texttt{Retrieve}(\kb, \doc)
\end{equation}
We retrieve a set of knowledge base entries $R = \{(f_{i_0}, H_{i_0}),\cdots (f_{i_m}, H_{i_m})\}\subset K$.
Here we assume that the facts most likely to require \emph{editing} in response to $\doc$ are those most similar to $\doc$.\footnote{For efficiency, we retrieve facts relevant to the entire document in this step, rather than first parsing the document into facts, then retrieving facts relevant to each extracted fact.}
\gncom{This part was a little bit confusing to me, so just to confirm: previous facts are retrieved according to the embedding of $\doc$, not according to the embedding of the facts parsed from $\doc$ (e.g. $f_j$)? With the description up to this point, I would have expected us to try to match new facts $f_j$ with old facts $f_i$ instead of trying to match $\doc$ with $f_i$. It might be nice to add a footnote explaining this design decision for clarity.}
Following most modern RAG approaches~\citep{RAG}, \ourmethod performs \textbf{dense vector retrieval}, using a learned embedding model $\embed$ to assign documents and facts vector representations, then retrieve a set of $m$ to optimize:
\begin{equation}
    \texttt{Retrieve}(\kb, \doc) = \argtopk_{(\fact_j, \facthist_j) \in \kb} ~\embed(\doc)^\top \embed(\fact_j) ~ .
\end{equation}


\paragraph{Step 2: Update retrieved facts.} 
\begin{align}
    &\forall (\fact_j, \facthist_j) \in R, \, (\fact'_j, \facthist'_j) \leftarrow \texttt{Update}(\fact_j,\facthist_j,\doc,\timestamp) \nonumber \\
    & \kb \gets \kb \cup \{(\fact'_j, \facthist'_j)\}
\end{align}
We update the knowledge base by modifying each retrieved fact $f_i\in R$ in one of the following ways:
\begin{itemize}
    \item \textbf{Reinforce fact}: If the fact $\fact$ is supported by $\doc$, we add $(\texttt{true}, \timestamp)$ to $\facthist$. An example of such a case would be $\fact =$ \textit{Mary works in a warehouse} and $\doc =$ \textit{Mary came back from her job at UPS where she loaded and sorted packages all day}.
    
    \item \textbf{Keep fact unchanged}: If $\doc$ is irrelevant to $\fact$ or does not affect the truth value of $\fact$, then we do nothing and let $\fact' = \fact$ and $\facthist' = \facthist$. An example of such a case would be $\fact =$ \textit{Mary works in a warehouse} and $\fact = $ \textit{Mary took a jog in the park}.

    \item \textbf{Make fact false}: If $\fact$ is contradicted by $\doc$, we add $(\texttt{false}, \tau)$ to $\facthist'$. An example of such a case would be $\fact =$ \textit{Mary works in a warehouse} and $\doc = $ \textit{Mary got fired from her warehouse job}.

    \item \textbf{Rewriting}: Alternatively, if $\fact$ is contradicted by $\doc$, we may \textit{rewrite} it into a new expression $\fact'$ that is inferrably true from $\doc$ and the subset of retrieved facts $\subset R$ that have been \textit{reinforced} or \textit{kept unchanged}. %
    We then replace the old KB entry $(\fact, H)$ with a new KB entry $(\fact', [(\texttt{true},\tau)])$.
\end{itemize}

For all operations above, we prompt an LM (which may be the same LM used for prediction) to classify each retrieved fact into one of \textit{reinforce, no change, make false}.\footnote{The task in the first pass is similar to a fuzzy version of natural language inference classification. Inputs that make facts more likely (even if they do not exactly entail those facts) are classified as \textit{support}, and inputs that make facts less likely (even if they do not exactly contradict those facts) are classified as \textit{make false}.}
We then iterate through all facts classified as \textit{make false}, and ask the LM if it can rewrite the fact into a true expression. In this second phase, the LM is allowed to condition on facts that it classified as \textit{reinforce} or \textit{no change}, allowing it to potentially handle multi-hop edits. The full details of this procedure can be found in~\cref{app:update_prompts}.






\paragraph{Step 3: Add new facts.} 

\begin{equation}
    \kb \leftarrow \kb \cup \texttt{Add\_facts}(T)
\end{equation}
We add all new facts by conditioning on $\doc$ and prompting the LM to extract atomic facts $\fact$. The prompt we use can be found in~\Cref{app:fact_extraction}.
Analogously, ~\citet{chen2023dense} used a \textit{propositionizer} to decompose articles into propositions. 


\paragraph{Prediction:}
To use an \ourmethod system after updating, generation is performed using a standard RAG pipeline described in step 1.
We condition on both the retrieved facts and their corresponding history in context. The full prompt can be found in~\Cref{app:infer_prompt}.


\begin{figure*}
    \centering
    \includegraphics[width=\linewidth,trim={0 0 0 0},clip]{figures/dataset3.pdf}
    \caption{Sample data from our datasets. The News dataset consists of factual questions whose answers change over time, with the associated source inducing that change. The Conversations dataset consists of conversations between two personas with evolving life facts. The single-hop subset directly states all facts that are changed, while the multi-hop subset requires reasoning about previous chunks of conversation to infer all changes.}
    \label{fig:dataset}
\end{figure*}


\section{Dataset}
\label{s:dataset}



We construct two datasets to evaluate \ourmethod.
We acquire a set of natural-language texts $L_t$, a set of ground truth world states $W_t$ and a series of questions $q_0\cdots q_n$ associated with $W_t$.
We focus on questions that \textit{update} over time: the set of questions we ask at each timestep are the same, but each question is associated with a list of timestamped answers $(q_i, \{(a_{i0}, t_{i0}), (a_{i1}, t_{i1}),\cdots\})$. %
The datasets span two domains where continual learning is useful: one about the evolving state of the world, and one about the evolving state of agents in a conversation. Samples from each dataset can be found in~\Cref{fig:dataset}. An overview of state transitions and questions in these two datasets can be found in \autoref{app:dataset_stats}.



\subsection{News Articles}


\paragraph{World States}
In this domain, world states are expressed in the form \texttt{(subj, rel, obj)}: for instance, \texttt{(Elizabeth II, position held, monarch of the United Kingdom)}. We mine these triples from Wikidata.\footnote{\url{https://www.wikidata.org/}, which is public domain. Its license can be found at \url{https://www.wikidata.org/wiki/Wikidata:Licensing}.} As Wikidata is updated over time, each fact is also associated with a start and end date. To find changed facts, we extract \texttt{(subj, rel)} pairs for which there are at least two distinct fact relations at different timestamps between November 2021 and April 2024. Through this process, we obtain 1,174 triples for 10 unique relations, summarized in \Cref{tab:news_relations}.

\paragraph{Documents}
For each world state \texttt{(subj, rel, obj, start\_ts, end\_ts)}, where the start and end timestamps are extracted from Wikidata, we obtain an English article confirming that fact between the start and end timestamps, validated by crowd workers. Through this process, annotators collected a total of \textbf{1149} articles.\footnote{Note 1149 $<$ 1174, meaning at least a few articles were shared across relations -- these represent difficult cases where a single article makes multiple relation changes.} See \Cref{app:wikidata} for details. These documents---rather than raw relation triples---are the input to \ourmethod.


\paragraph{Questions and Answers}
We automate the generation of questions and answers from $W$ by writing templates for each relation and generating questions and answers from those templates. We generated a total of \textbf{1409} questions. The full list of templates can be found in~\Cref{app:wikidata}.






\subsection{Synthetic Conversations}
Following prior work~\cite{maharana2024lococmo}, we construct a synthetic conversation domain by placing two LLMs with different personas in conversation with each other. Conversations are engineered to reflect changing facts in the agents' simulated lives. A detailed overview of dataset construction can be found in \cref{app:convos}.  To validate the LM generations, three authors manually examined 3 conversations (1008 questions) in total and got an average of 95\% accuracy on these questions.

This synthetic domain allows us to rigorously control and evaluate forms of reasoning that may be hard to isolate in natural data like news articles. 

\paragraph{World States}
We generate an independent world for each conversation.
We model the world underlying a conversation as a Markov chain with states $S$, described by a list of \texttt{(subj, rel, obj)} relations, and allowable transitions $T(S)$. States $S$ are defined by entities including people, companies, jobs, hobbies, along with mutable and immutable relations between them.
Transitions $t\in T(S)$
change one or more relation in the state: for example, \textit{Bob changed jobs to work at Google} changes the \textit{employees} of Google, the set of \textit{coworkers} of Bob, the set of \textit{coworkers} of all Google employees, and the set of \textit{coworkers} of all employees of Bob's former company, etc. At each timestep, we sample a transition from $T(S)$ uniformly at random.
The full list of entities, relations, and transitions and their downstream effects can be found in~\Cref{app:convos}.

\paragraph{Conversations}
We generate conversations by sampling two people in the world $p_1$ and $p_2$ and prompting two LLMs with their corresponding personas and the initial world state $S$.
We then generate twelve conversation ``chunks''---separated by time---by sampling state transitions between \textit{every other} 
chunk and having people converse about the facts that have changed after each transitions.

We also construct a challenge set of \textit{multi-hop} updates in this domain, which require propagating changes to multiple downstream facts and reasoning about global coherence between facts. For example, Bob may mention that he has changed his job but may not mention that \textit{Jane is no longer his coworker} or that \textit{Mary (who works at Google) is now his coworker}. The LM must make multi-hop inferences to update the latter two facts.

We generate \textbf{100} conversations (50 single-hop, 50 multi-hop) in total. Conversations were on average \textbf{11045} tokens long in the single-hop subset and \textbf{11069} tokens long in the multi-hop subset. Detailed statistics may be found in Appendix \Cref{fig:convos_stats}.


\paragraph{Questions and Answers}
Given a world state at time $t$, we query \textit{all} facts about the world. Similar to the news setting, we automate generation of questions and answers through templates. We generate \textbf{140} questions per conversation.














\begin{figure}[t!]
    \centering
    \includegraphics[width=.8\linewidth,trim={1cm 1cm 13.5cm 1cm},clip]{figures/News_Mixtral-8x7B.png} \\
    \includegraphics[width=.8\linewidth,trim={1cm 0.5cm 13.5cm 1cm},clip]{figures/News_Llama-3-8B.png}
    \caption{Mixtral-8x7B (top) and Llama-3-8B (bottom) results on the news article domain. %
    \ourmethod outperforms RAG, RAG with fact-level granularity, and even long-context models, especially in later timesteps as more new information is learned.}
    \label{fig:wiki_results}
\end{figure}

\section{Experiments}
\label{s:experiments}

In our experiments, we present to a LM articles or conversational turns in chronological order, and periodically ask questions about the state of the world (as described by input documents) at that point in time.

\subsection{Evaluation and Metrics}
\paragraph{News articles}
We present the model with a stream of articles
ordered by timestamp. As all answers are dated with a start and end timestamp, we always know which answer is true for a given timestamp.\footnote{Note that this does not correspond to when these facts became true and false in the real world, but rather to when the article introducing the changed fact was written and read.}
We ask questions at regular intervals, at timesteps corresponding to when 20\%, 40\%, 60\%, 80\%, and 100\% of the total world state changes have been revealed to the model.
Because it is too expensive to ask every question at every timestep, we ask \textit{all questions whose answers have changed} $Q$, then sample a subset of \textit{questions whose answers have not changed} $Q'$, such that $|Q'| = |Q|$.
We design each question as a multiple choice question, where the model is asked to select between all answers that have been true for the question in the past, present, or future.
This ensures that the negative options are sufficiently difficult, and allows us to probe for the models' updating capabilities.
We report exact-match accuracies between the model-predicted answer to the true answer.

\paragraph{Conversation} We evaluate each conversation independently, and report the mean and standard error of scores over each conversation.
We stream in \textit{chunks} of conversations into the model, and 
ask questions after each conversation chunk. Similarly to the news domain, we subsample questions whose answers have not changed, such that at each timestep we are asking the same number of questions whose answers have changed as those whose answers haven't changed. 
For questions that have multiple true answers (e.g. \texttt{List all siblings of Liam}), we measure the set equality between the generated and true sets of answers.
Otherwise, we use the same exact match accuracy as we use for the news articles domain.




\begin{table*}[t!]
    \centering
    \footnotesize
    \begin{tabular}{|p{1cm}|l|ccc|ccc|}
    \hline
        & & \multicolumn{6}{c|}{Data Subset} \\
        & & \multicolumn{3}{c}{Single-hop} & \multicolumn{3}{c|}{Multi-hop} \\
         & & 0 updates & 1 update & \multicolumn{1}{c}{2+ updates} & 0 updates & 1 update & 2+ updates \\
    \hline
        \multirow{4}{1cm}{Mixtral-8x7B} 
        & RAG \cite{RAG} & $\mathbf{86.0_{\pm 0.7}}$ & $56.7_{\pm 1.8}$ & $50.9_{\pm 3.2}$ & $\mathbf{84.5_{\pm 0.8}}$ & $\mathbf{20.9_{\pm 1.4}}$ & $20.0_{\pm 2.3}$ \\
        & Fact-RAG \cite{chen2023dense} & $82.7_{\pm 0.8}$ & $51.5_{\pm 1.8}$ & $52.7_{\pm 3.1}$ & $81.8_{\pm 0.8}$ & $18.0_{\pm 1.3}$ & $\mathbf{30.2_{\pm 2.7}}$ \\
        & \ourmethod (Ours) & $82.0_{\pm 0.8}$ & $\mathbf{59.1_{\pm 1.8}}$& $\mathbf{57.9_{\pm 3.1}}$ & $81.5_{\pm 0.8}$ & $\mathbf{20.1_{\pm 1.4}}$ & $27.2_{\pm 2.6}$ \\
        \cdashline{2-8}
        & Full Context & $88.8_{\pm 0.6}$ & $71.6_{\pm 1.6}$ & $75.7_{\pm 2.4}$ & $88.4_{\pm 0.6}$ & $43.2_{\pm 1.7}$ & $54.3_{\pm 2.8}$ \\
    \hline
        \multirow{3}{1cm}{Llama-3-8B} & RAG \cite{RAG} & $\mathbf{84.4 _{\pm 0.7}}$ & $57.8_{\pm 1.8}$ & $55.2_{\pm 3.1}$& $\mathbf{83.6_{\pm 0.8}}$ & $22.2_{\pm 0.1}$ & $26.8_{\pm 2.6}$  \\
        & Fact-RAG \cite{chen2023dense} & $82.6_{\pm 0.8}$ & $62.6_{\pm 1.7}$ & $62.0_{\pm 3.0}$ & $81.2_{\pm 0.8}$ & $\mathbf{26.4_{\pm 1.6}}$ & $\mathbf{32.1_{\pm 2.8}}$ \\
        & \ourmethod (Ours) & $82.0_{\pm 0.8}$ & $\mathbf{65.3_{\pm 1.7}}$ & $\mathbf{65.2_{\pm 2.9}}$ & $81.0_{\pm 0.8}$ & $\mathbf{26.5_{\pm 0.2}}$ & $\mathbf{31.7_{\pm 2.7}}$ \\
    \hline
    \end{tabular}
    \caption{Results on the synthetic conversation domain. Full context serves as a skyline in this domain as the full conversation fits into the context window. We compare against other retrieval-based methods. In \textbf{bold} are results that are the \textbf{statistically significantly best} out of all other methods in the same setting (model, data subset, \# updates). While \ourmethod significantly improves single-hop edits in both models, it still struggles with multi-hop edits. Small LMs make errors in multi-hop reasoning during the overwriting stage, and suspect that as LMs improve multi-hop reasoning, we will see greater gains with \ourmethod. \\
    \footnotesize{* We merge 2+ updates as generally there is a long tail of questions with more updates. Only 27 questions total have 3+ updates.} 
    }
    \label{tab:convo_results}
\end{table*}

\subsection{Models}
We use a Mixtral 8x7b Instruct model (56B parameters; \citealp{jiang2024mixtral}), queried using Together AI\footnote{\url{https://www.together.ai/}}, and a local copy of Meta's Llama-3 8b Instruct model (8B parameters ; \citealp{llama3modelcard}) run on one NVIDIA A100 GPU.\footnote{Llama-3 8b has knowledge cutoff of March 2023. Mixtral's has not been published, but appears to be around late 2022 or early 2023.} 
For all prompts during inference and update-time, we sample from the LM with temperature 0.
We use GTR (T5-large; 770M parameters; \citealp{ni-etal-2022-large}) as $\embed$ to encode queries and documents for dense retrieval, both in the inference stage and the retrieval step of updating. We use %
a fast inner-product search datastructure for efficient retrieval~\citep{FAISS}. 
For prompting during the updating stage, we use the same LM that we are using for inference.
We restrict the context window to 4096 for the news domain and 2048 for the conversation domain.\footnote{Note this is smaller than the original context windows for these models, both to run our experiments efficiently, and to test out a (realistic) scenario where the total number of new world changes cannot fit into the context window of a language model.}
Inference and updating took a few hours to complete for both models and for all method.
At inference time, we allow all models to perform zero-shot chain-of-thought, giving them an additional ability to reason about inconsistent facts at inference time. 

\subsection{Baselines}
We compare \ourmethod to three baselines:

\paragraph{RAG} RAG~\cite{RAG} stores and retrieves text at the granularity of \textit{passages}. We save each article and conversation chunk as a separate passage in the knowledge base. For long articles and conversation chunks, we divide them into passages of length \verb|context_window / 2|.

\paragraph{Fact-RAG} To isolate the effects of \textit{editing}, we benchmark against a version of RAG that stores and retrieves \textit{facts} in the knowledge base, akin to~\citet{chen2023dense}. 
We implement this baseline by prompting LMs to extract facts from passages, i.e.\ step 3 of \ourmethod, which 
outperformed the propositionizer from~\citet{chen2023dense}.

\paragraph{Long context LMs} Mixtral-8x7B has a long context window of 32k. 
We run an in-context learning baseline %
by conditioning Mixtral %
on all news articles or conversation chunks, presented in chronological order. These texts are timestamped, and Mixtral is able to condition on the most recent set of texts up to its context limit when making predictions. In the Conversations domain, this condition serves as a skyline since conversations fit completely into the context window.




\section{Results}
\label{sec:results}
\Cref{fig:wiki_results} and \Cref{tab:convo_results} show results for the news and conversation domains respectively.

\paragraph{\ourmethod improves over standard RAG with passage retrieval.} For both Mixtral and Llama-3 in both domains, we see significant improvements using \ourmethod over RAG, particuarly as the number of edits increases. For example, in the news domain, at the final timestamp after reading all articles, Mixtral with \ourmethod is 13 points better than Mixtral with RAG, while Llama with \ourmethod is about 6 points better than Llama with RAG.
We see similar trends on the single-hop subset of the conversation domain: for questions with 2+ updates, \ourmethod is 7 and 10 points better than RAG, using Mixtral and Llama respectively.

\paragraph{Editing existing facts improves beyond RAG with fact retrieval.}
For both Mixtral and Llama-3, \ourmethod substantially improves performance over Fact-RAG as the number of edits increases, on both the news domain and the single-hop subset of the conversation domain.
Improving knowledge base consistency helps, \textit{even with step-by-step reasoning} at inference-time. %

\paragraph{In the news domain, \ourmethod improves over long-context modeling.} 
In~\Cref{fig:wiki_results}, we plot Mixtral with its full context window on the news domain. Long-context models are unable to scale as more articles are added. However, we find that \ourmethod (and retrieval methods generally) are unable to compete against fitting full conversations in the context window~\Cref{tab:convo_results}. That said, the cost of conditioning on full conversations is greater than the cost of conditioning on simply retrieved facts, especially as the number of queries per conversation increases.\footnote{Conditioning Mixtral on full conversations costs 7.3K tokens per query, whereas retrieval costs $\sim$ 1.7K tokens per query $+$ a fixed cost of $\sim$ 42k tokens per conversation chunk. Generally in the real world that the number of queries far outflanks the number of documents generated about changes in the world. In our dataset without subsampling, full context would cost 102M tokens while ours would cost 28M tokens.}

\paragraph{Multi-hop retrieval and editing is still challenging.}
Both LMs struggle with the multi-hop subset of the conversation dataset.
We believe this isn't a drawback of fact editing itself, but of our implementation of it: a qualitative examination of failure cases (see \Cref{sec:multihop_errors} for some examples) revealed that 
our retrieval model often failed to retrieve all downstream facts that need to be edited,
and language models on the scale of Mixtral-8x7b and Llama-3-8b struggled with reasoning about multi-hop edits, failing to make those edits when necessary.
A more powerful retrieval and editing model may be able to avoid these errors.





\section{Conclusion}
This paper introduced \ourmethod, an approach for \textit{editing existing facts} in a knowledge base when new documents are being inserted. We also introduced two datasets for testing the ability of models to update their knowledge, accompanied by documents that induce those changes. Editing existing facts brings significant improvements to RAG-based models. Even if future models become better at reasoning about inconsistencies with scale, fact editing is useful for amortizing the cost of reasoning about consistency \textit{at insertion time}, rather than having to re-evaluate consistency each time a fact is queried.
Future work can focus on improving any part of the update pipeline, particularly focusing on retrieving downstream facts (step 1) that will be affected by an input (which is different from retrieving simply \textit{relevant} facts), and improving LM ability to perform multi-hop updates (step 2).


\subsubsection*{Limitations}
As noted in ~\Cref{sec:results}, \ourmethod is still subpar for multi-hop updates, largely due to retrieval model's inability to retrieve all the necessary facts and the LMs' inability to reason about multi-hop edits. We believe that this limitation can be mitigated with better retrieval models and better LMs.

Second, because LMs have a tendency to hallucinate, allowing LMs to directly edit the knowledge base may introduce noise into the knowledge base. While our results found that the utility of propagation was greater than any hindrance due to such noise, this noise has the potential to snowball on long timescales as the number of new passages and edits grows beyond tens of thousands, hundreds of thousands, or millions.
That said, we do not believe this limitation is inherent to knowledge-base editing: future work can explore more principled and rigorous approaches to editing with guarantees around what edits are made and to how many facts.
Furthermore, we believe that for any approach to model editing, there is a natural tradeoff between noise and edit coverage. 

Finally, having to process each document and update the knowledge base is less efficient than simply adding it to the retrieval store. We justify this cost by assuming that the number of insertions is far fewer than the number of queries. (For example, Forbes reports that 252,000 websites are created per day,\footnote{\url{https://www.forbes.com/advisor/business/software/website-statistics/}} while Google receives about 8.5 billion searches daily.\footnote{\url{https://seo.ai/blog/how-many-people-use-google}}) Thus, by shifting the cost of reasoning about consistency from query-time to insertion-time, \ourmethod is arguably \textit{more efficient} in practice than RAG.


\subsubsection*{Ethical Considerations}
Being able to interpretably edit models is useful for improving the safety and trustworthiness of models. If there is misinformation in the knowledge base, our method allows these facts to be corrected quickly and these corrections to propagate through the knowledge base.
Our method magnifies the effect of each change, making it easy for system designers to keep knowledge up-to-date and remove any stale or incorrect knowledge.
Conversely however, this could also empower malicious actors to insert false facts, which will also be propagated through the knowledge base.
There will need to be safeguards in place to ensure that any inserted and propagated knowledge is from reliable sources, with potential vetting of each inserted article. One of the pros of \ourmethod is that we can see every LM operation occurring in real time: any update operation can be examined manually to ensure that the changes are desirable.



\bibliography{custom}

\newpage

\appendix
\appendix
\section{Compression Mechanism}
\label{appendix:CompressionMechanism}
% We use a single linear layer (with no non-linearity) to project $F$-dimensional features to $F_C$ dimensional features on the sender side, where $F_C < F$. On the receiver side, we use another linear layer to project back to the original $F$-dimensional feature space, i.e, $x_{decompress} = {\bf W}_{decoder}{\bf W}_{encoder} x$, where $x$ and $x_{decompress}$ are the original and reconstructed feature vectors, respectively. ${\bf W}_{decoder}$ and ${\bf W}_{encoder}$ are the $F \times F_C$, and $F_C \times F$ decoder and encoder projections, respectively. ${\bf W}_{decoder}$ and ${\bf W}_{encoder}$ are trained, together with the GNN parameters, to minimize task loss.

% In the variale compression scenario, we gradually decrease the compression ratio as training progresses. The transmittted compressed feature tensor's size increases by  $\Delta F$ every few training epochs. We achieve this by increasing the number of rows of ${\bf W}_{encoder}$ and the number of columns of ${\bf W}_{decoder}$ by $\Delta F$. The new rows in ${\bf W}_{encoder}$ are initialized using Glorot initialization. To avoid a sudden jump in the network response, we initialize the new columns of ${\bf W}_{decoder}$ to zeros.  

For the compression mechanism, we communicate the total number of elements in the feature vector and intermediate activations divided by the compression ratio. Which values of the vectors to communicate are chosen at random at the encoder's end. 
For the decoder to know which element of the vector corresponds to the true values, a random key generator is shared a priori. The decoder simply places the values communicated in the corresponding position and sets a $0$ on the rest of the non-communicated values.






% \subsection{Deterministic Sliding Window}
% \subsection{Random Sampling}


% \begin{figure*}[h]
	% 	\centering
	% 	\begin{subfigure}{\textwidth}
		% 		\includegraphics[width = \textwidth]{figures/Architecture.png} 
		% 		\subcaption{}
		% 		\label{fig:arch_fixed}
		% 	\end{subfigure}
	% 	% \begin{subfigure}{\textwidth}
		% 		% 	\includegraphics[width = \textwidth]{figures/variable_compression.png} 
		% 		% 	\subcaption{}
		% 		% 	\label{fig:arch_variable}
		% 		% \end{subfigure}
	% 	\caption{Model architecture for $f_{compress}$ with (a) fixed compression ratio, and, (b) variable compression ratio.}
	% \end{figure*}
% \subsection{Network Architecture for $f_{compress}$ with fixed $r$}
% We introduce a learnable feature-based compression-decompression routine $f_{compress} = g^{-1} \circ g$ which employs an autoencoder architecture, with the encoder approximating the compressor function $g$ and the decoder approximating the decompressor function $g^{-1}$. We adopt a multilayer perceptron design for both the encoder and decoder, which is shared across all nodes in the graph. In this routine, when the worker $W_i$ requests for the node features $x_{v^i_j} \in \reals^m$ to a remote worker $W_j$, $W_j$ first employs the encoder function $g$ to embed the node features into a latent space $z_{v^i_j} \in \reals^n$ and then sends them to $W_i$. Upon receiving, $W_i$ uses the decoder function $g^{-1}$ on $z_{v^i_j}$ to decompress the received features into ${x}_{v^i_j}$. To train $g$ and $g^{-1}$ in an end-to-end manner, we use the original downstream task loss (e.g. cross-entropy loss for classification task) instead of the typical reconstruction loss utilized in an autoencoder training. In fact, the reconstruction loss can't be computed because the original feature vectors $x_{v^i_j}$ are not shared with machine $W_i$.

% \subsection{Network Architecture for $f_{compress}$ with variable $r$}
% For variable compression ratio $r=r_t$, the encoder needs to generate output vectors of variable sizes $n=n_t$ where $t$ is the training time. To facilitate this into the autoencoder architecture, we add an extra dropout layer at the end of the encoder with a variable dropout probability $p_t = 1 - \frac{1}{r_t}$. On the decompressor side, we add an extra linear layer at the front of the decoder and scale the activations by $\frac{1}{1-p_t} = r_t$. This layer ensures that the expected input values to the decoder don't change with $r_t$ and thus the training is stable.




\subsection{Scheduler}
\label{appendix:scheduler}
Several strategies can be utilized to increase the compress rate as we learn. A simple strategy is to increase it a fixed rate $r_{k+1}=r_k+R$, where $R$ is the fixed rate. Another strategy is to implement linear increase $r_{k}=\alpha k+r_0$, where $\alpha>0$ is the increasing slope. Another strategy is to implement an exponential increase $r_k=\frac{1}{\beta^{K-k+1}}$, with $\beta$ being the base of the exponential increase, and $K$ the total number of steps. In all cases, the scheduler is a monotone-increasing function. 

In our experiments, we considered $6$ different types of variable compression mechanisms based on the following equation, 

\begin{align}
	c =\min\bigg(c_{max} - a \frac{ c_{max} - c_{min}}{K}k, c_{min} \bigg)
\end{align}
We considered the slope $a=\{2,3,4,5,6,7\}$ and in all cases $c_{max}=128$, $c_{min}=1$. 
%In Figure \ref{fig:comm_rate_per_epoch} we show the communication rate per epoch for each mechanism. In Figure \ref{fig:comm_floats_per_epoch}, compute the number of communicated floating points per edge in a GNN with feature size $128$ and $256$ activations in the middle layers. In Figure \ref{fig:aggregated_comm_floats} we compute the total number of floating points communicated throughout a whole training session of $300$ epochs.
% \begin{figure*}
% 	\begin{subfigure}{0.33\textwidth}
% 		\centering
% 		\includegraphics[width = \textwidth]{figures/schedulers/comm_rate_per_epoch.pdf} 
% 		\caption{Compression rate per epoch}
% 		\label{fig:comm_rate_per_epoch}
% 	\end{subfigure}%
% 	\begin{subfigure}{0.33\textwidth}
% 		\includegraphics[width = \textwidth]{figures/schedulers/comm_floats_per_epoch.pdf} 
% 		\caption{Floating points comm. per epoch}
% 		\label{fig:comm_floats_per_epoch}
% 	\end{subfigure}
% 	\begin{subfigure}{0.33\textwidth}
% 		\includegraphics[width = \textwidth]{figures/schedulers/comm_floats_aggregated.pdf} 
% 		\caption{Aggregated floating points}
% 		\label{fig:aggregated_comm_floats}
% 	\end{subfigure}
% 	\caption{Compression rate and floating point communicated per epoch.}
% \end{figure*}
% \begin{figure*}
% 	\begin{subfigure}{0.5\textwidth}
% 		\centering
% 		\includegraphics[width = \textwidth]{figures/AccVsEdges/Acc_vs_PercentCrossEdges_arxiv.pdf} 
% 		\caption{Arxiv}
% 		\label{fig:acc_vs_self_arxiv}
% 	\end{subfigure}%
% 	\begin{subfigure}{0.5\textwidth}
% 		\includegraphics[width = \textwidth]{figures/AccVsEdges/Acc_vs_PercentCrossEdges_prods.pdf} 
% 		\caption{Products}
% 		\label{fig:acc_vs_self_prods}
% 	\end{subfigure}
% 	\caption{Accuracy as a function of the percentage of self-edges.}
%         \label{fig:acc_vs_self}
% \end{figure*}

%\begin{figure*}[h]
%	\centering
%	\begin{subfigure}{0.33\textwidth}
%		\includegraphics[width = \textwidth]{figures/schedulers/comm_rate_per_epoch.pdf} 
%		\subcaption{Compression rate per epoch}
%		\label{fig:comm_rate_per_epoch}
%	\end{subfigure}
%	\begin{subfigure}{0.33\textwidth}
%		\includegraphics[width = \textwidth]{figures/schedulers/comm_floats_per_epoch.pdf} 
%		\subcaption{Floating points per epoch}
%		\label{fig:comm_floats_per_epoch}
%	\end{subfigure}
%	\caption{\ref{fig:comm_rate_per_epoch} Compression rate per epoch for the difference training mechanisms. \ref{fig:comm_floats_per_epoch} Floating point numbers communicated per epoch per edge between machines.}
%\end{figure*}



%\begin{figure}
%	\centering
%	\includegraphics[width =0.6 \textwidth]{figures/schedulers/comm_floats_aggregated.pdf} 
%	\caption{Accumulated number of communicated floating point numbers for the different training mechanisms.}
%	\label{fig:accuracy_epoch}
%\end{figure}



%\subsection{Robustness to Scheduler Selection}
%Our Algorithm $\algo$ shows a solid robustness to the choice of slope $a$, and $c_{min}$. We tested several choices of schedulers, and in all of them, the accuracy of the learned GNN matches the one of the no communication, at a fraction of the time. 


% Please add the following required packages to your document preamble:
% \usepackage{multirow}
% Please add the following required packages to your document preamble:
% \usepackage{multirow}
% Please add the following required packages to your document preamble:
% \usepackage{multirow}
% Please add the following required packages to your document preamble:
% \usepackage{multirow}
% Please add the following required packages to your document preamble:
% \usepackage{multirow}
% Please add the following required packages to your document preamble:
% \usepackage{multirow}
% Please add the following required packag
\section{Partitioning Details}
In all cases, the partitions had the same number of nodes in each partition. In Table \ref{table:edges} we show the number of edges in each server, and across servers. As can be seen, the number of cross edges in METIS partitioning is always smaller than random, which makes sense given the objective of the METIS algorithm. Another important aspect is that as the number of partitions increases, the cross-partition number of edges increases and correspondingly the self-partition decreases. This is why the degradation happens, local graphs are smaller, and more communication is needed.
% Please add the following required packages to your document preamble:
% \usepackage{multirow}
\begin{table*}
  \small
\centering
\begin{tabular}{cc|cccc|}
\hline
\multicolumn{1}{c|}{\multirow{3}{*}{\begin{tabular}[c]{@{}c@{}}Edge\\ Type\end{tabular}}} & \multirow{3}{*}{Partitioning} & \multicolumn{4}{c}{Number of Servers}                                                               \\ \cline{3-6} 
\multicolumn{1}{c|}{}                              &                               & \multicolumn{4}{c}{OGBN-Products}                                                                   \\ \cline{3-6} 
\multicolumn{1}{c|}{}                              &                               & \multicolumn{1}{c|}{$2$}     & \multicolumn{1}{c|}{$4$}     & \multicolumn{1}{c|}{$8$}     & \multicolumn{1}{c}{$16$}     \\ \hline
\multicolumn{1}{c|}{Self }                & METIS            & \multicolumn{1}{c|}{$122019051(96.71\%)$} & \multicolumn{1}{c|}{$118533121(93.95\%)$} & \multicolumn{1}{c|}{$113962769(90.33\%)$} & \multicolumn{1}{c}{$110067019(87.24\%)$} \\ \hline
\multicolumn{1}{c|}{Self }                & Random                    & \multicolumn{1}{c|}{$64302907(50.97\%)$} & \multicolumn{1}{c|}{$33378937(26.46\%)$} & \multicolumn{1}{c|}{$17913873(14.2\%)$} & \multicolumn{1}{c}{$10179253(8.07\%)$} \\ \hline
\multicolumn{1}{c|}{Cross }               & METIS                     & \multicolumn{1}{c|}{$4148258(3.29\%)$} & \multicolumn{1}{c|}{$7634188(6.05\%)$} & \multicolumn{1}{c|}{$12204540(9.67\%)$} & \multicolumn{1}{c}{$16100290(12.76\%)$} \\ \hline
\multicolumn{1}{c|}{Cross }               & Random                     & \multicolumn{1}{c|}{$61864402(49.03\%)$} & \multicolumn{1}{c|}{$92788372(73.54\%)$} & \multicolumn{1}{c|}{$108253436(85.8\%)$} & \multicolumn{1}{c}{$115988056(91.93\%)$} \\ \hline
\multicolumn{2}{c|}{}                                                               & \multicolumn{4}{c}{OGBN-Arxiv}                                                                      \\ \hline
\multicolumn{1}{c|}{Self }                & METIS                      & \multicolumn{1}{c|}{$2173087(87.45\%)$} & \multicolumn{1}{c|}{$2038291(82.03\%)$} & \multicolumn{1}{c|}{$1864471(75.03\%)$} & \multicolumn{1}{c}{$1677943(67.52\%)$} \\ \hline
\multicolumn{1}{c|}{Self }                & Random                    & \multicolumn{1}{c|}{$1326581(53.38\%)$} & \multicolumn{1}{c|}{$749367(30.16\%)$} & \multicolumn{1}{c|}{$459233(18.48\%)$} & \multicolumn{1}{c}{$314967(12.68\%)$} \\ \hline
\multicolumn{1}{c|}{Cross }               & METIS                      & \multicolumn{1}{c|}{$311854(12.55\%)$} & \multicolumn{1}{c|}{$446650(17.97\%)$} & \multicolumn{1}{c|}{$620470(24.97\%)$} & \multicolumn{1}{c}{$806998(32.48\%)$} \\ \hline
\multicolumn{1}{c|}{Cross }               & Random               & \multicolumn{1}{c|}{$1158360(46.62\%)$} & \multicolumn{1}{c|}{$1735574(69.84\%)$} & \multicolumn{1}{c|}{$2025708(81.52\%)$} & \multicolumn{1}{c}{$2169974(87.32\%)$} \\ \hline
\end{tabular}
\caption{\label{table:edges} Number of self-edges and cross-edges for the different settings considered. }

\end{table*}

% \section{Cross-Edge analysis}

% In this section, we study the impact of the number of cross-edges on the graph. We define a cross-edge, as an edge that connects two nodes that are in different partitions. 
% Note that the number of cross-edges is related to the number of partitions, the density of the graph, and the partition method. In Table \ref{table:edges} we show the number and percentage of cross-edges for each graph, partition method, and number of partitions. It can be seen that METIS has a significantly smaller number of cross-edges than random. It is also true, that the number of cross-edges increases with the number of partitions. Also note that OGBN-Products graph has an average degree of $25.26$, whereas OBGN-Papers $6.89$. This means that OBGN-Products graph is sparser than Arxiv, which explains why using METIS partitioning, the number of cross-edges is smaller in OGBN-Products. 

% In Figure \ref{fig:acc_vs_self}, we plot the accuracy as a function of the percentage of cross-edges. In this plot, we only considered the percentage of cross-edges, and therefore, the partition mechanism and number of partitions are not stated. But, as Table \ref{table:edges} indicates, the $4$ points with more cross-edges correspond to random partitions with $2,4,8,16$ partitions respectively. Likewise, the $4$ points with less cross-edges correspond to METIS partitioning with $2,4,8,16$ partitions respectively.



% A salient conclusion is that in all cases as the percentage of self-edges increases the accuracy deteriorates. This is related to the fact that as the number of cross-edges increases the local data is less representative of the whole graph. Another conclusion drawn from \ref{fig:acc_vs_self} is that there is an ordering in the performance from less compression (Fixed compression $2$) to no communication. This is related to the fact that less compression transmits more information over the cross-edges than no communication. The greater the number of cross-edges the role of the compression becomes more important. 

% Regarding the sparsity of the datasets, the two plots show different behaviors. In the sparser graph \ref{fig:acc_vs_self_arxiv}, as edges are added, the accuracy increases, almost linearly. This is related to the fact that the cross-edges are not redundant in the information they bring from far-away nodes. On the other hand, the denser graph \ref{fig:acc_vs_self_prods} shows a saturation around $50\%$. That is to say, as the number of cross-edges decreases below $50\%$, the improvement in accuracy is not significant. This is related to the density of the graph, as given its large degree, nodes tend to be redundant. 



\subsection{Accuracy}
The variable compression mechanism recovers the no communication accuracy in all cases considered. There is no difference in the accuracy obtained with variable compression, and full communication. This is true, for all numbers of servers considered, and all partitions as can be seen in Tables \ref{table:results_random}, and \ref{table:results_metis} for METIS and random partition respectively. 

% Please add the following required packages to your document preamble:
% \usepackage{multirow}
\begin{table*}
\centering
	\begin{tabular}{c|cccc|cccc}
		\hline
		\multirow{3}{*}{Algorithm}    & \multicolumn{4}{c|}{OGBN-Products}                                                                   & \multicolumn{4}{c}{OGBN-Arxiv}                                                                      \\ \cline{2-9} 
		& \multicolumn{4}{c}{Number of Servers}                                                               & \multicolumn{4}{c}{Number of Servers}                                                               \\ \cline{2-9} 
		& \multicolumn{1}{c|}{$2$}     & \multicolumn{1}{c|}{$4$}     & \multicolumn{1}{c|}{$8$}     & $16$    & \multicolumn{1}{c|}{$2$}     & \multicolumn{1}{c|}{$4$}     & \multicolumn{1}{c|}{$8$}     & $16$    \\ \hline
Full Comm& \multicolumn{1}{c|}{$78.40$} & \multicolumn{1}{c|}{$78.41$} & \multicolumn{1}{c|}{$78.31$} & \multicolumn{1}{c|}{$78.19$} & \multicolumn{1}{c|}{$69.20$} & \multicolumn{1}{c|}{$69.86$} & \multicolumn{1}{c|}{$69.75$} & \multicolumn{1}{c}{$69.16$} \\ \hline
No Comm& \multicolumn{1}{c|}{$77.08$} & \multicolumn{1}{c|}{$74.96$} & \multicolumn{1}{c|}{$72.22$} & \multicolumn{1}{c|}{$69.10$} & \multicolumn{1}{c|}{$64.67$} & \multicolumn{1}{c|}{$61.22$} & \multicolumn{1}{c|}{$55.95$} & \multicolumn{1}{c}{$54.52$} \\ \hline
\textbf{Variable Comp. Slope $2$(ours)}& \multicolumn{1}{c|}{$78.28$} & \multicolumn{1}{c|}{$78.32$} & \multicolumn{1}{c|}{$78.13$} & \multicolumn{1}{c|}{$78.20$} & \multicolumn{1}{c|}{$69.21$} & \multicolumn{1}{c|}{$69.30$} & \multicolumn{1}{c|}{$69.89$} & \multicolumn{1}{c}{$69.80$} \\ \hline
\textbf{Variable Comp. Slope $3$(ours)}& \multicolumn{1}{c|}{$78.56$} & \multicolumn{1}{c|}{$78.81$} & \multicolumn{1}{c|}{$78.61$} & \multicolumn{1}{c|}{$78.79$} & \multicolumn{1}{c|}{$69.24$} & \multicolumn{1}{c|}{$69.48$} & \multicolumn{1}{c|}{$70.21$} & \multicolumn{1}{c}{$70.07$} \\ \hline
\textbf{Variable Comp. Slope $4$(ours)}& \multicolumn{1}{c|}{$78.47$} & \multicolumn{1}{c|}{$78.81$} & \multicolumn{1}{c|}{$78.53$} & \multicolumn{1}{c|}{$78.64$} & \multicolumn{1}{c|}{$69.47$} & \multicolumn{1}{c|}{$69.51$} & \multicolumn{1}{c|}{$69.95$} & \multicolumn{1}{c}{$69.90$} \\ \hline
\textbf{Variable Comp. Slope $5$(ours)}& \multicolumn{1}{c|}{$78.67$} & \multicolumn{1}{c|}{$78.62$} & \multicolumn{1}{c|}{$78.14$} & \multicolumn{1}{c|}{$78.67$} & \multicolumn{1}{c|}{$69.93$} & \multicolumn{1}{c|}{$69.80$} & \multicolumn{1}{c|}{$70.12$} & \multicolumn{1}{c}{$69.99$} \\ \hline
\textbf{Variable Comp. Slope $6$(ours)}& \multicolumn{1}{c|}{$78.71$} & \multicolumn{1}{c|}{$78.66$} & \multicolumn{1}{c|}{$78.56$} & \multicolumn{1}{c|}{$78.55$} & \multicolumn{1}{c|}{$69.49$} & \multicolumn{1}{c|}{$69.59$} & \multicolumn{1}{c|}{$70.09$} & \multicolumn{1}{c}{$70.02$} \\ \hline
\textbf{Variable Comp. Slope $7$(ours)}& \multicolumn{1}{c|}{$78.38$} & \multicolumn{1}{c|}{$78.70$} & \multicolumn{1}{c|}{$78.45$} & \multicolumn{1}{c|}{$78.71$} & \multicolumn{1}{c|}{$69.21$} & \multicolumn{1}{c|}{$69.89$} & \multicolumn{1}{c|}{$69.90$} & \multicolumn{1}{c}{$69.81$} \\ \hline
Fixed Comp Rate $2$& \multicolumn{1}{c|}{$78.35$} & \multicolumn{1}{c|}{$78.13$} & \multicolumn{1}{c|}{$78.00$} & \multicolumn{1}{c|}{$77.85$} & \multicolumn{1}{c|}{$66.04$} & \multicolumn{1}{c|}{$64.97$} & \multicolumn{1}{c|}{$64.34$} & \multicolumn{1}{c}{$62.68$} \\ \hline
Fixed Comp Rate $4$& \multicolumn{1}{c|}{$78.60$} & \multicolumn{1}{c|}{$77.50$} & \multicolumn{1}{c|}{$76.08$} & \multicolumn{1}{c|}{$74.62$} & \multicolumn{1}{c|}{$66.30$} & \multicolumn{1}{c|}{$63.93$} & \multicolumn{1}{c|}{$63.79$} & \multicolumn{1}{c}{$62.21$} \\ \hline
% Fixed Comp Rate $8$& \multicolumn{1}{c|}{$78.98$} & \multicolumn{1}{c|}{$76.80$} & \multicolumn{1}{c|}{$72.91$} & \multicolumn{1}{c|}{$69.57$} & \multicolumn{1}{c|}{$65.83$} & \multicolumn{1}{c|}{$63.38$} & \multicolumn{1}{c|}{$61.81$} & \multicolumn{1}{c}{$61.37$} \\ \hline
% Fixed Comp Rate $16$& \multicolumn{1}{c|}{$78.15$} & \multicolumn{1}{c|}{$75.62$} & \multicolumn{1}{c|}{$70.73$} & \multicolumn{1}{c|}{$68.88$} & \multicolumn{1}{c|}{$66.90$} & \multicolumn{1}{c|}{$63.50$} & \multicolumn{1}{c|}{$63.39$} & \multicolumn{1}{c}{$61.06$} \\ \hline
% Fixed Comp Rate $32$& \multicolumn{1}{c|}{$78.04$} & \multicolumn{1}{c|}{$74.22$} & \multicolumn{1}{c|}{$71.42$} & \multicolumn{1}{c|}{$67.73$} & \multicolumn{1}{c|}{$66.78$} & \multicolumn{1}{c|}{$65.66$} & \multicolumn{1}{c|}{$60.65$} & \multicolumn{1}{c}{$61.66$} \\ \hline
% Fixed Comp Rate $64$& \multicolumn{1}{c|}{$78.10$} & \multicolumn{1}{c|}{$75.11$} & \multicolumn{1}{c|}{$69.51$} & \multicolumn{1}{c|}{$65.86$} & \multicolumn{1}{c|}{$65.79$} & \multicolumn{1}{c|}{$63.79$} & \multicolumn{1}{c|}{$62.15$} & \multicolumn{1}{c}{$62.77$} \\ \hline
	\end{tabular}
\caption{\label{table:results_random} Accuracy results when training GNNs with full-communication, no communication, fixed and variable compression in both OGBN-Arxiv, and OGBN-Products. We test our Algorithm with $2,4,8$ and $16$ clients with \textbf{random partitioning} of the graph. }
\end{table*}

% Please add the following required packages to your document preamble:
% \usepackage{multirow}
\begin{table*}
\centering
\begin{tabular}{c|cccccccc}
\hline
\multirow{3}{*}{Algorithm}    & \multicolumn{4}{c}{OGBN-Products}                                                                                        & \multicolumn{4}{|c}{OGBN-Arxiv}                                                                      \\ \cline{2-9} 
                              & \multicolumn{8}{c}{Number of Servers}                                                                                                                                                                                           \\ \cline{2-9} 
                              & \multicolumn{1}{c|}{$2$}     & \multicolumn{1}{c|}{$4$}     & \multicolumn{1}{c|}{$8$}     & \multicolumn{1}{c|}{$16$}    & \multicolumn{1}{c|}{$2$}     & \multicolumn{1}{c|}{$4$}     & \multicolumn{1}{c|}{$8$}     & $16$    \\ \hline
Full Comm& \multicolumn{1}{c|}{$78.61$} & \multicolumn{1}{c|}{$78.79$} & \multicolumn{1}{c|}{$78.53$} & \multicolumn{1}{c|}{$78.25$} & \multicolumn{1}{c|}{$68.81$} & \multicolumn{1}{c|}{$69.63$} & \multicolumn{1}{c|}{$68.88$} & \multicolumn{1}{c}{$68.98$} \\ \hline
No Comm& \multicolumn{1}{c|}{$78.01$} & \multicolumn{1}{c|}{$78.39$} & \multicolumn{1}{c|}{$78.13$} & \multicolumn{1}{c|}{$77.62$} & \multicolumn{1}{c|}{$67.90$} & \multicolumn{1}{c|}{$66.77$} & \multicolumn{1}{c|}{$67.69$} & \multicolumn{1}{c}{$65.63$} \\ \hline
\textbf{Variable Comp. Slope $2$(ours)}& \multicolumn{1}{c|}{$78.43$} & \multicolumn{1}{c|}{$78.12$} & \multicolumn{1}{c|}{$78.80$} & \multicolumn{1}{c|}{$78.57$} & \multicolumn{1}{c|}{$69.25$} & \multicolumn{1}{c|}{$68.72$} & \multicolumn{1}{c|}{$69.08$} & \multicolumn{1}{c}{$69.14$} \\ \hline
\textbf{Variable Comp. Slope $3$(ours)}& \multicolumn{1}{c|}{$78.50$} & \multicolumn{1}{c|}{$78.26$} & \multicolumn{1}{c|}{$78.21$} & \multicolumn{1}{c|}{$78.31$} & \multicolumn{1}{c|}{$69.89$} & \multicolumn{1}{c|}{$69.48$} & \multicolumn{1}{c|}{$68.86$} & \multicolumn{1}{c}{$69.58$} \\ \hline
\textbf{Variable Comp. Slope $4$(ours)}& \multicolumn{1}{c|}{$78.73$} & \multicolumn{1}{c|}{$78.01$} & \multicolumn{1}{c|}{$78.13$} & \multicolumn{1}{c|}{$78.58$} & \multicolumn{1}{c|}{$69.39$} & \multicolumn{1}{c|}{$68.96$} & \multicolumn{1}{c|}{$68.75$} & \multicolumn{1}{c}{$69.16$} \\ \hline
\textbf{Variable Comp. Slope $5$(ours)}& \multicolumn{1}{c|}{$78.59$} & \multicolumn{1}{c|}{$78.51$} & \multicolumn{1}{c|}{$78.03$} & \multicolumn{1}{c|}{$78.35$} & \multicolumn{1}{c|}{$69.49$} & \multicolumn{1}{c|}{$69.04$} & \multicolumn{1}{c|}{$69.11$} & \multicolumn{1}{c}{$68.78$} \\ \hline
\textbf{Variable Comp. Slope $6$(ours)}& \multicolumn{1}{c|}{$78.21$} & \multicolumn{1}{c|}{$78.61$} & \multicolumn{1}{c|}{$78.50$} & \multicolumn{1}{c|}{$78.68$} & \multicolumn{1}{c|}{$69.33$} & \multicolumn{1}{c|}{$68.60$} & \multicolumn{1}{c|}{$69.41$} & \multicolumn{1}{c}{$69.36$} \\ \hline
\textbf{Variable Comp. Slope $7$(ours)}& \multicolumn{1}{c|}{$78.32$} & \multicolumn{1}{c|}{$78.39$} & \multicolumn{1}{c|}{$78.36$} & \multicolumn{1}{c|}{$78.56$} & \multicolumn{1}{c|}{$69.24$} & \multicolumn{1}{c|}{$69.28$} & \multicolumn{1}{c|}{$68.12$} & \multicolumn{1}{c}{$69.05$} \\ \hline
Fixed Comp Rate $2$& \multicolumn{1}{c|}{$78.49$} & \multicolumn{1}{c|}{$78.49$} & \multicolumn{1}{c|}{$78.18$} & \multicolumn{1}{c|}{$78.46$} & \multicolumn{1}{c|}{$67.68$} & \multicolumn{1}{c|}{$67.76$} & \multicolumn{1}{c|}{$66.73$} & \multicolumn{1}{c}{$65.78$} \\ \hline
Fixed Comp Rate $4$& \multicolumn{1}{c|}{$78.62$} & \multicolumn{1}{c|}{$78.31$} & \multicolumn{1}{c|}{$78.42$} & \multicolumn{1}{c|}{$78.34$} & \multicolumn{1}{c|}{$67.75$} & \multicolumn{1}{c|}{$66.64$} & \multicolumn{1}{c|}{$66.96$} & \multicolumn{1}{c}{$66.09$} \\ \hline
% Fixed Comp Rate $8$& \multicolumn{1}{c|}{$78.44$} & \multicolumn{1}{c|}{$78.24$} & \multicolumn{1}{c|}{$78.83$} & \multicolumn{1}{c|}{$78.72$} & \multicolumn{1}{c|}{$68.16$} & \multicolumn{1}{c|}{$67.97$} & \multicolumn{1}{c|}{$66.09$} & \multicolumn{1}{c}{$67.19$} \\ \hline
% Fixed Comp Rate $16$& \multicolumn{1}{c|}{$78.71$} & \multicolumn{1}{c|}{$78.72$} & \multicolumn{1}{c|}{$78.39$} & \multicolumn{1}{c|}{$78.44$} & \multicolumn{1}{c|}{$68.32$} & \multicolumn{1}{c|}{$67.65$} & \multicolumn{1}{c|}{$67.13$} & \multicolumn{1}{c}{$67.08$} \\ \hline
% Fixed Comp Rate $32$& \multicolumn{1}{c|}{$78.65$} & \multicolumn{1}{c|}{$78.67$} & \multicolumn{1}{c|}{$78.45$} & \multicolumn{1}{c|}{$78.03$} & \multicolumn{1}{c|}{$68.98$} & \multicolumn{1}{c|}{$66.68$} & \multicolumn{1}{c|}{$67.11$} & \multicolumn{1}{c}{$67.01$} \\ \hline
% Fixed Comp Rate $64$& \multicolumn{1}{c|}{$78.55$} & \multicolumn{1}{c|}{$78.48$} & \multicolumn{1}{c|}{$78.34$} & \multicolumn{1}{c|}{$78.68$} & \multicolumn{1}{c|}{$67.88$} & \multicolumn{1}{c|}{$67.56$} & \multicolumn{1}{c|}{$67.81$} & \multicolumn{1}{c}{$67.00$} \\ \hline
\end{tabular}
\caption{\label{table:results_metis} Accuracy results when training GNNs with full-communication, no communication, fixed and variable compression in both OGBN-Arxiv, and OGBN-Products. We test our Algorithm with $2,4,8$ and $16$ clients with \textbf{METIS partitioning} of the graph. }
\end{table*}


\subsection{Proof of Proposition \ref{prop:fixed_compression}}
\label{appendix:proposition_fixed_compression}
To begin with, we need to show these three lemmas. 

\begin{lemma}[GNN Function Difference]\label{lemma:func_diff}
	Under the assumptions of Proposition \ref{prop:fixed_compression}, the output of an $L$-layer GNN with $F$ and coefficients and $K$ filter taps per layer can be bounded by,
	\begin{align}
		||\Phi(\bbX_1,\bbS;\ccalH) - \Phi(\bbX_2,\bbS;\ccalH)  ||\leq \lambda_{\max}^L ||\bbX_1-\bbX_2||
	\end{align}
\end{lemma}
\begin{proof}[of Lemma \ref{lemma:func_diff}]
	Starting with the first layer of the GNN, and considering a single feature $||\bbx_{l1}-\bbx_{l2}||$, we can look into the difference between the successive layers as follows, 
	\begin{align}
		||\bbX_{l1}-\bbX_{l2}|| =& ||\non\bigg(\sum_{k=0}^{K-1}\bbH_k \bbS^k\bbx_{1}\bigg)-\non\bigg(\sum_{k=0}^{K-1}\bbH_k \bbS^k\bbX_{2}\bigg)||\\
		\leq& ||\sum_{k=0}^{K-1}\bbH_k \bbS^k\bbX_{1}-\sum_{k=0}^{K-1}\bbH_k \bbS^k\bbX_{2}||\label{eqn:normalized_lips}\\
  % \text{ normalized Lipschitz assumption \ref{as:normalized_lipschitz}}\\
		\leq& \lambda_{\max}||\bbX_{1}-\bbX_{2}|| \label{eqn:normalized_lips_filters}%\text{ normalized filters assumption \ref{as:filter_bounded}}
	\end{align}
 Where \eqref{eqn:normalized_lips} holds by normalized Lipschitz assumption \ref{as:normalized_lipschitz}, and \eqref{eqn:normalized_lips_filters} holds by the normalized filters assumption \ref{as:filter_bounded}
	By repeating the recursion over $L$ layers we attain the desired result.
\end{proof}
%%%%%%%%%%%%%%%%%%%%%%%%%%%%%%%%

\begin{lemma}[GNN Gradient Difference]\label{lemma:grad_diff} Under the assumptions of Proposition \ref{prop:fixed_compression}, the output of an $L$-layer GNN with $F$ and coefficients and $K$ filter taps per layer can be bounded by,
	\begin{align}
		&||\nabla_\ccalH\Phi(\bbX_1,\bbS;\ccalH) - \nabla_\ccalH\Phi(\bbX_2,\bbS;\ccalH)  ||\\
  &\leq 2\lambda_{\max} \sqrt{KFL} ||\bbX_1-\bbX_2|| \nonumber
	\end{align}
\end{lemma}
\begin{proof}[of Lemma \ref{lemma:grad_diff}]
	Starting with the first layer of the GNN, note that the derivative of the GNN with respect to any parameter in the first layer is the value of the polynomial. By denoting $h_v$ an element on the first layer of the GNN, the derivative with respect to the first layer is, 
	\begin{align}
		\nabla_{h_v}\non\bigg(\sum_{k=0}^{K-1}\bbH_k \bbS^k\bbX_{1}\bigg)=\non\bigg(\sum_{k=0}^{K-1}\bbH_k \bbS^k\bbX_{1}\bigg)  \bbS^k\bbX_{1}.
	\end{align}
	By taking the difference we get, 
	\begin{align}
		&||\nabla_{k_v}\non\bigg(\sum_{k=0}^{K-1}\bbH_k \bbS^k\bbx_{1}\bigg)-\nabla_{k_v}\non\bigg(\sum_{k=0}^{K-1}\bbH_k \bbS^k\bbX_{2}\bigg)||\\
		&=||\non\bigg(\sum_{k=0}^{K-1}\bbH_k \bbS^k\bbX_{1}\bigg)  \bbS^k\bbX_{1}-\non\bigg(\sum_{k=0}^{K-1}\bbH_k \bbS^k\bbX_{2}\bigg)  \bbS^k\bbX_{2}||\\
		&\leq ||\non\bigg(\sum_{k=0}^{K-1}\bbH_k \bbS^k\bbX_{1}\bigg) \bigg(  \bbS^k\bbX_{1}- \bbS^k\bbX_{2}\bigg)||\\
  % \text{ triangle inequality }\\
		&+||\bigg(\non\bigg(\sum_{k=0}^{K-1}\bbH_k \bbS^k\bbX_{1}\bigg) -\non\bigg( \sum_{k=0}^{K-1}\bbH_k \bbS^k\bbX_{2}\bigg)\bigg)  \bigg( \bbS^k\bbX_{2}\bigg)||
	\end{align}
	Now, given that the activation is normalized Lipschitz by assumption \ref{as:normalized_lipschitz}, the signals are normalized, and that the filter is normalized by assumption \ref{as:filter_bounded}, we can bound this term by, 
	\begin{align}
		&||\nabla_{k_v}\non\bigg(\sum_{k=0}^{K-1}\bbH_k \bbS^k\bbX_{1}\bigg)-\nabla_{k_v}\non\bigg(\sum_{k=0}^{K-1}\bbH_k \bbS^k\bbX_{2}\bigg)||\nonumber\\
  &\leq 2 \lambda_{\max}||\bbX_1 - \bbX_2 || 
	\end{align}
	By repeating the previous result for all layers, and all features and considering that the GNN has $KFL$ coefficients, we complete the proof. 
\end{proof}

%%%%%%%%%%%%%%%%%%%%%%%%%%%%%%%%
\begin{lemma}[Lipschitz Gradients with respect to the parameters]\label{lemma:lipschitz_loss_wrt_params}Under the assumptions of Proposition \ref{prop:fixed_compression}, the output of an $L$-layer GNN with $F$ and coefficients and $K$ filter taps per layer can be bounded by,
	\begin{align}
		&||\nabla_\ccalH\ell(\bby,\Phi(\bbx,\bbS;\ccalH_1)) - \nabla_\ccalH\ell(\bby,\Phi(\bbx,\bbS;\ccalH_2))  ||\nonumber\\
  &\leq 2ML ||\ccalH_1-\ccalH_2||
	\end{align}
\end{lemma}
\begin{proof}[of Lemma \ref{lemma:lipschitz_loss_wrt_params}] We begin by using the chain rule as follows, 
	\begin{align}
		&||\nabla_\ccalH\ell(\bby,\Phi(\bbx,\bbS;\ccalH_1)) - \nabla_\ccalH\ell(\bby,\Phi(\bbx,\bbS;\ccalH_2))  ||\\
		&=||\nabla\ell(\bby,\Phi(\bbx,\bbS;\ccalH_1))\nabla_\ccalH\Phi(\bbx,\bbS;\ccalH_1)\nonumber\\
  &- \nabla\ell(\bby,\Phi(\bbx,\bbS;\ccalH_2))\nabla_\ccalH\Phi(\bbx,\bbS;\ccalH_2)  ||\\
		&\leq||\bigg(\nabla\ell(\bby,\Phi(\bbx,\bbS;\ccalH_1)) \nonumber \\&-\nabla\ell(\bby,\Phi(\bbx,\bbS;\ccalH_2))\bigg)\nabla_\ccalH\Phi(\bbx,\bbS;\ccalH_2)  ||\\% \text{ triangle inequality }\\
		&+||\bigg(\nabla_\ccalH\Phi(\bbx_1,\bbS;\ccalH)-\nabla_\ccalH\Phi(\bbx,\bbS;\ccalH_2)\bigg)\nonumber\\
  &\nabla\ell(\bby,\Phi(\bbx,\bbS;\ccalH_1))|| 
	\end{align}
	Note that we consider the filters $\ccalH$ as a vector, where the coefficients have been concatenated. We can now use Cauchy-Schwartz to obtain, 
	\begin{align}
		&||\nabla_\ccalH\ell(\bby,\Phi(\bbx,\bbS;\ccalH_1)) - \nabla_\ccalH\ell(\bby,\Phi(\bbx,\bbS;\ccalH_2))  ||\\
		&\leq||\nabla\ell(\bby,\Phi(\bbx,\bbS;\ccalH_1))- \nabla\ell(\bby,\Phi(\bbx,\bbS;\ccalH_2))|| \nonumber\\
  &||\nabla_\ccalH\Phi(\bbx,\bbS;\ccalH_2)  || \\
		&+||\nabla_\ccalH\Phi(\bbx,\bbS;\ccalH_1)-\nabla_\ccalH\Phi(\bbx,\bbS;\ccalH_2)|| \nonumber\\
  &||\nabla\ell(\bby,\Phi(\bbx,\bbS;\ccalH_1))|| .
	\end{align}
	We can now use Assumptions \ref{as:Loss_Grad_Lipschitz}, and \ref{as:GNN_lipschitz}, to obtain
	\begin{align}
		&||\nabla_\ccalH\ell(\bby,\Phi(\bbx,\bbS;\ccalH_1)) - \nabla_\ccalH\ell(\bby,\Phi(\bbx,\bbS;\ccalH_2))  ||\nonumber\\
  &\leq 2ML ||\ccalH_1-\ccalH_2||
	\end{align}
	By denoting $L_\nabla = 2ML$ we complete the proof. 
	% Now we can use Assummption \ref{as:Loss_Grad_Lipschitz} and Assumption \ref{lemma:lipschitz_loss_wrt_params} for the first term, and Lemma \ref{lemma:grad_diff} for the second term to obtain, 
	% \begin{align}
		%     &||\nabla_\ccalH\ell(\bby,\Phi(\bbx_1,\bbS;\ccalH)) - \nabla_\ccalH\ell(\bby,\Phi(\bbx_2,\bbS;\ccalH))  ||\\
		% &\leq||\nabla\ell(\bby,\Phi(\bbx_1,\bbS;\ccalH))- \nabla\ell(\bby,\Phi(\bbx_2,\bbS;\ccalH))|| ||\nabla_\ccalH\Phi(\bbx_2,\bbS;\ccalH)  || \\
		% &+||\nabla_\ccalH\Phi(\bbx_1,\bbS;\ccalH)-\nabla_\ccalH\Phi(\bbx_2,\bbS;\ccalH)|| ||\nabla\ell(\bby,\Phi(\bbx_1,\bbS;\ccalH))|| 
		% \end{align}
\end{proof}

\begin{lemma}[Submartingale]\label{lemma:submartingale} 
	Consider the iterates generated by equation \ref{eqn:SGD} where the input vector $\bbx$ is compressed with error $\epsilon$ (cf. Definition \ref{eqn:compress_decompress}). Let the step-size  be $\eta\leq 1/\lipGrad$, if the compression error is such that, 
	\begin{align}\label{eqn:prop_submartingale_condition}
		\mbE_\ccalD[||\nabla_\ccalH \ell (y,\Phi(x,\bbS;\ccalH_t)) ||^2] \geq \lipGrad^2\epsilon^2
	\end{align}
	then the iterates satisfy that, 
	\begin{align}
		\mbE[\ell(y,\Phi(x,\bbS;\ccalH_{t+1}))] \leq \mbE[\ell (y,\Phi(x,\bbS;\ccalH_t))]
	\end{align}
\end{lemma}

\begin{proof}[of Lemma \ref{lemma:submartingale}]
	This proof follows the lines of \cite{bertsekas2000gradient}, and we start by defining a continuous function $g(\alpha)$ as follows, 
	\begin{align}
		g(\alpha)=\mbE[\ell(\bby,\Phi(\bbx,\bbS;\ccalH_t-\alpha\eta_t\nabla\ell(\bby,\Phi(\tilde \bbx,\bbS;\ccalH_t))))].
	\end{align}
	Note that, $g(0)=\mbE[\ell(\bby,\Phi(\bbx,\bbS;\ccalH_t))]$ and 
 
 $g(1)=\mbE[\ell(\bby,\Phi(\bbx,\bbS;\ccalH_{k+1}))]$, and also that the integral of $\frac{\partial}{\partial \alpha} g(\alpha)$ satisfies
	\begin{align}
		&g(1)-g(0)\nonumber\\
  &= \int_{0}^1 \frac{\partial}{\partial \alpha} g(\alpha) d\alpha \\
		&=  - \mbE[\eta\nabla_\ccalH\ell(\bby,\Phi(\tilde \bbx,\bbS;\ccalH_t))^\intercal \int_{0}^1\nabla_\ccalH\ell(\bby,\Phi(\bbx,\bbS;\ccalH_t\nonumber\\
  &-\alpha\eta\nabla_\ccalH\ell(\bby,\Phi(\tilde \bbx,\bbS;\ccalH_t))))d\alpha]\label{eqn:prop_submartingale_chain_rule}\\
		&=  - \mbE[\eta\nabla_\ccalH\ell(\bby,\Phi(\tilde \bbx,\bbS;\ccalH_t))^\intercal \int_{0}^1\nabla_\ccalH\ell(\bby,\Phi(\bbx,\bbS;\ccalH_t\nonumber\\
  &-\alpha\eta\nabla_\ccalH\ell(\bby,\Phi(\tilde \bbx,\bbS;\ccalH_t))))\\
		&+\nabla_\ccalH\ell(\bby,\Phi(\bbx,\bbS;\ccalH_t))-\nabla_\ccalH\ell(\bby,\Phi(\bbx,\bbS;\ccalH_t))d\alpha
		]\label{eqn:prop_submartingale_chain_rule_add_subtract}\\
		&=- \mbE[\eta\nabla_\ccalH\ell(\bby,\Phi(\tilde \bbx,\bbS;\ccalH_t)) ^\intercal\nabla_\ccalH\ell(\bby,\Phi(\bbx,\bbS;\ccalH_t))\nonumber\\
		&+\eta\nabla_\ccalH\ell(\bby,\Phi(\tilde \bbx,\bbS;\ccalH_t)) ^\intercal\int_{0}^1\nabla_\ccalH\ell(\bby,\Phi(\bbx,\bbS;\ccalH_t\nonumber\\
  &-\alpha\eta\nabla_\ccalH\ell(\bby,\Phi(\tilde \bbx,\bbS;\ccalH_t))))\nonumber\\
		&\quad\quad\quad\quad\quad-\nabla_\ccalH\ell(\bby,\Phi(\bbx,\bbS;\ccalH_t))d\alpha
		]\label{eqn:prop_submartingale_chain_rule_add_subtract_organize},
	\end{align}
	where \eqref{eqn:prop_submartingale_chain_rule} holds by the chain rule, \eqref{eqn:prop_submartingale_chain_rule_add_subtract} holds as we are adding and subtracting the same term, and \eqref{eqn:prop_submartingale_chain_rule_add_subtract_organize} is a rearrangement of terms. We can now utilize Cauchy-Schwartz to bound the difference as follows, 
	
	\begin{align}
		&\mbE_\ccalD[\ell(\bby,\Phi(\bbx,\bbS;\ccalH_{k+1}))-\ell(\bby,\Phi(\bbx,\bbS;\ccalH_t))]\nonumber\\
		&\leq - \mbE[\eta\nabla_\ccalH\ell(\bby,\Phi(\tilde \bbx,\bbS;\ccalH_t)) ^\intercal\nabla_\ccalH\ell(\bby,\Phi(\bbx,\bbS;\ccalH_t))\\
		&+\frac{\eta }{2}||\nabla_\ccalH\ell(\bby,\Phi(\tilde\bbx,\bbS;\ccalH_t))|| || \nabla_\ccalH\ell(\bby,\Phi(\bbx,\bbS;\ccalH_t\nonumber\\
  &-\eta\nabla_\ccalH\ell(\bby,\Phi(\tilde \bbx,\bbS;\ccalH_t))))\nonumber\\
  &-\nabla_\ccalH\ell(\bby,\Phi(\bbx,\bbS;\ccalH_t))||.\nonumber 
	\end{align}
	where the previous inequality holds given that $\int_0^1 \alpha^2 d\alpha=\frac{1}{2}$. We can utilize Lemma \ref{lemma:lipschitz_loss_wrt_params} to bound the difference between the gradients as follows, 
	\begin{align}
		&\mbE[\ell(\bby,\Phi(\bbx,\bbS;\ccalH_{k+1}))-\ell(\bby,\Phi(\bbx,\bbS;\ccalH_t))]\\
		&\leq \mbE[-\eta\nabla_\ccalH\ell(\bby,\Phi(\tilde \bbx,\bbS;\ccalH_t)) ^\intercal\nabla_\ccalH\ell(\bby,\Phi(\bbx,\bbS;\ccalH_t))\nonumber\\
  &+\frac{\lipGrad\eta ^2}{2}||\nabla_\ccalH\ell(\bby,\Phi(\tilde\bbx,\bbS;\ccalH_t))||^2]\nonumber .
	\end{align}
	% \red{hereRESUME}
	% Now, we can rearrange as follows,
	% \begin{align}
		%    &\mbE[\ell(\bby,\Phi(\bbx,\bbS;\ccalH_{k+1}))-\ell(\bby,\Phi(\bbx,\bbS;\ccalH_t))]\\
		%    &\leq - \mbE[2\langle\sqrt{\frac{\lipGrad\eta ^2}{2}}\nabla_\ccalH\ell(\bby,\Phi(\tilde \bbx,\bbS;\ccalH_t)) , \sqrt{\frac{2}{\lipGrad}}\nabla_\ccalH\ell(\bby,\Phi(\bbx,\bbS;\ccalH_t)) \rangle\\
		%    &+\langle\sqrt{\frac{\lipGrad\eta ^2}{2}}\nabla_\ccalH\ell(\bby,\Phi(\tilde\bbx,\bbS;\ccalH_t)),\sqrt{\frac{\lipGrad\eta ^2}{2}}\nabla_\ccalH\ell(\bby,\Phi(\tilde\bbx,\bbS;\ccalH_t))\rangle]\nonumber .
		% \end{align}
	% Knowing that for any two vectors, $\bba,\bbb$, $||\bba-\bbb||^2-||\bbb||^2=||\bba||^2-2\bba^\intercal\bbb$ given that the norm is induced by the inner product we obtain, 
	% \begin{align}
		%    &\mbE[\ell(\bby,\Phi(\bbx,\bbS;\ccalH_{k+1}))-\ell(\bby,\Phi(\bbx,\bbS;\ccalH_t))]\\
		%    &\leq - \frac{2}{\lipGrad}\mbE[||\nabla_\ccalH\ell(\bby,\Phi(\bbx,\bbS;\ccalH_t))||^2-||\frac{\lipGrad\eta }{2}\nabla_\ccalH\ell(\bby,\Phi(\tilde \bbx,\bbS;\ccalH_t)) - \nabla_\ccalH\ell(\bby,\Phi(\bbx,\bbS;\ccalH_t)) ||^2]\nonumber .
		% \end{align}
	
	
	% \red{here}
	
	Now, we can factor $-\eta/2$, and we obtain, 
	\begin{align}
		&\mbE[\ell(\bby,\Phi(\bbx,\bbS;\ccalH_{k+1}))-\ell(\bby,\Phi(\bbx,\bbS;\ccalH_t))]\\
		&\leq  \frac{-\eta}{2}\mbE[2\nabla_\ccalH\ell(\bby,\Phi(\tilde \bbx,\bbS;\ccalH_t)) ^\intercal\nabla_\ccalH\ell(\bby,\Phi(\bbx,\bbS;\ccalH_t))\nonumber\\
  &-||\nabla_\ccalH\ell(\bby,\Phi(\tilde\bbx,\bbS;\ccalH_t))||^2] \\
		&+\mbE[\frac{\lipGrad\eta ^2-\eta}{2}||\nabla_\ccalH\ell(\bby,\Phi(\tilde\bbx,\bbS;\ccalH_t))||^2]\nonumber .
	\end{align}
	Now by imposing the condition that $\eta<\frac{1}{\lipGrad}$, the second term can be ignored. Knowing that for any two vectors, $\bba,\bbb$, $||\bba-\bbb||^2-||\bbb||^2=||\bba||^2-2\bba^\intercal\bbb$ given that the norm is induced by the inner product we obtain, 
	\begin{align}
		&\mbE[\ell(\bby,\Phi(\bbx,\bbS;\ccalH_{k+1}))-\ell(\bby,\Phi(\bbx,\bbS;\ccalH_t))]\nonumber\\
		&\leq  \frac{-\eta}{2}\mbE[||\nabla_\ccalH\ell(\bby,\Phi(\bbx,\bbS;\ccalH_t))||^2\nonumber\\
  &-||\nabla_\ccalH\ell(\bby,\Phi(\tilde\bbx,\bbS;\ccalH_t))-\nabla_\ccalH\ell(\bby,\Phi(\bbx,\bbS;\ccalH_t))||^2] \nonumber .
	\end{align}
%	Now, we can partition the last element by adding and subtracting $\nabla_\ccalH\ell(\bby,\Phi(\bbx,\bbS;\ccalH_t))$ as follows, 
%\begin{align}
%&\mbE[\ell(\bby,\Phi(\bbx,\bbS;\ccalH_{k+1}))-\ell(\bby,\Phi(\bbx,\bbS;\ccalH_t))]\label{eqn:add_subtract}\\
%		&\leq  \frac{-\eta}{2}\bigg(\mbE[||\nabla_\ccalH\ell(\bby,\Phi(\bbx,\bbS;\ccalH_t))||^2] \nonumber\\
%  &-\mbE[||\nabla_\ccalH\ell(\bby,\Phi(\tilde\bbx,\bbS;\ccalH_t))-\nabla_\ccalH\ell(\bby,\Phi(\bbx,\bbS;\ccalH_t))+\nabla_\ccalH\ell(\bby,\Phi(\bbx,\bbS;\ccalH_t))-\nabla_\ccalH\ell(\bby,\Phi(\bbx,\bbS;\ccalH_t))||^2]\bigg)\nonumber .
%	\end{align}
% Now note that we consider the vectorized tensor $\ccalH$, and the norm in \ref{eqn:add_subtract} is induced by the innner product. Therefore, we can use the property $||a+b||^2\leq 3||a||^2 + 3||b||^2$, as follows,  
%	\begin{align}
%		&\mbE[\ell(\bby,\Phi(\bbx,\bbS;\ccalH_{k+1}))-\ell(\bby,\Phi(\bbx,\bbS;\ccalH_t))]\\
%		&\leq  \frac{-\eta}{2}\bigg(\mbE[||\nabla_\ccalH\ell(\bby,\Phi(\bbx,\bbS;\ccalH_t))||^2]-3\mbE[||\nabla_\ccalH\ell(\bby,\Phi(\tilde\bbx,\bbS;\ccalH_t))-\nabla_\ccalH\ell(\bby,\Phi(\bbx,\bbS;\ccalH_t))||^2] \\
%		&-3\mbE[||\nabla_\ccalH\ell(\bby,\Phi(\bbx,\bbS;\ccalH_t))-\nabla_\ccalH\ell(\bby,\Phi(\bbx,\bbS;\ccalH_t))||^2]\bigg)\nonumber ,
%	\end{align}
%	note that the cross terms are equal to zero given that $\mbE[\nabla_\ccalH\ell(\bby,\Phi(\bbx,\bbS;\ccalH_t))]=\mbE[\nabla_\ccalH \ell(\bby,\Phi(\bbx,\bbS;\ccalH_t))]$. 
 Finally, by Lemma \ref{lemma:grad_diff}, and compression mechanism \ref{def:CompressionDecompression}, 
	\begin{align}
		&\mbE[\ell(\bby,\Phi(\bbx,\bbS;\ccalH_{k+1}))-\ell(\bby,\Phi(\bbx,\bbS;\ccalH_t))]\\
		&\leq  \frac{-\eta}{2}\bigg(\mbE[||\nabla_\ccalH\ell(\bby,\Phi(\bbx,\bbS;\ccalH_t))||^2]-\lipGrad^2\epsilon^2\bigg)\nonumber .
	\end{align}
	
	
	By imposing the condition in \ref{eqn:prop_submartingale_condition} we complete the proof. 
\end{proof}


\begin{proof}[of Proposition \ref{prop:fixed_compression}]
	To begin with, for every $\beta$ we define the stopping time $K$ as
	\begin{align}
		K=\min_{k\geq 0} \{\mbE[||\nabla_\ccalH \ell(y,\Phi(\bbx,\bbS;\ccalH_t)) ||^2\leq \lipGrad^2\epsilon_k^2 +\beta^2]\}
	\end{align}
	We need to show that $\mbE[k^*]$ is of order $\ccalO(1/\beta)$. To do so, we start by taking the difference between the last iterate $K$ and the first one as follows, 
	\begin{align}
		&\mbE[\ell(\bby,\Phi(\bbx,\bbS;\ccalH_0))-\ell(\bby,\Phi(\bbx,\bbS;\ccalH_t))]\\
		&=\mbE_{K}[\mbE[\sum_{k=1}^K\ell(\bby,\Phi(\bbx,\bbS;\ccalH_{k-1}))-\ell(\bby,\Phi(\bbx,\bbS;\ccalH_t)) ]]\\
		&=\sum_{t=0}^\infty\mbE[\sum_{k=1}^t\ell(\bby,\Phi(\bbx,\bbS;\ccalH_{k-1}))\nonumber\\
  &-\ell(\bby,\Phi(\bbx,\bbS;\ccalH_t)) ]P(K=t)\label{eqn:propostion_convergence_termwise_summation}
	\end{align}
	Now, we know that for all $t\leq K$, we have that, 
	\begin{align}
		\mbE[\sum_{k=1}^K\ell(\bby,\Phi(\bbx,\bbS;\ccalH_{k-1}))-\ell(\bby,\Phi(\bbx,\bbS;\ccalH_t)) ]\geq \eta \beta .\label{eqn:propostion_convergence_difference_beta}
	\end{align}
	We can now substitute condition \ref{eqn:propostion_convergence_difference_beta} into equation \ref{eqn:propostion_convergence_termwise_summation} to obtain, 
	\begin{align}
		&\mbE[\ell(\bby,\Phi(\bbx,\bbS;\ccalH_0))-\ell(\bby,\Phi(\bbx,\bbS;\ccalH_t))]\\
		&\geq\sum_{t=0}^\infty \eta \beta K P(K=t)\geq\beta \eta \mbE[K].
	\end{align}
	Given that the loss function is non-negative, and dividing in both sides of the previous inequality by $\beta \eta$, we complete the proof.
	
\end{proof}



\section{Proof of Proposition \ref{prop:scheduler}}\label{appendix:proof_scheduler}
The sketch of this proof is as follows, first, we construct a martingale by multiplying the norm of the gradient by the condition that we want to satisfy. Second, we show that this construction is effectively a martingale. Third, we show that it converges. Finally, we show what the limit of this convergent martingale is.

% To begin with, an alternative way of showing that Proposition \ref{prop:scheduler} is true, is by showing that the inferior limit (i.e. $\lim\inf$) of the expected value of the norm of the gradient is $0$. Therefore, if 
% \begin{align}
	% 	\lim \inf_{k\to\infty} \mbE[|| \nabla_\ccalH \ell (y,\phi(x,\bbS;\ccalH_t))] = 0,
	% \end{align}
% we can show by contradiction, that for every $\delta>0$, and $k_0$, there exists a value of $K$, such that $\mbE[|| \nabla_\ccalH \ell (y,\phi(x,\bbS;\ccalH_t))] \leq \delta$. 

To begin with, we define the filtration $\ccalF_t$ by iterates generated according to \eqref{eqn:SGD}, and the sequence $X_t$ as follows, 
\begin{align}
	&X_t = ||\nabla_\ccalH \ell (y,\phi(x,\bbS;\ccalH_t))||^2\bbone[||\nabla_\ccalH \ell (y,\phi(x,\bbS;\ccalH_t))||^2\nonumber\\
 &\geq  L_\nabla^2 \epsilon^2_{t^{'}}+\sigma, t^{'}\leq t],
\end{align}
where $\bbone[\cdot]$ is the indicator function. The expected value of $|X_t|$ is bounded by Assumption \ref{as:Loss_Grad_Lipschitz}, and $X_t$ is adapted to the filtration generated by the iterates of \ref{eqn:SGD}. 
By \cite{durrett2019probability}, to show that $X_n$ is a super-martingale, we require, 
\begin{align}
	\mbE[X_{t+1}|\ccalF_t]\leq X_t.
\end{align}
By contradiction, we can argue that $\mbE[X_{t+1}|\ccalF_t]> X_t$. Now, if $\mbE[X_{t+1}|\ccalF_t]> X_t$ for every $t>t_0$, then it must be the case that $X_{t_0}>L_\nabla^2 \epsilon^2_{t_0}+\sigma$. If this is not the case, the indicator function will make the sequence equal to $0$ for all $t>t_0$, disproving the contradiction. Given that $X_{t_0}>L_\nabla^2 \epsilon^2_{t^{'}}+\sigma$, we can fix $\epsilon_0$, and by Proposition \ref{prop:fixed_compression}, we arrive at a contradiction, and therefore $X_n$ is a super-martingale.

Given that the $X_t$ is bounded below by $0$, by the  Martingale Convergence Theorem \cite[Theorem 4.2.1]{durrett2019probability}, $X_t$ converges. 

Now it remains to show that the limit of $\lim_{t\to\infty}X_t=0$. We can assume that this is not true. We can therefore assume that $\lim_{t\to\infty}X_t=A<\infty$ with $A>\sigma$. Now, we know that the scheduler decreases, and therefore $\exists t^*:L^2_\nabla\epsilon^2_{t^*}+\sigma<A$. In this case again, $X_t$ cannot be larger that  $L^2_\nabla\epsilon^2_{t^*}+\sigma$ forever by Proposition \ref{prop:fixed_compression}. Therefore, there is no $A>\sigma$ such that $X_t$ converges to. Which implies that $X_n\to 0$.

To finalize, given that we made no assumptions over the initial time $t$, $||\nabla_\ccalH \ell (y,\phi(x,\bbS;\ccalH_t))||^2\leq \sigma$ happens infinitely often, completing the proof. 

%%%%%%%%%%%%%%%%%%%%%%%%%%%%%%%%%%%%%%%%%%%%%%%%%%%%%%%%%%%%%%%%%%%%%%%%%%%%%%%%%%%%%%%%%%%%%%%%%%%%%%%%%%%%%%%%%%%%%%%%%%%%%%%%%%%%%%%%%%%%%%%%%%%%%%%%%%%%%%%%%%%%%%%%%%%%%%%%%%%%%%%%%%%%%%%%%%%%%%%%%%%%%%%%%%%%%%%%%%%%%%%%%%%%%%%%%%%%%%%%%%%%%%%%%%%%%%%%%%%%%%%%%%%%%%%%%%%%%%%%%%%%%%%%%%%%%%%%%%%%%%%%%%%%%%%%%%%%%%%%%%%%%%%%%%%%%%%%%%%%%%%%%%%%%%%%%%%%%%%%%%%%%%%%%%%%%%%%%%%%%%%%%%%%%%%%%%%%%%%%%%%%%%%%%%%%%%%%%%%%

% \newpage

% \section{Varying Compression Rates}


% In this work, we propose to utilize varying compression rates while learning the GNN. At the beginning of the training, we utilize a large compression ratio and we reduce it as we train. Intuitively, our method proposes to increase the fidelity of the estimator of the gradient as the GNN approaches convergence. 

% Given that edges of the graph $\bbS$ exist between nodes own between different workers, when a gradient at worker $W_i$ is computed, data from workers $W_j$'s that posses nodes adjacent to the ones owned by $W_i$ are needed. In order to reduce the communication costs between workers, we propose to compress the information sent between them. To this end, in this work, we propose to communicate the activation between agents.
% %
% \begin{definition}
	% 	The compression and decompression mechanism $g_{\epsilon,r},g_{\epsilon^{-1},r}$ with compression error $\epsilon$, and rate $r$,  satisfies that given a set of parameters $x$, when compressed and decompressed, the following relation holds i.e.,
	% 	\begin{align}
		% 		&a= g_{\epsilon,r} (x),  \text{ and }\tilde x = g_\epsilon^{-1}(g_\epsilon( x)) \text{ and }\mbE[\tilde x - x]=0 \text{ with }\mbE[||\tilde x - x||^2]\leq\epsilon^2,
		% 	\end{align}
	% 	where $a\in \reals^m$ is the compressed signal with rate $r$, $\frac{m}{n}=r$. If $\epsilon=0$ we say that we compute a loss-less compression. 
	% \end{definition}
% %

% Returning to \ref{eqn:SGD}, in this paper we propose to make the updates on $\ccalH$ based on the decompressed signals $\tilde x$. To this end, each $W_i$ compresses its activation to obtain $a_j$. Next, it transmits the compressed activation to its neighbours. Upon receiving all the compressed activations, each worker decompresses  the information and computes the backward pass to obtain the stochastic gradient with which it computes the gradient step. In this paper, we propose to vary the compression rate $r_k$ across iterations. A more succinct description of the procedure can be found in Algorithm \ref{alg:varying_compr_rates}.




% The advantage of our procedure relies on the fact that the compressed data is transmitted between agents reducing the costs of communication. Given that the bottleneck is given by the communication times, compressing and decompressing information locally adds less overhead than transmitting the vector $x$.

% %%%%%%%%%%%%%%%%%%%%%%%%%%%%%%%%%%%%%%%%%%%%%%%%%%%%%%%%%%%%%%%%%%%%%%%%%%%%%%%%%%%%%%%%%%%%%%%%%%%%%%%%%%%%%%%%%%%%%%%%%%%%%%%%%%%%%%%%%%%%%%%%%%%%%%%%%%%%%%%%%%%%%%%%%%%%%%%%%%%%%%%%%%%%%%%%%%%%%%%%%%%%%%%%%%%%%%%%%%%%%%%%%%%%%%%%%%%%%%%%%%%%%%%%%%%%%%%%%%%%%%%%%%%%%%%%%%%%%%%%%%%%%%%%%%%%%%%%%%%%%%%%%%%%%%%%%%%%%%%%%%%%%%%%%%%%%%%%%%%%%%%%%%%%%%%%%%%%%%%%%%%%%%%%%%%%%%%%%%%%%%%%%%%%%%%%%%%%%%%%%%%%%%%%%%%%%%%%%%%%

% %\section{experiments}


% \section{Algorithm Convergence}
% In order to show convergence of Algorithm \ref{alg:varying_compr_rates}, we need to introduce three assumptions. 

% \begin{assumption}
	% 	The positive loss $\ell$ function has $L$ Lipschitz continuous gradients i.e., $||\nabla\ell(\bby_1,\bbz)- \nabla\ell(\Phi(\bby_2,\bbz)||\leq L||\bby_1-\bby_2||$.
	% \end{assumption}
% \begin{assumption}
	% 	The empirical estimator of the gradient $ \nabla_\ccalH \ell (y_i,\Phi(x_i,\bbS;\ccalH_t))$ is an unbiased estimator of the gradient $\nabla_\ccalH \ell (y_i,\Phi(x_i,\bbS;\ccalH_t))$, and the variance can be controlled by the number of samples in the batch $B$ as follows, 
	% 	\begin{align}
		% 		\mbE\bigg[|| \frac{1}{B}\sum_{i=1}^B\nabla_\ccalH \ell (y_i,\Phi(x_i,\bbS;\ccalH_t))- \nabla_\ccalH \ell (y,\Phi(x,\bbS;\ccalH_t))||^2\bigg]\leq \frac{\sigma^2}{B}
		% 	\end{align}
	% 	with $\infty>\sigma>0$ being the variance of the estimator. 
	% \end{assumption}
% \begin{assumption}
	% 	The graph convolutional filters in every layer of the graph neural network are bounded, i.e.
	% 	\begin{align}
		% 		||h_{*\bbS}x|| \leq ||x|| \lambda_{max} \bigg(\sum_{t=0}^T h_t \bbS^t\bigg),\text{ with } \lambda_{max} \bigg(\sum_{t=0}^T h_t \bbS^t\bigg)<\infty.
		% 	\end{align}
	% \end{assumption}


% Proposition \ref{prop:compression_rate} shows that the iterates generated by Algorithm \ref{alg:varying_compr_rates} form a supermartingale. If we run Algorithm \ref{alg:varying_compr_rates} sufficiently it will converge to a first order stationary point. Intuitively, as we reduce the value of the loss $\ell$, so does the magnitude of the gradient $||\nabla_\ccalH \ell (f_{\ccalH_{k+1}})||$, therefore, as we increase the number of epochs, we need to reduce the compression error. 


% \begin{proposition}[Convergence of \algo]
	% 	Consider the iterates generates by equation \ref{eqn:compressed_SGD} where the gradient is compressed with compression rate $r_k$ (cf. Definition \ref{eqn:compress_decompress}). Let the step-size  be $\eta\leq 1/\lipGrad$, if the compression error is such that at every step $k$, 
	% 	\begin{align}
		% 		\mbE[||\nabla_\ccalH \ell (y,\Phi(x,\bbS;\ccalH_t)) ||^2] \geq \frac{\lipGrad^2\epsilon_k^2}{B}+\frac{\sigma^2}{B} + \beta,
		% 	\end{align}
	% 	Then \algo\ converges to a first order stationary point in $K\leq \ccalO(\frac{1}{\beta})$ iterations. ,i.e., 
	% 	\begin{align}
		% 		\mbE[|| \nabla_\ccalH \ell (y,\Phi(x,\bbS;\ccalH_t))||^2]\leq \beta^2
		% 	\end{align}
	% \end{proposition}



% \subsection{Scheduler} \label{subsec:scheduler}




\end{document}
