% RLC main.tex Version 2024.4

\documentclass[10pt]{article} % For LaTeX2e
%\usepackage{rlc}
% If accepted, instead use the following line for the camera-ready submission:
\usepackage[accepted]{rlc}
% To de-anonymize and remove mentions to RLC (for example, for posting to preprint servers), instead use the following:
% \usepackage[preprint]{rlc}

\usepackage{booktabs} % for professional tables
\newcommand{\algorithmautorefname}{Algorithm}



% Optional math commands from https://github.com/goodfeli/dlbook_notation.
\input{math_commands.tex}
\newcommand\blfootnote[1]{%
  \begingroup
  \renewcommand\thefootnote{}\footnote{#1}%
  \addtocounter{footnote}{-1}%
  \endgroup
}

\usepackage{hyperref}
\hypersetup{colorlinks,linkcolor={blue},citecolor={magenta},urlcolor={red}}  

\newcommand{\psc}[1]{\textcolor{blue}{\textbf{[psc: }#1\textbf{]}}}
\newcommand{\johan}[1]{\textcolor{red}{\textbf{[johan: }#1\textbf{]}}}
\newcommand{\joao}[1]{\textcolor{green}{\textbf{[joao: }#1\textbf{]}}}
\newcommand\sbullet[1][.5]{\mathbin{\vcenter{\hbox{\scalebox{#1}{$\bullet$}}}}}
\usepackage{graphicx}
\usepackage{subfigure}
\usepackage{wrapfig}

\usepackage{algorithm}
\usepackage{algpseudocode}

\usepackage{bbm}

\usepackage{hyperref}
\usepackage{cleveref}
\usepackage{url}


\title{On the consistency of hyper-parameter selection in value-based deep reinforcement learning}

% Authors must not appear in the submitted version. They should be hidden
% as long as the tmlr package is used without the [accepted] or [preprint] options.
% Non-anonymous submissions will be rejected without review.
% \author{Johan Obando-Ceron\footnotemark[1]\\%\thanks{Equal contribution}
%       jobando0730@gmail.com \\
%       Mila - Québec AI Institute \\
%       Universit\'e de Montr\'eal \\
%       Google DeepMind
%       \And
%       João G.M. Araújo \thanks{Equal contribution}\\%\printfnsymbol{1}\\
%       joaogui@google.com \\
%       Google DeepMind
%       \And
%       Aaron Courville \\
%       aaron.courville@umontreal.ca \\
%       Mila - Québec AI Institute, Universit\'e de Montr\'eal \\
%       \And
%       Pablo Samuel Castro\\
%       psc@google.com\\
%       Google DeepMind\\
%       Mila - Québec AI Institute, Universit\'e de Montr\'eal \\
%       }


\author{Johan Obando-Ceron\(^{*1,2,3}\), João G.M. Araújo\(^{*3}\), Aaron Courville\(^{1,2}\), \\\textbf{Pablo Samuel Castro\(^{1,2,3}\)
}\\\\
Mila - Québec AI Institute\(^{1}\) \\
Universit\'e de Montr\'eal\(^{2}\) \\
Google DeepMind\(^{3}\)  \\
}


% The \author macro works with any number of authors. Use \AND 
% to separate the names and addresses of multiple authors.

\newcommand{\fix}{\marginpar{FIX}}
\newcommand{\new}{\marginpar{NEW}}

\def\month{June}  % Insert correct month for camera-ready version
\def\year{2024} % Insert correct year for camera-ready version
\def\openreview{\url{https://openreview.net/pdf?id=szUyvvwoZB}} % Insert correct link to OpenReview for camera-ready version


\begin{document}


\maketitle
\blfootnote{*Authors contributed equally. Correspondence to \texttt{jobando0730@gmail.com},\texttt{[joaogui,psc]@google.com}}

\begin{abstract}
Deep reinforcement learning (deep RL) has achieved tremendous success on various domains through a combination of algorithmic design and careful selection of hyper-parameters. Algorithmic improvements are often the result of iterative enhancements built upon prior approaches, while hyper-parameter choices are typically inherited from previous methods or fine-tuned specifically for the proposed technique. Despite their crucial impact on performance, hyper-parameter choices are frequently overshadowed by algorithmic advancements. This paper conducts an extensive empirical study focusing on the reliability of hyper-parameter selection for value-based deep reinforcement learning agents, including the introduction of a new score to quantify the consistency and reliability of various hyper-parameters. Our findings not only help establish which hyper-parameters are most critical to tune, but also help clarify which tunings remain {\em consistent} across different training regimes.

\end{abstract}

\section{Introduction}
\label{sec:introduction}

Sequential decision making is generally considered an essential ingredient for generally capable agents. The ability to plan ahead and adapt to changing circumstances is synonymous with the concept of {\em agency}. For decades, the field of reinforcement learning (RL) has worked on developing methods, or agents, for precisely this purpose. This research has borne impressive results, such as developing agents which can play difficult Atari games \citep{mnih2015humanlevel}, control stratospheric balloons \citep{Bellemare2020AutonomousNO}, control a tokamak fusion reactor \citep{Degrave2022MagneticCO}, among others. These are all examples of {\em deep reinforcement learning} (DRL), which combines the theory of reinforcement learning with the expressiveness and flexibility of deep neural networks.

The success of these methods built on years of academic research, where novel algorithms and techniques were introduced and showcased on academic benchmarks such as the ALE \citep{bellemare2012ale}, MuJoCo \citep{todorov2012mujoco}, and others. These benchmarks typically consist of a suite of environments that have varied transition and reward dynamics. Their common usage provides us with a familiarity which affords us a sense of interpretability, a consistency in evaluation that grants us a sense of reliability, and their variety yields a sense of generalizability. Unfortunately, this promise often fails to materialize: their reliability has been brought into question by numerous works which demonstrate their fickleness \citep{Henderson2017DeepRL,agarwal2021deep}, while there is a general sentiment that researchers have ``overfit’’ to these benchmarks, bringing into question their generalizability. A critical aspect to these challenges is the difficulty in training neural networks in an RL setting \citep{ostrovski2021the,lyle2022learning,sokar2023dormant}.

Although the successes above built on prior methods, they were not taken ``as is’’: it took large teams of researchers many months and lots of compute to adapt prior work to their specific problem. These adaptations include changes to the network architectures, designing reward functions to induce the desired behaviours, and careful tuning of the many hyper-parameters. This last point is indeed {\em essential} to the success of any DRL method: improper hyper-parameter choices can cause a theoretically sound method to drastically underperform, while careful hyper-parameter selection can dramatically increase the performance of an otherwise sub-optimal method.

As an example of this dichotomy, we examine how DER \citep{hasselt19when}, a method that has become a common baseline for the Atari $100$k benchmark \citep{kaiser2020modelbased}, came to be. DQN, considered to be the start of the field of DRL research, was introduced by showcasing its super-human performance on the ALE \citep{bellemare2012ale}, a suite of $57$ Atari $2600$ games. This suite became one of the most popular benchmarks on which to evaluate new methods over $200$ million environment frames\footnote{See \citep{machado2018revisiting} for more details on ALE evaluation standards.}. A few years later, when \citet{kaiser2020modelbased} introduced the SiMPLe algorithm as a sample-efficient method, they argued for evaluating it only on $100$k agent actions\footnote{The standard for ALE agents is to use frame-skipping, where $4$ environment frames occur for every agent action. This results in frustratingly confusing nomenclature, as $200$M is specified in environment frames (or $500$k agent actions), while $100$k is specified in agent actions (or $400$k environment frames).} with a subset of $26$ games, so as to properly test the sample-efficiency of new methods. The authors demonstrated that their proposed method outperformed Rainbow \citep{Hessel2018RainbowCI}, the state-of-the-art method of the time. In response, \citet{hasselt19when} introduced Data Efficient Rainbow (DER), which outperformed SiMPLe even though it was the same Rainbow algorithm, but {\em with a careful tuning of the hyper-parameters for the $100$k training regime}.

One could argue that the hyper-parameters of Rainbow were overly-tuned to the $200$M benchmark, while the hyper-parameters of DER were overly-tuned to the $100$k benchmark. More importantly, what this story highlights is that, despite careful evaluation it is quite likely that a new method {\em will not work as intended when deployed on a different environment from which it was trained on}, and that a significant  amount of hyper-parameter tuning will be necessary. This flies in the face of the supposed generalizability of DRL academic research, and makes it difficult for groups without large computational budgets to successfully apply prior work to applied problems.

It thus behooves the community to develop a better understanding of the {\em transferability} and {\em consistency} of hyper-parameter selection across different training regimes, and to build a better shared understanding of the relative importance of the many possible hyper-parameters to tune. In this work, we take a stride towards this by conducting an exhaustive empirical investigation of the various hyper-parameters affecting DRL agents. We focus our attention on two value-based agents developed for the Atari $100$k suite: DER mentioned above, and DrQ($\epsilon$), a variant of DQN that was optimized for the $100$k suite. Although developed for the $100$k suite, we also train these agents for $40$M million environment frames. Our intent is to examine the transferability of various hyper-parameter choices across different training regimes. Specifically, we investigate:
{\bf Across data regimes:} Do hyper-parameters selected in the $100$k regime work well in a larger data regime? {\bf Across agents:} Do hyper-parameters selected for one agent work well in another? {\bf Across environments:} Do hyper-parameters tuned in one set of environments work well in others?

In total, we investigated $12$ hyper-parameters with different values for $2$ agents over $26$ environments, each for $5$ seeds, resulting in a total of $108$k independent training runs. This breadth of experimentation results in an overwhelming amount of data which complicates their analyses. We address this challenge in two ways: \textit{(i)} We introduce a new score which provides us with an aggregate value for the considerations mentioned above. \textit{(ii)} We provide an interactive website where others may easily navigate the large number of experimental figures we have generated.

The score provides us with a high-level overview of our findings, while the website grants us a fine-grained mechanism to analyze the results. We hope this effort provides the community with useful tools so as to develop not just better DRL algorithms, but better methodologies to evaluate their interpretability, reliability, and generalizability.
\section{Background}
\label{sec:brackground}

% We cover the necessary background in a little more detail than is traditionally done, so as to be able to draw a direct connection to the hyper-parameters considered. 

The field of reinforcement learning studies algorithms for sequential decision-making problems. In these settings, an algorithm (or agent) interacts with an {\em environment} by transitioning between {\em states} and making action choices at discrete timesteps; the environment responds to each action by (possibly) changing the agent's state and yielding a numerical reward or cost. The goal of the agent is to maximize the cumulative rewards (or minimize the cost) throughout its lifetime.
This is typically formalized as a Markov decision process (MDP) \citep{puterman2014markov} $\langle \mathcal{X}, \mathcal{A}, \mathcal{P}, \mathcal{R}, \gamma \rangle$, where $\mathcal{X}$ is the set of states, $\mathcal{A}$ is the set of available actions, $\mathcal{P}:\mathcal{X}\times\mathcal{A}\rightarrow \Delta(\mathcal{X})$\footnote{$\Delta(X)$ denotes a distribution over the set $X$.} is the transition function, $\mathcal{R}:\mathcal{X}\times\mathcal{A}\rightarrow\mathbb{R}$ is the reward function, and $\gamma\in [0, 1)$ is a discount factor. An agent's behaviour is formalized by a policy $\pi:\mathcal{X}\rightarrow\Delta(\mathcal{A})$, whose {\em value} from any state $x\in\mathcal{X}$ is given by the Bellman recurrence 
$V^{\pi}(x) := \mathbb{E}_{a\sim\pi(x)}\left[\mathcal{R}(x, a) + \gamma \mathbb{E}_{x'\sim\mathcal{P}(x, a)}V^{\pi}(x')\right]$. $Q$-functions allow us to measure the value of taking any action $a\in\mathcal{A}$ from a state $x\in\mathcal{X}$ and following $\pi$ afterwards: $Q^{\pi}(x, a) := \mathcal{R}(x, a) + \gamma \mathbb{E}_{x'\sim\mathcal{P}(x, a)} V^{\pi}(x')$. %This admits a simple algorithm for defining a new policy $\pi^\prime$ that is at least as good as $\pi$: $\pi^\prime(x) = \arg\max_{a\in\mathcal{A}}Q^{\pi}(x, a)$. 
A policy $\pi^*$ is considered optimal if for any policy $\pi$, $V^* := V^{\pi^*} \geq V^{\pi}$.

Solving for the equations discussed above would require access to both $\mathcal{R}$ and $\mathcal{P}$, which are usually unknown. Instead, RL typically assumes the agent has access to transitions $\tau := (x, a, r, x')\in\mathcal{X}\times\mathcal{A}\times\mathbb{R}\times\mathcal{X}$, arising from interactions with the environment. Given such a transition,  $Q$-learning \citep{Watkins1992qlearning} updates its estimate of $Q$ via: $Q_{t+1}(x, a) \leftarrow  Q_t(x, a) + {\alpha} TD(Q, \tau)$, where {$\alpha$} is a learning rate and $TD$ is the {\em temporal-difference error}, given by $TD(Q_t, \tau) := r + \gamma \max_{a' \in\mathcal{A}} Q_t(x', a') - Q_t(x, a)$. If the state and action spaces are small, one can store all the $Q$-values in a table of size $|\mathcal{X}|\times |\mathcal{A}|$. For most problems of interest, however, state spaces are very large (and possibly infinite). In these cases, one can use a function approximator, such as a neural network, parameterized by $\theta$: $Q_\theta\approx Q$. Indeed, in order to achieve super-human performance on the Arcade Learning Environment (ALE) \citep{bellemare2012ale}, \citet{mnih2015humanlevel} used a neural network consisting of three convolutional layers (Conv layers), followed by two multi-layer perceptrons (Dense layers) with $|\mathcal{A}|$ outputs in the final layer (representing the $Q$-value estimates for each action). With the exception of the final layer, a ReLU non-linearity follows each layer.

Updating $Q_{\theta}$ thus corresponds to updating the parameters $\theta$, which may be done by using optimization algorithms such as Adam \citep{kingma15adam} to minimize the temporal-difference error. At a high-level, this yields an update of the form: $\theta_{t+1} \leftarrow \theta_t + \alpha\nabla_{\theta_t} \mathbb{E}_{\tau\sim\mathcal{D}} TD(Q_{\theta_t}, \tau)$.
The expectation can be approximated using a batch of $m$ transitions drawn from a distribution $\mathcal{D}$, which can be computed efficiently on specialized hardware such as GPUs and TPUs. Additionally, \citet{mnih2015humanlevel} argued that using $\bar{\theta}$, a less-frequently updated copy of the parameters, when computing TD helps with training stability. A common approach introduced by \citet{mnih2015humanlevel} is to clip the rewards at $(-1, 1)$. The TD term thus becomes:
  $TD(Q_{\theta}, \tau) := clip(r, (-1, 1)) + \gamma\max_{a'\in\mathcal{A}}Q_{\bar{\theta}}(x', a') - Q_{\theta}(x, a)$.

Although DQN benchmarked on the 57 ALE games with the same set of hyper-parameters, \citet{anschel2017averageddqn} 
%and \citet{cini2020deep} 
demonstrated that in some environments it can result in degraded performance.
A number of papers have proposed improvements to increase stability and performance, which \citet{Hessel2018RainbowCI} combined into a single agent they called \emph{Rainbow}. Specifically, they combined DQN with double Q-learning \citep{hasselt2015doubledqn}, prioritized experience replay \citep{Schaul2016PrioritizedER}, dueling networks \citep{wang16dueling}, multi-step learning \citep{sutton88learning}, noisy nets \citep{fortunato18noisy}, and distributional reinforcement learning \citep{Bellemare2017ADP}.


\section{THC Score}
\label{sec:thc_metric}

Statistical metrics play a crucial role in assessing and evaluating the performance of DRL algorithms. 
They provide valuable insights into the strengths and weaknesses of different approaches, guiding researchers and practitioners in the development of more effective reinforcement learning systems.
For example, some the metrics focus on the mean reward obtained by an agent per time step (Average Reward), the percentage of episodes in which the agent achieves a predefined goal or task (success rate) among others \citep{agarwal2021deep, chan2020measuring, Henderson2017DeepRL}. 

Measuring the transferability/consistency of hyper-parameters in DRL is challenging, as existing metrics fall short in capturing the nuanced aspects of how well hyper-parameter settings generalize across different environments or agents. Developing such a metric would enhance the ability to systematically compare and select hyper-parameter configurations that exhibit robust performance across a range of application domains.

To understand the consistency of hyper-parameters we focus on their ranking consistency across experimental settings. Put another way: if a given hyper-parameter value is optimal/pessimal in a setting, is it still optimal/pessimal in another? And so we analyse, for each hyper-parameter, whether their values lead to the same ranking order for different experimental settings, where the ranking is on final performance. 

We compute ranking agreement for three setups: 
{\bf $1$) Varying algorithms} while keeping the environment and data regime fixed (e.g. when proposing a new value-based algorithm but not having enough compute to run a comprehensive hyper-parameter search). {\bf $2$) Varying environments} while keeping the algorithm and data regime fixed (e.g. when using a state of the art algorithm in a new domain).
{\bf $3$) Varying data regimes} while keeping the environment and algorithm fixed (e.g. when adapting a new algorithm to a new data regime \citep{hasselt19when}).
Concretely, our desire is to have a metric that yields a high value score would indicate that the hyper-parameter in question is {\em important}, in the sense that it will likely require retuning; conversely, a low score suggests the hyper-parameter value can likely be kept as is.

Kendall's Tau \citep{kendall38measure} and Kendall's W \citep{10.1214/aoms/1177732186} are natural choices, but these metrics were developed for situations where the rankings were based on a single score, instead of a range of possible scores, and they can result in degenerate values when two settings have similar performance or when two settings alternate between optimal and pessimal rankings. For these reasons, we introduce the \textbf{T}uning \textbf{H}yperparameter \textbf{C}onsistency ({\bf THC}) score. Consider a set of $n$ hyper-parameters $\lbrace H_1,\ldots,H_n\rbrace$, each with its set of values $\lbrace\lbrace h_{11},h_{12},\ldots,h_{1m_1}\rbrace, \ldots,\lbrace h_{n1},h_{n2},\ldots,h_{nm_n}\rbrace\rbrace$ (e.g. hyper-parameter $H_i$ has $m_i$ values). The THC score involves three computations: (i) rankings for each hyper-parameter setting (\autoref{alg:computeRankings}); (ii) normalized peak-to-peak value for each hyper-parameter setting (Eqn.~\ref{eqn:ptp} below); and (iii) overall THC score for the hyper-parameter (see Eqn.~\ref{eqn:thc} below).

If we run multiple independent runs for each hyper-parameter setting $h_{ij}$, we can compute the mean $\mu_{ij}$ and standard deviation $\sigma_{ij}$ for these runs\footnote{One may also use confidence intervals instead of standard deviations.}. For each hyper-parameter setting $h_{ij}$ we then compute an initial ranking $r'_{ij}$ based on the upper bound ($\mu_{ij}+\sigma_{ij}$), with the lower bound ($\mu_{ij}-\sigma_{ij}$) used to break ties. We then define a set containing hyper-parameter settings with overlapping values:
\begin{align*}
    I_{ij} := \{k \vert (\mu_{ij} - \sigma_{ij} < \mu_{ik} + \sigma_{ik} &\text{ and } \mu_{ij} - \sigma_{ij} > \mu_{ik} - \sigma_{ik}) \\ &\text{ or } \\ \break (\mu_{ij} + \sigma_{ij} > \mu_{ik} - \sigma_{ik} &\text{ and } \mu_{ij} + \sigma_{ij} < \mu_{ik} + \sigma_{ik}) \}
\end{align*}



\begin{algorithm}[!t]
\caption{Compute rankings}\label{alg:computeRankings}
\begin{algorithmic}[1]
\Require Multiple runs for various settings of hyper-parameter $H_i$: $\lbrace h_{i1},h_{i2},\ldots,h_{im_i}\rbrace$, aggregate metrics $\mu_i$: $\lbrace \mu_{i1},\mu_{i2},\ldots,\mu_{im_i}\rbrace$ and measure of spread $\sigma_i$: $\lbrace \sigma_{i1},\sigma_{i2},\ldots,\sigma_{im_i}\rbrace$
\For{$i$ in $1 \ldots n$}
    \State $r'_{i} = \textrm{argsort}(\mu_i + \sigma_i)$ \Comment{Gets the index of each value as if the array was sorted}
    \State $\mu'_i, \sigma'_i = \mu_i[r'_{i}], \sigma_i[r'_{i}]$ \Comment{Sorted versions of aggregate and spread metrics}
    \For{$j$ in $1 \ldots m_{i}$} 
        \State $u_{j} = \textrm{binary\_search}(\mu'_i - \sigma'_i, \mu_{ij} + \sigma_{ij})$ \Comment{highest rank whose lower bound overlaps with j}
        \State $l_{j} = \textrm{binary\_search}(\mu'_i + \sigma'_i, \mu_{ij} - \sigma_{ij})$ \Comment{lowest rank whose upper bound overlaps with j}
    \EndFor
    \State $\bf{r_{i}} = \frac{u + l}{2}$ \Comment{The average rank in $l_j,l_j+1, \ldots, u_j$ is the average of $l_j$ and $u_j$}
\EndFor

\end{algorithmic}
\end{algorithm}

The final ranking of each hyper-parameter is $r_{ij} = \frac{\sum_{k \in I_{ij}} r
'_{ik}}{\vert I_{ij} \vert}$, 
as \autoref{alg:computeRankings} details. These rankings are for {\em one} training regime; however, as mentioned in the introduction, we are interested in quantifying the {\em consistency} of a hyper-parameter $H$ across varying training regimes. Consider four training regimes $A, B, C, D$, and let $\lbrace \mathfrak{R}^A,\ldots,\mathfrak{R}^D\rbrace$ denote their respective rankings. For each hyper-parameter value $h_x\in H$ we compute its normalized ``peak-to-peak''\footnote{Inspired by numpy's peak-to-peak function numpy.ptp \citep{harris2020array}.} value $\overline{\textrm{ptp}}$, which quantifies its variance in ranking, as follows: First compute the $\textrm{ptp}$ value $\textrm{ptp}(h_x) = \max\left(\lbrace \mathfrak{R}^A(h_x),\ldots,\mathfrak{R}^D(h_x)\rbrace\right) - \min\left(\lbrace \mathfrak{R}^A(h_x),\ldots,\mathfrak{R}^D(h_x)\rbrace\right)$, then normalize:
\begin{align}
    \overline{\textrm{ptp}}(h_x) = \frac{\textrm{ptp}(h_x)}{\sum_{h_y\in H}\textrm{ptp}(h_y)}
    \label{eqn:ptp}
\end{align}

Notably, hyper-parameter settings that have consistent rankings across training regimes will have a normalized $\textrm{ptp}$ value of zero. Finally, the $\textrm{THC}$ score for hyper-parameter $H$ is defined as:
\begin{align}
    \textrm{THC}(H) = \frac{\sum_{h_x\in H}\overline{\textrm{ptp}}(h_x)}{|H|}.
    \label{eqn:thc}
\end{align}

This score will result in low values for hyper-parameters whose varying settings have consistent ranking across various training regimes, and high values when these rankings vary. Intuitively, {\em hyper-parameters with high values will most likely require re-tuning when switching training regimes}. See \autoref{sec:appendixTHC} for more examples of computing the score, as well as the source code provided with this submission.



\section{Hyper-parameters considered} 
\label{sec:hyper-parameter_selection}

We describe the set of hyper-parameters explored in this work, with the values used for each listed in \autoref{sec:list_hyperparameters}. Unless otherwise specified, these are examined for both Conv and Dense layers.

{\bf Activation functions:} 
Non-linear activation functions are a fundamental part of neural networks, as their removal effectively turns the network into a linear function approximator.
While various activation functions have been proposed \citep{devlin2019bert, Elfwing2018SigmoidWeightedLU, 10.5555/3305381.3305478}, there have been few works comparing their performance \citep{Shamir2020SmoothAA}; to the best of our knowledge, there are no previous examples of such a comparison in the RL setting.


{\bf Normalization: }
Normalization plays an important role in supervised learning \citep{tan2020efficientnet, xie2017aggregated} but is relatively rare in deep reinforcement learning, with a few exceptions \citep{gogianu2021spectral, bhatt2019crossnorm, arpit2019initialize, alphaZero}. We explore {\em batch normalization} \citep{ioffe2015batch} and {\em layer normalization} \citep{ba2016layer}.

{\bf Network capacity: } 
``Scaling laws'' have been central to the growth of capabilities in large language/vision models, but have mostly eluded reinforcement learning agents, with a few exceptions \citep{schwarzer23a, taiga2022investigating, farebrother2022proto,obando2024mixtures,obandoceron2024pruned,farebrother2024stop}. 
To investigate the impact of network size, we vary the {\em depth} (e.g. the number of hidden layers) and the {\em width} (e.g. the number of neurons of each hidden layer).

{\bf Optimizer hyper-parameters: }
\label{sec:optimizerHypers}
We explore three hyper-parameters of Adam \citep{kingma15adam}, which has become the standard optimizer used by most: {\em learning rate}, {\em epsilon} and {\em weight decay}.
\emph{Learning rate} determines the step size at which the algorithm adjusts the model's parameters during each iteration.
$\epsilon$ represents a small constant value that is added to the denominator of the update rule to avoid numerical instabilities.
\emph{Weight decay} adds a penalty term to the loss function during training that discourages the model from assigning excessively increasing weight magnitudes.


{\bf $\epsilon$-greedy exploration: } 
$\epsilon$-greedy exploration is a simple and popular exploration technique which picks actions greedily with probability $1-\epsilon$, and a random action with probability $\epsilon$. Traditionally, experiments on the ALE use a linear decay strategy to decay $\epsilon$ from $1.0$ to its target value.

{\bf Reward clipping: } 
Most ALE experiments clip rewards at $(-1, 1)$ \citep{mnih2015humanlevel}.

{\bf Discount factor: } 
The multiplicative factor $\gamma$ discounts future rewards and its importance has been observed in a number of recent works \citep{amit2020discount, hessel19inductive, gelada2019off, vanseijen2019using, francoislavet2016discount,schwarzer23a}.

{\bf Replay buffer: }  
DRL agents  store past experiences in a replay buffer, to sample from during learning. The {\em replay capacity} parameter refers to the amount of data experiences stored in the buffer. 
It is common practice to only begin sampling from the replay buffer when a minimum number of transitions have been stored, referred to as the {\em minimum replay history}.


{\bf Batch size: } 
The number of stored transitions that are sampled for learning at each training step.

{\bf Update horizon: }
Multi-step learning \citep{sutton88learning} computes the temporal difference error using multi-step transitions, instead of a single step. DQN uses a single-step update by default, whereas Rainbow chose a 3-step update \citep{Hessel2018RainbowCI}. The update horizon has been argued to trade-off between the bias and the variance of the return estimate \citep{biasandvariance_kea}. 


{\bf Target Update periods: }
Value based agents often employ an online and a {\em target} Q-network, the latter which is updated less frequently by directly syncing (or Polyak-averaging) from the online network; the {\em target updated period} determines how frequently this occurs.


{\bf Update periods: }
The online network parameters are updated after every {\em update period} environment steps, with a value of $4$ used in standard ALE training.

{\bf Number of atoms: } 
In distributional reinforcement learning \citep{Bellemare2017ADP}, the output layer predicts the distribution of the returns for each action $a$ in a state $s$, instead of the mean $Q^{\pi}(s, a)$. A popular approach is to model the return as a categorical distribution parameterized by a certain number of 'atoms' over a pre-specified support. 


\begin{figure}[!t]
    \centering
  \includegraphics[width=\linewidth]{figures/this_score_all.pdf}%
    \caption{Tuning hyper-parameter Consistency (THC Score, see \cref{sec:thc_metric}) evaluated across agents (\textbf{left panel}), data regimes (\textbf{center panel}), and environments  (\textbf{right panel}). Different colors indicate different data regimes (left panel) and different agents (center and right panels); grey bars/titles indicate hyper-parameters which are not comparable across the considered transfer settings.
    \label{fig:this_score_all}%
    }%
\end{figure}



\section{Experimental results} 
\label{exp_results}
As mentioned in the introduction, there already exist two data regimes for evaluating agents on the ALE suite: the (low-data regime) $100$k \citep{kaiser2020modelbased} and the original $200$M benchmark \citep{mnih2015humanlevel}. The $100$k benchmark includes only $26$ games from the original suite, so we focus on these for our evaluation. For computational considerations, we follow \citet{graesser2022state} and use $40$M million environment frames as our large-data regime.
We use the settings of DrQ($\epsilon$) (introduced by \citet{agarwal2021deep} as an improvement over the DrQ of \citet{yarats2021image}), and 
Data Efficient Rainbow (DER) introduced by \citet{hasselt19when}. All experiments were run on a Tesla P100 GPU and took around $2$-$4$ hours ($100$k) and $1$-$2$ days ($40$M) per run.
Both algorithms are implemented in the Dopamine library \citep{castro18dopamine}. Since the $100$k setting is cheaper, we evaluated a larger set of hyper-parameter values there and manually picked the most informative subset for running in the $40$M setting. For all our experiments we ran 5 independent seeds and followed the guidelines suggested by \citet{agarwal2021deep} for more statistically meaningful comparisons. Specifically, we computed aggregate human-normalized scores and report interquantile mean (IQM) with $95\%$ stratified bootstrap CIs. 

In \autoref{fig:this_score_all} we present the computed THC score for all the hyper-parameters discussed in \cref{sec:hyper-parameter_selection}, and we discuss their consistency across agents in Section~\ref{sec:acrossAlgorithms}, across data regimes in Section~\ref{sec:acrossData}, and  across environments in Section~\ref{sec:acrossEnvironments}. More detailed discussions are provided in \autoref{sec:finerGrainedExperiments} and a set of interesting findings in \autoref{sec:imf}. It is worth recalling that higher THC scores indicate less consistency, which suggests a likely need to re-tune the respective hyper-parameters when changing training configurations.


\subsection{Optimal hyper-parameters mostly Transfer Across Agents}
\label{sec:acrossAlgorithms}
We find that optimal hyper-parameters for DrQ($\epsilon$) agree quite often with DER, which is somewhat expected given that they're based on the same classical RL algorithm of Q-learning, and have the same number of updates in the same environments. Looking at THC values between the two agents for different data regimes we see that all values are below $0.5$, and in the $100$k regime tend to be even lower. Nevertheless, comparing the results of the two rows in \cref{fig:drq_eps_batch_sizes,fig:per_game} demonstrate that there can still be strong differences between the two. In the $40$M regime, the hyper-parameters with the highest THC are batch size and update horizon, consistent with the findings of \cite{obandoceron2023small}, where these two hyper-parameters proved crucial to boosting agent performance.


\begin{figure}[!t]
    \centering
  \includegraphics[width=0.8\linewidth]{figures/DER_adam_eps.pdf}%
    \caption{
     \textbf{Measured IQM of human-normalized scores on the $26$ $100$k benchmark games, with varying Adam's $\epsilon$} for DER. We evaluate performance at 100k agent steps (or 400k environment frames), and at $40$ million environment frames. The ordering of the best hyper-parameters switches between the two data regimes.
    }
    \label{fig:der_adam_eps}
\end{figure}

\subsection{Optimal hyper-parameters mostly do not Transfer Across Data Regimes}
\label{sec:acrossData}
We find that optimal hyper-parameters for Atari 100k mostly do not transfer once you move to 40M updates, showing that even when keeping algorithms and environment constant one may still need to tune hyper-parameters should they change the amount of data their agent can train on. Of the hyper-parameters considered, {\em Adam's $\epsilon$} and {\em update period} seem to be the most critical to re-tune (see \autoref{fig:der_adam_eps} for results on DER for Adam's $\epsilon$). The results with Adam's $\epsilon$ are surprising, as the purpose of this hyper-parameter is mostly for numerical stability. The update horizon results are consistent with what is done in practice between these two data regimes (e.g. Rainbow uses an update horizon of $3$, while DER uses $10$).

\begin{figure}[!h]
    \centering
  \includegraphics[width=0.8\linewidth]{figures/DrQ_eps_subs.pdf}
  \includegraphics[width=0.8\linewidth]{figures/DER_subs.pdf}
    \caption{\textbf{Measured returns with varying batch size} for DrQ($\epsilon$) (top) and DER (bottom) at $40$M environment frames for four representative games, demonstrating that the ranking of the hyper-parameter values can drastically change from one game to the next. All results averaged over $5$ seeds, shaded areas represent $95\%$ confidence intervals.
    }%
    \label{fig:drq_eps_batch_sizes}%
\end{figure}


\subsection{Optimal hyper-parameters do not Transfer Across Environments}
\label{sec:acrossEnvironments}
Our experiments show that hyper-parameters that perform well on some games lead to lackluster final performance in others. Indeed, in \autoref{fig:this_score_all} we can see that the THC score is highest when evaluating across environments. This strongly suggests that, when using an existing agent in a new environment, most of the hyper-parameters would need extra tuning.
\autoref{fig:drq_eps_batch_sizes} displays the results when varying batch size, where we can see that the rankings can sometimes be complete opposites across games (compare Kangaroo and Gopher).
 


\section{A web-based appendix} 
\label{web_results}
We have run an extensive number of experiments (around 108k) for this work, which would render a traditional appendix unwieldy. Instead, we provide an interactive website\footnote{Website available at \href{https://consistent-hyperparameters.streamlit.app/}{\emph{https://consistent-hparams.streamlit.app/}}.} which facilitates navigating the full set of results. Presenting empirical research results in this manner offers a range of benefits that enhance accessibility, engagement, and comprehension. 
This dynamic presentation allows readers to more easily make comparisons over different games, agents, and parameters. 


The website's main page presents aggregate IQM results for all hyper-parameters investigated in both data regimes (e.g. \autoref{fig:der_adam_eps}), while sub-pages present detailed performance comparisons when sliced by game (\autoref{fig:drq_eps_batch_sizes} presents a subset of this) and hyper-parameter (\autoref{fig:per_game} presents a subset of this).
The added level of granularity provided by the sub-pages can be crucial for understanding the specific strengths and weaknesses of an algorithm in various scenarios. All results averaged over 5 seeds, shaded areas represent 95\% confidence intervals.

\begin{figure}[!t]
    \centering
   \includegraphics[width=\textwidth]{figures/DrQ_eps_game_subs.pdf}
   \includegraphics[width=\textwidth]{figures/DER_game_subs.pdf}
  
    \caption{\textbf{Measured returns with various hyper-parameter variations on Asterix} for DrQ($\epsilon$) (top) and DER (bottom) at 40M environment frames. Displaying eight representative hyper-parameters, enabling per-game analyses for hyper-parameter selection.}%
    \label{fig:per_game}%
    \vspace{-1em}
\end{figure}
%\section{Environment properties}
\label{sec:environment_properties}

The Arcade Learning Environment (ALE) is a platform designed for evaluating and comparing the performance of reinforcement learning algorithms on classic arcade games. ALE has emerged as the benchmark for evaluating the capabilities of reinforcement learning (RL) algorithms in tackling intricate discrete control tasks. Since its release in 2013 \citep{bellemare2012ale}, the benchmark has gained thousands of citations and almost all state-of-the-art RL algorithms have featured it in their work. However, results generated from the full benchmark have typically been limited to a few large research groups.

The cost of producing evaluations on the full dataset is not feasible for many researchers and not necessary if you want to evaluate some specific algorithm capabilities, like long context games. Therefore, here we provide some key properties of the environments based on our previous findings which will allow the RL community select a small but representative subsets of environments when evaluating specific algorithm capabilities.
\section{Related work}
\label{related_work}

While RL as a field has seen many innovations in the last
years, small changes to the algorithm or its implementation can have a big impact on its results \citep{engstrom2020implementation, joajo2021lifting}.  Deep reinforcement learning approaches are often notoriously sensitive to their hyperparamaters and demonstrate brittle convergence properties \citep{haarnoja2018soft}. This is particularly true for off-policy approaches that use a replay buffer to leverage past experiences \citep{duan2016benchmarking}.


\cite{Henderson2017DeepRL} investigate the effects of existing degrees of variability between various RL setups and their effects on algorithm performance. Although restricted to the domain of existing environments, \cite{Henderson2017DeepRL} propose more robust performance estimators for RL learning algorithms. \cite{islam2017reproducibility} and \cite{shengyi2022the37implementation} have shown the difficulty in reproducing policy gradient algorithms due to the variance.
\cite{andrychowicz2020matters} did a deep dive in algorithmic choices on policy-based algorithms. Their analyses covered differences in hyper-parameters, algorithms, and implementation details.


In an effort to consolidate innovations in deep RL, several papers have examined the effect of smaller design decisions like the loss function or policy regularization for on-policy algorithms \cite{andrychowicz2020matters}, DQN agents \citep{obando2020revisiting}, imitation learning \citep{hussenot2021hyperparameter} and offline RL \citep{paine2020hyperparameter, lu2021revisiting}. AutoRL methods, on the other hand, have focused on automating and abstracting some of these decisions \citep{parker2022automated, eimer2023hyperparameters} by using data-driven approaches to learn various algorithmic components or even entire RL algorithms \citep{co2021evolving,lu2022discovered}. All these works have demonstrated that hyperparameters in deep reinforcement learning warrant more attention from the research community than they currently receive. Underreported tuning practices can distort algorithm evaluations, and overlooked hyperparameters may lead to suboptimal performance.


% Despite all the efforts, there are still some mysteries that have been not understood yet. Many of these new and unexpected discoveries have remained concealed due to limited parameter exploration caused by the substantial computational resources they demand. Investigate the impact of design choices and hyper-parameter in deep RL algorithms with large state environments is very challenging and almost impossible to explore on academic labs. Therefore, we decide to focus on exploring the relationship between hyparparemeters and  value-based methods, as opposed to the greater focus in Actor-Critic and Policy-based of the previous works \citep{andrychowicz2020matters}.

\section{Discussion}
\label{sec:discussion}
One of the central challenges in reinforcement learning research is the non-stationarity during training in the inputs (due to self-collected data) and targets (due to bootstrapping). This is in direct contrast with supervised learning settings, where datasets and labels are typically fixed throughout training. This non-stationarity may be largely to blame for some of the ranking inconsistencies observed under different training regimes (e.g. \autoref{fig:der_adam_eps}), and why different hyper-parameter tunings are required for different settings (e.g. DER versus Rainbow).

Hyper-parameters are commonly tuned on a subset of environments (e.g. 3-5 games) and then evaluated on the full suite. Our findings suggest that this approach may not be the most rigorous, as hyper-parameter selection can vary dramatically from one game to the next (c.f. \cref{fig:drq_eps_batch_sizes,fig:per_game}).
While aggregate results (e.g. IQM) provide a succinct summary of performance, they unfortunately gloss over substantial differences in the individual environments. If our hope as researchers is to be able to use these algorithms beyond academic benchmarks, understanding these differences is {\em essential}, in particular in real-world applications such as healthcare and autonomous driving.

We have conducted a large number of experiments to investigate the impact of various hyper-parameter choices. While the THC score (\autoref{fig:this_score_all}) provides a high-level view of the transferability of hyper-parameter choices, our collective results suggest that a {\em single} set of hyper-parameter choices will never suffice to achieve strong performance across all environments. The ability to dynamically adjust hyper-parameter values during training is one way to address this; to properly do so would require quantifiable measures of environment characteristics that go beyond coarse specifications (such as sparse versus dense reward systems). The per-game results we present here may serve as an initial step in this direction. In Appendix~\ref{sec:gopher} we provide a fine-grained analysis of DER on Gopher as an example of the type of analyses enabled by our website. We hope our analyses, results, and website prove useful to RL researchers in developing robust and  transferable algorithms to handle increasingly complex problems.\\

\subsubsection*{Acknowledgements}

The authors would like to thank Jesse Farebrother, Gopeshh Subbaraj, Doina Precup, Hugo Larochelle, and the rest of the Google DeepMind Montreal team for valuable discussions during the preparation of this work.  Jesse Farebrother deserves a special mention for providing us valuable feed-back on an early draft of the paper. We thank the anonymous reviewers for their valuable help in improving our manuscript. We would also like to thank the Python community \cite{van1995python, 4160250} for developing tools that enabled this work, including NumPy \cite{harris2020array}, Matplotlib \cite{hunter2007matplotlib}, Jupyter \cite{2016ppap}, Pandas \cite{McKinney2013Python} and JAX \cite{bradbury2018jax}.

\subsubsection*{Broader Impact Statement}

Although the work presented here is mostly academic, it aids in the development of more capable and reliable autonomous agents. While our contributions do not directly contribute to any negative societal impacts, we urge the community to consider these when building on our research.

\bibliography{main}
\bibliographystyle{rlc}

\newpage
\appendix
\appendix
\section{Compression Mechanism}
\label{appendix:CompressionMechanism}
% We use a single linear layer (with no non-linearity) to project $F$-dimensional features to $F_C$ dimensional features on the sender side, where $F_C < F$. On the receiver side, we use another linear layer to project back to the original $F$-dimensional feature space, i.e, $x_{decompress} = {\bf W}_{decoder}{\bf W}_{encoder} x$, where $x$ and $x_{decompress}$ are the original and reconstructed feature vectors, respectively. ${\bf W}_{decoder}$ and ${\bf W}_{encoder}$ are the $F \times F_C$, and $F_C \times F$ decoder and encoder projections, respectively. ${\bf W}_{decoder}$ and ${\bf W}_{encoder}$ are trained, together with the GNN parameters, to minimize task loss.

% In the variale compression scenario, we gradually decrease the compression ratio as training progresses. The transmittted compressed feature tensor's size increases by  $\Delta F$ every few training epochs. We achieve this by increasing the number of rows of ${\bf W}_{encoder}$ and the number of columns of ${\bf W}_{decoder}$ by $\Delta F$. The new rows in ${\bf W}_{encoder}$ are initialized using Glorot initialization. To avoid a sudden jump in the network response, we initialize the new columns of ${\bf W}_{decoder}$ to zeros.  

For the compression mechanism, we communicate the total number of elements in the feature vector and intermediate activations divided by the compression ratio. Which values of the vectors to communicate are chosen at random at the encoder's end. 
For the decoder to know which element of the vector corresponds to the true values, a random key generator is shared a priori. The decoder simply places the values communicated in the corresponding position and sets a $0$ on the rest of the non-communicated values.






% \subsection{Deterministic Sliding Window}
% \subsection{Random Sampling}


% \begin{figure*}[h]
	% 	\centering
	% 	\begin{subfigure}{\textwidth}
		% 		\includegraphics[width = \textwidth]{figures/Architecture.png} 
		% 		\subcaption{}
		% 		\label{fig:arch_fixed}
		% 	\end{subfigure}
	% 	% \begin{subfigure}{\textwidth}
		% 		% 	\includegraphics[width = \textwidth]{figures/variable_compression.png} 
		% 		% 	\subcaption{}
		% 		% 	\label{fig:arch_variable}
		% 		% \end{subfigure}
	% 	\caption{Model architecture for $f_{compress}$ with (a) fixed compression ratio, and, (b) variable compression ratio.}
	% \end{figure*}
% \subsection{Network Architecture for $f_{compress}$ with fixed $r$}
% We introduce a learnable feature-based compression-decompression routine $f_{compress} = g^{-1} \circ g$ which employs an autoencoder architecture, with the encoder approximating the compressor function $g$ and the decoder approximating the decompressor function $g^{-1}$. We adopt a multilayer perceptron design for both the encoder and decoder, which is shared across all nodes in the graph. In this routine, when the worker $W_i$ requests for the node features $x_{v^i_j} \in \reals^m$ to a remote worker $W_j$, $W_j$ first employs the encoder function $g$ to embed the node features into a latent space $z_{v^i_j} \in \reals^n$ and then sends them to $W_i$. Upon receiving, $W_i$ uses the decoder function $g^{-1}$ on $z_{v^i_j}$ to decompress the received features into ${x}_{v^i_j}$. To train $g$ and $g^{-1}$ in an end-to-end manner, we use the original downstream task loss (e.g. cross-entropy loss for classification task) instead of the typical reconstruction loss utilized in an autoencoder training. In fact, the reconstruction loss can't be computed because the original feature vectors $x_{v^i_j}$ are not shared with machine $W_i$.

% \subsection{Network Architecture for $f_{compress}$ with variable $r$}
% For variable compression ratio $r=r_t$, the encoder needs to generate output vectors of variable sizes $n=n_t$ where $t$ is the training time. To facilitate this into the autoencoder architecture, we add an extra dropout layer at the end of the encoder with a variable dropout probability $p_t = 1 - \frac{1}{r_t}$. On the decompressor side, we add an extra linear layer at the front of the decoder and scale the activations by $\frac{1}{1-p_t} = r_t$. This layer ensures that the expected input values to the decoder don't change with $r_t$ and thus the training is stable.




\subsection{Scheduler}
\label{appendix:scheduler}
Several strategies can be utilized to increase the compress rate as we learn. A simple strategy is to increase it a fixed rate $r_{k+1}=r_k+R$, where $R$ is the fixed rate. Another strategy is to implement linear increase $r_{k}=\alpha k+r_0$, where $\alpha>0$ is the increasing slope. Another strategy is to implement an exponential increase $r_k=\frac{1}{\beta^{K-k+1}}$, with $\beta$ being the base of the exponential increase, and $K$ the total number of steps. In all cases, the scheduler is a monotone-increasing function. 

In our experiments, we considered $6$ different types of variable compression mechanisms based on the following equation, 

\begin{align}
	c =\min\bigg(c_{max} - a \frac{ c_{max} - c_{min}}{K}k, c_{min} \bigg)
\end{align}
We considered the slope $a=\{2,3,4,5,6,7\}$ and in all cases $c_{max}=128$, $c_{min}=1$. 
%In Figure \ref{fig:comm_rate_per_epoch} we show the communication rate per epoch for each mechanism. In Figure \ref{fig:comm_floats_per_epoch}, compute the number of communicated floating points per edge in a GNN with feature size $128$ and $256$ activations in the middle layers. In Figure \ref{fig:aggregated_comm_floats} we compute the total number of floating points communicated throughout a whole training session of $300$ epochs.
% \begin{figure*}
% 	\begin{subfigure}{0.33\textwidth}
% 		\centering
% 		\includegraphics[width = \textwidth]{figures/schedulers/comm_rate_per_epoch.pdf} 
% 		\caption{Compression rate per epoch}
% 		\label{fig:comm_rate_per_epoch}
% 	\end{subfigure}%
% 	\begin{subfigure}{0.33\textwidth}
% 		\includegraphics[width = \textwidth]{figures/schedulers/comm_floats_per_epoch.pdf} 
% 		\caption{Floating points comm. per epoch}
% 		\label{fig:comm_floats_per_epoch}
% 	\end{subfigure}
% 	\begin{subfigure}{0.33\textwidth}
% 		\includegraphics[width = \textwidth]{figures/schedulers/comm_floats_aggregated.pdf} 
% 		\caption{Aggregated floating points}
% 		\label{fig:aggregated_comm_floats}
% 	\end{subfigure}
% 	\caption{Compression rate and floating point communicated per epoch.}
% \end{figure*}
% \begin{figure*}
% 	\begin{subfigure}{0.5\textwidth}
% 		\centering
% 		\includegraphics[width = \textwidth]{figures/AccVsEdges/Acc_vs_PercentCrossEdges_arxiv.pdf} 
% 		\caption{Arxiv}
% 		\label{fig:acc_vs_self_arxiv}
% 	\end{subfigure}%
% 	\begin{subfigure}{0.5\textwidth}
% 		\includegraphics[width = \textwidth]{figures/AccVsEdges/Acc_vs_PercentCrossEdges_prods.pdf} 
% 		\caption{Products}
% 		\label{fig:acc_vs_self_prods}
% 	\end{subfigure}
% 	\caption{Accuracy as a function of the percentage of self-edges.}
%         \label{fig:acc_vs_self}
% \end{figure*}

%\begin{figure*}[h]
%	\centering
%	\begin{subfigure}{0.33\textwidth}
%		\includegraphics[width = \textwidth]{figures/schedulers/comm_rate_per_epoch.pdf} 
%		\subcaption{Compression rate per epoch}
%		\label{fig:comm_rate_per_epoch}
%	\end{subfigure}
%	\begin{subfigure}{0.33\textwidth}
%		\includegraphics[width = \textwidth]{figures/schedulers/comm_floats_per_epoch.pdf} 
%		\subcaption{Floating points per epoch}
%		\label{fig:comm_floats_per_epoch}
%	\end{subfigure}
%	\caption{\ref{fig:comm_rate_per_epoch} Compression rate per epoch for the difference training mechanisms. \ref{fig:comm_floats_per_epoch} Floating point numbers communicated per epoch per edge between machines.}
%\end{figure*}



%\begin{figure}
%	\centering
%	\includegraphics[width =0.6 \textwidth]{figures/schedulers/comm_floats_aggregated.pdf} 
%	\caption{Accumulated number of communicated floating point numbers for the different training mechanisms.}
%	\label{fig:accuracy_epoch}
%\end{figure}



%\subsection{Robustness to Scheduler Selection}
%Our Algorithm $\algo$ shows a solid robustness to the choice of slope $a$, and $c_{min}$. We tested several choices of schedulers, and in all of them, the accuracy of the learned GNN matches the one of the no communication, at a fraction of the time. 


% Please add the following required packages to your document preamble:
% \usepackage{multirow}
% Please add the following required packages to your document preamble:
% \usepackage{multirow}
% Please add the following required packages to your document preamble:
% \usepackage{multirow}
% Please add the following required packages to your document preamble:
% \usepackage{multirow}
% Please add the following required packages to your document preamble:
% \usepackage{multirow}
% Please add the following required packages to your document preamble:
% \usepackage{multirow}
% Please add the following required packag
\section{Partitioning Details}
In all cases, the partitions had the same number of nodes in each partition. In Table \ref{table:edges} we show the number of edges in each server, and across servers. As can be seen, the number of cross edges in METIS partitioning is always smaller than random, which makes sense given the objective of the METIS algorithm. Another important aspect is that as the number of partitions increases, the cross-partition number of edges increases and correspondingly the self-partition decreases. This is why the degradation happens, local graphs are smaller, and more communication is needed.
% Please add the following required packages to your document preamble:
% \usepackage{multirow}
\begin{table*}
  \small
\centering
\begin{tabular}{cc|cccc|}
\hline
\multicolumn{1}{c|}{\multirow{3}{*}{\begin{tabular}[c]{@{}c@{}}Edge\\ Type\end{tabular}}} & \multirow{3}{*}{Partitioning} & \multicolumn{4}{c}{Number of Servers}                                                               \\ \cline{3-6} 
\multicolumn{1}{c|}{}                              &                               & \multicolumn{4}{c}{OGBN-Products}                                                                   \\ \cline{3-6} 
\multicolumn{1}{c|}{}                              &                               & \multicolumn{1}{c|}{$2$}     & \multicolumn{1}{c|}{$4$}     & \multicolumn{1}{c|}{$8$}     & \multicolumn{1}{c}{$16$}     \\ \hline
\multicolumn{1}{c|}{Self }                & METIS            & \multicolumn{1}{c|}{$122019051(96.71\%)$} & \multicolumn{1}{c|}{$118533121(93.95\%)$} & \multicolumn{1}{c|}{$113962769(90.33\%)$} & \multicolumn{1}{c}{$110067019(87.24\%)$} \\ \hline
\multicolumn{1}{c|}{Self }                & Random                    & \multicolumn{1}{c|}{$64302907(50.97\%)$} & \multicolumn{1}{c|}{$33378937(26.46\%)$} & \multicolumn{1}{c|}{$17913873(14.2\%)$} & \multicolumn{1}{c}{$10179253(8.07\%)$} \\ \hline
\multicolumn{1}{c|}{Cross }               & METIS                     & \multicolumn{1}{c|}{$4148258(3.29\%)$} & \multicolumn{1}{c|}{$7634188(6.05\%)$} & \multicolumn{1}{c|}{$12204540(9.67\%)$} & \multicolumn{1}{c}{$16100290(12.76\%)$} \\ \hline
\multicolumn{1}{c|}{Cross }               & Random                     & \multicolumn{1}{c|}{$61864402(49.03\%)$} & \multicolumn{1}{c|}{$92788372(73.54\%)$} & \multicolumn{1}{c|}{$108253436(85.8\%)$} & \multicolumn{1}{c}{$115988056(91.93\%)$} \\ \hline
\multicolumn{2}{c|}{}                                                               & \multicolumn{4}{c}{OGBN-Arxiv}                                                                      \\ \hline
\multicolumn{1}{c|}{Self }                & METIS                      & \multicolumn{1}{c|}{$2173087(87.45\%)$} & \multicolumn{1}{c|}{$2038291(82.03\%)$} & \multicolumn{1}{c|}{$1864471(75.03\%)$} & \multicolumn{1}{c}{$1677943(67.52\%)$} \\ \hline
\multicolumn{1}{c|}{Self }                & Random                    & \multicolumn{1}{c|}{$1326581(53.38\%)$} & \multicolumn{1}{c|}{$749367(30.16\%)$} & \multicolumn{1}{c|}{$459233(18.48\%)$} & \multicolumn{1}{c}{$314967(12.68\%)$} \\ \hline
\multicolumn{1}{c|}{Cross }               & METIS                      & \multicolumn{1}{c|}{$311854(12.55\%)$} & \multicolumn{1}{c|}{$446650(17.97\%)$} & \multicolumn{1}{c|}{$620470(24.97\%)$} & \multicolumn{1}{c}{$806998(32.48\%)$} \\ \hline
\multicolumn{1}{c|}{Cross }               & Random               & \multicolumn{1}{c|}{$1158360(46.62\%)$} & \multicolumn{1}{c|}{$1735574(69.84\%)$} & \multicolumn{1}{c|}{$2025708(81.52\%)$} & \multicolumn{1}{c}{$2169974(87.32\%)$} \\ \hline
\end{tabular}
\caption{\label{table:edges} Number of self-edges and cross-edges for the different settings considered. }

\end{table*}

% \section{Cross-Edge analysis}

% In this section, we study the impact of the number of cross-edges on the graph. We define a cross-edge, as an edge that connects two nodes that are in different partitions. 
% Note that the number of cross-edges is related to the number of partitions, the density of the graph, and the partition method. In Table \ref{table:edges} we show the number and percentage of cross-edges for each graph, partition method, and number of partitions. It can be seen that METIS has a significantly smaller number of cross-edges than random. It is also true, that the number of cross-edges increases with the number of partitions. Also note that OGBN-Products graph has an average degree of $25.26$, whereas OBGN-Papers $6.89$. This means that OBGN-Products graph is sparser than Arxiv, which explains why using METIS partitioning, the number of cross-edges is smaller in OGBN-Products. 

% In Figure \ref{fig:acc_vs_self}, we plot the accuracy as a function of the percentage of cross-edges. In this plot, we only considered the percentage of cross-edges, and therefore, the partition mechanism and number of partitions are not stated. But, as Table \ref{table:edges} indicates, the $4$ points with more cross-edges correspond to random partitions with $2,4,8,16$ partitions respectively. Likewise, the $4$ points with less cross-edges correspond to METIS partitioning with $2,4,8,16$ partitions respectively.



% A salient conclusion is that in all cases as the percentage of self-edges increases the accuracy deteriorates. This is related to the fact that as the number of cross-edges increases the local data is less representative of the whole graph. Another conclusion drawn from \ref{fig:acc_vs_self} is that there is an ordering in the performance from less compression (Fixed compression $2$) to no communication. This is related to the fact that less compression transmits more information over the cross-edges than no communication. The greater the number of cross-edges the role of the compression becomes more important. 

% Regarding the sparsity of the datasets, the two plots show different behaviors. In the sparser graph \ref{fig:acc_vs_self_arxiv}, as edges are added, the accuracy increases, almost linearly. This is related to the fact that the cross-edges are not redundant in the information they bring from far-away nodes. On the other hand, the denser graph \ref{fig:acc_vs_self_prods} shows a saturation around $50\%$. That is to say, as the number of cross-edges decreases below $50\%$, the improvement in accuracy is not significant. This is related to the density of the graph, as given its large degree, nodes tend to be redundant. 



\subsection{Accuracy}
The variable compression mechanism recovers the no communication accuracy in all cases considered. There is no difference in the accuracy obtained with variable compression, and full communication. This is true, for all numbers of servers considered, and all partitions as can be seen in Tables \ref{table:results_random}, and \ref{table:results_metis} for METIS and random partition respectively. 

% Please add the following required packages to your document preamble:
% \usepackage{multirow}
\begin{table*}
\centering
	\begin{tabular}{c|cccc|cccc}
		\hline
		\multirow{3}{*}{Algorithm}    & \multicolumn{4}{c|}{OGBN-Products}                                                                   & \multicolumn{4}{c}{OGBN-Arxiv}                                                                      \\ \cline{2-9} 
		& \multicolumn{4}{c}{Number of Servers}                                                               & \multicolumn{4}{c}{Number of Servers}                                                               \\ \cline{2-9} 
		& \multicolumn{1}{c|}{$2$}     & \multicolumn{1}{c|}{$4$}     & \multicolumn{1}{c|}{$8$}     & $16$    & \multicolumn{1}{c|}{$2$}     & \multicolumn{1}{c|}{$4$}     & \multicolumn{1}{c|}{$8$}     & $16$    \\ \hline
Full Comm& \multicolumn{1}{c|}{$78.40$} & \multicolumn{1}{c|}{$78.41$} & \multicolumn{1}{c|}{$78.31$} & \multicolumn{1}{c|}{$78.19$} & \multicolumn{1}{c|}{$69.20$} & \multicolumn{1}{c|}{$69.86$} & \multicolumn{1}{c|}{$69.75$} & \multicolumn{1}{c}{$69.16$} \\ \hline
No Comm& \multicolumn{1}{c|}{$77.08$} & \multicolumn{1}{c|}{$74.96$} & \multicolumn{1}{c|}{$72.22$} & \multicolumn{1}{c|}{$69.10$} & \multicolumn{1}{c|}{$64.67$} & \multicolumn{1}{c|}{$61.22$} & \multicolumn{1}{c|}{$55.95$} & \multicolumn{1}{c}{$54.52$} \\ \hline
\textbf{Variable Comp. Slope $2$(ours)}& \multicolumn{1}{c|}{$78.28$} & \multicolumn{1}{c|}{$78.32$} & \multicolumn{1}{c|}{$78.13$} & \multicolumn{1}{c|}{$78.20$} & \multicolumn{1}{c|}{$69.21$} & \multicolumn{1}{c|}{$69.30$} & \multicolumn{1}{c|}{$69.89$} & \multicolumn{1}{c}{$69.80$} \\ \hline
\textbf{Variable Comp. Slope $3$(ours)}& \multicolumn{1}{c|}{$78.56$} & \multicolumn{1}{c|}{$78.81$} & \multicolumn{1}{c|}{$78.61$} & \multicolumn{1}{c|}{$78.79$} & \multicolumn{1}{c|}{$69.24$} & \multicolumn{1}{c|}{$69.48$} & \multicolumn{1}{c|}{$70.21$} & \multicolumn{1}{c}{$70.07$} \\ \hline
\textbf{Variable Comp. Slope $4$(ours)}& \multicolumn{1}{c|}{$78.47$} & \multicolumn{1}{c|}{$78.81$} & \multicolumn{1}{c|}{$78.53$} & \multicolumn{1}{c|}{$78.64$} & \multicolumn{1}{c|}{$69.47$} & \multicolumn{1}{c|}{$69.51$} & \multicolumn{1}{c|}{$69.95$} & \multicolumn{1}{c}{$69.90$} \\ \hline
\textbf{Variable Comp. Slope $5$(ours)}& \multicolumn{1}{c|}{$78.67$} & \multicolumn{1}{c|}{$78.62$} & \multicolumn{1}{c|}{$78.14$} & \multicolumn{1}{c|}{$78.67$} & \multicolumn{1}{c|}{$69.93$} & \multicolumn{1}{c|}{$69.80$} & \multicolumn{1}{c|}{$70.12$} & \multicolumn{1}{c}{$69.99$} \\ \hline
\textbf{Variable Comp. Slope $6$(ours)}& \multicolumn{1}{c|}{$78.71$} & \multicolumn{1}{c|}{$78.66$} & \multicolumn{1}{c|}{$78.56$} & \multicolumn{1}{c|}{$78.55$} & \multicolumn{1}{c|}{$69.49$} & \multicolumn{1}{c|}{$69.59$} & \multicolumn{1}{c|}{$70.09$} & \multicolumn{1}{c}{$70.02$} \\ \hline
\textbf{Variable Comp. Slope $7$(ours)}& \multicolumn{1}{c|}{$78.38$} & \multicolumn{1}{c|}{$78.70$} & \multicolumn{1}{c|}{$78.45$} & \multicolumn{1}{c|}{$78.71$} & \multicolumn{1}{c|}{$69.21$} & \multicolumn{1}{c|}{$69.89$} & \multicolumn{1}{c|}{$69.90$} & \multicolumn{1}{c}{$69.81$} \\ \hline
Fixed Comp Rate $2$& \multicolumn{1}{c|}{$78.35$} & \multicolumn{1}{c|}{$78.13$} & \multicolumn{1}{c|}{$78.00$} & \multicolumn{1}{c|}{$77.85$} & \multicolumn{1}{c|}{$66.04$} & \multicolumn{1}{c|}{$64.97$} & \multicolumn{1}{c|}{$64.34$} & \multicolumn{1}{c}{$62.68$} \\ \hline
Fixed Comp Rate $4$& \multicolumn{1}{c|}{$78.60$} & \multicolumn{1}{c|}{$77.50$} & \multicolumn{1}{c|}{$76.08$} & \multicolumn{1}{c|}{$74.62$} & \multicolumn{1}{c|}{$66.30$} & \multicolumn{1}{c|}{$63.93$} & \multicolumn{1}{c|}{$63.79$} & \multicolumn{1}{c}{$62.21$} \\ \hline
% Fixed Comp Rate $8$& \multicolumn{1}{c|}{$78.98$} & \multicolumn{1}{c|}{$76.80$} & \multicolumn{1}{c|}{$72.91$} & \multicolumn{1}{c|}{$69.57$} & \multicolumn{1}{c|}{$65.83$} & \multicolumn{1}{c|}{$63.38$} & \multicolumn{1}{c|}{$61.81$} & \multicolumn{1}{c}{$61.37$} \\ \hline
% Fixed Comp Rate $16$& \multicolumn{1}{c|}{$78.15$} & \multicolumn{1}{c|}{$75.62$} & \multicolumn{1}{c|}{$70.73$} & \multicolumn{1}{c|}{$68.88$} & \multicolumn{1}{c|}{$66.90$} & \multicolumn{1}{c|}{$63.50$} & \multicolumn{1}{c|}{$63.39$} & \multicolumn{1}{c}{$61.06$} \\ \hline
% Fixed Comp Rate $32$& \multicolumn{1}{c|}{$78.04$} & \multicolumn{1}{c|}{$74.22$} & \multicolumn{1}{c|}{$71.42$} & \multicolumn{1}{c|}{$67.73$} & \multicolumn{1}{c|}{$66.78$} & \multicolumn{1}{c|}{$65.66$} & \multicolumn{1}{c|}{$60.65$} & \multicolumn{1}{c}{$61.66$} \\ \hline
% Fixed Comp Rate $64$& \multicolumn{1}{c|}{$78.10$} & \multicolumn{1}{c|}{$75.11$} & \multicolumn{1}{c|}{$69.51$} & \multicolumn{1}{c|}{$65.86$} & \multicolumn{1}{c|}{$65.79$} & \multicolumn{1}{c|}{$63.79$} & \multicolumn{1}{c|}{$62.15$} & \multicolumn{1}{c}{$62.77$} \\ \hline
	\end{tabular}
\caption{\label{table:results_random} Accuracy results when training GNNs with full-communication, no communication, fixed and variable compression in both OGBN-Arxiv, and OGBN-Products. We test our Algorithm with $2,4,8$ and $16$ clients with \textbf{random partitioning} of the graph. }
\end{table*}

% Please add the following required packages to your document preamble:
% \usepackage{multirow}
\begin{table*}
\centering
\begin{tabular}{c|cccccccc}
\hline
\multirow{3}{*}{Algorithm}    & \multicolumn{4}{c}{OGBN-Products}                                                                                        & \multicolumn{4}{|c}{OGBN-Arxiv}                                                                      \\ \cline{2-9} 
                              & \multicolumn{8}{c}{Number of Servers}                                                                                                                                                                                           \\ \cline{2-9} 
                              & \multicolumn{1}{c|}{$2$}     & \multicolumn{1}{c|}{$4$}     & \multicolumn{1}{c|}{$8$}     & \multicolumn{1}{c|}{$16$}    & \multicolumn{1}{c|}{$2$}     & \multicolumn{1}{c|}{$4$}     & \multicolumn{1}{c|}{$8$}     & $16$    \\ \hline
Full Comm& \multicolumn{1}{c|}{$78.61$} & \multicolumn{1}{c|}{$78.79$} & \multicolumn{1}{c|}{$78.53$} & \multicolumn{1}{c|}{$78.25$} & \multicolumn{1}{c|}{$68.81$} & \multicolumn{1}{c|}{$69.63$} & \multicolumn{1}{c|}{$68.88$} & \multicolumn{1}{c}{$68.98$} \\ \hline
No Comm& \multicolumn{1}{c|}{$78.01$} & \multicolumn{1}{c|}{$78.39$} & \multicolumn{1}{c|}{$78.13$} & \multicolumn{1}{c|}{$77.62$} & \multicolumn{1}{c|}{$67.90$} & \multicolumn{1}{c|}{$66.77$} & \multicolumn{1}{c|}{$67.69$} & \multicolumn{1}{c}{$65.63$} \\ \hline
\textbf{Variable Comp. Slope $2$(ours)}& \multicolumn{1}{c|}{$78.43$} & \multicolumn{1}{c|}{$78.12$} & \multicolumn{1}{c|}{$78.80$} & \multicolumn{1}{c|}{$78.57$} & \multicolumn{1}{c|}{$69.25$} & \multicolumn{1}{c|}{$68.72$} & \multicolumn{1}{c|}{$69.08$} & \multicolumn{1}{c}{$69.14$} \\ \hline
\textbf{Variable Comp. Slope $3$(ours)}& \multicolumn{1}{c|}{$78.50$} & \multicolumn{1}{c|}{$78.26$} & \multicolumn{1}{c|}{$78.21$} & \multicolumn{1}{c|}{$78.31$} & \multicolumn{1}{c|}{$69.89$} & \multicolumn{1}{c|}{$69.48$} & \multicolumn{1}{c|}{$68.86$} & \multicolumn{1}{c}{$69.58$} \\ \hline
\textbf{Variable Comp. Slope $4$(ours)}& \multicolumn{1}{c|}{$78.73$} & \multicolumn{1}{c|}{$78.01$} & \multicolumn{1}{c|}{$78.13$} & \multicolumn{1}{c|}{$78.58$} & \multicolumn{1}{c|}{$69.39$} & \multicolumn{1}{c|}{$68.96$} & \multicolumn{1}{c|}{$68.75$} & \multicolumn{1}{c}{$69.16$} \\ \hline
\textbf{Variable Comp. Slope $5$(ours)}& \multicolumn{1}{c|}{$78.59$} & \multicolumn{1}{c|}{$78.51$} & \multicolumn{1}{c|}{$78.03$} & \multicolumn{1}{c|}{$78.35$} & \multicolumn{1}{c|}{$69.49$} & \multicolumn{1}{c|}{$69.04$} & \multicolumn{1}{c|}{$69.11$} & \multicolumn{1}{c}{$68.78$} \\ \hline
\textbf{Variable Comp. Slope $6$(ours)}& \multicolumn{1}{c|}{$78.21$} & \multicolumn{1}{c|}{$78.61$} & \multicolumn{1}{c|}{$78.50$} & \multicolumn{1}{c|}{$78.68$} & \multicolumn{1}{c|}{$69.33$} & \multicolumn{1}{c|}{$68.60$} & \multicolumn{1}{c|}{$69.41$} & \multicolumn{1}{c}{$69.36$} \\ \hline
\textbf{Variable Comp. Slope $7$(ours)}& \multicolumn{1}{c|}{$78.32$} & \multicolumn{1}{c|}{$78.39$} & \multicolumn{1}{c|}{$78.36$} & \multicolumn{1}{c|}{$78.56$} & \multicolumn{1}{c|}{$69.24$} & \multicolumn{1}{c|}{$69.28$} & \multicolumn{1}{c|}{$68.12$} & \multicolumn{1}{c}{$69.05$} \\ \hline
Fixed Comp Rate $2$& \multicolumn{1}{c|}{$78.49$} & \multicolumn{1}{c|}{$78.49$} & \multicolumn{1}{c|}{$78.18$} & \multicolumn{1}{c|}{$78.46$} & \multicolumn{1}{c|}{$67.68$} & \multicolumn{1}{c|}{$67.76$} & \multicolumn{1}{c|}{$66.73$} & \multicolumn{1}{c}{$65.78$} \\ \hline
Fixed Comp Rate $4$& \multicolumn{1}{c|}{$78.62$} & \multicolumn{1}{c|}{$78.31$} & \multicolumn{1}{c|}{$78.42$} & \multicolumn{1}{c|}{$78.34$} & \multicolumn{1}{c|}{$67.75$} & \multicolumn{1}{c|}{$66.64$} & \multicolumn{1}{c|}{$66.96$} & \multicolumn{1}{c}{$66.09$} \\ \hline
% Fixed Comp Rate $8$& \multicolumn{1}{c|}{$78.44$} & \multicolumn{1}{c|}{$78.24$} & \multicolumn{1}{c|}{$78.83$} & \multicolumn{1}{c|}{$78.72$} & \multicolumn{1}{c|}{$68.16$} & \multicolumn{1}{c|}{$67.97$} & \multicolumn{1}{c|}{$66.09$} & \multicolumn{1}{c}{$67.19$} \\ \hline
% Fixed Comp Rate $16$& \multicolumn{1}{c|}{$78.71$} & \multicolumn{1}{c|}{$78.72$} & \multicolumn{1}{c|}{$78.39$} & \multicolumn{1}{c|}{$78.44$} & \multicolumn{1}{c|}{$68.32$} & \multicolumn{1}{c|}{$67.65$} & \multicolumn{1}{c|}{$67.13$} & \multicolumn{1}{c}{$67.08$} \\ \hline
% Fixed Comp Rate $32$& \multicolumn{1}{c|}{$78.65$} & \multicolumn{1}{c|}{$78.67$} & \multicolumn{1}{c|}{$78.45$} & \multicolumn{1}{c|}{$78.03$} & \multicolumn{1}{c|}{$68.98$} & \multicolumn{1}{c|}{$66.68$} & \multicolumn{1}{c|}{$67.11$} & \multicolumn{1}{c}{$67.01$} \\ \hline
% Fixed Comp Rate $64$& \multicolumn{1}{c|}{$78.55$} & \multicolumn{1}{c|}{$78.48$} & \multicolumn{1}{c|}{$78.34$} & \multicolumn{1}{c|}{$78.68$} & \multicolumn{1}{c|}{$67.88$} & \multicolumn{1}{c|}{$67.56$} & \multicolumn{1}{c|}{$67.81$} & \multicolumn{1}{c}{$67.00$} \\ \hline
\end{tabular}
\caption{\label{table:results_metis} Accuracy results when training GNNs with full-communication, no communication, fixed and variable compression in both OGBN-Arxiv, and OGBN-Products. We test our Algorithm with $2,4,8$ and $16$ clients with \textbf{METIS partitioning} of the graph. }
\end{table*}


\subsection{Proof of Proposition \ref{prop:fixed_compression}}
\label{appendix:proposition_fixed_compression}
To begin with, we need to show these three lemmas. 

\begin{lemma}[GNN Function Difference]\label{lemma:func_diff}
	Under the assumptions of Proposition \ref{prop:fixed_compression}, the output of an $L$-layer GNN with $F$ and coefficients and $K$ filter taps per layer can be bounded by,
	\begin{align}
		||\Phi(\bbX_1,\bbS;\ccalH) - \Phi(\bbX_2,\bbS;\ccalH)  ||\leq \lambda_{\max}^L ||\bbX_1-\bbX_2||
	\end{align}
\end{lemma}
\begin{proof}[of Lemma \ref{lemma:func_diff}]
	Starting with the first layer of the GNN, and considering a single feature $||\bbx_{l1}-\bbx_{l2}||$, we can look into the difference between the successive layers as follows, 
	\begin{align}
		||\bbX_{l1}-\bbX_{l2}|| =& ||\non\bigg(\sum_{k=0}^{K-1}\bbH_k \bbS^k\bbx_{1}\bigg)-\non\bigg(\sum_{k=0}^{K-1}\bbH_k \bbS^k\bbX_{2}\bigg)||\\
		\leq& ||\sum_{k=0}^{K-1}\bbH_k \bbS^k\bbX_{1}-\sum_{k=0}^{K-1}\bbH_k \bbS^k\bbX_{2}||\label{eqn:normalized_lips}\\
  % \text{ normalized Lipschitz assumption \ref{as:normalized_lipschitz}}\\
		\leq& \lambda_{\max}||\bbX_{1}-\bbX_{2}|| \label{eqn:normalized_lips_filters}%\text{ normalized filters assumption \ref{as:filter_bounded}}
	\end{align}
 Where \eqref{eqn:normalized_lips} holds by normalized Lipschitz assumption \ref{as:normalized_lipschitz}, and \eqref{eqn:normalized_lips_filters} holds by the normalized filters assumption \ref{as:filter_bounded}
	By repeating the recursion over $L$ layers we attain the desired result.
\end{proof}
%%%%%%%%%%%%%%%%%%%%%%%%%%%%%%%%

\begin{lemma}[GNN Gradient Difference]\label{lemma:grad_diff} Under the assumptions of Proposition \ref{prop:fixed_compression}, the output of an $L$-layer GNN with $F$ and coefficients and $K$ filter taps per layer can be bounded by,
	\begin{align}
		&||\nabla_\ccalH\Phi(\bbX_1,\bbS;\ccalH) - \nabla_\ccalH\Phi(\bbX_2,\bbS;\ccalH)  ||\\
  &\leq 2\lambda_{\max} \sqrt{KFL} ||\bbX_1-\bbX_2|| \nonumber
	\end{align}
\end{lemma}
\begin{proof}[of Lemma \ref{lemma:grad_diff}]
	Starting with the first layer of the GNN, note that the derivative of the GNN with respect to any parameter in the first layer is the value of the polynomial. By denoting $h_v$ an element on the first layer of the GNN, the derivative with respect to the first layer is, 
	\begin{align}
		\nabla_{h_v}\non\bigg(\sum_{k=0}^{K-1}\bbH_k \bbS^k\bbX_{1}\bigg)=\non\bigg(\sum_{k=0}^{K-1}\bbH_k \bbS^k\bbX_{1}\bigg)  \bbS^k\bbX_{1}.
	\end{align}
	By taking the difference we get, 
	\begin{align}
		&||\nabla_{k_v}\non\bigg(\sum_{k=0}^{K-1}\bbH_k \bbS^k\bbx_{1}\bigg)-\nabla_{k_v}\non\bigg(\sum_{k=0}^{K-1}\bbH_k \bbS^k\bbX_{2}\bigg)||\\
		&=||\non\bigg(\sum_{k=0}^{K-1}\bbH_k \bbS^k\bbX_{1}\bigg)  \bbS^k\bbX_{1}-\non\bigg(\sum_{k=0}^{K-1}\bbH_k \bbS^k\bbX_{2}\bigg)  \bbS^k\bbX_{2}||\\
		&\leq ||\non\bigg(\sum_{k=0}^{K-1}\bbH_k \bbS^k\bbX_{1}\bigg) \bigg(  \bbS^k\bbX_{1}- \bbS^k\bbX_{2}\bigg)||\\
  % \text{ triangle inequality }\\
		&+||\bigg(\non\bigg(\sum_{k=0}^{K-1}\bbH_k \bbS^k\bbX_{1}\bigg) -\non\bigg( \sum_{k=0}^{K-1}\bbH_k \bbS^k\bbX_{2}\bigg)\bigg)  \bigg( \bbS^k\bbX_{2}\bigg)||
	\end{align}
	Now, given that the activation is normalized Lipschitz by assumption \ref{as:normalized_lipschitz}, the signals are normalized, and that the filter is normalized by assumption \ref{as:filter_bounded}, we can bound this term by, 
	\begin{align}
		&||\nabla_{k_v}\non\bigg(\sum_{k=0}^{K-1}\bbH_k \bbS^k\bbX_{1}\bigg)-\nabla_{k_v}\non\bigg(\sum_{k=0}^{K-1}\bbH_k \bbS^k\bbX_{2}\bigg)||\nonumber\\
  &\leq 2 \lambda_{\max}||\bbX_1 - \bbX_2 || 
	\end{align}
	By repeating the previous result for all layers, and all features and considering that the GNN has $KFL$ coefficients, we complete the proof. 
\end{proof}

%%%%%%%%%%%%%%%%%%%%%%%%%%%%%%%%
\begin{lemma}[Lipschitz Gradients with respect to the parameters]\label{lemma:lipschitz_loss_wrt_params}Under the assumptions of Proposition \ref{prop:fixed_compression}, the output of an $L$-layer GNN with $F$ and coefficients and $K$ filter taps per layer can be bounded by,
	\begin{align}
		&||\nabla_\ccalH\ell(\bby,\Phi(\bbx,\bbS;\ccalH_1)) - \nabla_\ccalH\ell(\bby,\Phi(\bbx,\bbS;\ccalH_2))  ||\nonumber\\
  &\leq 2ML ||\ccalH_1-\ccalH_2||
	\end{align}
\end{lemma}
\begin{proof}[of Lemma \ref{lemma:lipschitz_loss_wrt_params}] We begin by using the chain rule as follows, 
	\begin{align}
		&||\nabla_\ccalH\ell(\bby,\Phi(\bbx,\bbS;\ccalH_1)) - \nabla_\ccalH\ell(\bby,\Phi(\bbx,\bbS;\ccalH_2))  ||\\
		&=||\nabla\ell(\bby,\Phi(\bbx,\bbS;\ccalH_1))\nabla_\ccalH\Phi(\bbx,\bbS;\ccalH_1)\nonumber\\
  &- \nabla\ell(\bby,\Phi(\bbx,\bbS;\ccalH_2))\nabla_\ccalH\Phi(\bbx,\bbS;\ccalH_2)  ||\\
		&\leq||\bigg(\nabla\ell(\bby,\Phi(\bbx,\bbS;\ccalH_1)) \nonumber \\&-\nabla\ell(\bby,\Phi(\bbx,\bbS;\ccalH_2))\bigg)\nabla_\ccalH\Phi(\bbx,\bbS;\ccalH_2)  ||\\% \text{ triangle inequality }\\
		&+||\bigg(\nabla_\ccalH\Phi(\bbx_1,\bbS;\ccalH)-\nabla_\ccalH\Phi(\bbx,\bbS;\ccalH_2)\bigg)\nonumber\\
  &\nabla\ell(\bby,\Phi(\bbx,\bbS;\ccalH_1))|| 
	\end{align}
	Note that we consider the filters $\ccalH$ as a vector, where the coefficients have been concatenated. We can now use Cauchy-Schwartz to obtain, 
	\begin{align}
		&||\nabla_\ccalH\ell(\bby,\Phi(\bbx,\bbS;\ccalH_1)) - \nabla_\ccalH\ell(\bby,\Phi(\bbx,\bbS;\ccalH_2))  ||\\
		&\leq||\nabla\ell(\bby,\Phi(\bbx,\bbS;\ccalH_1))- \nabla\ell(\bby,\Phi(\bbx,\bbS;\ccalH_2))|| \nonumber\\
  &||\nabla_\ccalH\Phi(\bbx,\bbS;\ccalH_2)  || \\
		&+||\nabla_\ccalH\Phi(\bbx,\bbS;\ccalH_1)-\nabla_\ccalH\Phi(\bbx,\bbS;\ccalH_2)|| \nonumber\\
  &||\nabla\ell(\bby,\Phi(\bbx,\bbS;\ccalH_1))|| .
	\end{align}
	We can now use Assumptions \ref{as:Loss_Grad_Lipschitz}, and \ref{as:GNN_lipschitz}, to obtain
	\begin{align}
		&||\nabla_\ccalH\ell(\bby,\Phi(\bbx,\bbS;\ccalH_1)) - \nabla_\ccalH\ell(\bby,\Phi(\bbx,\bbS;\ccalH_2))  ||\nonumber\\
  &\leq 2ML ||\ccalH_1-\ccalH_2||
	\end{align}
	By denoting $L_\nabla = 2ML$ we complete the proof. 
	% Now we can use Assummption \ref{as:Loss_Grad_Lipschitz} and Assumption \ref{lemma:lipschitz_loss_wrt_params} for the first term, and Lemma \ref{lemma:grad_diff} for the second term to obtain, 
	% \begin{align}
		%     &||\nabla_\ccalH\ell(\bby,\Phi(\bbx_1,\bbS;\ccalH)) - \nabla_\ccalH\ell(\bby,\Phi(\bbx_2,\bbS;\ccalH))  ||\\
		% &\leq||\nabla\ell(\bby,\Phi(\bbx_1,\bbS;\ccalH))- \nabla\ell(\bby,\Phi(\bbx_2,\bbS;\ccalH))|| ||\nabla_\ccalH\Phi(\bbx_2,\bbS;\ccalH)  || \\
		% &+||\nabla_\ccalH\Phi(\bbx_1,\bbS;\ccalH)-\nabla_\ccalH\Phi(\bbx_2,\bbS;\ccalH)|| ||\nabla\ell(\bby,\Phi(\bbx_1,\bbS;\ccalH))|| 
		% \end{align}
\end{proof}

\begin{lemma}[Submartingale]\label{lemma:submartingale} 
	Consider the iterates generated by equation \ref{eqn:SGD} where the input vector $\bbx$ is compressed with error $\epsilon$ (cf. Definition \ref{eqn:compress_decompress}). Let the step-size  be $\eta\leq 1/\lipGrad$, if the compression error is such that, 
	\begin{align}\label{eqn:prop_submartingale_condition}
		\mbE_\ccalD[||\nabla_\ccalH \ell (y,\Phi(x,\bbS;\ccalH_t)) ||^2] \geq \lipGrad^2\epsilon^2
	\end{align}
	then the iterates satisfy that, 
	\begin{align}
		\mbE[\ell(y,\Phi(x,\bbS;\ccalH_{t+1}))] \leq \mbE[\ell (y,\Phi(x,\bbS;\ccalH_t))]
	\end{align}
\end{lemma}

\begin{proof}[of Lemma \ref{lemma:submartingale}]
	This proof follows the lines of \cite{bertsekas2000gradient}, and we start by defining a continuous function $g(\alpha)$ as follows, 
	\begin{align}
		g(\alpha)=\mbE[\ell(\bby,\Phi(\bbx,\bbS;\ccalH_t-\alpha\eta_t\nabla\ell(\bby,\Phi(\tilde \bbx,\bbS;\ccalH_t))))].
	\end{align}
	Note that, $g(0)=\mbE[\ell(\bby,\Phi(\bbx,\bbS;\ccalH_t))]$ and 
 
 $g(1)=\mbE[\ell(\bby,\Phi(\bbx,\bbS;\ccalH_{k+1}))]$, and also that the integral of $\frac{\partial}{\partial \alpha} g(\alpha)$ satisfies
	\begin{align}
		&g(1)-g(0)\nonumber\\
  &= \int_{0}^1 \frac{\partial}{\partial \alpha} g(\alpha) d\alpha \\
		&=  - \mbE[\eta\nabla_\ccalH\ell(\bby,\Phi(\tilde \bbx,\bbS;\ccalH_t))^\intercal \int_{0}^1\nabla_\ccalH\ell(\bby,\Phi(\bbx,\bbS;\ccalH_t\nonumber\\
  &-\alpha\eta\nabla_\ccalH\ell(\bby,\Phi(\tilde \bbx,\bbS;\ccalH_t))))d\alpha]\label{eqn:prop_submartingale_chain_rule}\\
		&=  - \mbE[\eta\nabla_\ccalH\ell(\bby,\Phi(\tilde \bbx,\bbS;\ccalH_t))^\intercal \int_{0}^1\nabla_\ccalH\ell(\bby,\Phi(\bbx,\bbS;\ccalH_t\nonumber\\
  &-\alpha\eta\nabla_\ccalH\ell(\bby,\Phi(\tilde \bbx,\bbS;\ccalH_t))))\\
		&+\nabla_\ccalH\ell(\bby,\Phi(\bbx,\bbS;\ccalH_t))-\nabla_\ccalH\ell(\bby,\Phi(\bbx,\bbS;\ccalH_t))d\alpha
		]\label{eqn:prop_submartingale_chain_rule_add_subtract}\\
		&=- \mbE[\eta\nabla_\ccalH\ell(\bby,\Phi(\tilde \bbx,\bbS;\ccalH_t)) ^\intercal\nabla_\ccalH\ell(\bby,\Phi(\bbx,\bbS;\ccalH_t))\nonumber\\
		&+\eta\nabla_\ccalH\ell(\bby,\Phi(\tilde \bbx,\bbS;\ccalH_t)) ^\intercal\int_{0}^1\nabla_\ccalH\ell(\bby,\Phi(\bbx,\bbS;\ccalH_t\nonumber\\
  &-\alpha\eta\nabla_\ccalH\ell(\bby,\Phi(\tilde \bbx,\bbS;\ccalH_t))))\nonumber\\
		&\quad\quad\quad\quad\quad-\nabla_\ccalH\ell(\bby,\Phi(\bbx,\bbS;\ccalH_t))d\alpha
		]\label{eqn:prop_submartingale_chain_rule_add_subtract_organize},
	\end{align}
	where \eqref{eqn:prop_submartingale_chain_rule} holds by the chain rule, \eqref{eqn:prop_submartingale_chain_rule_add_subtract} holds as we are adding and subtracting the same term, and \eqref{eqn:prop_submartingale_chain_rule_add_subtract_organize} is a rearrangement of terms. We can now utilize Cauchy-Schwartz to bound the difference as follows, 
	
	\begin{align}
		&\mbE_\ccalD[\ell(\bby,\Phi(\bbx,\bbS;\ccalH_{k+1}))-\ell(\bby,\Phi(\bbx,\bbS;\ccalH_t))]\nonumber\\
		&\leq - \mbE[\eta\nabla_\ccalH\ell(\bby,\Phi(\tilde \bbx,\bbS;\ccalH_t)) ^\intercal\nabla_\ccalH\ell(\bby,\Phi(\bbx,\bbS;\ccalH_t))\\
		&+\frac{\eta }{2}||\nabla_\ccalH\ell(\bby,\Phi(\tilde\bbx,\bbS;\ccalH_t))|| || \nabla_\ccalH\ell(\bby,\Phi(\bbx,\bbS;\ccalH_t\nonumber\\
  &-\eta\nabla_\ccalH\ell(\bby,\Phi(\tilde \bbx,\bbS;\ccalH_t))))\nonumber\\
  &-\nabla_\ccalH\ell(\bby,\Phi(\bbx,\bbS;\ccalH_t))||.\nonumber 
	\end{align}
	where the previous inequality holds given that $\int_0^1 \alpha^2 d\alpha=\frac{1}{2}$. We can utilize Lemma \ref{lemma:lipschitz_loss_wrt_params} to bound the difference between the gradients as follows, 
	\begin{align}
		&\mbE[\ell(\bby,\Phi(\bbx,\bbS;\ccalH_{k+1}))-\ell(\bby,\Phi(\bbx,\bbS;\ccalH_t))]\\
		&\leq \mbE[-\eta\nabla_\ccalH\ell(\bby,\Phi(\tilde \bbx,\bbS;\ccalH_t)) ^\intercal\nabla_\ccalH\ell(\bby,\Phi(\bbx,\bbS;\ccalH_t))\nonumber\\
  &+\frac{\lipGrad\eta ^2}{2}||\nabla_\ccalH\ell(\bby,\Phi(\tilde\bbx,\bbS;\ccalH_t))||^2]\nonumber .
	\end{align}
	% \red{hereRESUME}
	% Now, we can rearrange as follows,
	% \begin{align}
		%    &\mbE[\ell(\bby,\Phi(\bbx,\bbS;\ccalH_{k+1}))-\ell(\bby,\Phi(\bbx,\bbS;\ccalH_t))]\\
		%    &\leq - \mbE[2\langle\sqrt{\frac{\lipGrad\eta ^2}{2}}\nabla_\ccalH\ell(\bby,\Phi(\tilde \bbx,\bbS;\ccalH_t)) , \sqrt{\frac{2}{\lipGrad}}\nabla_\ccalH\ell(\bby,\Phi(\bbx,\bbS;\ccalH_t)) \rangle\\
		%    &+\langle\sqrt{\frac{\lipGrad\eta ^2}{2}}\nabla_\ccalH\ell(\bby,\Phi(\tilde\bbx,\bbS;\ccalH_t)),\sqrt{\frac{\lipGrad\eta ^2}{2}}\nabla_\ccalH\ell(\bby,\Phi(\tilde\bbx,\bbS;\ccalH_t))\rangle]\nonumber .
		% \end{align}
	% Knowing that for any two vectors, $\bba,\bbb$, $||\bba-\bbb||^2-||\bbb||^2=||\bba||^2-2\bba^\intercal\bbb$ given that the norm is induced by the inner product we obtain, 
	% \begin{align}
		%    &\mbE[\ell(\bby,\Phi(\bbx,\bbS;\ccalH_{k+1}))-\ell(\bby,\Phi(\bbx,\bbS;\ccalH_t))]\\
		%    &\leq - \frac{2}{\lipGrad}\mbE[||\nabla_\ccalH\ell(\bby,\Phi(\bbx,\bbS;\ccalH_t))||^2-||\frac{\lipGrad\eta }{2}\nabla_\ccalH\ell(\bby,\Phi(\tilde \bbx,\bbS;\ccalH_t)) - \nabla_\ccalH\ell(\bby,\Phi(\bbx,\bbS;\ccalH_t)) ||^2]\nonumber .
		% \end{align}
	
	
	% \red{here}
	
	Now, we can factor $-\eta/2$, and we obtain, 
	\begin{align}
		&\mbE[\ell(\bby,\Phi(\bbx,\bbS;\ccalH_{k+1}))-\ell(\bby,\Phi(\bbx,\bbS;\ccalH_t))]\\
		&\leq  \frac{-\eta}{2}\mbE[2\nabla_\ccalH\ell(\bby,\Phi(\tilde \bbx,\bbS;\ccalH_t)) ^\intercal\nabla_\ccalH\ell(\bby,\Phi(\bbx,\bbS;\ccalH_t))\nonumber\\
  &-||\nabla_\ccalH\ell(\bby,\Phi(\tilde\bbx,\bbS;\ccalH_t))||^2] \\
		&+\mbE[\frac{\lipGrad\eta ^2-\eta}{2}||\nabla_\ccalH\ell(\bby,\Phi(\tilde\bbx,\bbS;\ccalH_t))||^2]\nonumber .
	\end{align}
	Now by imposing the condition that $\eta<\frac{1}{\lipGrad}$, the second term can be ignored. Knowing that for any two vectors, $\bba,\bbb$, $||\bba-\bbb||^2-||\bbb||^2=||\bba||^2-2\bba^\intercal\bbb$ given that the norm is induced by the inner product we obtain, 
	\begin{align}
		&\mbE[\ell(\bby,\Phi(\bbx,\bbS;\ccalH_{k+1}))-\ell(\bby,\Phi(\bbx,\bbS;\ccalH_t))]\nonumber\\
		&\leq  \frac{-\eta}{2}\mbE[||\nabla_\ccalH\ell(\bby,\Phi(\bbx,\bbS;\ccalH_t))||^2\nonumber\\
  &-||\nabla_\ccalH\ell(\bby,\Phi(\tilde\bbx,\bbS;\ccalH_t))-\nabla_\ccalH\ell(\bby,\Phi(\bbx,\bbS;\ccalH_t))||^2] \nonumber .
	\end{align}
%	Now, we can partition the last element by adding and subtracting $\nabla_\ccalH\ell(\bby,\Phi(\bbx,\bbS;\ccalH_t))$ as follows, 
%\begin{align}
%&\mbE[\ell(\bby,\Phi(\bbx,\bbS;\ccalH_{k+1}))-\ell(\bby,\Phi(\bbx,\bbS;\ccalH_t))]\label{eqn:add_subtract}\\
%		&\leq  \frac{-\eta}{2}\bigg(\mbE[||\nabla_\ccalH\ell(\bby,\Phi(\bbx,\bbS;\ccalH_t))||^2] \nonumber\\
%  &-\mbE[||\nabla_\ccalH\ell(\bby,\Phi(\tilde\bbx,\bbS;\ccalH_t))-\nabla_\ccalH\ell(\bby,\Phi(\bbx,\bbS;\ccalH_t))+\nabla_\ccalH\ell(\bby,\Phi(\bbx,\bbS;\ccalH_t))-\nabla_\ccalH\ell(\bby,\Phi(\bbx,\bbS;\ccalH_t))||^2]\bigg)\nonumber .
%	\end{align}
% Now note that we consider the vectorized tensor $\ccalH$, and the norm in \ref{eqn:add_subtract} is induced by the innner product. Therefore, we can use the property $||a+b||^2\leq 3||a||^2 + 3||b||^2$, as follows,  
%	\begin{align}
%		&\mbE[\ell(\bby,\Phi(\bbx,\bbS;\ccalH_{k+1}))-\ell(\bby,\Phi(\bbx,\bbS;\ccalH_t))]\\
%		&\leq  \frac{-\eta}{2}\bigg(\mbE[||\nabla_\ccalH\ell(\bby,\Phi(\bbx,\bbS;\ccalH_t))||^2]-3\mbE[||\nabla_\ccalH\ell(\bby,\Phi(\tilde\bbx,\bbS;\ccalH_t))-\nabla_\ccalH\ell(\bby,\Phi(\bbx,\bbS;\ccalH_t))||^2] \\
%		&-3\mbE[||\nabla_\ccalH\ell(\bby,\Phi(\bbx,\bbS;\ccalH_t))-\nabla_\ccalH\ell(\bby,\Phi(\bbx,\bbS;\ccalH_t))||^2]\bigg)\nonumber ,
%	\end{align}
%	note that the cross terms are equal to zero given that $\mbE[\nabla_\ccalH\ell(\bby,\Phi(\bbx,\bbS;\ccalH_t))]=\mbE[\nabla_\ccalH \ell(\bby,\Phi(\bbx,\bbS;\ccalH_t))]$. 
 Finally, by Lemma \ref{lemma:grad_diff}, and compression mechanism \ref{def:CompressionDecompression}, 
	\begin{align}
		&\mbE[\ell(\bby,\Phi(\bbx,\bbS;\ccalH_{k+1}))-\ell(\bby,\Phi(\bbx,\bbS;\ccalH_t))]\\
		&\leq  \frac{-\eta}{2}\bigg(\mbE[||\nabla_\ccalH\ell(\bby,\Phi(\bbx,\bbS;\ccalH_t))||^2]-\lipGrad^2\epsilon^2\bigg)\nonumber .
	\end{align}
	
	
	By imposing the condition in \ref{eqn:prop_submartingale_condition} we complete the proof. 
\end{proof}


\begin{proof}[of Proposition \ref{prop:fixed_compression}]
	To begin with, for every $\beta$ we define the stopping time $K$ as
	\begin{align}
		K=\min_{k\geq 0} \{\mbE[||\nabla_\ccalH \ell(y,\Phi(\bbx,\bbS;\ccalH_t)) ||^2\leq \lipGrad^2\epsilon_k^2 +\beta^2]\}
	\end{align}
	We need to show that $\mbE[k^*]$ is of order $\ccalO(1/\beta)$. To do so, we start by taking the difference between the last iterate $K$ and the first one as follows, 
	\begin{align}
		&\mbE[\ell(\bby,\Phi(\bbx,\bbS;\ccalH_0))-\ell(\bby,\Phi(\bbx,\bbS;\ccalH_t))]\\
		&=\mbE_{K}[\mbE[\sum_{k=1}^K\ell(\bby,\Phi(\bbx,\bbS;\ccalH_{k-1}))-\ell(\bby,\Phi(\bbx,\bbS;\ccalH_t)) ]]\\
		&=\sum_{t=0}^\infty\mbE[\sum_{k=1}^t\ell(\bby,\Phi(\bbx,\bbS;\ccalH_{k-1}))\nonumber\\
  &-\ell(\bby,\Phi(\bbx,\bbS;\ccalH_t)) ]P(K=t)\label{eqn:propostion_convergence_termwise_summation}
	\end{align}
	Now, we know that for all $t\leq K$, we have that, 
	\begin{align}
		\mbE[\sum_{k=1}^K\ell(\bby,\Phi(\bbx,\bbS;\ccalH_{k-1}))-\ell(\bby,\Phi(\bbx,\bbS;\ccalH_t)) ]\geq \eta \beta .\label{eqn:propostion_convergence_difference_beta}
	\end{align}
	We can now substitute condition \ref{eqn:propostion_convergence_difference_beta} into equation \ref{eqn:propostion_convergence_termwise_summation} to obtain, 
	\begin{align}
		&\mbE[\ell(\bby,\Phi(\bbx,\bbS;\ccalH_0))-\ell(\bby,\Phi(\bbx,\bbS;\ccalH_t))]\\
		&\geq\sum_{t=0}^\infty \eta \beta K P(K=t)\geq\beta \eta \mbE[K].
	\end{align}
	Given that the loss function is non-negative, and dividing in both sides of the previous inequality by $\beta \eta$, we complete the proof.
	
\end{proof}



\section{Proof of Proposition \ref{prop:scheduler}}\label{appendix:proof_scheduler}
The sketch of this proof is as follows, first, we construct a martingale by multiplying the norm of the gradient by the condition that we want to satisfy. Second, we show that this construction is effectively a martingale. Third, we show that it converges. Finally, we show what the limit of this convergent martingale is.

% To begin with, an alternative way of showing that Proposition \ref{prop:scheduler} is true, is by showing that the inferior limit (i.e. $\lim\inf$) of the expected value of the norm of the gradient is $0$. Therefore, if 
% \begin{align}
	% 	\lim \inf_{k\to\infty} \mbE[|| \nabla_\ccalH \ell (y,\phi(x,\bbS;\ccalH_t))] = 0,
	% \end{align}
% we can show by contradiction, that for every $\delta>0$, and $k_0$, there exists a value of $K$, such that $\mbE[|| \nabla_\ccalH \ell (y,\phi(x,\bbS;\ccalH_t))] \leq \delta$. 

To begin with, we define the filtration $\ccalF_t$ by iterates generated according to \eqref{eqn:SGD}, and the sequence $X_t$ as follows, 
\begin{align}
	&X_t = ||\nabla_\ccalH \ell (y,\phi(x,\bbS;\ccalH_t))||^2\bbone[||\nabla_\ccalH \ell (y,\phi(x,\bbS;\ccalH_t))||^2\nonumber\\
 &\geq  L_\nabla^2 \epsilon^2_{t^{'}}+\sigma, t^{'}\leq t],
\end{align}
where $\bbone[\cdot]$ is the indicator function. The expected value of $|X_t|$ is bounded by Assumption \ref{as:Loss_Grad_Lipschitz}, and $X_t$ is adapted to the filtration generated by the iterates of \ref{eqn:SGD}. 
By \cite{durrett2019probability}, to show that $X_n$ is a super-martingale, we require, 
\begin{align}
	\mbE[X_{t+1}|\ccalF_t]\leq X_t.
\end{align}
By contradiction, we can argue that $\mbE[X_{t+1}|\ccalF_t]> X_t$. Now, if $\mbE[X_{t+1}|\ccalF_t]> X_t$ for every $t>t_0$, then it must be the case that $X_{t_0}>L_\nabla^2 \epsilon^2_{t_0}+\sigma$. If this is not the case, the indicator function will make the sequence equal to $0$ for all $t>t_0$, disproving the contradiction. Given that $X_{t_0}>L_\nabla^2 \epsilon^2_{t^{'}}+\sigma$, we can fix $\epsilon_0$, and by Proposition \ref{prop:fixed_compression}, we arrive at a contradiction, and therefore $X_n$ is a super-martingale.

Given that the $X_t$ is bounded below by $0$, by the  Martingale Convergence Theorem \cite[Theorem 4.2.1]{durrett2019probability}, $X_t$ converges. 

Now it remains to show that the limit of $\lim_{t\to\infty}X_t=0$. We can assume that this is not true. We can therefore assume that $\lim_{t\to\infty}X_t=A<\infty$ with $A>\sigma$. Now, we know that the scheduler decreases, and therefore $\exists t^*:L^2_\nabla\epsilon^2_{t^*}+\sigma<A$. In this case again, $X_t$ cannot be larger that  $L^2_\nabla\epsilon^2_{t^*}+\sigma$ forever by Proposition \ref{prop:fixed_compression}. Therefore, there is no $A>\sigma$ such that $X_t$ converges to. Which implies that $X_n\to 0$.

To finalize, given that we made no assumptions over the initial time $t$, $||\nabla_\ccalH \ell (y,\phi(x,\bbS;\ccalH_t))||^2\leq \sigma$ happens infinitely often, completing the proof. 

%%%%%%%%%%%%%%%%%%%%%%%%%%%%%%%%%%%%%%%%%%%%%%%%%%%%%%%%%%%%%%%%%%%%%%%%%%%%%%%%%%%%%%%%%%%%%%%%%%%%%%%%%%%%%%%%%%%%%%%%%%%%%%%%%%%%%%%%%%%%%%%%%%%%%%%%%%%%%%%%%%%%%%%%%%%%%%%%%%%%%%%%%%%%%%%%%%%%%%%%%%%%%%%%%%%%%%%%%%%%%%%%%%%%%%%%%%%%%%%%%%%%%%%%%%%%%%%%%%%%%%%%%%%%%%%%%%%%%%%%%%%%%%%%%%%%%%%%%%%%%%%%%%%%%%%%%%%%%%%%%%%%%%%%%%%%%%%%%%%%%%%%%%%%%%%%%%%%%%%%%%%%%%%%%%%%%%%%%%%%%%%%%%%%%%%%%%%%%%%%%%%%%%%%%%%%%%%%%%%%

% \newpage

% \section{Varying Compression Rates}


% In this work, we propose to utilize varying compression rates while learning the GNN. At the beginning of the training, we utilize a large compression ratio and we reduce it as we train. Intuitively, our method proposes to increase the fidelity of the estimator of the gradient as the GNN approaches convergence. 

% Given that edges of the graph $\bbS$ exist between nodes own between different workers, when a gradient at worker $W_i$ is computed, data from workers $W_j$'s that posses nodes adjacent to the ones owned by $W_i$ are needed. In order to reduce the communication costs between workers, we propose to compress the information sent between them. To this end, in this work, we propose to communicate the activation between agents.
% %
% \begin{definition}
	% 	The compression and decompression mechanism $g_{\epsilon,r},g_{\epsilon^{-1},r}$ with compression error $\epsilon$, and rate $r$,  satisfies that given a set of parameters $x$, when compressed and decompressed, the following relation holds i.e.,
	% 	\begin{align}
		% 		&a= g_{\epsilon,r} (x),  \text{ and }\tilde x = g_\epsilon^{-1}(g_\epsilon( x)) \text{ and }\mbE[\tilde x - x]=0 \text{ with }\mbE[||\tilde x - x||^2]\leq\epsilon^2,
		% 	\end{align}
	% 	where $a\in \reals^m$ is the compressed signal with rate $r$, $\frac{m}{n}=r$. If $\epsilon=0$ we say that we compute a loss-less compression. 
	% \end{definition}
% %

% Returning to \ref{eqn:SGD}, in this paper we propose to make the updates on $\ccalH$ based on the decompressed signals $\tilde x$. To this end, each $W_i$ compresses its activation to obtain $a_j$. Next, it transmits the compressed activation to its neighbours. Upon receiving all the compressed activations, each worker decompresses  the information and computes the backward pass to obtain the stochastic gradient with which it computes the gradient step. In this paper, we propose to vary the compression rate $r_k$ across iterations. A more succinct description of the procedure can be found in Algorithm \ref{alg:varying_compr_rates}.




% The advantage of our procedure relies on the fact that the compressed data is transmitted between agents reducing the costs of communication. Given that the bottleneck is given by the communication times, compressing and decompressing information locally adds less overhead than transmitting the vector $x$.

% %%%%%%%%%%%%%%%%%%%%%%%%%%%%%%%%%%%%%%%%%%%%%%%%%%%%%%%%%%%%%%%%%%%%%%%%%%%%%%%%%%%%%%%%%%%%%%%%%%%%%%%%%%%%%%%%%%%%%%%%%%%%%%%%%%%%%%%%%%%%%%%%%%%%%%%%%%%%%%%%%%%%%%%%%%%%%%%%%%%%%%%%%%%%%%%%%%%%%%%%%%%%%%%%%%%%%%%%%%%%%%%%%%%%%%%%%%%%%%%%%%%%%%%%%%%%%%%%%%%%%%%%%%%%%%%%%%%%%%%%%%%%%%%%%%%%%%%%%%%%%%%%%%%%%%%%%%%%%%%%%%%%%%%%%%%%%%%%%%%%%%%%%%%%%%%%%%%%%%%%%%%%%%%%%%%%%%%%%%%%%%%%%%%%%%%%%%%%%%%%%%%%%%%%%%%%%%%%%%%%

% %\section{experiments}


% \section{Algorithm Convergence}
% In order to show convergence of Algorithm \ref{alg:varying_compr_rates}, we need to introduce three assumptions. 

% \begin{assumption}
	% 	The positive loss $\ell$ function has $L$ Lipschitz continuous gradients i.e., $||\nabla\ell(\bby_1,\bbz)- \nabla\ell(\Phi(\bby_2,\bbz)||\leq L||\bby_1-\bby_2||$.
	% \end{assumption}
% \begin{assumption}
	% 	The empirical estimator of the gradient $ \nabla_\ccalH \ell (y_i,\Phi(x_i,\bbS;\ccalH_t))$ is an unbiased estimator of the gradient $\nabla_\ccalH \ell (y_i,\Phi(x_i,\bbS;\ccalH_t))$, and the variance can be controlled by the number of samples in the batch $B$ as follows, 
	% 	\begin{align}
		% 		\mbE\bigg[|| \frac{1}{B}\sum_{i=1}^B\nabla_\ccalH \ell (y_i,\Phi(x_i,\bbS;\ccalH_t))- \nabla_\ccalH \ell (y,\Phi(x,\bbS;\ccalH_t))||^2\bigg]\leq \frac{\sigma^2}{B}
		% 	\end{align}
	% 	with $\infty>\sigma>0$ being the variance of the estimator. 
	% \end{assumption}
% \begin{assumption}
	% 	The graph convolutional filters in every layer of the graph neural network are bounded, i.e.
	% 	\begin{align}
		% 		||h_{*\bbS}x|| \leq ||x|| \lambda_{max} \bigg(\sum_{t=0}^T h_t \bbS^t\bigg),\text{ with } \lambda_{max} \bigg(\sum_{t=0}^T h_t \bbS^t\bigg)<\infty.
		% 	\end{align}
	% \end{assumption}


% Proposition \ref{prop:compression_rate} shows that the iterates generated by Algorithm \ref{alg:varying_compr_rates} form a supermartingale. If we run Algorithm \ref{alg:varying_compr_rates} sufficiently it will converge to a first order stationary point. Intuitively, as we reduce the value of the loss $\ell$, so does the magnitude of the gradient $||\nabla_\ccalH \ell (f_{\ccalH_{k+1}})||$, therefore, as we increase the number of epochs, we need to reduce the compression error. 


% \begin{proposition}[Convergence of \algo]
	% 	Consider the iterates generates by equation \ref{eqn:compressed_SGD} where the gradient is compressed with compression rate $r_k$ (cf. Definition \ref{eqn:compress_decompress}). Let the step-size  be $\eta\leq 1/\lipGrad$, if the compression error is such that at every step $k$, 
	% 	\begin{align}
		% 		\mbE[||\nabla_\ccalH \ell (y,\Phi(x,\bbS;\ccalH_t)) ||^2] \geq \frac{\lipGrad^2\epsilon_k^2}{B}+\frac{\sigma^2}{B} + \beta,
		% 	\end{align}
	% 	Then \algo\ converges to a first order stationary point in $K\leq \ccalO(\frac{1}{\beta})$ iterations. ,i.e., 
	% 	\begin{align}
		% 		\mbE[|| \nabla_\ccalH \ell (y,\Phi(x,\bbS;\ccalH_t))||^2]\leq \beta^2
		% 	\end{align}
	% \end{proposition}



% \subsection{Scheduler} \label{subsec:scheduler}




\end{document}
