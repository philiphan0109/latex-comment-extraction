\section{Conclusion}
Depression is a major mental health disorder. As such, detecting depression can have significant impacts across several domains. Towards this goal, we propose an affective mobile system that allows us to collect facial behavior primitives from faces by opportunistically capturing user faces by observing the user interaction with their phone in a naturalistic setting. To this end, we build a depressive episode prediction model that achieves 81\% of AUROC. Our regression model that estimates PHQ-9 scores reached a moderate level of accuracy, exhibiting an MAE of 3.08. Based on the results from our cross-validation, we found our model produces reliable performance from several weeks of data to detect depressive episodes. We highlight key behavior primitives differentiating depressive and non-depressive episodes and use case scenarios regarding how the system could be applicable in detecting mental and neurological disorders for researchers and stakeholders. Lastly, we discuss privacy and ethical considerations in deploying such a system.