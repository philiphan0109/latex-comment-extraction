\appendix

\section{Survey}

\subsection{PHQ-9}
Participants were asked: "Over the last two weeks, how often have you been bothered by the following problems?" This questionnaire is part of the standard assessment to gauge the severity of depressive symptoms. Below is the PHQ-9 questionnaire (Table \ref{tab:phq9}) used in the study:

\begin{table}[h]
\centering
\small
\caption{PHQ-9 Questionnaire Items}
\label{tab:phq9}
\begin{tabular}{>{\raggedright}p{0.5cm} >{\raggedright\arraybackslash}p{12cm}}
\toprule
\textbf{No.} & \textbf{Question} \\
\midrule
1 & Little interest or pleasure in doing things \\
2 & Feeling down, depressed, or hopeless \\
3 & Trouble falling or staying asleep, or sleeping too much \\
4 & Feeling tired or having little energy \\
5 & Poor appetite or overeating \\
6 & Feeling bad about yourself - or that you are a failure or have let yourself or your family down \\
7 & Trouble concentrating on things, such as reading the newspaper or watching television \\
8 & Moving or speaking so slowly that other people could have noticed. Or the opposite - being so fidgety or restless that you have been moving around a lot more than usual \\
9 & Thoughts that you would be better off dead, or of hurting yourself \\
\bottomrule
\end{tabular}
\end{table}

\subsection{Study Feedback} \label{A_Study_Feedback}
 Table \ref{tab:user_feedback} lists the user feedback questions administered at the end of the study to gauge participants' perceptions of the FacePsy app, focusing on aspects of consent, data collection triggers, impact on privacy, understanding of feature extraction, and long-term acceptance. The questions were designed to understand the participants' experiences throughout their interaction with the app and to gather suggestions for future improvements.

\begin{table}[h]
\centering
\small
\caption{User Feedback Questions}
\label{tab:user_feedback}
\begin{tabular}{>{\raggedright}p{0.5cm} >{\raggedright\arraybackslash}p{12cm}}
\toprule
\textbf{No.} & \textbf{Question} \\
\midrule
1 & When you first started using the FacePsy app, what were your initial thoughts about the facial data collection, especially when unlocking your phone or using specific apps? \\
2 & How were you informed about the data collection process, and did you feel that the consent process adequately addressed your concerns about privacy and data usage? \\
3 & Can you describe how you felt the first few times the app activated upon unlocking your phone or opening trigger apps? Did it become more acceptable over time, or did it remain a concern? \\
4 & Were there any particular trigger apps that made you more uncomfortable when the FacePsy app activated? How did this affect your usage of those apps? \\
5 & Did you notice any changes in your phone usage habits due to the app’s data collection methods? For example, did you use your phone less frequently or avoid certain apps? \\
6 & What are your thoughts on the app automatically discarding images after 20 seconds? Did this feature influence your comfort level with the ongoing data collection? \\
7 & How well do you understand the process of facial feature extraction by the app? Was there enough information provided about what data is extracted and how it is used? \\
8 & Do you trust that the facial data collected remains on your device and is not uploaded elsewhere? What could increase your trust in the system’s handling of your data? \\
9 & How has your perception of the FacePsy app and its data collection practices changed during the course of the study? \\
10 & What improvements or changes would you suggest for the app, especially regarding user control over data collection and privacy? \\
\bottomrule
\end{tabular}
\end{table}

