\section{Field Study: Understanding and Detecting Depressive Episodes}
To obtain the feasibility of our proposed FacePsy framework in the field study, we address the following question: (RQ1) can the facial behavior features collected by our mobile sensing system be effectively utilized to detect depressive episodes in a naturalistic environment? (Section \ref{RQ1}). To understand sets of depression-related biomarkers that could be used in the wild we address (RQ2) What’s the significance of different biomarkers on depression detection differentiating depressive vs. non-depressive episodes in real-world settings? We introduce the top 27 facial behavior features.

\begin{table}[h]
\centering
\small
\caption{\label{tab:corr}Summary of Feature Correlations with Depressive Episodes}  % <-- Your table caption here
\setlength{\tabcolsep}{4pt}  % adjust to fit your needs
% \small  % or \footnotesize if needed
\begin{tabular}{lp{1cm}rp{2.5cm}p{3.1cm}}
\toprule
Feature & p-value (<0.05) & r-value & Depressive Episode Mean (SD) & Non-Depressive Episode Mean (SD) \\
\midrule
                 ear\_right\_sum\_morning &                   0.00 &     0.35 &                23.34 (28.87) &                      8.26 (11.1) \\
                  ear\_left\_sum\_morning &                   0.00 &     0.34 &                21.36 (25.82) &                     7.99 (10.83) \\
          headEulerAngle\_Y\_sum\_morning &                   0.00 &    -0.33 &             -404.03 (719.02) &                  -33.33 (332.65) \\
    leftEyeOpenProbability\_sum\_morning &                   0.00 &     0.33 &                55.41 (69.64) &                     21.04 (28.1) \\
   rightEyeOpenProbability\_sum\_morning &                   0.00 &     0.31 &                 48.74 (62.6) &                    19.99 (25.73) \\
                    AU15\_min\_afternoon &                   0.00 &     0.27 &                  0.09 (0.14) &                      0.04 (0.04) \\
   rightEyeOpenProbability\_std\_evening &                   0.00 &     0.26 &                  0.24 (0.07) &                       0.2 (0.07) \\
        smilingProbability\_sum\_morning &                   0.00 &     0.26 &                  5.52 (9.62) &                      1.98 (3.52) \\
 rightEyeOpenProbability\_std\_afternoon &                   0.00 &     0.23 &                  0.24 (0.07) &                       0.2 (0.07) \\
                     AU17\_sum\_midnight &                   0.00 &    -0.23 &                  6.36 (13.6) &                    21.02 (34.59) \\
                   AU12\_median\_morning &                   0.00 &    -0.22 &                  0.29 (0.21) &                       0.4 (0.27) \\
                      AU07\_std\_evening &                   0.00 &     0.22 &                  0.26 (0.08) &                      0.22 (0.07) \\
                      AU02\_std\_evening &                   0.00 &     0.21 &                  0.21 (0.07) &                      0.17 (0.08) \\
        smilingProbability\_max\_evening &                   0.00 &     0.21 &                   0.35 (0.2) &                      0.25 (0.21) \\
                   AU07\_median\_morning &                   0.00 &    -0.21 &                  0.59 (0.24) &                       0.68 (0.2) \\
        smilingProbability\_max\_morning &                   0.00 &     0.21 &                   0.33 (0.2) &                       0.24 (0.2) \\
      smilingProbability\_mean\_midnight &                   0.00 &     0.21 &                  0.14 (0.12) &                      0.09 (0.09) \\
                       AU12\_q3\_morning &                   0.00 &    -0.21 &                  0.39 (0.23) &                       0.5 (0.27) \\
                     AU12\_mean\_morning &                   0.00 &    -0.21 &                  0.31 (0.19) &                      0.41 (0.24) \\
    leftEyeOpenProbability\_std\_evening &                   0.00 &     0.20 &                  0.23 (0.07) &                       0.2 (0.07) \\
          headEulerAngle\_X\_std\_evening &                   0.00 &     0.20 &                   3.86 (1.4) &                      3.25 (1.33) \\
                      AU06\_max\_morning &                   0.01 &    -0.20 &                  0.56 (0.27) &                      0.66 (0.23) \\
        smilingProbability\_std\_evening &                   0.00 &     0.20 &                  0.09 (0.05) &                      0.06 (0.06) \\
      smilingProbability\_std\_afternoon &                   0.00 &     0.20 &                  0.08 (0.06) &                      0.06 (0.06) \\
     smilingProbability\_mean\_afternoon &                   0.00 &     0.20 &                  0.11 (0.09) &                      0.07 (0.09) \\
                     AU06\_mean\_morning &                   0.01 &    -0.20 &                  0.29 (0.19) &                      0.38 (0.21) \\
       smilingProbability\_mean\_evening &                   0.00 &     0.20 &                   0.1 (0.08) &                      0.07 (0.07) \\
\bottomrule
\end{tabular}
\end{table}



\subsection{Statistical difference between depressive and non-depressive episodes}

The analysis evaluated the correlation of various features with depressive episodes (See Table \ref{tab:corr}). The features were ranked based on the strength of their correlation (r-value) with the target variable, which indicates the presence of a depressive episode. Out of 1280 only 158 features had a p-value less than 0.05. Selecting features with at least a weak correlation (r-value >= abs(0.20)), streamlining analysis by prioritizing meaningful relationships, and enhancing model interpretability and efficiency while reducing noise and the risk of overfitting. Temporal dynamics of depressive episodes suggest features measured in the morning often show significant correlations, pointing to the potential impact of depression on morning routines or states, such as reduced facial expressiveness or specific eye movement patterns. The feature \textit{headEulerAngle\_Y\_sum\_morning} has the strongest negative correlation with depressive episodes, with an r-value of -0.33. This suggests that as the value of this feature decreases, the likelihood of a depressive episode increases. The Y rotation of the head translates to the yaw of the Euler angle. Other features with notable negative correlations include \textit{AU17\_sum\_midnight}, \textit{AU12\_median\_morning}, \textit{AU12\_q3\_morning} \textit{AU07\_median\_morning}, \textit{AU06\_max\_morning} and \textit{AU06\_mean\_morning}. The absence of \textit{AU06}, associated with expressing emotions related to happiness or joy, correlates with depression. Furthermore, \textit{AU07}, \textit{AU12} and \textit{AU17} have been linked to depression severity, supporting existing evidence \cite{song2020spectral, gavrilescu2019predicting}.

The feature \textit{ear\_right\_sum\_morning} and \textit{ear\_left\_sum\_morning} shows a strong positive correlation with depressive episodes, with an r-value of 0.35 and 0.34, respectively. This indicates that as the value of this feature increases, the likelihood of a depressive episode also increases. It's very important to consider the temporal dynamics of these features. Other features with significant positive correlations include \textit{leftEyeOpenProbability\_sum\_morning}, \textit{rightEyeOpenProbability\_sum\_morning}, \textit{rightEyeOpenProbability\_std\_evening} and \textit{leftEyeOpenProbability\_std\_evening}. The presence of strong EAR, eye open probability related to high alertness \cite{abe2023perclos} in morning and evening could be explained as the “eveningness–morningness” dimension in depression \cite{chelminski1999analysis}. The preference for morning or evening can largely be attributed to the reduction of depressive symptoms such as low energy, avoidance of social interaction, and loss of interest in previously pleasurable activities \cite{putilov2017state}. 

The presence of a positive correlation (r-values of 0.26, 0.21 for morning sum and maximum, respectively; and similarly positive correlations for evening and midnight measures) between \textit{smilingProbability} and depressive episodes suggests that higher smiling probabilities are associated with an increased likelihood of depressive episodes. This interpretation may seem counterintuitive since it's expected that depressive episodes would be associated with less smiling. However, this unexpected positive correlation doesn't necessarily imply that smiling more leads to depression or vice versa. It might reflect complex underlying behaviors or compensatory mechanisms, such as "smiling depression," where individuals might smile or maintain a facade of happiness in social situations despite experiencing depressive symptoms internally \cite{vanswearingen1999specific}. However, as a limitation of our study we are not able to confirm if participants are going such cases. Previous research suggests that depression is not only associated with sad facial expressions but also with “a total lack of facial expression corresponding to the lack of affective experience” \cite{ellgring2007non}. Since we collect short segments (10 sec) of data, it can be interpreted in various ways, e.g., a smile may be a result of feeling happy or feeling helpless, as suggested by prior research \cite{song2020spectral}. While a positive correlation between \textit{smilingProbability} and depressive episodes seems paradoxical, it highlights the complexity of depressive behaviors and the importance of considering broader psychological and situational contexts when interpreting these findings.



\subsection{Model development from data in the field study } \label{RQ1}

\subsubsection{Universal model}
Table \ref{tab:results_features_lopo_lgbm} summarizes the predictive performance of universal models. To understand how different subsets of facial behavior features contribute to detecting depression, we evaluated nine different models, each with a different face feature set, using LightGBM. The model using the most significant features from the correlation analysis performed the best, followed by the model that included feature selection. This approach enhances the model's interpretability and comprehension. The TSF model achieved 51\% accuracy, with a precision of 40\%, indicating that it correctly predicted depression 40\% of the time. A recall of 96\% suggests that out of all the depressive episode cases in the dataset, the model successfully identifies 96\% of them as positive. This is particularly important in depression detection, where missing out on positive cases leads to missing out on opportunities to intervene. The model's reliability is also reflected in an AUROC score of 0.67 (Fig. \ref{fig:roc_features_lopo_lgbm}). In terms of regression metrics, the MAE for each model also provides insights into the quantitative accuracy of depression severity estimation, showing the lowest error (3.26) for the Action Units model, which suggests its superior ability to estimate the severity of depression correctly compared to other models, where TSF model got an MAE of 5.13. While this indicates limited ability to distinguish between depression and no-depression classes, it represents better agreement between the model's predictions and actual observations than a random classifier.

\begin{figure}[h]
    \includegraphics[scale=0.3]{Figs/ROC-feature-lopo-lgbm.png}
    \caption{The ROC plots show the universal model performance of each feature type model.}
    \label{fig:roc_features_lopo_lgbm}
\end{figure}

\begin{table}[h]
\caption{\label{tab:results_features_lopo_lgbm} Universal Model Performance: We trained eight LGBM models for predicting depression, including a different feature subset. The model trained using all features showed the best results in predicting depression} 

\footnotesize
\centering  
    \begin{tabular}{p{3.4cm}p{1cm}p{1cm}p{1cm}p{1cm}p{1cm}p{1cm}p{1.7cm}}
    \toprule
    \textbf{Model} & \textbf{MAE} & \textbf{Accuracy} & \textbf{Precision} & \textbf{Recall} & \textbf{F1} & \textbf{AUROC} & \textbf{No. of Features} \\ 
    \midrule
    
    Eye Open Probability (EOP)  & 5.20 & 0.33\ & 0.31\ & 0.83\ & 0.45\ & 0.33\  &  64 \\
    
    Smiling Probability (SP) & 5.32 & 0.37\ & 0.33\ & 0.87\ & 0.48\ & 0.52\ & 32 \\
    
    Head Euler Angle (HEA) & 4.71 & 0.29\ & 0.28\ & 0.71\ & 0.40\ & 0.27\ & 96 \\

    Action Units (AU) & \textbf{3.26} & 0.38\ & 0.30\ & 0.66\ & 0.42\ & 0.45\ & 384 \\

    Eye-aspect ratio (EAR) & 5.31 & 0.32\ & 0.30\ & 0.80\ & 0.44\ & 0.35\ & 64 \\

    Inter-vector angle (IVA) & 4.59 & 0.40\ & 0.31\ & 0.66\ & 0.42\ & 0.43\ &  640 \\
    
    Top Significant Features (TSF) & 5.13 & \textbf{0.51}\ & \textbf{0.40}\ & \textbf{0.96}\ & \textbf{0.56}\ & \textbf{0.67}\ & 27 \\

    Feature Selection (FS) & 4.04 & 0.50\ & 0.38\ & 0.77\ & 0.51\ & 0.57\ &  46 \\
    
    All features & 3.77 & 0.40\ & 0.28\ & 0.51\ & 0.36\ & 0.40\ & 1280 \\
    
    \bottomrule
 
\end{tabular}
\end{table}

\subsubsection{Hybrid model}
Table \ref{tab:results_features_lopdo} summarizes the predictive performance of hybrid models. The model with the best performance is the one using feature selection with LightGBM. In the context of detecting depressive episodes, the model demonstrated a commendable performance with an accuracy of 69\%. Notably, when predicting a depressive episode, it was correct 57\% of the time, as indicated by a precision for the depressive class. Furthermore, it successfully identified 62\% of all actual depressive episodes, reflected by a recall. The F1-score, a measure of the model's balance between precision and recall, was 0.67 for depressive episodes, suggesting a harmonized performance despite the inherent class imbalance. The model's reliability was also underscored by an AUROC of 0.81, indicating a strong ability to distinguish between the two classes and a good agreement between the model's predictions and actual observations. The regression results further enhance our understanding, with the model achieving an MAE of 3.08 on the PHQ-9 scale, which ranges from 0 to 27. This indicates that the model's depression severity predictions are typically within approximately three points of the actual clinical assessments, showing relatively moderate accuracy in quantifying the severity of depressive symptoms.

\begin{figure}[h]
    \includegraphics[scale=0.3]{Figs/ROC-feature-lopdo-lgbm.png}
    \caption{The ROC plots show the hybrid model performance of each feature type model.}
    \label{fig:roc_features_lopdo_lgbm}
\end{figure}

\begin{table}[h]
\caption{\label{tab:results_features_lopdo} Hybrid Model Performance : We trained eight LGBM models for predicting depression, including a different feature subset. The model trained using all features showed the best results in predicting depression} 

\footnotesize
\centering  
    \begin{tabular}{p{3.4cm}p{1cm}p{1cm}p{1cm}p{1cm}p{1cm}p{1cm}p{1.7cm}}
    \toprule
    \textbf{Model} & \textbf{MAE} & \textbf{Accuracy} & \textbf{Precision} & \textbf{Recall} & \textbf{F1} & \textbf{AUROC} & \textbf{No. of Features} \\ 
    \midrule
    
    Eye Open Probability (EOP) & 3.16 & 0.67\ & 0.50\ & 0.48\ & 0.49\ & 0.66\  &  64 \\

    Smiling Probability (SP) & 3.26 & 0.64\ & 0.46\ & 0.43\ & 0.44\ & 0.63\ &  32 \\
    
    Head Euler Angle (HEA) & 3.08 & \textbf{0.72}\ & 0.60\ & 0.51\ & 0.55\ & 0.75\ &  96 \\

    Action Units (AU) & 3.02 & 0.67\ & 0.50\ & 0.35\ & 0.41\ & 0.67\ &  384 \\
    
    Eye-aspect ratio (EAR) & 3.37 & 0.64\ & 0.45\ & 0.41\ & 0.43\ & 0.63\ &  64 \\

    Inter-vector angle (IVA) & 3.57 & 0.59\ & 0.35\ & 0.28\ & 0.31\ & 0.55\ &  640 \\

    Top Significant Features (TSF) & 3.18 & 0.70\ & 0.55\ & 0.55\ & 0.55\ & 0.77\ & 27 \\

    Feature Selection (FS) & 3.08 & 0.69\ & \textbf{0.57}\ & \textbf{0.62}\ & \textbf{0.67}\ & \textbf{0.81}\ &  46 \\

    All features & \textbf{2.81} & 0.71\ & 0.59\ & 0.39\ & 0.47\ & 0.75\ &  1280 \\
    
    \bottomrule
 
\end{tabular}
\end{table}


Overall, the performance of the models varied, but the one using selected features showed the best results in predicting depression. From the AUROC plot (Fig. \ref{fig:roc_features_lopdo_lgbm}), we can observe even though model with HEA achieved better results in terms of accuracy of 72\%, the model itself is stable when combined with other features its yields much better results with more predictive performance stabilization.


\subsubsection{Minimum number of days needed to produce reliable detection}



The AUROC is a metric used to evaluate the performance of a diagnostic test, with values ranging from 0.5 to 1. A value greater than 0.5 is necessary for the test to be meaningful, and an AUROC of 0.7 or above is generally considered acceptable. In the context of a depression detection model, the performance was better than random guessing on day 1, with an AUROC of greated than 0.5. On day 1, the model's performance improved to a fair level with an AUROC of 62.4\%. Remarkably, starting from day 7, the model achieved an acceptable performance with an AUROC of 71.4\%. This progression illustrates (Fig \ref{fig:NumberofDaysPerformance}) a significant enhancement in the model's ability to accurately detect depression over the weeks.

\begin{figure}[h]
    \includegraphics[scale=0.35]{Figs/NumberofDaysPerformance-lopdo-lgbm.png}
    \caption{Minimum number of days needed to produce reliable detection}
    \label{fig:NumberofDaysPerformance}
\end{figure}

