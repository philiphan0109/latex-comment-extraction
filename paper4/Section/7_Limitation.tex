\section{Limitation and Future Work}
Even though we were able to get valuable insights about modeling depression and the proposed subset of facial and physiological signals, there are still improvements to be made for this system to be applicable in clinical settings. Although we successfully built a population model to detect depression, however, there might be individual patterns that our population model cannot capture, and thus may limit the generalizability of our model. In our future work, we will collect a larger dataset per participant and investigate the use of more personalized individual models. While we only register 11 categories of app use there could be more categories of app use that could work as the best avenue for data collection, further research should examine if such categories exist. The current limitation of our app, FacePsy, lies in its inadvertent triggering of data collection during intra-app navigation, such as moving between pages within the same app, leading to multiple data captures in a single session. In future work, we plan to refine the app's architecture to discern and limit data collection to significant user interactions, thereby enhancing the efficiency and relevance of the data collection process. Furthermore, in our future work, we want to integrate the pupillary response measurement module \cite{islam2024pupilsense} in our processing pipeline for in-app measurement of pupillary response by using Android Native libraries. In addition, we aim to enrich FacePsy by integrating it with systems like AWARE and Fitbit, combining rich emotional signals extracted from visual data with other data such as GPS, heart rate, and EDA. We also plan to add contextual layers, like categorizing apps that trigger data collection in model development which we collect as part of the FacePsy triggering mechanism, to deepen the understanding of facial behavior in context.



Our depression labeling strategy may raise concerns regarding the data's clarity and characteristics: Each session represents a unique data point with specific features. It's important to note that a single day could encompass multiple sessions. However, each session linked to a particular user will predict the different depressive episode labels span over two weeks observation period. This methodology presents two potential challenges. Firstly, the variability between sessions might make it challenging to identify a consistent pattern, which could affect the accuracy of predicting depressive episodes. Secondly, if sessions appear too homogenous, it could suggest that the model might be detecting implicit user characteristics rather than their actual risk of depression.







